% -*- latex -*- This is a LaTeX document.
% $Id: proposal.tex,v 1.6 1999-09-03 19:50:22 cananian Exp $
%%%%%%%%%%%%%%%%%%%%%%%%%%%%%%%%%%%%%%%%%%%%%%%%%%%%%%%%
\documentclass[12pt,twoside]{article}
%\usepackage{pdffonts}
\usepackage{beton}\usepackage{euler}
\usepackage{comdef}

% the name of the project
\linespread{1.1}

\title{{\large Master's Thesis proposal for}\\
FLEX: An Optimizing Java Compiler using SSI Form}
\author{C. Scott Ananian}
\date{March 1999\\\vspace{5pt}}

\begin{document}
\begin{titlepage}
\newcommand{\nl}{\\[0.5\baselineskip]}
\begin{centering}\large
Massachusetts Institute Of Technology\\
Department of Electrical Engineering and Computer Science\\
\vspace{0.5cm}
Proposal for Thesis Research in Partial Fulfillment\\
of the Requirements for the Degree of\\
Master of Engineering\\
\end{centering}
\vspace{0.25cm}
\begin{tabbing}
Title: \textbf{FLEX: An Optimizing Java Compiler using SSI Form}\nl
Submitted By: \= C. Scott Ananian \hspace{3cm}\=\rule{6cm}{0.5pt}\\
              \> Ashdown House, Room 510A     \>(signature)\\
              \> Cambridge, MA 02139\nl
Date of Submission: March 1999\nl
Expected Date of Completion: September 1999\nl
Laboratory where thesis will be done: Laboratory for Computer
Science%\nl
\end{tabbing}
Brief Statement of the Problem:

An investigation into the use of novel intermediate representations is
proposed in the context of a sophisticated compiler for the Java
programming language.  The compiler will be constructed and both
theoretical and practical methods will be used to evaluate the
effectiveness of the intermediate representation.  Analysis and
optimization techniques based on the representation will be
implemented and presented.

~\\Supervision Agreement:

The program outlined in the proposal is adequate for a Master's
thesis.  The supplies and facilities are available, and I am willing
to supervise the research and evaluate the thesis report.

\begin{tabbing}
\hspace{3.5in}\=\kill
\>\rule{2.5in}{0.5pt}\\
\>M. Rinard, Prof. of Comp. Sci.\\
\end{tabbing}

\end{titlepage}

The FLEX project is an ongoing effort to create intelligent
compiler technology to address embedded system and hardware design
issues for safe high-level languages, in particular, Java.
Current compilers cannot understand enough of the high-level structure
of a program to do reasonable optimization of threading and memory
management code, especially when pointers are present. Among the goals
of the FLEX project are automatic code partitioning for distributed
systems, memory allocation strategy conversion for embedded systems,
and the use of Java as a high-level hardware description language.

This thesis concentrates on the code representation used in the
FLEX compiler; specifically, on Static Single Information (SSI)
form, which is introduced as a more general replacement for the
prevalent SSA form.  SSI form is symmetrical for both forward and
reverse dataflow analysis, and represents information generated both
at variable definitions and at branches in control flow.  A rigorous
definition of minimal SSI form is presented, and algorithms to
generate this form in time linear to program size are
presented based on cycle equivalence analysis on the underlying
control flow graph.  The correctness of these algorithms is proved.

SSI form enables simple and efficient code optimization, with obvious
benefits over SSA form for constructs (such as null-pointer and
array-bounds checks) common in Java code.  In addition, a novel
bitwidth analysis is presented which is suitable for both hardware
synthesis and code generation for vector units such as Intel's MMX.
Loop analyses, reverse dataflow analysis, and constraint systems are
reexamined using SSI form, and a connection to dataflow computing is
shown.

Numerical results will be given supporting the time complexity of the
algorithms presented and showing the optimization improvements
possible using SSI form.

Finally, future work on the FLEX project is briefly discussed, and
SSI form is placed in the context of the project as a whole.

\end{document}
