% -*- latex -*- This is a LaTeX document.
% $Id: proposal.tex,v 1.3 1999-04-28 15:17:00 cananian Exp $
%%%%%%%%%%%%%%%%%%%%%%%%%%%%%%%%%%%%%%%%%%%%%%%%%%%%%%%%
\documentclass[12pt,notitlepage,twocolumn,twoside]{article}

% A breaking non-space for URLs
\newcommand{\bp}{\hspace{0pt}}

% phi and sigma functions
\newcommand{\phifunction}{$\phi$-function}
\newcommand{\sigfunction}{$\sigma$-function}

% the name of the project
\newcommand{\magic}{\textsc{Magic}}

\title{{\large Master's Thesis proposal for}\\
\magic: An Optimizing Java Compiler using SSI Form}
\author{C. Scott Ananian}
\date{\today \\ $ $Revision: 1.3 $ $}

\begin{document}
\pagestyle{myheadings}\markboth{$ $Revision: 1.3 $ $}{$ $Revision: 1.3 $ $}

\maketitle

The \magic\ project is an ongoing effort to create intelligent
compiler technology to address embedded system and hardware design
issues for safe high-level languages, in particular, Java.
Current compilers cannot understand enough of the high-level structure
of a program to do reasonable optimization of threading and memory
management code, especially when pointers are present. Among the goals
of the \magic\ project are automatic code partitioning for distributed
systems, memory allocation strategy conversion for embedded systems,
and the use of java as a high-level hardware description language.

This thesis concentrates on the code representation used in the
\magic\ compiler; specifically, on Static Single Information (SSI)
form, which is introduced as a more general replacement for the
prevalent SSA form.  SSI form is symmetrical for both forward and
reverse dataflow analysis, and represents information generated both
at variable definitions and at branches in control flow.  A rigorous
definition of minimal SSI form is presented, and algorithms to
generate this form in time linear to program size are
presented based on cycle equivalence analysis on the underlying
control flow graph.  The correctness of these algorithms is proved.

SSI form enables simple and efficient code optimization, with obvious
benefits over SSA form for constructs (such as null-pointer and
array-bounds checks) common in Java code.  In addition, a novel
bitwidth analysis is presented which is suitable for both hardware
synthesis and code generation for vector units such as Intel's MMX.
Loop analyses, reverse dataflow analysis, and constraint systems are
reexamined using SSI form, and a connection to dataflow computing is
shown.

Numerical results will be given supporting the time complexity of the
algorithms presented and showing the optimization improvements
possible using SSI form.

Finally, future work on the \magic\ project is briefly discussed, and
SSI form is placed in the context of the project as a whole.

\end{document}
