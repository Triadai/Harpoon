% CVSId: $Id: oqe.tex,v 1.1 2002-04-17 23:59:14 cananian Exp $
\documentclass[%
pdf,
colorBG,
slideColor,
%nocolorBG,
%slideBW,
%%%%%%%%%%%%%%%%%%%%%
%draft,
oqe
%frames
%azure
%contemporain
%nuancegris
%troispoints
%lignesbleues
%darkblue
%alienglow
%autumn
]{prosper}
\usepackage{amsmath}
\hypersetup{pdfpagemode=FullScreen}

\title{Size Optimizations for Java Programs}
%\author{\href{http://cscott.net}{C.~Scott~Ananian}}
\author{C.~Scott~Ananian}
\institution{%
Laboratory for Computer Science\\
Massachusetts Institute of Technology\\
cananian@lcs.mit.edu}

\begin{document}
\maketitle

%---------------------------------------------------------------------- SLIDE -
% \begin{slide}{The quest for $\pi$}
% \begin{itemize}
% \item The following formula computes $8$ correct digits per iteration 
%   (Ramanujan):
% \end{itemize}
%   \begin{small}
%   \begin{equation*}
%     \frac{1}{\pi}=\sum_{n=0}^\infty \frac{(\frac{1}{4})_n(\frac{2}{4})_n(\frac{3}{4})_n}{n!^3}\bigl(2\sqrt{2}(1103+26390n)\bigr)\frac{1}{(99^2)^{2n+1}}
%   \end{equation*}
%   \end{small}
% \end{slide}

%---------------------------------------------------------------------- SLIDE -
\begin{slide}{Our Goal}
\begin{center}
Reduce the memory consumption of object-oriented programs

\vspace{0.5cm}
\fontTitle{By}
\vspace{0.5cm}

Using program analysis to identify opportunities to reduce the space
required to store objects,

\vspace{0.5cm}
\fontTitle{Then}
\vspace{0.5cm}

Applying transformations to reduce the memory consumption of the program.
\end{center}
\end{slide}

%---------------------------------------------------------------------- SLIDE -
\begin{slide}{Structure of a Java Object}
\begin{itemize}
\item Typical of many O-O languages.
\end{itemize}
figure goes here.
\end{slide}

\end{document}

%%% Local Variables: 
%%% mode: latex
%%% TeX-master: t
%%% End: 
