% -*- latex -*- This is a LaTeX document.
% $Id: pldi99-tech.tex,v 1.6 1999-07-10 08:20:02 cananian Exp $
%%%%%%%%%%%%%%%%%%%%%%%%%%%%%%%%%%%%%%%%
\section{Technical Content} % RENAME ME!
% Technical Contents (definitions, algorithms, etc)
%  Program Representation
%   Sigmas, placement algorithms, theorems.
%  Constraint extraction and framework.
%  Contraint systems and resolution analysis (for example)
%   Bitwidth
%   Array bounds analysis, etc.

\subsection{Definition of SSI Form}
SSI form is an extension of the SSA form introduced in \cite{cytron89:ssa}.
Building SSI form involves adding pseudo-assignments for a variable $V$:
\begin{enumerate}
\item[$(\phi)$] at a control-flow merge when disjoint paths from a
conditional branch come together and at least one of the paths
contains a definition of $V$; and
\item[$(\sigma)$] at locations where control-flow splits and at least
one of the disjoint paths from the split uses the value of $V$.
\end{enumerate}

\subsection{Criteria for inserting \sigfunction{s}}
To minimize the number of \sigfunction{s}, there should be a
\sigfunction{} for variable $a$ at node $z$ of the flowgraph exactly
when:
\begin{enumerate}
\item node $x$ contains a use of $a$,
\item node $y$ contains a use of $a$,
\item there is a nonempty path $P_{zx}$ of edges from $z$ to $x$,
\item there is a nonempty path $P_{zy}$ of edges from $z$ to $y$, and
\item paths $P_{zx}$ and $P_{zy}$ do not have any node in common
except $z$ (that is, $z$ is the point of divergence for these paths).
\end{enumerate}
We will call this the \textit{path-convergence criterion} for
inserting \sigfunction{s}.  We consider the end node to contain an
implicit definition of every variable.

Note that this criterion is very similar to the path-con\-ver\-gence
criterion for inserting \phifunction{s} described in
\cite{appel:modern,cytron91:ssa}.  And, just in the \phifunction{}
case, the above criterion is open-ended: since the \sigfunction{}
itself counts as a use of $a$, we must iterate the above to determine
the final set of \sigfunction{s}.

Upon examination, we see that the path-con\-ver\-gence criteria for $\phi$- and
\sigfunction{s} interact.  Since \sigfunction{s} are variable
definitions and \phifunction{s} are variable uses, the set of
equations defined by the respective criteria must be iterated together 
in order to find the necessary function sets.  The total number of
$\phi$- and \sigfunction{s} remains linear, however: we can only place 
a single $\phi$- and/or \sigfunction{} per variable at any given
flowgraph node, so the total number of added functions is limited to 
$2 \cdot N \cdot V$.

Cytron et al.\ \cite{cytron91:ssa} has shown that the iterated
path-con\-ver\-gence criterion for \phifunction{s} is equivalent to the
\textit{iterated dominance frontier criterion}: whenever a node $x$
contains a definition of a variable $a$, then any node $z$ in the
dominance frontier of $x$ needs a \phifunction{} for a; nodes in the
dominance frontier of any added \phifunction{} similarly need
\phifunction{s}.  The equivalent statement for \sigfunction{s}
references \textit{uses} of the variable $a$ and nodes in its
post-dominance frontier. 

\subsection{Placement algorithms}
Sreedhar and Gao have shown \cite{sreedhar95:lintime} that it is
possible to place \phifunction{s} in time proportional to the size of
the program.  With appropriate modifications to the algorithm, it can
be used to place \sigfunction{s}.  However, as noted above, $\phi$-
and \sigfunction{} placement is not independent: the placement of
\phifunction{s} necessitates additional \sigfunction{} placement, and
vice versa.  Thus, the (linear time) placement algorithms can be run
iteratively to find a fixed point.  Since the maximum number of
$\phi$- or \sigfunction{s} is proportional to the size of the program,
it is obvious that no more than $N$ iterations will be required,
resulting in a worst-case running time of $O(N^2)$.  In practice, and
for structured control flow, running time is linear.

\begin{figure}
\begin{center}
\newcommand*{\figscale}{0.25}
\input{Figures/evil}
\end{center}
\caption{Pure Evil.}
\end{figure}
