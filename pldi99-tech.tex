% -*- latex -*- This is a LaTeX document.
% $Id: pldi99-tech.tex,v 1.7 1999-11-09 21:22:45 cananian Exp $
%%%%%%%%%%%%%%%%%%%%%%%%%%%%%%%%%%%%%%%%
\section{The Static Single Information form}
In this section we will provide a formal specification of SSI form and
its \emph{minimal} and \emph{pruned} variants.  We will also provide
efficient algorithms for constructing these representations.

% Technical Contents (definitions, algorithms, etc)
%  Program Representation
%   Sigmas, placement algorithms, theorems.
%  Constraint extraction and framework.
%  Contraint systems and resolution analysis (for example)
%   Bitwidth
%   Array bounds analysis, etc.

\subsection{Definition of SSI form}
SSI form is an extension of the SSA form introduced in \cite{cytron89:ssa}.
Building SSI form involves adding pseudo-assignments for a variable $V$:
\begin{enumerate}
\item[$(\phi)$] at a control-flow merge when disjoint paths from a
conditional branch come together and at least one of the paths
contains a definition of $V$; and
\item[$(\sigma)$] at locations where control-flow splits and at least
one of the disjoint paths from the split uses the value of $V$.
\end{enumerate}

\subsection{Criteria for inserting \sigfunction{s}}
To minimize the number of \sigfunction{s}, there should be a
\sigfunction{} for variable $a$ at node $z$ of the flowgraph exactly
when:
\begin{enumerate}
\item node $x$ contains a use of $a$,
\item node $y$ contains a use of $a$,
\item there is a nonempty path $P_{zx}$ of edges from $z$ to $x$,
\item there is a nonempty path $P_{zy}$ of edges from $z$ to $y$, and
\item paths $P_{zx}$ and $P_{zy}$ do not have any node in common
except $z$ (that is, $z$ is the point of divergence for these paths).
\end{enumerate}
We will call this the \textit{path-convergence criterion} for
inserting \sigfunction{s}.  We consider the start node to contain an
implicit definition of every variable, and the end node to contain an
implicit use of every variable.

Note that this criterion is very similar to the path-con\-ver\-gence
criterion for inserting \phifunction{s} described in
\cite{appel:modern,cytron91:ssa}.  And, just in the \phifunction{}
case, the above criterion is open-ended: since the \sigfunction{}
itself counts as a use of $a$, we must iterate the above to determine
the final set of \sigfunction{s}.

Upon examination, we see that the path-con\-ver\-gence criteria for $\phi$- and
\sigfunction{s} interact.  Since \sigfunction{s} are variable
definitions and \phifunction{s} are variable uses, the set of
equations defined by the respective criteria must be iterated together 
in order to find the necessary function sets.  The total number of
$\phi$- and \sigfunction{s} remains linear, however: we can only place 
a single $\phi$- and/or \sigfunction{} per variable at any given
flowgraph node, so the total number of added functions is limited to 
$2 \cdot N \cdot V$.

Cytron et al.\ \cite{cytron91:ssa} has shown that the iterated
path-con\-ver\-gence criterion for \phifunction{s} is equivalent to the
\textit{iterated dominance frontier criterion}: whenever a node $x$
contains a definition of a variable $a$, then any node $z$ in the
dominance frontier of $x$ needs a \phifunction{} for $a$; nodes in the
dominance frontier of any added \phifunction{} similarly need
\phifunction{s}.  The equivalent statement for \sigfunction{s}
references \textit{uses} of the variable $a$ and nodes in its
post-dominance frontier. 

\subsection{Variable renaming after \phisigfunction insertion}

Renaming variables after \phisigfunction insertion is done so as to
satisfy the following two conditions:\fixme{The first is similar to
the SSA form constraint on \phifunction{s}.  Skip it?}
\begin{enumerate}
\item For every node $x$ containing a definition of a variable $a$ in
the renamed program and node $y$ containing a use of that variable, there
exists at least one non-empty path $P_{xy}$ of edges from $x$ to $y$
and no such path contains a definition of $a$ other than at $x$.
\item For every pair of nodes $x$ and $y$ containing uses of a
variable $a$ defined at a node $z$ in the renamed program, either
every nonempty path $P_{zx}$ of edges from $z$ to $x$ must contain
node $y$, or every nonempty path $P_{zy}$ of edges from $z$ to $y$
must contain $x$.
\end{enumerate}

In addition, correctness requres that:
\begin{enumerate}
\item Along any possible control-flow path in a program being executed
consider any use of a variable $a$ in the original program and the
corresponding use of $a_i$ in the renamed program.  Then, at every
occurrence of the use on the path, $a$ and $a_i$ have the same value.
The path need not be cycle-free.
\end{enumerate}

\subsection{Minimal and pruned SSI forms}

% write me!

\section{SSI construction algorithms}

Construction of SSI form takes place in two phases.  First, the required
\phisigfunction{s} for each variable are inserted at control-flow
merge and split points.  Then renaming is performed to create a valid
SSI form program.

\subsection{Placement algorithms}
Sreedhar and Gao have shown \cite{sreedhar95:lintime} that it is
possible to place \phifunction{s} in time proportional to the size of
the program.  With appropriate modifications to the algorithm, it can
be used to place \sigfunction{s}.  However, as noted above, $\phi$-
and \sigfunction{} placement is not independent: the placement of
\phifunction{s} necessitates additional \sigfunction{} placement, and
vice versa.  Thus, the (linear time) placement algorithms can be run
iteratively to find a fixed point.  Since the maximum number of
$\phi$- or \sigfunction{s} is proportional to the size of the program,
it is obvious that no more than $N$ iterations will be required,
resulting in a worst-case running time of $O(N^2)$.  In practice, and
for structured control flow, running time is linear.
% rewrite last sentence to reference studies on control-flow depth in
% actual programs.

% refer to program structure tree and cycle-equivalency algorithms.
% copy algorithm 5.3 from thesis here.
\begin{myalgorithm}\small
% algorithm for phi/sigma placement.

Place($G$:~CFG) =
\begin{myalgorithmic}
\LET $r$ be the top-level region for $G$
\FOREACH{variable $v$ in $G$}
\STATE PlaceOne($r$, $v$, \code{false}) \COMMENT{place phis}
\STATE PlaceOne($r$, $v$, \code{true})  \COMMENT{place sigmas}
\ENDFOR
\end{myalgorithmic}
~\\
PlaceOne($r$:~region, $v$:~variable, $ps$:~boolean):~boolean =
\begin{myalgorithmic}
\STATE \COMMENT{Post-order traversal}
\STATE flag $\gets$ false
\FOREACH{child region $r'$}
  \IF{PlaceOne($r'$, $v$, $ps$)}
    \STATE flag $\gets$ true
   \ENDIF 
\ENDFOR
                             \EMPTYLINE
\FOREACH{node $n$ in region $r$ not contained in a child region}
  \IF{$ps$ is \code{false} and $n$ contains a definition of $v$}
    \STATE flag $\gets$ true
  \ENDIF
  \IF{$ps$ is \code{true} and $n$ contains a use of $v$}
    \STATE flag $\gets$ true
  \ENDIF
\ENDFOR
                             \EMPTYLINE
\STATE \COMMENT{add phis/sigmas to merges/splits where $v$ may be live}
\IF{flag = true}
  \FOREACH{node $n$ in region $r$ not contained in a child region}
    \IF{MaybeLive($n$) = true}
      \IF{$ps$ is \code{false} and the input arity of $n$ exceeds 1}
        \STATE place a phi function for $v$ at $n$
      \ENDIF
      \IF{$ps$ is \code{true} and the output arity of $n$ exceeds 1}         
        \STATE place a sigma function for $v$ at $n$
      \ENDIF
    \ENDIF
  \ENDFOR
\ENDIF
                             \EMPTYLINE
\RETURN flag
\end{myalgorithmic}

\caption{Placing \phisigfunction{s}.}\label{alg:SSIplace}
\end{myalgorithm}
