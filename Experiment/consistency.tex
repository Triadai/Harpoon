\documentclass{article}
\usepackage{amsmath}
\usepackage{amssymb}
\usepackage{dsfont}
\usepackage{epsfig}
\usepackage{bbold}

\title{}
\author{Consistency constraints}
\date{July 30, 2003}

\setlength{\topmargin}{0in}
\setlength{\oddsidemargin}{0in}
\setlength{\evensidemargin}{0in}
\setlength{\textwidth}{6.6in}
\setlength{\textheight}{9in}
\setlength{\parindent}{0.5in}



\begin{document}
\maketitle
\begin{flushleft}
\thispagestyle{empty}

\vspace{0.5in}


The tool we designed receives a set of consistency constraints,
and signals an error if any of these constraints is violated.

\vspace{0.1in}
These consistency constraints are based on a model that represents the 
data structures used by the program under analysis.  The model is mathematical 
and is composed of sets and functions that describe the properties of these 
data structures.  These sets and functions can make reference to the actual
data structures used in the program.

\vspace{0.1in}
The sets and functions in the model offer at least two views or perspectives 
of each data structure property considered, and in this way we can make 
statements concerning consistency. More precisely, if the tool finds any 
discrepancies between two perspectives of the same property, a consistency 
error is reported. 

\vspace{0.2in}
We provide below the descriptions of the sets and functions that are used for 
constructing the model for the file system benchmarks. We use the following
two notations: $f.x$ to denote $f(x)$, and $f.\sim x$ to denote $f^{-1}(x)$.

\begin{itemize}
  \item{{\bf SuperBlock:} {\it set of integers} 
   \\ {\it SuperBlock} is a set that contains a single element, the number of 
the block containing the super block.}

  \item{{\bf GroupBlock:} {\it set of integers}
   \\ {\it GroupBlock} is a set that contains a single element, the number of 
the block containing the group block.}

  \item{{\bf InodeTableBlock:} {\it set of integers} 
  \\ {\it InodeTableBlock} is a set that contains a single element, the number 
of the block containing the inode table.}

  \item{{\bf InodeBitmapBlock:} {\it set of integers} 
  \\ {\it InodeBitmapBlock} is a set that contains a single element, the number
of the block containing the inode bitmap.}

  \item{{\bf BlockBitmapBlock:} {\it set of integers} 
  \\ {\it BlockBitmapBlock} is a set that contains a single element, the number
of the block containing the block bitmap.}

  \item{{\bf RootDirectoryInode:} {\it set of integers}
  \\ {\it RootDirectoryInode} is a set that contains a single element, the 
number of the inode containing the root directory.}


  \item{{\bf UsedInode:} {\it set of integers}
  \\If {\it i} is an inode number, then {\it i} belongs to the set 
{\it UsedInode} if at least one of the following two conditions is fulfilled:
      \begin{itemize}
         \item{{\it i} is the only element of the set {\it RootDirectoryInode}}
         \item{There exists a directory entry {\it de} in the program that 
          points to inode {\it i}}
      \end{itemize} }

  \item{{\bf FreeInode} {\it set of integers}
  \\If {\it i} is an inode, then {\it i} belongs to the set {\it FreeInode} 
if and only if {\it i} doesn't belong to the set {\it UsedInode}}

  \item{{\bf UsedBlock} {\it set of integers}
  \\If {\it b} is a block number, then {\it b} belongs 
to the set {\it UsedBlock} if at least one of the following conditions is 
fulfilled:
      \begin{itemize}
         \item{{\it b} is the only element of the set {\it SuperBlock}}
         \item{{\it b} is the only element of the set {\it GroupBlock}}
         \item{{\it b} is the only element of the set {\it InodeTableBlock}}
         \item{{\it b} is the only element of the set {\it InodeBitmapBlock}}
         \item{{\it b} is the only element of the set {\it BlockBitmapBlock}}
         \item{There exists an inode {\it i} in the set {\it UsedInode} such 
               that the entry corresponding to inode {\it i} in the inode 
               table points to block {\it b}}
      \end{itemize} }

  \item{{\bf FreeBlock} {\it set of integers}
  \\If {\it b} is a block, then {\it b} belongs to the set {\it FreeBlock} if 
and only if {\it b} doesn't belong to the set {\it UsedBlock}}


  \item{{\bf referencecount} {\it function referencecount: set of Inode's in the program  $\rightarrow$  set of integers}
  \\ If {\it i} is an inode, then {\it i.referencecount} denotes the value of 
the {\it referencecount} field in the respective inode structure.}

  \item{{\bf inodeof} {\it function inodeof: set of DirectoryEntry's in the 
program  $\rightarrow$  UsedInode}
  \\ If {\it de} is a directory entry, then {\it de.inodeof} denotes the value 
  of the {\it inodenumber} field in the respective directory entry structure.}

  \item{{\bf inodestatus} {\it function inodestatus: set of Inode's in the 
program  $\rightarrow$  \{Used, Free\}}
  \\ If {\it i} is an inode, {\it i.inodestatus} can have two values: 
  {\it Used} and {\it Free}.  We have {\it i.inodestatus=Used} if and only if 
  the inode {\it i} is marked as used in the inode bitmap.}

  \item{{\bf blockstatus} {\it function blocktatus: set of integers  
$\rightarrow$  \{Used, Free\}}
  \\ If {\it b} is a block number, {\it b.blockstatus} can have two values: 
  {\it Used} and {\it Free}. We have {\it b.blockstatus=Used} if and only if 
  the block {\it b} is marked as used in the block bitmap.}


\end{itemize}


\end{flushleft}
\end{document}
