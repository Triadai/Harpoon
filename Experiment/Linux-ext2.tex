\documentclass{article}
\usepackage{amsmath}
\usepackage{amssymb}
\usepackage{dsfont}
\usepackage{epsfig}
\usepackage{url}
\usepackage{bbold}

\title{}
\author{Short Description of the Linux ext2 File System \footnote{Adapted from the documents found at http://www.nongnu.org/ext2-doc and http://infocom.cqu.edu.au/Units/aut2000/85321/ Assessment/Exam/1995\_Supp\_Exam\_Solutions}}

\date{July 30, 2003}

\setlength{\topmargin}{0in}
\setlength{\oddsidemargin}{0in}
\setlength{\evensidemargin}{0in}
\setlength{\textwidth}{6.6in}
\setlength{\textheight}{9in}
\setlength{\parindent}{0.5in}



\begin{document}
\maketitle
\begin{flushleft}

\thispagestyle{empty}

\vspace{0.5in}



In the Linux ext2 File System, the disk can be regarded as an array of blocks 
of identical size. In our toy version, the block size is 8192 bytes, 
and the file system has 1024 blocks.


\vspace{0.1in}
To read the information from a file, the file system needs to know where on 
the disk the blocks that hold the information in the file are stored.  
To accomplish this, the ext2 file system uses some additional structures, 
called inodes.  More precisely, every physical file on the disk is represented 
by an inode. Each inode keeps information about the size of the file, the 
blocks it contains (up to a maximum of 12 blocks in our toy version), and the 
number of entries in the root directory pointing to this inode (see below).

\vspace{0.1in}
As you might have noticed already, the only information about a file not 
included in an inode is its name.  The names of the files are stored in a 
special structure, called the root directory.  

\vspace{0.1in}
The root directory is stored in inode 0, and is composed of directory
entries.  Each directory entry occupies 128 bytes and keeps information about 
the name of a file and the inode number where this file is stored.  

\vspace{0.1in}
Note that  we can have multiple directory entries referencing the same inode.  
In this case, we say that the corresponding  file names are links for one 
another.

\vspace{0.1in}

Finally, the ext2 file system contains the following special blocks:

\subsection*{SuperBlock}

The superblock is the structure on an ext2 disk containing the very basic 
information about the file system. It is stored in block 0, and is laid out 
in the following form:

\begin{verbatim}
struct SuperBlock {
  int FreeBlockCount;
  int FreeInodeCount;
  int NumberofBlocks;
  int NumberofInodes;
  int RootDirectoryInode;
  int blocksize;
};
\end{verbatim}

where:


\begin{itemize}
  \item{FreeBlockCount: indicates the total number of free blocks on the disk;}
  \item{FreeInodeCount: indicates the total number of free inodes on the disk;}
  \item{NumberofBlocks: indicates the total number of blocks on the disk;}
  \item{NumberofInodes: indicates the total number of inodes on the disk;}
  \item{RootDirectoryInode: indicates the inode where the root directory is 
kept;}
  \item{blocksize: stores the block size;}
\end{itemize}


\subsection*{GroupBlock}

The file system is split into {\it block groups}.  However, our toy version 
contains a single block group.   The groupblock is stored in block 1 and keeps 
the following information:
\begin{verbatim}
struct GroupBlock {
  int BlockBitmapBlock;
  int InodeBitmapBlock;
  int InodeTableBlock;
  int GroupFreeBlockCount;
  int GroupFreeInodeCount;
};
\end{verbatim}

where:


\begin{itemize}
  \item{BlockBitmapBlock:  indicates the block in which the block bitmap is stored;}
  \item{InodeBitmapBlock:  indicates the block in which the inode bitmap is stored;}
  \item{InodeTableBlock:  indicates the block in which the inode table is stored;}
  \item{GroupFreeBlockCount: indicates the total number of free blocks in the block group;}
  \item{GroupFreeInodeCount: indicates the total number of free inodes in the block group;}

\end{itemize}


\subsection*{BlockBitmap}

The block bitmap is an array of bits, which keeps information about which 
blocks are free and which are used. It is kept in block 2.


\subsection*{InodeBitmap}

The inode bitmap is an array of bits, which keeps information about which 
inodes are free and which are used. It is kept in block 3.


\subsection*{InodeTable}
The inode table is an array of inodes, used to keep track of every inode.
It is kept in block 4.

\vspace{0.2in}
The next two diagrams show graphically the organization of the disk and 
the organization of the InodeTable:

\vspace{0.2in}

\begin{center}
\epsfxsize=5.5in
\epsfbox{disk.eps}

\epsfxsize=3in
\epsfbox{InodeTable.eps}
\end{center}

\vspace{0.1in}
Here are the contents of the file {\it file.h}, which keeps the structure 
definitions, and which is almost identical for all the three variants of 
the program:

\begin{verbatim}
#ifndef FILE_H
#define FILE_H

#define BLOCKSIZE 8192
#define NUMBLOCK 1024
#define LENGTH BLOCKSIZE*NUMBLOCK
#define NUMINODES BLOCKSIZE/56
#define NUMFILES 200

struct block {
  char array[BLOCKSIZE];
};

struct SuperBlock {
  int FreeBlockCount;
  int FreeInodeCount;
  int NumberofBlocks;
  int NumberofInodes;
  int RootDirectoryInode;
  int blocksize;
};

struct GroupBlock {
  int BlockBitmapBlock;
  int InodeBitmapBlock;
  int InodeTableBlock;
  int GroupFreeBlockCount;
  int GroupFreeInodeCount;
};

struct BlockBitmap {
  char blocks[NUMBLOCK/8+1];
};

struct InodeBitmap {
  char inode[NUMINODES/8+1];
};

struct Inode {
  int filesize;
  int Blockptr[12];
  int referencecount;
};

struct InodeBlock {
  struct Inode entries[NUMINODES];
};

#define DIRECTORYENTRYSIZE 128
struct DirectoryEntry {
  char name[124];
  int inodenumber;
};

struct DirectoryBlock {
  struct DirectoryEntry entries[BLOCKSIZE/128];
};


void createdisk();
void createfile(struct block *ptr,char *filename, char *buf,int buflen);
void addtode(struct block *ptr, int inode, char * filename);
int getinode(struct block *ptr);
int getblock(struct block * ptr);
void removefile(char *filename, struct block *ptr);
void createlink(struct block *ptr,char *filename, char *linkname);
struct block * chmountdisk(char *filename);
void chunmountdisk(struct block *vptr);
struct block * mountdisk(char *filename);
void unmountdisk(struct block *vptr);
void closefile(struct block *ptr, int fd);
bool writefile(struct block *ptr, int fd, char *s);
int writefile(struct block *ptr, int fd, char *s, int len);
char readfile(struct block *ptr, int fd);
int readfile(struct block *ptr, int fd, char *buf, int len);
int openfile(struct block *ptr, char *filename);

void calltool(char *text);

#define MAXFILES 300
struct filedesc {
  int inode;
  int offset;
  bool used;
};
#endif
\end{verbatim}


\end{flushleft}
\end{document}