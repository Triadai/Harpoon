% -*- latex -*- This is a LaTeX document.
% $Id: quads.tex,v 1.4 1998-09-08 14:33:05 cananian Exp $
%%%%%%%%%%%%%%%%%%%%%%%%%%%%%%%%%%%%%%%%
\documentclass[11pt,notitlepage,twocolumn,twoside]{article}

% What's the name of this project this week?
\newcommand{\Harpoon}{Harpoon}%{ActiveWorld}
\newcommand{\harpoon}{harpoon}%{activeWorld}

\author{C.~Scott~Ananian}
\title{\Harpoon\ Project Compiler Intermediate Representation}
\date{\today \\ $ $Revision: 1.4 $ $}

% PDF-friendly fonts:
\renewcommand{\rmdefault}{ptm}
\renewcommand{\sfdefault}{phv}
\renewcommand{\ttdefault}{pcr}

\newcommand{\phifunction}{$\phi$-function}

\begin{document}
\bibliographystyle{abbrv}
\maketitle

\section{Conceptual Overview}

The \Harpoon\ project compiler front-end translates java bytecode files
into a class-oriented intermediate representation which is intended to
be easier to analyze and manipulate than bytecode assembly language.
The intermediate representation is control-flow-graph structured, with
all control flow explicit.  It is also maximally factored and in
static single assignment (SSA) form.  Internally the intermediate
representation is known as ``QuadSSA,'' referring
to its derivation from quadruple-style IRs and its SSA form.

\subsection{Quadruples}
Unlike expression-tree structured intermediate representations, where
every operand can optionally be the root of an expression, quadruple
representations are flat.  The ``typical'' statement is of the form
$a \leftarrow b \oplus c$; the name \textit{quadruple} comes from the fact
that there are four components ($a$, $b$, $c$, $\oplus$)
\cite{appel:modern}.\footnote{We use $\oplus$ to stand for an arbitrary
binary operator.}  Obviously an IR expressive enough to represent the
entire Java language needs more than a four component operation
statement, but we have attempted to retain the atomic simplicity
of the quadruple form.

\subsection{Static Single-Assignment Form}

Quoting from Appel in \cite{appel:modern}:
\begin{quote}
Many dataflow analyses need to find the use-sites of each defined
variable, or the definition-sites of each variable used in an
expression.  The \textit{def-use chain} is a data structure that makes
this efficient: for each statement in the flow graph, the compiler can
keep a list of pointers to all the \textit{use} sites of variables
defined there, and a list of pointers to all \textit{definition} sites
of the variables used there.\ldots

An improvement on the idea of def-use chains is \textit{static
single-assignment form}, or \textit{SSA form}, an intermediate
representation in which each variable has only one definition in the
program text.  The one (static) definition-site may be in a loop that
is executed many (dynamic) times, thus the name \textit{static}
single-assignment form.
\end{quote}

An example of the use of SSA form is shown in figure \ref{fig:ssa1}.
Simple variable renaming suffices to transform straight-line code into
SSA form.  Subscripts are used to emphasize the relationship of the
renamed variables to the original variables.  A benefit of SSA form
which is obvious from the example is that unrelated uses of the same variable
in the source program ($\tt i_1$, $\tt i_2$) become different
variables in the SSA form, eliminating false dependencies.

\begin{figure}[t]
\begin{tabular}{lll}
Conventional    && Static Single Assignment \\ \hline
$\tt \ldots$    && $\tt \ldots$ \\
$\tt i=0$       && $\tt i_0 = 0$ \\
$\tt i=i+1$     && $\tt i_1 = i_0 + 1$ \\
$\tt j=func(i)$ && $\tt j_0 = func(i_1)$ \\
$\tt i=2$       && $\tt i_2 = 2$ \\
$\tt \ldots$    && $\tt \ldots$ \\
\\
\end{tabular}
\caption{SSA transformation of straight-line code.}
\label{fig:ssa1}
\end{figure}

SSA form becomes more complex when we
introduce branches and loops.  Figure \ref{fig:ssa2} shows the
necessary transformation.  You will notice the introduction of
\textit{phi functions} at locations where control flow merges.  The
\phifunction{} ``magically'' chooses a value from among its arguments
based on the control flow path used to reach it.  Note that, although
\phifunction{s} are impossible to implement directly in an instruction set
(due to their magical properties), they can be replaced by \texttt{move}
instructions along each control flow edge leading to the
\phifunction{}.  Doing so violates the static single assignment
constraints, but leads to code executable by real processors.

Unless you are implementing code generator backends, it is unlikely
you will need to so replace \phifunction{s} or view them as anything
but magical n-ary operators.  However, it is important to observe and 
maintain the ordering relationship between control-flow edges and 
\phifunction{} arguments during transformation and analysis.

\begin{figure}[t]
\begin{tabular}{lll}
Conventional    && Static Single Assignment \\ \hline
$\tt \ldots$    && $\tt \ldots$ \\
$\tt i=0$       && $\tt i_0 = 0$ \\
$\tt if$ $x$ $\tt then$ && $\tt if$ $x$ $\tt then$ \\
~~$\tt i=1$     && ~~$\tt i_1 = 1$ \\
                && $\tt i_2 = \phi(i_0, i_1)$ \\
$\tt j=func(i)$ && $\tt j_0 = func(i_2)$ \\
$\tt \ldots$    && $\tt \ldots$ \\
\end{tabular}
\caption{SSA transformation of branching code.}
\label{fig:ssa2}
\end{figure}

Analysis, transformation, and optimization of the
IR is simplified by its SSA form.
In addition, the QuadSSA form is \textit{maximally factored}.
Constants are not allowed as quadruple operands (except for a special
\texttt{const} operation); which creates a unique mapping from
variable names to values in the computation.  This simplifies value
analysis.

Appel describes several other benefits of the SSA form in \cite{appel:modern}:

\begin{quote}
%\begin{enumerate}
---%\item 
If a variable has $N$ uses and $M$ definitions (which occupy
about $N+M$ instructions in a program), it takes space (and time)
proportional to $N\cdot M$ to represent def-use chains---a quadratic
blowup.  For almost all realistic programs, the size of the SSA form
is linear in the size of the original program.

---%\item 
Uses and defs of variables in SSA form relate in a useful way to
the dominator structure of the control flow graph, which simplifies
algorithms such as interference-graph construction.
%\end{enumerate}
\end{quote}

\subsection{Exception handling}
Exception handling in the Java language complicates control-flow.
Operations (such as division, possibly by zero) may implicitly throw
exceptions that radically redirect the flow of control.  To facilitate
analysis, exception handling and its associated control-flow is made
explicit in the intermediate representation.  For example, null
pointer and array bounds checks are inserted before object and array
references, and division by zero is explicitly checked before every
division operation.  These explicit and comprehensive checks are
intended to be targeted by agressive optimizations designed to
removed those cases with are redundant, impossible, or otherwise
unnecessary.  The design goal of the IR is that \textit{no} IR
statement should throw an implicit exception under any 
circumstance.\footnote{This goal is actual impossible for the java
bytecode opcode \texttt{aastore}, which throws an
\texttt{ArrayStoreException} under circumstances which cannot be
otherwise tested for in java code.  Thus array stores are
currently the single exception to our design goals; the lack of exception
handling for this one case is not likely to cause difficulties for most
applications.}  Pursuing that goal involves changing the semantics of method
invocation slightly: so that the \texttt{call} statement not throw an
implicit exception, the IR \texttt{call} has been defined to return
\textit{two} values.  In addition to the conventional (and optional)
method return value, an ``exception return value'' is defined to hold
the exception thrown by a method, or \texttt{null} if the method
completed without throwing an exception.  Explicit tests of the
exception return value can then be added after the \texttt{call}
statement, and control-flow made explicit as with the other IR
operations.  The \texttt{throw} statement in the IR is thus stripped
of its special meaning and becomes simply an alternate \texttt{return}
statement for the exception return value.\footnote{Try, catch, and
finally blocks are, of course, taken into consideration when a
bytecode \texttt{athrow} is translated to an IR \texttt{THROW}.}

\section{Implementation Details}

The IR described in these notes is defined in the Java package
\texttt{harpoon.IR.QuadSSA}.  The \texttt{harpoon.IR.QuadSSA.Code}
class provides a means to access the QuadSSA form of a given method;
see the definition of superclass \texttt{harpoon.ClassFile.HCode} and
the example code in \texttt{harpoon.Main.Main} for details.  The
QuadSSA statements are called \texttt{Quad}s and are subclasses of
\texttt{harpoon.IR.QuadSSA.Quad}; they are graph-structured and
doubly-linked to enable both forward and reverse traversal of the
control-flow graph.  The parent class \texttt{Quad} contains the
graph-oriented methods of the objects.  Its superinterface
\texttt{harpoon.ClassFile.HCodeElement} defines standard methods to
get object ID numbers and source file information which
ought to be implemented for elements of any intermediate representation.
An enumeration of
\texttt{Quad} types and their uses is provided in figure
\ref{fig:quads}.  It may be observed that the representation uses both
header and footer nodes, in the \texttt{HEADER} and \texttt{FOOTER}
classes.  \texttt{HEADER} nodes contain a special link to the
\texttt{FOOTER} to allow this node to be easily identified, and a
special subclass of \texttt{HEADER}, \texttt{METHODHEADER}, provides
information on the assignment of method arguments to compiler
temporary variables at the start of method code.

\begin{figure*}[!p]
\begin{tabular}{p{1.25in}p{4.5in}} 
\small\bf Class Name & \small\bf Description \\ \hline
\tt HEADER & Start node for control-flow graph.  Performs no operation.\\
\tt METHODHEADER & Subclass of \texttt{HEADER} with additional 
                   method-argument information.\\
\tt FOOTER & End node for control-flow graph.  Performs no operation.\\ \hline
\tt AGET & Fetch from an indexed array element.\\
\tt ALENGTH & Access the length of an array. \\
\tt ANEW & Create a new array object (uninitialized). \\
\tt ASET & Assign a value to an indexed array element. \\
\tt CALL & Invoke an object method. \\
\tt CJMP & Conditional jump based on a boolean value. \\
\tt CONST & Assign numeric or string constants to 
            compiler temporary variables. \\
\tt GET & Fetch the value of an object field. \\
\tt INSTANCEOF & Determine whether an object reference is an instance of a
                 given class; result is a boolean value. \\
\tt JMP & Unconditional branch.  Unused. \\
%MONITOR 
\tt MOVE & Assign one compiler temporary to another. \\
\tt NEW  & Create a new class object (uninitialized). \\
\tt NOP  & Do nothing. \\
\tt OPER & Perform a n-ary operation on set of compiler temporaries.\\
\tt PHI  & \phifunction{} representation.\\
\tt RETURN & Return a value for this method invocation.\\
\tt SET  & Assign a value to an object field. \\
\tt SWITCH & Jump to one of multiple targets, depending on a key value. \\
\tt THROW & Return an exception for this method invocation. \\
\hline
\end{tabular}
\caption{Class names of QuadSSA statement types.
\small (Public classes in \texttt{harpoon.IR.QuadSSA})}
\label{fig:quads}
\end{figure*}

\newcommand{\cls}[1]{\item[\normalfont #1]}
\newcommand{\expn}[1]{{\normalfont\it #1}}
\begin{figure*}
\texttt{%
\begin{description}
\setlength{\parskip}{0em}
\setlength{\itemsep}{0em}
\setlength{\labelsep}{0em}
\item[] package harpoon.IR.QuadSSA;
\item[] import harpoon.ClassFile.*;
\item[]
\item[] \expn{Quadruple statements}
\item[] abstract class Quad implements HCodeElement
\cls{AGET}(HCodeElement from, Temp dst, Temp objectref, Temp index)
\cls{ALENGTH}(HCodeElement from, Temp dst, Temp objectref)
\cls{ANEW}(HCodeElement from, Temp dst, HClass hclass, Temp dims[])
\cls{ASET}(HCodeElement from, Temp objectref, Temp index, Temp src)
\cls{CALL}(HCodeElement from, HMethod method, Temp objectref, Temp
params[], Temp retval, Temp retex) \expn{// objectref, retval may be null}
\cls{CJMP}(HCodeElement from, Temp test)
\cls{CONST}(HCodeElement from, Temp dst, Object value, HClass type)
\cls{GET}(HCodeElement from, Temp dst, HField field, \\ Temp objectref)
         \expn{// objectref may be null}
\cls{HEADER}(HCodeElement from, FOOTER footer)
\cls{INSTANCEOF}(HCodeElement from, Temp dst, Temp src, HClass hclass)
\cls{JMP}(HCodeElement from)
\cls{METHODHEADER}(HCodeElement from, FOOTER footer, Temp params[])
\cls{MOVE}(HCodeElement from, Temp dst, Temp src)
\cls{NEW}(HCodeElement from, Temp dst, HClass hclass)
\cls{NOP}(HCodeElement from)
\cls{OPER}(HCodeElement from, String opcode, Temp dst, Temp operands[])
\cls{PHI}(HCodeElement from, Temp dst[], int arity)
\cls{PHI}(HCodeElement from, Temp dst[], Temp src[][], int arity)
\cls{RETURN}(HCodeElement from, Temp retval) \expn{// retval may be null}
\cls{SET}(HCodeElement from, HField field, Temp objectref, \\ Temp src)
      \expn{// objectref may be null}
\cls{SWITCH}(HCodeElement from, Temp index, int keys[])
\cls{THROW}(HCodeElement from, Temp throwable)
\item[]
\item[] \expn{String constants for} opcode \expn{field of} OPER
\item[] \{  \}
\end{description}
}
Only the constructors are shown; the object field variables correspond
exactly to the names of the constructor arguments.
\caption{caption}
\label{fig:classes}
\end{figure*}

\bibliography{harpoon}

\end{document}
