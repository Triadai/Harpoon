\begin{abstract}

Debugging parallel programs is complicated by the
potential for data races between parallel threads.
A data race occurs when two parallel threads
access the same memory location and one of the
accesses is a write. Data races are especially
problematic because they introduce the possibility 
of nondeterministic behavior, which complicates
the debugging process by making it difficult
to reproduce and eliminate bugs.

We have developed a static analysis algorithm that is 
capable of determining if a parallel program is
free of data races. This algorithm is designed 
for a general and complex 
class of programs: divide and conquer
programs with recursively generated concurrency
patterns in which the parallel threads access
disjoint sections of arrays using pointer variables.

Our race detector uses several advanced program
analyses. It uses a novel pointer analysis algorithm
for multithreaded programs to disambiguate accesses
via pointers; it uses a novel symbolic analysis 
algorithm for pointer into arrays
to extract the array sections accessed
by recursive parallel procedures. These
analyses are useful for more than just static
race detection. In particular, they can be used
to statically detect array bounds violations and
to automatically parallelize sequential implementations
of divide and conquer algorithms. 

We have implemented our race detector and used it
to analyze a range of divide and conquer algorithms
written in the Cilk parallel programming language.

\end{abstract}
