% -*-latex-*- This is a LaTeX document.
% $Id: ptran.tex,v 1.26 2002-07-24 23:59:33 cananian Exp $
\documentclass[11pt,notitlepage]{article}
%\documentclass{acmconf}
\usepackage{comdef} % common definitions
\usepackage{pdffonts} % PDF-friendly fonts
\usepackage{stmaryrd} % for \rrbracket and \llbracket
\usepackage{amsmath}
\author{C.~Scott~Ananian}
\title{Points-To Analysis with Companion Objects}
\date{\today \\ $ $Revision: 1.26 $ $}

\begin{document}
\bibliographystyle{abbrv}
\maketitle
\section{Singular Values}

We say a program variable $v$ is \newterm{singular} at a statement $s$
if $v$ will have a different value at every execution of $s$.  An
allocation site $a = \text{new }A(\ldots)$ produces a value which is singular
at the site; other defintions of $a$ might reach other statements
$s_x$ in the program, so $a$ is not guaranteed to be singular at $s_x$.

A variable $v_1$ singular at a statement $s_1$ is
\newterm{pairwise-singular} with a variable $v_2$ singular at a statement
$s_2$ iff no value of $v_1$ at an execution $s_1$ is ever equal to any
value of $v_2$ at an execution of $s_2$.

Now we can say that a set $S$ of statements is
\newterm{mutually-singular} iff
\begin{multline*}
\forall s_1 \in S, s_2 \in S, v_1 \in \text{use}(s_1), v_2 \in \text{use}(s_2),
  s_1 \neq s_2 \vee v_1 \neq v_2:
\\
v_1 \text{ at } s_1 \text{ is pairwise-singular with } v_2 \text{ at } s_2
\end{multline*}

We can compute a conservative approximation to singularity as follows.
We say a variable $v$ is singular at a use $s$ iff:
\begin{itemize}
\item all reaching definitions of $v$ are either allocation statements 
$v = \text{new }T(\ldots)$ or moves $v = x$ where $x$ is singular at
the move, and
\item every non-trivial path from $s$ to $s$ redefines $v$.
\end{itemize}

Let $\text{Rd}(s,v)$ be the set of reaching definitions of $v$ at $s$.
We define the set of \newterm{generation sites} for a variable $v$ singular at
use $s$ as:
\begin{displaymath}
\text{GenSite}(s,v) %\text{ where } v\in\text{use}(s)
 =
\bigcup_{d \in \text{Rd}(s,v)} g(d)
\text{ where }
g(d) = \begin{cases}
                \text{GenSite}(d,x) &\text{when } d \text{ is } v=x \\
                \{d\}&\text{when } d \text{ is } v=\text{new } T(\ldots)
       \end{cases}
\end{displaymath}

We can then approximate pairwise-singularity by saying $v_1$ at $s_1$
is pairwise-singular with $v_2$ at $s_2$ if
\begin{displaymath}
\neg\exists d \in \text{GenSite}(s_1, v_1) \cap \text{GenSite}(s_2, v_2)
     : \text{Path}_{\mathcal{G}-d}(s_1, s_2) \vee
       \text{Path}_{\mathcal{G}-d}(s_2, s_1)
\end{displaymath}
where $\text{Path}_{\mathcal{G}-z}(x, y)$ is true iff there is a path
from $x$ to $y$ in directed control-flow graph $\mathcal{G}$ which
does not pass through $z$.  Note that
$\text{Path}_\mathcal{G}(x,y)$ is true if $x=y$.
An infeasible path analysis \cite{267921} could be used to refine
this predicate by excluding paths never traversed by any
execution of the program.
% note that a statement ... = a + b *could* however have a=b and still
% be singular.  does this need to be fixed?

\begin{myfigure}[p]
\begin{samplecode}[2]%
while (\ldots) \{         &while (\ldots) \{         \\
\>v = new T(\ldots)       &\>v1 = new T(\ldots)      \\
\>if (\ldots)             &\>if (\ldots)             \\
\>\>\ldots = v;  /* s1 */ &\>\>v2 = v1;              \\
\>else                    &\>if (\ldots)             \\
\>\>\ldots = v;  /* s2 */ &\>\>\ldots = v2;  /* s3 */ \\
\}                        &\>else                    \\
                          &\>\>\ldots = v2;  /* s4 */ \\
                          &\}                        \\
\end{samplecode}%
\caption{Statements $s_1$ and $s_2$ are pairwise-singular in the code
  on the left; statements $s_3$ and $s_4$ are not singular because
  there exist paths $s_3\pathplus s_3$ and $s_4\pathplus s_4$ which do not
  redefine \texttt{v2}.}
\label{fig:pairwise-singular}
\end{myfigure}

\subsection{Intraprocedural singular value analysis}

An intraprocedural analysis to compute $\text{GenSite}(s, v)$ and the
singularity of individual variables can be built using the standard
dataflow framework:
\begin{eqnarray*}
\text{Out}(n)&=&\left(\text{In}(n)-\text{Kill}(n)\right)\cup\text{Gen}(n) \\
\text{In}(n)&=&\bigcup_{p\in\text{pred}(n)} \text{Out}(p)\\
\end{eqnarray*}

We will perform the analysis simultaneously on three relations
$\text{Ru}$, $\text{Nd}$, and $\text{Gs}$.
The relations are represented as
sets of pairs \tuple{v, X} where
$\mathcal{R}(v)=X$ iff $\tuple{v,X} \in R$.
Both sets are initialized
to the empty set for all $n$, and grow monotonically as the analysis
progresses.
We let
$\text{Ru}(v)$ be the set of reaching \textbf{uses} of $v$ at $n$,
$\text{Nd}(v)$ be the set of \textbf{non-}singular reaching definitions
of $v$ at $n$,
 and
$\text{Gs}(v)$ be $\text{GenSite}(v,n)$, the set of generation sites
of $v$ at $n$.
The appropriate Gen and Kill rules are:
\begin{eqnarray*}
% this is the standard reaching-def rule:
%\text{Gen}_\text{Rd}(n) &=&
%       \left\{ \tuple{v, n} | v \in \text{def}(n) \right\} \\
%\text{Kill}_\text{Rd}(n) &=&
%       \left\{ \tuple{v, s} | v \in \text{def}(n) \wedge
%                          \tuple{v,s} \in \text{In}_\text{Rd}(n) \right\} \\
%%%%%%%%%%%%%%%%%%%%%%%%
\text{Gen}_\text{Ru}(n) &=&
        \left\{ \tuple{v, n} | v \in \text{use}(n) \right\} \\
\text{Kill}_\text{Ru}(n) &=&
        \left\{ \tuple{v, s} | v \in \text{def}(n) \wedge
                          \tuple{v,s} \in \text{In}_\text{Ru}(n) \right\} \\
%
% non-singular reaching defs of n.
\text{Gen}_\text{Nd}(n) &=&
        \left\{ \tuple{v, n} | v \in \text{def}(n) \wedge
\left( \text{IsBad}(n) \vee
\exists u\in\text{use}(n):\text{Ns}(u,n)
 \right) \right\}
\\
\text{Kill}_\text{Nd}(n) &=&
        \left\{ \tuple{v, s} | v \in \text{def}(n) \wedge
                          \tuple{v,s} \in \text{In}_\text{Nd}(n) \right\} \\
% gensites!
\text{Gen}_\text{Gs}(n) &=&
       \left\{ \tuple{v, n} | v \in \text{def}(n)\wedge\text{IsNew}(n)\right\}
       \cup\\&&\quad
       \left\{ \tuple{v, s} | v \in \text{def}(n)\wedge\text{IsMove}(n) \wedge
                              \exists u\in\text{use}(n) :
                                 \tuple{u,s}\in \text{In}_\text{Gs}(n)\right\}
\\
\text{Kill}_\text{Gs}(n) &=&
       \left\{ \tuple{v, s} | v \in \text{def}(n) \wedge
                         \tuple{v,s} \in \text{In}_\text{Gs}(n) \right\} %\\
\end{eqnarray*}
where
\begin{eqnarray*}
\text{IsMove}(n) &=& \text{true iff }n\text{ is a move statement } v=x\\
\text{IsNew}(n) &=& \text{true iff }n\text{ is a allocation statement } v=\text{new }T(\ldots)\\
\text{IsBad}(n) &=& \neg\left(\text{IsMove}(n) \vee \text{IsNew}(n)\right) \\
\text{Ns}(v,n) &=&
%v\in\text{use}(n) \wedge \left(
\tuple{v,n} \in \text{In}_\text{Ru}(n) \vee
\exists n': \tuple{v,n'}\in\text{In}_\text{Nd}(n)
%\right)
\end{eqnarray*}
After the analysis is complete, $v$ is (conservatively) singular at
use $n$ if $\neg\text{Ns}(v,n)$, and 
$\text{GenSite}(s,v)=\left\{ d | \tuple{v,d}\in\text{In}_\text{Gs}(s) \right\}$.
Note that $\text{IsBad}(n)$ is true for fetches from the heap and
definitions of formal parameters.

This analysis can be performed using a sparse representation, such as
SSA or SSI, as well.

Variable $v_1$ singular at $s_1$ is pairwise-singular to $v_2$
singular at $s_2$ if
\begin{displaymath}
\neg\exists 
     d : \tuple{v_1, d} \in \text{In}_\text{Gs}(s_1) \wedge
         \tuple{v_2, d} \in \text{In}_\text{Gs}(s_2) \wedge
\left(\text{Path}_{\mathcal{G}-d}(s_1, s_2) \vee
      \text{Path}_{\mathcal{G}-d}(s_2, s_1) \right)
\end{displaymath}

The mutual singularity of a group of statements can be computed from
the variable singularity and generation site information using the
definitions in the previous section.

\textbf{XXX: Perhaps present algorithm for determining mutual
singularity using a single graph traversal, keeping track of which
defs we've passed and halting when we hit another node without
having passed through a redef point?}

\subsection{Interprocedural singular value analysis}

We turn our intraprocedural analysis into a compositional
interprocedural analysis by introducing ``conditional singularity''
for formal parameters and others which depend on them.  A
conditionally-singular variable is only singular if some set of actual
parameters is mutually-singular when the method is called.%
\footnote{Note that aliasing of parameters is one of the ways that
a set of parameters may fail to be mutually-singular.}
We define
$\text{CondParam}(v,n)$ to be the set of formal parameters which the
singularity of $v$ at $n$ is dependent on.  We can extend our dataflow
analysis of the previous section to compute this function.


Previously, the function $\text{In}_\text{Gs}$ was typed
$\mathcal{S\to V\times S}$
where $\mathcal{S}$ is the domain of statements
and $\mathcal{V}$ the domain of variables.
We expand that type to
$\mathcal{S\to V\times (S\cup P)}$ where $\mathcal{P}$ is a domain
representing formal parameters.  The set $P_m$ of formal parameters for
a method $m$ consists of unique tokens (possibly ordinal integers) for
each formal parameter of the method.  Let
$\text{formal}(p):\mathcal{P\to V}$ map parameter tokens to the
corresponding formal variables in a callee, and
$\text{actual}(p):\mathcal{P\to V}$ map parameter tokens to the
corresponding actual variables at a callsite.
Now, instead of initializing $\forall n :
\text{In}_\text{Gs}(n)=\emptyset$, we let
\begin{displaymath}
\text{In}_\text{Gs}(\text{\tt HEADER}) =
   \left\{ \tuple{v,p} | p\in P_m \wedge v=\text{formal}(p) \right\}
\end{displaymath}
We also extend the $\text{IsBad}$ predicate:
\begin{displaymath}
\text{IsBad}(n) = \neg\left(\text{IsMove}(n) \vee \text{IsNew}(n) \vee
                            \text n = \text{\tt HEADER} \right)
\end{displaymath}
so that definitions of formal parameters are not marked as
non-singular reaching definitions.  Now we can define
\begin{displaymath}
\text{CondParam}(v,n) =
    \left\{ p\in P_m | \tuple{v,p}\in \text{In}_\text{Gs}(n) \right\}
\end{displaymath}

A variable $v$ is singular at $n$ conditional on $P$ if
$P=\text{CondParam}(v,n)$ and $\neg\text{Ns}(v,n)$.  This
is identical to our former case of non-conditional singularity
when $P=\emptyset$.  Variable $v_1$ singular at $s_1$ conditional on
$P_1$ is pairwise-singular conditional on $P_1\cup P_2$ to variable
$v_2$ singular at $s_2$ conditional on $P_2$ if
\begin{displaymath}
\neg\exists 
     d : \tuple{v_1, d} \in \text{In}_\text{Gs}(s_1) \wedge
         \tuple{v_2, d} \in \text{In}_\text{Gs}(s_2) \wedge
\left(\text{Path}_{\mathcal{G}-d}(s_1, s_2) \vee
      \text{Path}_{\mathcal{G}-d}(s_2, s_1) \right)
\end{displaymath}
as before, where $\mathcal{G}-d=\mathcal{G}$ if $d\notin\mathcal{S}$.

Similarly, a set of statements $S$ is mutually-singular conditional on $P$
if
\begin{multline*}
\forall s_1 \in S, s_2 \in S, v_1 \in \text{use}(s_1), v_2 \in \text{use}(s_2),
  s_1 \neq s_2 \vee v_1 \neq v_2:
\\\shoveright{
v_1 \text{ singular at } s_1 \text{ conditional on } P_1
}\\\shoveright{{}\wedge
v_2 \text{ singular at } s_2 \text{ conditional on } P_2
}\\\shoveright{{}\wedge
v_1 \text{ at } s_1 \text{ on } P_1
\text{ is pairwise-singular with }
v_2 \text{ at } s_2 \text{ on } P_2
}\\{}\wedge
P_1 \cup P_2 \subseteq P
\end{multline*}

Now we can resolve conditional singularity compositionally by
examining actual parameters at a call-site $s_\text{call}$.
A set $S=S_1 \cup S_2$ of statements,
where $s_1\in S_1$ belong to the caller and $s_2\in S_2$ belong to the
callee, are mutually-singular conditional on $P$ if:
\begin{itemize}
\item $S_2$ are mutually-singular conditional on $P'$,
\item there is no path from $s_1 \in S_1$ to callsite or
             from callsite to $s_1\in S_1$,%
\footnote{Assuming METHODENTER in callee reaches all $s_2\in S_2$ and
          all $s_2\in S_2$ reach METHODEXIT in callee.}
and
\item all actual parameters corresponding to $P'$ at callsite $s_\text{call}$
  unioned with $S_1$ are mutually-singular conditional on $P$; that is, let
\begin{displaymath}
U = 
\left\{\tuple{\text{actual}(p), s_\text{call}} | p \in P' \right\}
    \cup
\left\{\tuple{v,s}|s\in S_1 \wedge v\in \text{use}(s)\right\}
\end{displaymath}
then
\begin{multline*}
\forall \left\{\tuple{v_1,s_1},\tuple{v_2,s_2}\right\} \subseteq U :
\\\shoveright{
v_1 \text{ singular at } s_1 \text{ conditional on } P_1
}\\\shoveright{{}\wedge
v_2 \text{ singular at } s_2 \text{ conditional on } P_2
}\\\shoveright{{}\wedge
v_1 \text{ at } s_1 \text{ on } P_1
\text{ is pairwise-singular with }
v_2 \text{ at } s_2 \text{ on } P_2
}\\{}\wedge
P_1 \cup P_2 \subseteq P
\end{multline*}
\end{itemize}

So we can throw away all information internal to the callee except
mutually-singular sets of statements and their sets of dependent
formal parameters.

%A compositional
%analysis is presented to compute interprocedural singularity.
%This might be in some ways similar to interprocedural reaching defs?
%Computing mutual singularity can rest on the fact that values
%generated at sites inside a method can never be the same as values generated
%at sites outside the method; we don't have to compare site-by-site.

\section{Compositional Pointer Analysis}

(Roughly) We use the Whaley/Rinard compositional pointer analysis
framework, except that we can mark edges as ``companion'' and
``possible companion''.  Annotate edges with the field set statements
used to create them.  If all field set statements associated with an
edge are mutually-singular, then the edge is a companion ---
``possibly companion'' if some of the edges are outside edges (thus
we don't know if the field set statements associated with these edges
are going to be mutually-singular with the rest).  ``Possibly
companion'' edges must still be mutually-singular internally.

\begin{myfigure}
\begin{tabular}{ll|ll}
$l\in L$ & local variables           &$n\in N$ & nodes\\
$p\in P$ & formal parameter variables&$N_I\subseteq N$ & inside nodes\\
$v\in V=L\cup P$& all variables      &$N_O\subseteq N$ & outside nodes\\
$c\in C$ & classes                   &$C\subseteq N$ & (static) class nodes\\
$m\in M$ & methods                   &$N_P\subseteq N_I$ & parameter nodes\\
$f\in F$ & fields                    &$N_G\subseteq N_I$ & global nodes\\
$s\in S$ & statements                &$N_L\subseteq N_I$ & load nodes\\
         &                           &$N_R\subseteq N_I$ & return nodes\\
\end{tabular}
\caption{Definitions}
\label{fig:defin}
\end{myfigure}

A points-to graph is a quintuple of the form \tuple{O,I,e,s,r}, where
\begin{itemize}
\item $O \subseteq (N\times F)\times(N_L \cup N_G)$ is a set of
outside edges.  Outside edges represent references created outside the
current analysis scope.
\item $I \subseteq (V\cup(N\times F))\times N$ is a set of inside
edges.  Inside edges represent references created inside the current
analysis scope.
\item $e:N\to 2^M$ is an escape function that records the set of
unanalyzed method invocation sites that a node escapes down into.
\item $s:(O\cup I)\to 2^S_\bot$ labels companion edges with an
  associated set of mutually-singular store statements.  A
  non-companion edge $e\in O\cup I$ has $s(e)=\bot$.
\item $r \subseteq N$ is a return set representing the set of objects
that may be returned by the currently analyzed method.
\end{itemize}


\subsection{Evaluating a call-site monomorphically}

present rules.  don't need to say how we choose to eval
monomorphically yet.  Just keep the companion edges accurate.

\subsection{Evaluating a call-site polymorphically}

present rules.  don't need to say how we choose to eval
polymorphically yet.  Just keep the companion edges accurate,
keep updating mutual-singularity of field gets,
and rename internal nodes by call-site.

\section{Allocating Polymorphism}

We evaluate sites polymorphically if they are factory methods, or
contain unbalanced companion objects.  Otherwise, the site is
evaluated monomorphically.

(Roughly) If an allocated (internal?) node escapes via the return
value, then this is a factory method.  Evaluate it polymorphically.
If we have ``possible companion'' edges --- that is, we allocate
a node internally which may be a companion of a node allocated
externally --- then evaluate it polymorphically.  [I'm guessing
that possible companion edges which touch only outside, not inside,
nodes are not sufficient grounds for polymorphic evaluation.]
Otherwise, evaluate it monomorphically.

\section{Related Work}

Bod{\'\i}k et al outlined a method for detecting infeasible paths
in \cite{267921} which could be used to extend the set of statements
we determine are mutually-singular (by enabling us to be less conservative).

Altucher et al renamed allocation sites for better alias analysis
precision on C programs with factory methods \cite{199466}.  They
do not handle Java or companion objects.

Collberg et al described a Java obfuscation technique that relies in
part on the intractability of alias analysis \cite{268962}.  The
analysis we describe is sufficient to break some of their predicates.
This could be used to construct a deobfuscator.  (These techniques
have been used in at least one commerical obfuscator \cite{humper02}.)


\bibliography{harpoon}
\end{document}
