\chapter{Introduction}\label{cha:intro}
\epigraphhead[70]{\epigraph{%
[Parallel programming] is hard.
}{\textit{Teen Talk Barbie}\\(apocryphal)}}
\index{Barbie!sayings inaccurately attributed to}
% see http://itre.cis.upenn.edu/~myl/languagelog/archives/002892.html
% for the real story and the accurate quote (``math class is tough'')

How can we fully utilize the coming generation of parallel systems?
The primitives available to today's programmers are largely
inadequate.  \defn{Transactions} have been proposed as an
alternative to the existing mechanisms for expressing concurrency and
synchronization, but current implementations of transactions are
either too limited or too 
inefficient for general use.  This thesis presents the design
and implementation of efficient and powerful transaction systems to
help address the challenges posed by current trends in computing
hardware.

\section{The rising challenge of multicore systems}
Processor technology is nearing its limits: even
though transistor quantities continue to grow exponentially, we are
now unable to effectively harness those vast quantities of transistors
to create speedier single processors.  The smaller transistors yield
relatively slower signal propagation times, dooming attempts to create
a single synchronized processor from all of those resources.
Instead, hardware manufacturers are providing tightly integrated
\defn{multicore} systems which integrate multiple parallel
processors on one chip.

The widespread adoption of parallel systems creates problems: how can
we ensure that operations occur in an appropriate order?
How can we ensure certain operations occur
\textit{atomically}, so that other components of the parallel system
only observe data structures in well-defined states?

\index{Lock(s)!problems with}
Atomicity in shared-memory multiprocessors is 
conventionally provided
via mutual-exclusion \defn{locks} (see, for example,
\cite[p.~35]{Tanenbaum92}).  Although locks are easy to
implement using test-and-set, compare-and-swap, or
load-linked/{\bp}store-conditional instructions, they introduce a host of
difficulties.  Protocols to avoid deadlock when locking multiple
objects often involve acquiring the locks in a consistent linear
order, making programming with locks error-prone and introducing
significant time and space overheads in the resulting code.  The
granularity of each lock must also be
explicitly chosen, as locks that are too fine introduce unnecessary
space and time overhead, while locks that are too coarse sacrifice
attainable parallelism (or may even deadlock).  Every access to a
shared object must hold some lock protecting that object, regardless
of whether another thread is actually attempting to access the same object.

\section{Advantages of transactions}

\defn{Transactions} are
an alternative means of providing concurrency control.
A transaction can be thought of as a sequence of loads and stores
performed as part of a program which either
\defn{commits} or \defn{aborts}.  If a transaction
commits, then all of the loads and stores appear to have run
atomically with respect to other transactions.  That is, the
transaction's operations are not interleaved with those of other
transactions.  If a transaction aborts, then none of its stores take
effect and the transaction may be restarted, with some mechanism to
ensure forward progress.

By structuring concurrency at a high level with transactions, human
programmers no longer need to manage the details required to ensure
atomicity.  A full mental model of the global concurrency structure
must be kept in mind when writing synchronization code, which
programmers have difficulty correctly maintaining.
The simpler ``global atomicity'' guaranteed under the
transactional model eliminates potential errors and simplifies the conceptual
model of the system, making future modifications safer as
well.

The transaction primitives presented in this thesis can exploit
optimistic concurrency, provide fault tolerance, and prevent delays by
using nonblocking synchronization.  Although transactions can be
implemented using mutual exclusion (locks), our algorithms
utilize nonblocking synchronization \index{Nonblocking
synchronization}\index{Synchronization!nonblocking}
\cite{Lamport77,Herlihy88,HerlihyLuMo03,MassalinPu91,GreenwaldCh96} to
exploit optimistic concurrency among transactions.  Nonblocking
synchronization offers several advantages; from the system
builder's perspective the principle advantage is fault tolerance.
In a traditionally constructed system, a process that
fails or pauses while holding a lock within a critical region can
prevent all other processes from making progress.  It is in general
not possible to restore locked data structures to a
consistent state after such a failure.  Nonblocking synchronization
offers a graceful solution, as non-progress or failure
of any one thread does not affect the progress or consistency of other
threads or the system.

Implementing transactions using
nonblocking synchronization offers performance benefits as well.
When mutual exclusion is used to enforce atomicity, page
faults, cache misses, context
switches, I/O, and other unpredictable events may result in delays to the
entire system. Nonblocking
synchronization allows undelayed processes or processors to continue
to make progress.
In real-time systems, the use of nonblocking
synchronization can prevent \defnlti{Priority inversion}
\index{Lock(s)!priority inversion} in the system
\cite{Jones97}, although \naive implementations may result in
starvation of low-priority tasks (see \secref{progress} for a discussion).

\section{Unlimited transactions}

The
\defni{Transactional memory} abstraction
\cite{Knight86,HerlihyMo93,StoneStHe93,RajwarGo02,ShavitTo95,HerlihyLuMoSc03},
has
been proposed as a general and flexible way to allow programs to read
and modify disparate primary memory locations as a single
operation (atomically), much as a database transaction can atomically modify many
records on disk.

\index{HTM|see{Hardware Transactional Memory}}
\index{STM|see{Software Transactional Memory}}
\index{Hardware Transactional Memory}
\defn{Hardware transactional memory} (HTM) supports atomicity through
architectural means, whereas \defn{software transactional memory}
\index{Software Transactional Memory}
(STM) supports atomicity through languages, compilers, and libraries.
I will present both software and hardware implementations of the
transaction model.

Researchers of both HTM and STM commonly express the opinion that
transactions need never touch many memory locations, and hence it is
reasonable to put a (small) bound on their
size~\cite{HerlihyMo93,HerlihyLuMoSc03}.\footnote{%
For example, \cite[section 5.2]{HerlihyMo93} states,
``Our [HTM] implementation relies on the assumption that transactions have
short durations and small data sets''; while 
the STM described in \cite{HerlihyLuMoSc03} has quadratic slowdown when
transactions touch many objects: performance is $O((R+W)R)$, where $R$
and $W$ are the number of objects opened for reading and writing,
respectively.}
For HTM implementations,
they conclude that a small piece of additional hardware---typically in
the form of a fixed-size content-addressable memory and supporting
logic---should suffice.  For STM implementations, some researchers
argue additionally that transactions occur infrequently, and hence the
software overhead would be dwarfed by the other processing done by an
application.
In contrast, this thesis supports transactions of
arbitrary size and duration.  Just as most programmers need not pay any
attention to the exact size and replacement policy of their
system's cache, programmers ought not concern themselves with limits
or implementation details of their transaction system.

My goal is to
make concurrent and fault-tolerant programming easier,
without incurring excessive overhead.  This thesis advocates
unbounded transactions because neither programmers nor compilers can
easily cope when an architecture imposes a hard limit on transaction
size.  An implementation might be optimized for transactions below a
certain size, but must still operate correctly for larger
transactions.  The size of transactional hardware should be an
implementation parameter, like cache size or memory size, which can
vary without affecting the portability of binaries.

In \charef{hybrid} I show how a fast hardware implementation for
frequent short transactions can gracefully fail over to a software
implementation designed to efficiently execute large long-lived
transactions.
The hybrid approach allows more sophisticated transaction models to be
implemented, while allowing a simpler hardware transaction mechanism
to provide speed in the common case.

\section{Strong atomicity}\label{sec:strongatom}
Blundell, Lewis, and Martin~\cite{BlundellLeMa05} distinguish
between \defn{strongly atomic}\index{Strong atomicity}\index{Atomicity!strong}
transaction systems, which protect transactions from interference from
nontransactional code, and \defn{weakly atomic} transaction
systems which do not afford this
 protection.\index{Weak atomicity}\index{Atomicity!weak}
Nearly all current software transaction systems are weakly
atomic,\footnote{There are some systems which use type systems to
disallow nontransactional access to objects with a ``shared'' type
\cite{RudysWa02, HerlihyLuMoSc03, HarrisMaPeHe05}, but to my knowledge
all systems which allow (or cannot prevent) access to shared objects
from nontransactional code do so unsafely
\cite{ShavitTo95,HarrisFr03,Herlihy05b,DiceShSh06,HerlihyLuMo06,FraserHa04}.}
however,
despite the pitfalls thus opened for the unwary programmer, because of
the perceived difficulty in efficiently implementing the required
protection.

Strong atomicity is clearly preferable, as the programmer will
inevitably overlook some nontransactional references to shared data;
we wish to preserve correctness in this common case.
Blundell \etal point out
that programs written for a weakly atomic model (to run on a current software
transaction system, say) may deadlock when run under strong atomicity
(for example, on a hardware transaction system).  The transaction
systems considered in this thesis preserve the
correct atomic behavior of transactions even in the face of
unsynchronized accesses from outside the transaction.


\punt{
Concurrency in our programs is presently managed with explicit locking
protocols.  We submit that this is inadequate, and propose
\textit{transactions} as an alternative.  Current adoption of
transactional programming is limited because transaction
implementations are inefficient.  This thesis demonstrates
techniques to create both software and hardware implementations of
transactions that are efficient and powerful enough to be usable.
Software transactions can be adopted now; hardware transaction systems
are more efficient, but can't be put to use until manufacturers have
produced chips with the needed capabilities.  Once hardware
transaction systems are available, hybrid systems can combine software
and hardware techniques to achieve the best qualities of both.

In this thesis I show how transactions can be integrated into an
object-oriented language and used for backtracking, exception
handling, and concurrency control in new programs, in addition to
expressing atomicity and fault-tolerance.

One of the common synchronization mistakes made in lock-based
object-oriented code is
the ``multiple object problem'': when lock A protects object A, and
lock B protects object B, it is easy to get into trouble when writing
routines that involve both objects A and B\@.  Either insufficient
locks are taken, or both locks are taken without carefully considering
deadlock, or else one object's locks are released and reacquired during the
operation allowing atomicity to be broken.
In \charef{safe-transactify} I describe a technique
for ``transactifying'' existing lock-based code
to identify or fix some existing concurrency bugs.


\section{Efficient Implementations of Transactions}

\section{Hybrid Transaction Systems}
}

%%%%%%%%%%%%% LIST OF CONTRIBUTIONS %%%%%%%%%%%%%%%%%%%
\section{Summary of contributions}
In summary, this thesis makes the following contributions:

\begin{itemize}
\item I provide efficient implementations of
\textit{strongly atomic} transaction primitives to enable their general use.
\punt{
In particular, my technique imposes as little as 15\% overhead on
nontransactional code in a software-only system.}
My experiments have
not shown an overhead of more than 150\% in real applications,
although proper compiler support should reduce that worst case
considerably.%
\note{Be sure to update this worst case when I get the final final \#s.}

\item The transaction
primitives presented in this thesis can exploit optimistic
concurrency, provide fault tolerance, and prevent delays by using
nonblocking synchronization.
\punt{
\item 
I show how transactions can be integrated into an
object-oriented language and used for backtracking, exception
handling, and concurrency control in new programs, in addition to
expressing atomicity and fault-tolerance.
\item In \charef{safe-transactify} I describe a technique
for ``transactifying'' existing lock-based code
to identify or fix existing concurrency bugs.
}

\item This thesis proposes systems which can support transactions of
arbitrary size and duration, sparing the programmer from detailed
knowledge of the system's implementation details.

\item I present both software and hardware implementations of the
transaction model.
The software transaction system runs real programs written in
Java; I discuss the practical implementation details encountered.
The hardware transaction systems require only small changes to the
processor core and cache.

\item I show how a fast hardware implementation for
frequent short transactions can gracefully fail over to a software
implementation designed to efficiently execute large long-lived
transactions.
\end{itemize}
%%%%%%%%%%%%% END LIST OF CONTRIBUTIONS %%%%%%%%%%%%%%%%%%%
%% ORGANIZATION OF THESIS:  XXX FIXME XXX

In \charef{examples} I provide some concurrent programming
examples that illustrate the limitations of current lock-based
methodologies.  I also provide examples illustrating the uses
(some novel) of a transaction system, and conclude with a brief
caution about the current limits of transactions.

In \charef{stm} I present the design of \apex, an efficient software-only
implementation of transactions.  Software-only transactions can be
implemented on current hardware, and can easily accommodate many
different transaction and nesting models.  Software transactions excel
at certain long-lived transactions, where the overhead is
small compared to the transaction execution time.  An early version of
\apex was published as \cite{Ananian05}.
I conclude by
presenting a microbenchmark that demonstrates the performance limits of
the design.

In \charef{stmimpl} I discuss the practical implementation of
\apex.  I present details of the compiler
analyses and transformations performed, as well as solutions to
problems that arise when implementing Java.  I then present
benchmark results using real applications, discuss the benefits and
limitations revealed, and describe how the limits could be overcome.

\Charef{largeobj} explores one such limitation in depth,
describing how large arrays fit into an object-oriented transaction
scheme.  I present a potential solution to the problem based on
fast functional arrays.

In \charef{htm} I present LTM and UTM, hardware systems that enable fast
short transactions.  The transaction model is more limited, but short
committing transactions may execute with no overhead.  The additional
hardware is small and easily added to current processor and memory
system designs.  This portion of the chapter is joint work with Krste
Asanovi\'c, 
Bradley C. Kuszmaul, Charles E. Leiserson, and Sean Lie, and has been
previously published as \cite{AnanianAsKuLeLi05,AnanianAsKuLeLi06}.

\punt{
In \charef{htmplus}, I present extensions to the hardware that allow
more capable virtualized transactions.  These extensions address the
major limitations of the scheme presented in \charef{htm}, but the
required hardware is more complicated.  Further, we must still chose
particular solutions to open problems in transactional memory systems
(for example, nesting); it may not be desirable to hard-wire specific
choices before the community has more experience with the
transactional approach to writing software.
}

% \secref{hybrid}
At the end of the chapter,
I present \hyx, a \defn{hybrid} transaction implementation
that builds on the strengths of simple hardware support while
allowing software fallback to support a robust and capable transaction
mechanism.  Unlike the extended hardware scheme, the transaction model
is still easy to change and update; the hardware primarily supports
fast small transactions and conflict checking in the common case.
I discuss some ways compilers can further
optimize software and hybrid transaction systems.  These opportunities
may not be available to pure-hardware implementations.

In \charef{chall} I discuss remaining challenges to the use of
ubiquitous transactions for synchronization, and present some ideas
toward solutions. \Charef{related} discusses related work, and my
final chapter summarizes my findings and draws conclusions.


%%%%%%%%%%%%%%%%%%%%%%%%%%%%%%%%%%%%%%%%%%%%%%%%%%%%%%%%%%%%%%%%%%%%%
\chapter{Transactional programming: The good, bad, and the ugly}
\markboth{CHAPTER 2.~~TRANSACTIONAL PROGRAMMING}{}
\label{cha:examples}
\epigraphhead[70]{\epigraph{%
You have a hardware or a software problem.
}{\textit{Service manual for Gestetner~3240}}}
\punt{
\epigraphhead[70]{\epigraph{%
A good stock of examples, as large as possible, is indispensable for a
thorough understanding of any concept.
}{\textit{Paul Halmos}}}
}

Before diving into the design of an efficient transaction system,
I motivate the transactional programming model by
presenting four common scenarios that are needlessly
difficult using lock-based concurrency.  I then present four novel
applications that a transactional model facilitates.  To ground the
discussion in reality, I conclude by enumerating a few cases where
transactions may \emph{not} be the best solution.

\section{Four old things you can't (easily) do with locks}
Locks engender poor modularity and composability, an inability to deal
gracefully with asynchronous events, and fragile and complex safety
protocols that are often expressed externally to their
implementations.  These limitations of locks are well-known
\cite{Herlihy05}.  I present four common tasks which locks make
unnecessarily difficult: making localized changes to synchronization,
performing atomic operations on containers, creating a thread-safe
double-ended queue, and handling asynchronous exceptions.

\subsection*{Tweak performance with localized changes}
% global lock orders, oh my!
Preventing deadlocks and races requires global protocols and non-local
reasoning.  It is not enough to simply specify a lock of a certain
granularity protecting certain data items; the order or circumstances
in which the lock may be acquired and released must also be specified
globally, in the context of all locks in the system, in order to
prevent deadlocks or unexpected races.  This requirement prevents the
programmer from easily tuning the system using localized changes:
every small change must be re-verified against the protocols specified
in the whole-program context in order to prevent races and/or
deadlock.

% usage rules embedded in comments
Furthermore, this whole-program protocol is not typically expressed
directly in code.  With common programming languages, acquire/release
ordering and guarantees must be expressed externally, often as
comments in the source code that easily drift out of sync with the
implementation.  For example, in \cite{AnanianAsKuLeLi05} we counted
the comments in the Linux filesystem layer, and found that about 15\%
of these relate to locking protocols; often describing global
invariants of the program which are difficult to verify.  Many
reported kernel bugs involve races and deadlocks.

% races, lost wakeups ?

\subsection*{Atomically move data between thread-safe containers}
% locks don't compose
Another common programming pitfall with locks is their
\defn{non-composability}.\index{Lock(s)!non-composability}
For example, given two thread-safe
container classes implemented with locks, it is impossible to
safely compose
the \textit{get} function of one with the \textit{put} function of
the other to create an atomic move.  We must peek inside the
implementation of the containers to synthesize an appropriate locking
mechanism for such an action---for example, to acquire the 
appropriate
container, element, or other
locks on \emph{both} containers---and even then,
we need to resort to some global lock ordering to guard against
deadlock.  Modularity must be broken in order to synthesize the
appropriate composed function, if it is possible at all.

In the Jade programming language, Rinard presents a partial
solution using ``implicitly synchronized''
objects~\cite[p14]{Rinard98}.  Lock acquisition for each module is
exposed in the module's API as an executable ``access declaration.''
Operation composition is accomplished by creating an access
declaration for the composed operation which invokes the appropriate
access declarations for the components.  The runtime system orders the
lock acquisitions to prevent deadlock.  This process suffices for
conservative mutual exclusion, but pre-declaration makes it difficult
to express optimistic locking protocols, where one or more locks are
only rarely required.

Transactions do not suffer from the composability problem
\cite{HarrisMaPeHe05}.  Because transactions only specify the atomicity
properties, not the locks required, the programmer's job is made easier and
implementations are free to optimistically synchronize in any way that
preserves atomicity.\footnote{\Secref{flow} presents a concrete example.}

\subsection*{Create a thread-safe double-ended queue}
% publishable result: michael/scott, 1996 PPoPP
Herlihy suggests creating a thread-safe double-ended queue using locks
as ``sadistic homework'' for computer science students \cite{Herlihy05}.
Although double-ended queues are a simple data structure,
creating a scalable locking protocol is a
non-trivial exercise.  One wants dequeue and enqueue operations to
complete concurrently when the ends of the queue are ``far enough''
apart, while safely handling the interference in the small-queue case.
In fact, the solution to this assignment was a publishable result, as
Michael and Scott demonstrated in 1996 \cite{MichaelSc96}.

The simple ``one lock'' solution to the double-ended queue
problem, ruled out as unscalable in the locking case, is scalable and
efficient for nonblocking transactions \cite{HerlihyLuMo03,Herlihy05}.

\subsection*{Handle asynchronous exceptions}
Properly handling asynchronous events is difficult with locks, because
it is impossible to safely go off to handle the event while holding an
arbitrary set of locks---and it is impossible to safely drop the
locks.  The solution implemented in the Real-Time Specification for
Java \cite{BollellaBrGoDiFuTu00}
and similar systems (see \cite[section 9]{MarlowPeMoRe01}) is to generally forbid asynchronous events within
locked regions, allowing the programmer to explicitly specify certain
points within the region at which execution can be interrupted,
dropping all locks in order to do so.  Maintaining the correctness
in the face of even explicitly declared interruption points is still
difficult.

Transactional atomic regions handle asynchronous exceptions
gracefully, by aborting the transaction to allow an
event to occur.

\section{Four new things transactions make easy}

I present four examples in this section, illustrating how transactions
can support fault tolerance and backtracking, simplify locking, and
provide a more intuitive means for specifying thread-safety
properties.  I first examine a destructive traversal algorithm,
showing how a transaction implementation can be treated as an
exception-handling mechanism.  I show how a variant of this mechanism
can be used
to implement backtracking search.  Using a network flow example, I
then show how the transaction mechanism can be used to simplify the
locking discipline required when synchronizing concurrent
modifications to multiple objects.  Finally, I show an existing race
in the Java standard libraries (in the class
\texttt{java.lang.StringBuffer}).  ``Transactification'' of the
existing class corrects this race.

\subsection*{Destructive traversal}\label{sec:destruct}
Many recursive data structures can be traversed without the use of a
stack by using pointer reversal.  This technique is widely used in
garbage collectors, and was first demonstrated in this context by
Schorr and Waite \cite{SchorrWa67}.  An implementation of a
pointer-reversal traversal of a simple singly-linked list is shown in
\figref{destruct-list}.
\begin{figure}\sis\fontsize{9}{10}\begin{verbatim}
// destructive list traversal.
void traverse(List l) {
  List last = null, t;
  
  /* zip through the list, reversing links */
  for (int i=0; i<2; i++) {
    do {
      if (i==0) visit(l); // visit node
      t = l.next;
      l.next = last;
      last = l;
      l = t;
    } while (l!=null);
    l = last;
    // now do again, backwards. (restoring links)
  }
}
\end{verbatim}
\caption{Destructive linked-list traversal.}
\label{fig:destruct-list}
\end{figure}

The \texttt{traverse()} function traverses the list, visiting nodes in
order and then reversing the {\tt next} pointer.  When the end of the
list is reached, the reversed links are traversed to restore the
list's original state.

Of course, I have chosen the simplest possible data structure here, but
the technique works for trees and graphs---and the reader may mentally
substitute their favorite hairy update on a complicated data
structure.

In normal execution, the data structure is left complete and intact
after the operation.  But
imagine that an exception or fault occurs inside the {\tt visit()} method
at some point during the traversal: an assertion fires, an exception
occurs, the hardware hiccups, or a thread is killed.  Control may
leave the {\tt traverse()} method, but the data structure is left in
shambles.  What is needed is some exception-handling procedure to
restore the proper state of the list.  This handler can be manually coded with
Java's existing {\tt try}/{\tt catch} construct, but the
exception-handling code must be tightly-coupled to the traversal if it
is going to undo the list mutations.

\note{Used to have a try/catch with explicit fail in here.}
Instead, I can provide a non-deterministic choice operator,
{\tt try}/{\tt else}, and write the recovery code at a higher level as:
\begin{inlinecode}
try {
  traverse(list);
} else { // try-else construct
  throw new Error();
}
\end{inlinecode}
\note{I'd prefer `fail t' here, but that raises the question of how
  to export objects from transactional contexts.}

The {\tt try}/{\tt else} block appears to make a non-deterministic
choice between executing the {\tt try} or the {\tt else} clause,
depending on whether the {\tt try} would succeed or not.
This construct can be straightforwardly implemented
with a transaction around the traversal,
always initially attempting
the {\tt try}.  Exceptions or faults cause the transaction to abort;
when it does so all the
heap side-effects of the {\tt try} block disappear.

\subsection*{Backtracking search}\label{sec:backtrack}
Introducing an explicit {\tt fail} statement allows us to use the same
{\tt try}/{\tt else} to facilitate backtracking
search.\index{Backtracking} Backtracking search is used to implement
practical\footnote{As opposed to the limited regular expressions
demonstrated in theory classes which are always neatly compiled to
deterministic finite automata \cite{Friedl02}.} regular expressions,
parsers, logic programming languages, Scrabble-playing
programs~\cite{AppelJa88}, and (in general) any problem with multiple
solutions or multiple solution techniques.
\note{Scrabble is an interesting case: supporting the multiple
  possible plays (and pruning the search or selecting the best)
  requires a \texttt{fail t} statement where we can send back the play
  and its score, allowing them to escape the transaction.  If we want
  to search through the opponent's possible responses to the play, then
  we need nested transactions, so that we can \texttt{fail t} the
  opponents different responses without (yet) exiting the search that
  yielded our own play.}

\begin{figure}\sis\fontsize{9}{10}\begin{verbatim}
char buffer[];
int  pos;

void eat(char token) {
  if (buffer[pos++] != token)
    fail;
}

int expr() {
   try {
     return sum();
   } else {
     return difference();
   }
}

int sum() {
   int a = number();
   eat(ADD);
   int b = number();
   return a+b;
}

int difference() {
   int a = number();
   eat(MINUS);
   int b = number();
   return a-b;
}
\end{verbatim}
\caption[A simple backtracking recursive-decent parser.]
{A simple backtracking recursive-decent parser, using transactional \texttt{try}/\texttt{else}/\texttt{fail}.}
\label{fig:backtrack-parser}
\end{figure}

As a simple example, let us consider a recursive-decent parser such as
that shown in \figref{backtrack-parser}.  We don't know whether to
apply the \texttt{sum()} or \texttt{difference()} production until
after we've parsed some common left prefix.  We can use backtracking
to attempt one rule (\texttt{sum}) and \texttt{fail} out of it
inside the \texttt{eat()} method, in the process undoing any data
structure updates performed on this path, and then attempt the other
possible production.

\subsection*{Optimistic synchronization}\label{sec:flow}\index{Network flow}

Let's now turn our attention now to parallel codes, the more
conventional application of transaction systems.
Consider a serial program for computing network flow (see, for
example, \cite[Chapter 26]{CormenLeRi01}).  The inner loop of the code
pushes flow across an edge by increasing the ``excess flow'' on one
vertex and decreasing it by the same amount on another vertex.  One
might see the following Java code:
\begin{inlinecode}
void pushFlow(Vertex v1, Vertex v2, double flow) {
  v1.excess += flow; /* Move excess flow from v1 */
  v2.excess -= flow; /* to v2.                   */
}
\end{inlinecode}

To parallelize this code, one must preclude multiple threads from
modifying the excess flow on those two vertices at the same time.
Locks provide one way to enforce this mutual exclusion: 
\begin{inlinecode}
void pushFlow(Vertex v1, Vertex v2, double f) {
  Object lock1, lock2;
  if (v1.id < v2.id) {       /* Avoid deadlock. */
    lock1 = v1; lock2 = v2;
  } else {
    lock1 = v2; lock2 = v1;
  }
  synchronized(lock1) {
    synchronized(lock2) {
      v1.excess += f; /* Move excess flow from v1 */
      v2.excess -= f; /* to v2.                   */
    } /* unlock lock2 */
  } /* unlock lock1 */
}
\end{inlinecode}

This code is surprisingly complicated and slow compared to the
original.  Space for each object's lock must be reserved.
To avoid deadlock, the code must acquire the locks in
a consistent linear order, resulting in an unpredictable branch in the
code.  In the code shown,
I have required the programmer to insert an \texttt{id} field into
each vertex object to maintain a total ordering.
The time required to acquire the locks may be
an order of magnitude larger than the time to
modify the excess flow.
\note{Using \flex, the locking code is over 11x
  slower than the no-locks code.  With Sun's JVM, this overhead falls
  to about 1.7x, because Sun is wicked smart about their lock
  implementations.}
What's more, all of this overhead is rarely
needed!  For a graph with thousands or millions of vertices, the
number of threads operating on the graph is likely to be less than a
hundred.  Consequently, the chances are quite small that two different
threads actually conflict.  Without the locks to implement mutual
exclusion, however, the program would occasionally fail.

Software transactions (and some language support) allow the
programmer to parallelize the original code using an \texttt{atomic}
keyword to indicate that the code block should appear to execute
atomically: 
\begin{inlinecode}
void pushFlow(Vertex v1, Vertex v2, double flow) {
  atomic { /* Transaction begin. */
    v1.excess += flow; /* Move excess flow from v1 */
    v2.excess -= flow; /* to v2.                   */
  } /* Transaction end. */
}
\end{inlinecode}

This {\tt atomic} region can be implemented as a transaction, and
with an appropriately nonblocking implementation, it
will scale better, execute faster, and use less memory than the locking version
\cite{AnanianAsKuLeLi05,HarrisFr03,GreenwaldCh96,MassalinPu91,HerlihyMo93,ShavitTo95}.
From the programmer's point of view, I have also eliminated the
convoluted locking protocol which must
be observed rigorously everywhere the related fields are accessed, if
deadlock and races are to be avoided.

Further, I can implement {\tt atomic} using the {\tt try}/{\tt else}
exception-handling mechanism I have already introduced:
\begin{inlinecode}
for (int b=0; ; b++) {
  try {
    // atomic actions
  } else {
    backOff(b);
    continue;
  }
  break; // success!
}
\end{inlinecode}

\note{How hard do I try to execute the {\tt try} block?}
I non-deterministically choose to execute the body of the {\tt
  atomic} block if and only if it will be observed by all to execute
atomically.  The same linguistic mechanism I introduced for
fault tolerance and backtracking provides atomic regions for
synchronization as well.
\note{Mention optimistic parallelism here?}

\subsection*{Bug fixing}\label{sec:stringbuffer}
The existing \defni{Monitor synchronization}\index{Synchronization!monitor} methodology for Java,
building on such features in progenitors such as Emerald
\cite{BlackHuJuLe86,JulSt91},\index{Emerald}\footnote{See \secref{emerald}.}
implicitly associates an 
lock with each object.
Data races are prevented by
requiring a thread to acquire an
object's lock before touching the object's shared fields.
The lack of races is not sufficient to prevent unanticipated
parallel behavior, however.

Flanagan and Qadeer \cite{FlanaganQa03} demonstrated this
insufficiency with an
actual bug they discovered in the Sun JDK 1.4.2 Java standard
libraries.  The \texttt{java.lang.StringBuffer}\index{java.lang.StringBuffer@\texttt{java.\bp{}lang.\bp{}StringBuffer}} class,
which implements a mutable string abstraction, is implemented as follows:
\label{pg:stringbuffer}\begin{inlinecode}
public final class StringBuffer ... {
  private char value[];
  private int count;
  ... 
  public synchronized
  StringBuffer append(StringBuffer sb) {
    ...
A:  int len = sb.length();
    int newcount = count + len; 
    if (newcount > value.length)
      expandCapacity(newcount);
    // next statement may use stale len
B:  sb.getChars(0, len, value, count);
    count = newcount;
    return this;
  }
  public synchronized int length() { return count; }
  public synchronized void getChars(...) { ... }
}
\end{inlinecode}

The library documentation indicates that the methods of this class are meant
to execute atomically, and the {\tt synchronized} modifiers on the
methods are meant to accomplish this.

The {\tt append()} method is \emph{not} atomic, however.  Another
thread may change the length of the parameter \texttt{sb} (by adding
or removing characters) between the call to \texttt{sb.length()} at
label A and the call to \texttt{sb.getChars(\ldots)} at label B\@.
This non-atomicity may cause incorrect data to be appended to the
target or a \texttt{StringIndexOutOfBoundException} to be thrown. 
Although the calls to
\texttt{sb.length()} and \texttt{sb.getChars()} are individually
atomic, they do not compose to form an atomic implementation of
\texttt{append()}.  
%The simple monitor synchronization scheme breaks
%down when operations touch multiple objects.

Replacing {\tt synchronized} with {\tt atomic} in
this code gives us the semantics
we desire: the atomicity of nested {\tt atomic} blocks is guaranteed
by the atomicity of the outermost block, ensuring that the entire
operation appears atomic.

Both the network flow example and this {\tt StringBuffer} example require
synchronization of
updates to more than one object.
Monitor synchronization is not
well-suited to this task.  Atomic regions implemented with
transactions can be used to simplify the locking discipline required
to synchronize multiobject mutation
and provide a more intuitive specification for the desired
concurrent properties.  Further, the {\tt StringBuffer} example shows
that simply replacing {\tt synchronized} with {\tt atomic} provides a
alternative semantics that may in fact correct existing
synchronization errors.
For many Java programs, the
semantics of {\tt atomic} and {\tt synchronized} are identical; see
\secref{semantic}.
\note{Can I make this rigorous?}

\section{Some things we still can't (easily) do}\label{sec:xlimit}
The transaction mechanism presented here is not a universal
replacement for all synchronization.  In particular, transactions
cannot replace mutual exclusion
required to serialize I/O, although the needed locks can certainly be
built with transactions.  
The challenge of integrating I/O within a
transactional context is discussed in \secref{io}.
Large programs---the
Linux kernel\note{TODO: Find citation}, for example---have been
written such that locks are
never held across context switches or I/O operations, however.  Transactions
provide a complete solution for this limited form of synchronization.

Blocking producer-consumer queues, or other options that require a
transaction to wait upon a condition variable, may introduce
complications into a transaction system: transactions cannot
immediately retry when they fail, but instead must wait for some
condition to become true.  \Secref{object-wait} describes some
solutions to this problem, ranging from \naive (keep retrying and
aborting with exponential backoff until the condition is finally true)
to clever; the clever solutions require additional transaction
machinery.


\note{
One of the original goals of this thesis was to investigate better integration
of transactional with inherently nontransactional operations.
One promising approach is the addition of explicit ``recovery'' blocks
to undo the effects of nontransactional regions embedded within
an aborting transaction.}

% LocalWords:  atomicity transactional linearization SMP LTM UTM TCC backoff
% LocalWords:  unscalable dequeue enqueue composability filesystem analyses STM
% LocalWords:  microbenchmark virtualized multicore undelayed HTM API runtime
% LocalWords:  unsynchronized Rinard Herlihy parallelize vertices JVM JDK
% LocalWords:  StringBuffer nontransactional transactifying Transactification
% LocalWords:  automata
