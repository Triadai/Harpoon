\section{Algorithms}
\label{app:algorithms}

\begin{myalgorithm}\small
% algorithm for phi/sigma placement.

Place($G$:~CFG) =
\begin{myalgorithmic}
\LET $r$ be the top-level region for $G$
\FOREACH{variable $v$ in $G$}
\STATE PlaceOne($r$, $v$, \code{false}) \COMMENT{place phis}
\STATE PlaceOne($r$, $v$, \code{true})  \COMMENT{place sigmas}
\ENDFOR
\end{myalgorithmic}
~\\
PlaceOne($r$:~region, $v$:~variable, $ps$:~boolean):~boolean =
\begin{myalgorithmic}
\STATE \COMMENT{Post-order traversal}
\STATE flag $\gets$ false
\FOREACH{child region $r'$}
  \IF{PlaceOne($r'$, $v$, $ps$)}
    \STATE flag $\gets$ true
   \ENDIF 
\ENDFOR
                             \EMPTYLINE
\FOREACH{node $n$ in region $r$ not contained in a child region}
  \IF{$ps$ is \code{false} and $n$ contains a definition of $v$}
    \STATE flag $\gets$ true
  \ENDIF
  \IF{$ps$ is \code{true} and $n$ contains a use of $v$}
    \STATE flag $\gets$ true
  \ENDIF
\ENDFOR
                             \EMPTYLINE
\STATE \COMMENT{add phis/sigmas to merges/splits where $v$ may be live}
\IF{flag = true}
  \FOREACH{node $n$ in region $r$ not contained in a child region}
    \IF{MaybeLive($n$) = true}
      \IF{$ps$ is \code{false} and the input arity of $n$ exceeds 1}
        \STATE place a phi function for $v$ at $n$
      \ENDIF
      \IF{$ps$ is \code{true} and the output arity of $n$ exceeds 1}         
        \STATE place a sigma function for $v$ at $n$
      \ENDIF
    \ENDIF
  \ENDFOR
\ENDIF
                             \EMPTYLINE
\RETURN flag
\end{myalgorithmic}

\caption{Placing \phisigfunction{s}.}\label{alg:SSIplace}
\end{myalgorithm}

\begin{myfigure}[p]\small
% data types for SSI renaming algorithm.
Data type \code{Environment}:
\begin{tightdesc}
\item[create():~Environment]
:\par   make an environment with no mappings.
\item[put($\mathcal{E}$:~Environment, $v_1$:~variable, $v_2$:~variable)]
:\par   extend environment $\mathcal{E}$ with a mapping from $v_1$ to $v_2$.
\item[get($\mathcal{E}$:~Environment, $v$:~variable):~variable]
:\par   return the current mapping in $\mathcal{E}$ for $v$.
\item[beginScope($\mathcal{E}$:~Environment)]
:\par   save the current mapping of $\mathcal{E}$ for later restoration.
\item[endScope($\mathcal{E}$:~Environment)]
:\par   restore the mapping of $\mathcal{E}$ to that present at the
        last \code{beginScope} on $\mathcal{E}$.
\end{tightdesc}

\caption{Environment datatype for the SSI renaming algorithm.}
\label{fig:SSIrename_data}
\end{myfigure}
\begin{myalgorithm}\small
% algorithm for phi/sigma renaming.

Rename($G$:~CFG) =
\begin{myalgorithmic}
\STATE $\text{Init}(G)$
\FOREACH{edge $e$ leaving \code{START}}
 \STATE $\text{Search}(e)$ \label{line:search1}
\ENDFOR
\end{myalgorithmic}
~\\
Init($G$:~CFG) =
\begin{myalgorithmic}
\FOREACH{edge $e$ in $G$}
 \STATE $\text{Marked}[e] \gets \code{false}$
\ENDFOR
\FOREACH{variable $V$ in $G$}
 \STATE $C(V) \gets 0$
\ENDFOR
\STATE $\mathcal{E} = \text{create}()$ \COMMENT{create a new environment}
\end{myalgorithmic}
~\\
Inc($\mathcal{E}$:~Environment, $V$:~variable):~variable =
\begin{myalgorithmic}
\STATE $i \gets C(V) + 1$
\STATE $C(v) \gets i$
\STATE $\mathcal{E}.\text{put}(V, V_i)$
\RETURN $V_i$
\end{myalgorithmic}

\caption{SSI renaming algorithm.}\label{alg:SSIrename1}
\end{myalgorithm}
\begin{myalgorithm}\small
% SSI renaming algorithm, cont.
Search($\tuple{s,d}$:~edge) =
\begin{myalgorithmic}
 \REQUIRE{$s$ to be a node containing \phisigfunction[or]{s}}
 \IF{$s$ is a node containing \phifunction{s}}
  \FOREACH{\phifunction $P$ in $s$}
   \STATE replace the destination $V$ of $P$ by $Inc(V)$
  \ENDFOR
 \ELSIF{$s$ is a node containing \sigfunction{s}}
  \FOREACH{\sigfunction $S$ in $s$}
   \STATE $j \gets \text{WhichSucc}(\tuple{s,d})$
   \STATE replace the $j$-th destination $V$ of $S$ by $Inc(V)$
  \ENDFOR
 \ENDIF
 \STATE\COMMENT{now rename inside basic block}
 \LOOP
  \IF{$d$ is a node containing \phifunction{s}}
   \FOREACH{\phifunction $P$ in $d$}
    \STATE $j\gets \text{WhichPred}(\tuple{s,d})$
    \STATE replace the $j$-th operand $V$ of $P$ by $Get(V)$
   \ENDFOR
   \STATE $\text{PushEdge}(\tuple{d,\text{Succ}[d]})$
   \RETURN
  \ELSIF{$s$ is a node containing \sigfunction{s}}
   \FOREACH{\sigfunction $S$ in $d$}
    \STATE replace the operand $V$ of $S$ by $Get(V)$
   \ENDFOR
   \STATE $\text{PushEdge}(\tuple{d,\text{Succ}[d]})$
   \RETURN
  \ENDIF
  \STATE\COMMENT{ordinary assignment}
  \FOREACH{variable $V$ in $LHS(d)$}
   \STATE replace $V$ by $Get(V)$ in $LHS(d)$
  \ENDFOR
  \FOREACH{variable $V$ in $RHS(d)$}
   \STATE replace $V$ by $Inc(V)$ in $RHS(d)$
  \ENDFOR
  \IF{$d$ has a successor}
   \STATE $s \gets d$
   \STATE $d \gets \text{Succ}[d]$
  \ELSE
   \RETURN
  \ENDIF
 \ENDLOOP
\end{myalgorithmic}

\caption{SSI renaming algorithm, cont.}\label{alg:SSIrename2}
\end{myalgorithm}

\begin{myfigure}\small
% data types for cycle-equivalency algorithm.
Operations on nodes:
\begin{tightdesc}
\item[NodeUseful(n:node):~boolean]
:	Whether the results of this node are ever used
\item[Uses(n:node):~set of variables]
:	Variables for which this node contains a use
\end{tightdesc}

Operations on variables:
\begin{tightdesc}
\item[VarUseful(v:variable):~boolean]
:	Whether there is some \textbf{n} for which 
        \textbf{Uses(n)} contains \textbf{v} and
        \textbf{NodeUseful(n)} is \texttt{true}
%:	$\exists n\in\domain{Nodes}\;.\;v\in\mbox{Uses}(n)\wedge\mbox{NodeUseful}(n)$
%:	Whether the value of this variable is ever used
\item[Definitions(v:variable):~set of nodes]
:	Nodes which contain a definition for~\textbf{v}
\end{tightdesc}

\caption{Datatypes and operations used in unused code elimination.}
\label{fig:deaddata}
\end{myfigure}

\begin{myalgorithm}\small\linespread{0.75}
% fast algorithm for dead code elimination (unused code elimination)
\begin{verbatim}
Procedure FindUseful (G: CFG) {
  let W be an empty work list;
  for each variable v, do
    VarUseful(v) := false;
  for each node n in G, in any order, do {
    NodeUseful(n) := false;
    if n is a CALL, RETURN, or other special node, then
      add n to W;
  }

  while W is not empty, do {
    let n be any element from W;
    remove n from W;
    MarkNodeUseful(n, W);
  }
}
Procedure MarkNodeUseful(n: node, W: WorkList) {
  NodeUseful(n) := true;
  /* everything used by a useful node is useful */
  for each v in Uses(n), do
    if not VarUseful(v), then
      MarkVarUseful(v, W);
}
Procedure MarkVarUseful(v: variable, W: WorkList) {
  VarUseful(v) := true;
  /* The definition of a useful variable is useful */
  for each n in Definitions(v), then
    /* size(Definitions(v)) <= 1 in SSI form */
    if not NodeUseful(n), then
      add n to W;
}
\end{verbatim}
\caption{Identifying unused code using SSI form.}
\label{alg:deadalg}
\end{myalgorithm}

\begin{myalgorithm}\small
Init($G$:CFG) =
\begin{algorithmic}[1]
\STATE $E_e \gets \emptyset$
\STATE $E_n \gets \emptyset$
\FOREACH {variable $v$ in $G$}
 \IF{some node $n$ defines $v$}
  \STATE $V[v] \gets \bot$
 \ELSE
  \STATE $V[v] \gets \top$
 \ENDIF
\ENDFOR
\end{algorithmic}
~\\
Analyze($G$:CFG) =
\begin{algorithmic}[1]
\LET $r$ be the start node of graph $G$
\STATE $E_n \gets E_n \cup \{ r \}$
\STATE $W_n \gets \{ r \}$
\STATE $W_v \gets \emptyset$
          \EMPTYLINE
\REPEAT
 \IF{$W_n$ is not empty}
  \STATE remove some node $n$ from $W_n$
  \IF{$n$ has only one outgoing edge $e$ leading to some node $n'$}
   \STATE RaiseE($e$)
  \ENDIF
  \STATE Visit($n$)
 \ENDIF
 \IF{$W_v$ is not empty}
  \STATE remove some variable $v$ from $W_v$
  \FOREACH{node $n$ containing a use of $v$}
   \STATE Visit($n$)
  \ENDFOR
 \ENDIF
\UNTIL{both $W_v$ and $W_n$ are empty}
\end{algorithmic}

\caption{SCC algorithm for SSA form.}\label{alg:scc}
\end{myalgorithm}
\begin{myalgorithm}\small
RaiseE($e$:edge) =
\begin{algorithmic}[1]
\STATE $E_e \gets E_e \cup \{ e \}$
\LET $n$ be the destination of edge $e$
\IF{$n \not\in E_n$}
\STATE $E_n \gets E_n \cup \{ n \}$
\STATE $W_n \gets W_n \cup \{ n \}$
\ENDIF
\end{algorithmic}
~\\
RaiseV($v$:variable, $L$:lattice value) =
\begin{algorithmic}[1]
\IF{$V[v] \sqsupset L$}
\STATE $V[v] \gets L$
\STATE $W_v \gets W_v \cup \{ v \}$
\ENDIF
\end{algorithmic}
~\\
Visit($n$:node) =
\begin{algorithmic}[1]
\FOREACH{assignment ``$v \gets x \oplus y$'' in $n$}
 \STATE RaiseV($v$, $V[x] \oplus V[y]$)
 \COMMENT{binop rule: see table~\ref{tab:sccmeet1}}
\ENDFOR
               \EMPTYLINE
\FOREACH{assignment ``$v \gets \text{MEM}(\ldots)$'' or 
                    ``$v \gets \text{CALL}(\ldots)$'' in $n$}
 \STATE RaiseV($v$, $\top$)
\ENDFOR
               \EMPTYLINE
\FOREACH{assignment ``$v \gets \phi(x_1,\ldots,x_n)$'' in $n$}
 \FOREACH{variable $x_i$ corresponding to predecessor edge $e_i$ of $n$}
  \IF{$e_i \in E_e$}
   \STATE RaiseV($v$, $V[v] \meet V[x_i]$)
   \COMMENT{meet rule: see table~\ref{tab:sccmeet1}}
  \ENDIF
 \ENDFOR
\ENDFOR
               \EMPTYLINE
\FOREACH{branch ``if $v$ goto $e_1$ else $e_2$'' in $n$}
 \STATE $L \gets V[v]$
 \IF{$L=\top$ or $L=c$ where $c$ signifies ``true''}
  \STATE RaiseE($e_1$)
 \ENDIF
 \IF{$L=\top$ or $L=c$ where $c$ signifies ``false''}
  \STATE RaiseE($e_2$)
 \ENDIF
\ENDFOR
\end{algorithmic}
\caption{SCC algorithm for SSA form, cont.}\label{alg:scc2}
\end{myalgorithm}

\section{Proofs}
\label{app:proofs}

\noindent
Proof of Lemma~\ref{lem:sese_child}.
\begin{proof}
Let us assume a \phifunction for $v$ is needed at some node $Z$
inside an SESE not containing a definition of $v$.  
Then by the path-convergence criterion for \phifunction{s},
there exist paths $X \pathplus Z$ and $Y \pathplus Z$
having no nodes but $Z$ in common where $X$ and $Y$ contain either
definitions of $v$ or \phisigfunction[or]{s} for $v$.  Choose any such
paths:
\begin{description}
\item[Case I:] Both $X$ and $Y$ are outside the SESE.  Then, as there
is only one entrance edge into the SESE, the paths $X \pathplus Z$ and
$Y \pathplus Z$ must contain some node in common other than Z.  But
this contradicts our choice of $X$ and $Y$.
\item[Case II:] At least one of $X$ and $Y$ must be inside the SESE.
If both $X$ and $Y$ are not definitions of $v$ but rather
\phisigfunction[or]{s} for $v$, then by recursive application of this proof
there must exist some choice of $X$, $Y$, and $Z$ inside this SESE
where at least one of $X$ and $Y$ is a definition.  But $X$ or $Y$
cannot be a definition of $v$ because they are inside the SESE of $Z$ which
was chosen to contain no definitions of $v$.
\end{description}

A symmetric argument holds for \sigfunction{s} for $v$, using
the path-convergence criterion for \sigfunction{s}, and the fact
that there exists one exit edge from the SESE.
\end{proof}

\noindent
Proof of Lemma~\ref{lem:ssi_place_dom}.
\begin{proof}
We will first prove that a node $N$ failing any one of the conditions does
not need a \phisigfunction[or].
\begin{itemize}
\item The path-convergence criteria for \phifunction{s} (\sigfunction{s})
require node $N$ to be the first convergence
(divergence) of some paths $X \pathplus N$ and $Y \pathplus N$ ($N
\pathplus X$ and $N \pathplus Y$).  If the input arity is less than 2
or there is no path from a definition of $v$, than it fails the
path-convergence criterion for \phifunction{s}.  If the output arity
is less than 2 or there is no path to a use of $v$, then it fails the
path-convergence criterion for \sigfunction{s}.
\item If there exists a SESE containing $N$ that does not contain any
definition, \phisigfunction[or]{} $D$ for $v$, then $N$ does not require a
\phisigfunction[or]{} for $v$ by lemma~\ref{lem:sese_child}.
\item Let us suppose every $D_i$ containing a definition,
\phisigfunction[or]{} for $v$ dominates $N$.  If $N$ requires a
\phifunction for $v$,  there exist paths $D_1 \pathplus N$ and
$D_2 \pathplus N$ containing no nodes in common but $N$.  We use these
paths to construct simple paths $\code{START}\pathplus D_1 \pathplus N$ and
$\code{START}\pathplus D_2 \pathplus N$.  By the definition of a
dominator, every path from \code{START} to $N$ must contain every
$D_i$.  But $D_1 \pathplus N$ cannot contain $D_2$, and if
$\code{START} \pathplus D_1$ contains $D_2$, we can make a path
$\code{START} \pathplus D_2 \pathplus N$ which does not contain $D_1$
by using the $D_1$-free path $D_2 \pathplus N$.  The assumption leads
to a contradiction; thus, there must exist some $D_i$ which does not
dominate $N$ if $N$ is required to have a \phifunction for $v$.  The
symmetric argument holds for post-dominance and \sigfunction{s}.
\end{itemize}
This proves that the conditions are necessary.  It is obvious from an
examination of the path convergence criteria for \phisigfunction{s}
and lemma \ref{lem:sese_child} that they are sufficient.
\end{proof}

The SSI renaming algorithm presented in Figures~\ref{alg:SSIrename1}
and \ref{alg:SSIrename2} requires
an \code{Environment} datatype which is defined in
Figure~\ref{fig:SSIrename_data}.  Using an imperative programming
style, it is possible to perform a sequence of any $N$ operations on
\code{Environment} as defined in the figure in $O(N)$ time; in a
functional programming style any $N$ operations can be completed in
$O(N \log N)$ time.\footnote{The curious reader is referred to section
5.1 of Appel \cite{appel:modern} for implementation details.}  As the
coarse structure of Algorithm~\ref{alg:SSIrename1} is a simple
depth-first search, it is easy to see that the \code{Search} procedure
can be invoked from line~\fullref{line:search1} and
line~\fullref{line:search2} a total of $O(E)$ times; likewise its
inner loop (lines~\ref{line:searchloop_start} to
\ref{line:searchloop_end}) can be executed a total of $E$ times across
all invocations of \code{Search}.  A total of $U_{SSA}+D_{SSA}$ calls
to the operations of the \code{Environment} datatype will be made
within all executions of \code{Search}.  For the imperative
implementation of \code{Environment} a total time bounds of
$O(E+U_{SSA}+D_{SSA})$ for the variable renaming algorithm is
obtained.

\begin{lemma}\label{lem:path_construct}
The stack trace of calls to \code{Search} defines a unique path
through $G$ from \code{START}.
\end{lemma}
\begin{proof}
We will prove this lemma by construction.  For every consecutive pair
of calls to \code{Search} we construct a path $X\pathplus Y$ starting with the
edge $\tuple{X,N_0}$ which is the argument of the first call, and
ending with the edge $\tuple{N_n, Y}$ which is the argument of the
second call.  From line~\ref{line:search_onesucc} of the \code{Search}
procedure on page~\pageref{line:search_onesucc} we note that every
edge $\tuple{N_i, N_{i+1}}$ between the first and last has exactly one
successor.  Furthermore, the call to search on line~\ref{line:search2}
defines a path starting with the edge which our segment $X\pathplus Y$
ends with; therefore the paths can be combined.  By so doing from the
bottom of the call stack to the top we construct a unique path from
\code{START}.
\end{proof}

For brevity, we will hereafter refer to the canonical path constructed
in the manner of lemma~\ref{lem:path_construct} corresponding to the
stack of calls to \code{Search} when an edge $e$ is first
encountered as $CP(e)$.  Every edge in the CFG is encountered exactly
once by \code{Search}, so $CP(e)$ exists and is unique for every edge
$e$ in the CFG.

\begin{lemma}\label{lem:renamephi}
SSI form property~\ref{crit_phiname} (\phifunction{} naming) holds
for variables renamed according to Algorithm~\ref{alg:SSIrename1}.
\end{lemma}
\begin{proof}
We restate SSI form property~\ref{crit_phiname} for reference:
\begin{quote}
For every node $X$ containing a definition of a variable $V$ in
the new program and node $Y$ containing a use of that variable, there
exists at least one path $X \pathplus Y$ and no such path contains a
definition of $V$ other than at $X$.
\end{quote}
We consider the canonical path 
$CP(\tuple{Y',Y})=\code{START}\pathstar Y' \path Y$
for some use of a variable $v$ at $Y$, constructed according to
lemma~\ref{lem:path_construct} 
from a stack trace of calls to \code{Search}.
is encountered.  This path is unique, although more than one canonical
path may terminate at $Y$ at nodes with more than one predecessor.
These paths are distinguished by the incoming edge to
$Y$.\footnote{Note that the notation \tuple{N,N'} for denoting edges
does not always denote an edge unambigiously; imagine a conditional
branch where both the \code{true} and \code{false} case lead to the
same label.  In such cases an additional identifier is necessary to
distinguish the edges.  Alternatively, one may split such edges to
remove the ambiguity.  We treat edges as uniquely identifiable and
leave the implementation to the reader.}  We identify each operand
$v_i$ of a \phifunction{} with the appropriate incoming edge $e$ to
ensure that $CP(e)$ is well defined and unique in the context of a
use of $v_i$.

The canonical path $\code{START}\pathplus Y$ must contain $X$, a definition of
$v$, if $Y$ uses a variable defined in $X$, as \code{Search} renames
all definitions (in lines \ref{line:rendef1}, \ref{line:rendef2}, and
\ref{line:rendef3}) and destroys the name mapping in $\mathcal{E}$
just before it returns.  The call to \code{Search} which creates the
definition of $v$ must therefore always be on the stack, and thus in
the path $CP(\tuple{Y',Y})$, for any use to receive a the name $v$.
Note that this is
true for \phifunction{s} as well, which receive names when the
appropriate incoming edge $\tuple{Y',Y}$ is traversed, not necessarily
when the node $Y$ containing the \phifunction{} is first encountered.

We have proved that $\code{START}\pathplus X\pathplus Y$ exists; now
we must prove that no other path from $X$ to $Y$ contains a definition
of $v$.  Call this other definition $D$.  Obviously $D$ cannot be on
our canonical path $\code{START}\pathplus X\pathplus Y$, or
line~\ref{line:rendef3} would have caused $Y$ to use a different name.
But as we just stated, all variable name mappings done by $D$ will be
removed when the call to \code{Search} which touched $D$ is taken off
the call stack.  So $D$ must be on the call stack, and thus on the
canonical path; a contradiction. %
%
Since assuming the existence of some other path $X\pathplus Y$
containing a definition of $v$ leads to contradiction no other such
path may exist, completing the proof of the lemma.
\end{proof}

\begin{lemma}\label{lem:renamesig}
SSI form property~\ref{crit_signame} (\sigfunction{} naming) holds for
variables renamed according to Algorithm~\ref{alg:SSIrename1}.
\end{lemma}
\begin{proof}
We restate SSI form property~\ref{crit_signame} for reference:
\begin{quote}
For every pair of nodes $X$ and $Y$ containing uses of a
variable $V$ defined at node $Z$ in the new program, either every path
$Z \pathplus X$ must contain $Y$ or every path $Z \pathplus Y$ must
contain $X$.
\end{quote}
Let us assume there are paths $Z\pathplus X$ and $Z\pathplus Y$
violating this condition; that is, let us chose nodes $X$ and $Y$
which use $V$ and $Z$ defining $V$ such that there exists a path $P_1$
from $Z$ to $X$ not containing $Y$ and a path $P_2$ from $Z$ to $Y$ not
containing $X$.  By the argument of the previous lemma, there exists
a canonical path $P_3=CP(e)$ from \code{START} to $X$ through $Z$
corresponding to a stack
trace of \code{Search}; note that $P_3$ need not contain $P_1$.
There are two cases:
\begin{description}
\item[Case I:] $P_3$ does not contains $Y$.  Then there is some last
node $N$ present on both $P_2: Z\pathstar N\pathplus Y$ and
$P_3: \code{START}\pathplus Z\pathstar N\pathplus X$.  By 
the path-convergence criterion for \sigfunction{s},
this node $N$ requires a \sigfunction{}
for $V$.  If $N\not=Z$ then line~\ref{line:rendef1} of
Algorithm~\ref{alg:SSIrename1} would rename $V$ along $P_3$
and $X$ would not use the same variable $Z$ defined; if
$N=Z$, then line~\ref{line:rendef2} would have ensured that $X$
and $Y$ used different names.  Either case contradicts our choices of
$X$, $Y$, and $Z$.
\item[Case II:] $P_3$ does contain $Y$.  Then consider the path
$\code{START}\pathplus Z\pathplus Y$ along $P_3$, which does not
contain $X$.  The argument of case I applies with $X$ and $Y$ reversed.
\end{description}
Any assumed violation of property~\ref{crit_signame} leads to
contradiction, proving the lemma.
\end{proof}

Every path $CP(e)$ corresponds to a execution state in a call to
\code{Search} at the point where $e$ is first encountered.  The value
of the environment mapping $\mathcal{E}$ at this point in the
execution of Algorithm~\ref{alg:SSIrename1} we will denote as
$\mathcal{E}^e$.  For a node $N$ having a single predecessor $N_p$ and
single successor $N_s$, we will denote
$\mathcal{E}^{\tuple{N_p,N}}$ as $\mathcal{E}_{\text{before}}^N$ and 
$\mathcal{E}^{\tuple{N,N_s}}$ as $\mathcal{E}_{\text{after}}^N$.
It is obvious that 
$\mathcal{E}_{\text{after}}^{N_p} = \mathcal{E}_{\text{before}}^{N  }$ and
$\mathcal{E}_{\text{after}}^{N  } = \mathcal{E}_{\text{before}}^{N_s}$
when $N_p$ and $N_s$, respectively, are also single-predecessor
single-successor nodes.

\begin{lemma}\label{lem:correctness}
SSI form property~\ref{crit_correct} (correctness) holds for
variables renamed according to Algorithm~\ref{alg:SSIrename1}.  That
is, along any possible control-flow path in a program being executed a
use of a variable $V_i$ in the new program will always have the same
value as a use of the corresponding variable $V$ in the original
program.
\end{lemma}
\begin{proof}
We will use induction along the path $N_0\path N_1\path\ldots\path N_n$.
We consider $e_k=\tuple{N_{k},N_{k+1}}$, the $(k+1)$th edge in the path,
and assume that, for all $j<k$, each variable $V$ in the original
program agrees with the value of $\mathcal{E}^{e_j}[V]=V_i$ in the new
program.  We show that $\mathcal{E}^{e_k}[V]$ agrees with $V$ at edge
$e_k$ in the path.
\begin{description}
\item[Case I:] $k=0$. The base case is trivial: the \code{START} node
($N_0$) contains no statements, and along each edge $e$ leaving start
$\mathcal{E}^e[V]=V_0$.  By definition $V_0$ agrees with $V$ at the
entry to the procedure.
\item[Case II:] $k>0$ and $N_k$ has exactly one predecessor and one successor.
If $N_k$ is single-entry single-exit, then it is not a \phisigfunction[or].
As an ordinary assignment, it will be handled by
lines~\ref{line:rename_ordinary1} to \ref{line:rename_ordinary2} of
Algorithm~\vref{alg:SSIrename2}.  By the induction hypothesis (which
tells us that the uses at $N_k$ correspond to the same values as the
uses in the original program) and the semantics of
assignment, the mapping $\mathcal{E}_{\text{after}}^{N_k}$ is easily
verified to be valid when $\mathcal{E}_{\text{before}}^{N_k}$ is
valid.  Thus the value of every original variable $V$ corresponds to
the value of the new variable 
$\mathcal{E}_{\text{after}}^{N_k}[V]=\mathcal{E}^{e_k}[V]$ on $e_k$.
\item[Case III:] $k>0$ and $N_k$ has multiple predecessors and one
successor.  In this case $N_k$ may have multiple \phifunction{s} in
the new program, and \mycomment{by the definition in section~\ref{sec:defs}} $N_k$
has no statements in the original program.  Thus the value of any
variable $V$ in the original program along edge $e_k$ is identical to
its value along edge $e_{k-1}$.  We need only show that the value of
the variable $\mathcal{E}^{e_{k-1}}[V]$ is the same as the value of
the variable $\mathcal{E}^{e_k}[V]$ in the new program.  For any
variable $V$ not mentioned in a \phifunction{} at $N_k$ this is
obvious.  Each variable defined in a \phifunction{} will get the value
of the operand corresponding to the incoming control-flow path edge.
The relevant lines in Algorithm~\ref{alg:SSIrename2} start with
\ref{line:phisrc1} and \ref{line:phisrc2}, where we see that the
operand corresponding to edge $e_{k-1}$ of a \phifunction{} for $V$
correctly gets $\mathcal{E}^{e_{k-1}}[V]$.  At
line~\ref{line:rendef1}, we see that the destination of the
\phifunction{} is correctly $\mathcal{E}^{e_k}[V]$.  Thus the value of
every original variable $V$ correctly correponds to
$\mathcal{E}^{e_k}[V]$ by the induction hyptothesis and the semantics
of the \phifunction{s}.
\item[Case IV:] $k>0$ and $N_k$ has one predecessor and multiple
successors.  Here $N_k$ may have multiple \sigfunction{s} in the new
program, and is empty in the original program.  The argument goes as
for the previous case.  It is obvious that variables not mentioned in
the \sigfunction{s} correspond at $e_k$ if they did at $e_{k-1}$.  For
variables mentioned in \sigfunction{s}, line~\ref{line:sigsrc} shows
that operands correctly get $\mathcal{E}^{e_{k-1}}[V]$ and
line~\ref{line:rendef2} shows that the destination corresponding to
$e_k$ correctly gets $\mathcal{E}^{e_k}[V]$.  Therefore the values of
original variables $V$ correspond to the value of
$\mathcal{E}^{e_k}[V]$ by the induction hypothesis and the semantics
of the \sigfunction{s}.
\mycomment{
\item[Case V:] $N_k$ has multiple predecessors and multiple
successors.  Forbidden by the CFG definition in section~\ref{sec:defs}.
some variable $V$) follows.
}
\end{description}
Therefore, on every edge of the chosen path, the values of the
original variables correspond to the values of the renamed SSI form
variables. The value correspondence at the path endpoint (a use of
\end{proof}

\section{Optimistic and Pessimistic Algorithms}
\label{app:optimistic}

In our experience, optimistic algorithms tend to
have poor time bounds because of the possibility of input graphs like
the one illustrated in Figure~\vref{fig:evil}.
Proving that all but two nodes require
\phisigfunction[and/or]{s} for the variable $a$ in this example seems to
inherently require $O(N)$ passes over the graph; each pass can prove
that \phisigfunction[or]{s} are required for only those nodes adjacent to
nodes tagged in the previous pass.  Starting with the circled node, the
\phisigfunction{s} spread one node left on each pass. On the other hand,
a pessimistic algorithm assumes the correct answer at the start, fails
to show that any \phisigfunction[or]{s} can be removed, and
terminates in one pass.\dontfixme{Are we \emph{sure} similar worst cases
don't exist for the pessimistic algorithm?}

\begin{myfigure}[t]
\centering\renewcommand{\figscale}{0.25}\input{Figures/evil}
\caption{A worst-case CFG for ``optimistic'' algorithms.}
\label{fig:evil}
\end{myfigure}
