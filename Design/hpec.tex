% -*- latex -*- This is a LaTeX document.
% $Id: hpec.tex,v 1.6 2004-05-31 02:37:23 cananian Exp $
%%%%%%%%%%%%%%%%%%%%%%%%%%%%%%%%%%%%%%%%
\documentclass[twoside,twocolumn,notitlepage,letterpaper,9pt]{article}
\usepackage{times}
\usepackage{supertech}
\usepackage{support}

\setlength{\oddsidemargin}{-0.25in}
\setlength{\evensidemargin}{-0.25in}
\setlength{\textwidth}{7in}
\setlength{\topmargin}{-0.7in}
\setlength{\textheight}{9.7in}
\addtolength{\topmargin}{-\headheight}
\addtolength{\topmargin}{-\headsep}
\setlength{\columnsep}{0.25in}

% areas:
% Algorithm Mapping to High Performance Architectures
% Software Architectures, Reusability, Scalability, and Standards
% Fault-Tolerant Hardware and Software Techniques
% Automated Tools for Embedded System Development

\title{Language-level Transactions for Modular Reliable Systems}
\gdef\and{\hspace*{4em}}%Tight title with one line for authors
\author{C.~Scott~Ananian \and Martin~Rinard \\
Computer Science and Artificial Intelligence Laboratory\\
Massachusetts Institute of Technology\\ 
Cambridge, MA 02139 \\
\texttt{\{cananian,rinard\}@csail.mit.edu}
}
\date{}

\begin{document}
\maketitle
\support{This research was supported by DARPA/AFRL Contract F33615-00-C-1692.}

The transaction model is a natural means to express atomicity,
fault-tolerance, synchronization, and exception handling in reliable
programs.  A (lightweight, in-memory) transaction can be thought of as
a sequence of program loads and stores which either \defn{commits} or
\defn{aborts}.  If a transaction commits, then all of the loads and
stores appear to have run atomically with respect to other
transactions.  That is, the transaction's operations appear not to
have been interleaved with those of other transactions or
non-transactional code.  If a transaction aborts, then none of its
stores take effect and the transaction can be safely restarted,
typically using a backoff algorithm to preclude live-lock.
A subset of the traditional ACID database semantics are provided.

Although transactions can be implemented using mutual exclusion
(locks), we present algorithms utilizing non-blocking synchronization
to exploit optimistic concurrency among transactions and provide
fault-tolerance.  A process which fails while holding a lock within a
critical region can prevent all other non-failing processes from ever
making progress.  It is in general not possible to restore the locked
data structures to a consistent state after such a failure.
Non-blocking synchronization offers a graceful solution to this
problem, as non-progress or failure of any one thread or module will
not affect the progress or consistency of other threads or the system.

Implementing transactions using non-blocking synchronization offers
performance benefits as well.  Even in a failure-free system, page
faults, cache misses, context switches, I/O, and other unpredictable
events may result in delays to the entire system when mutual exclusion
is used to guarantee the atomicity of operation sequences;
non-blocking synchronization allows undelayed processes or processors
to continue to make progress.  Similarly, in real-time systems, the
use of non-blocking synchronization can prevent \emph{priority
  inversion} in the system by allowing high priority threads to abort
lower priority threads at any point.
%\cite{Jones97}

We show how to integrate non-blocking transactions into an
object-oriented language, ``transactifying'' existing code to fix
existing concurrency bugs and using transactions for modular fault-tolerance,
backtracking, exception-handling, and concurrency control in new
programs.

We propose the use of compiler-supported ``atomic'' blocks to
specify synchronization.  This is less error-prone than manual
maintenance of a locking discipline: deadlocks may be introduced when
locks are not acquired and released in a highly disciplined manner,
and the specification of locking discipline cuts across module
boundaries.  Races are common when multiple shared objects are
involved in an operation, each with its own lock.  We provide several
examples of such problematic locking code.  A non-blocking transaction
implementation prevents inadvertent deadlocks, and \texttt{atomic}
declarations implemented with the transaction mechanism can extend
across method invocations and module boundaries to protect multiple
objects involved in an operation without allowing races between them.
An optimistic non-blocking implementation provides performance
improvements over locking strategies in some cases as well.

Language-level transactions are used as a general
exception-handling and backtracking mechanism.  Instead of forcing the
programmer to manually track changes made to program state in order to
implement proper fault recovery, we can handle the exception using
transaction rollback to automatically restore a safe program state,
even if the fault occurred in the middle of mutating shared objects.
An efficient and graceful transaction mechanism integrated into the
programming language encourages a robust programming style where
recovery and retry after an unexpected condition is made simple and
faults and recovery do not break abstraction boundaries.

We describe an efficient pure-software transaction mechanism we have
implemented for programs written in Java.  We also discuss our design
and simulation (with Asanovi\'c, Kuszmaul, Leiserson, and Lie) of
minimally-intrusive architecture extensions which allow most
transactions to complete with near-zero overhead.  Unlike previous
hardware approaches, our scheme is scalable and supports transactions
of unlimited size, although performance is best for transactions which
fit in local cache.  Finally, we describe our hybrid hardware-software
scheme combining the speed hardware provides for small
transactions with the flexibility the software implementation allows
for large or long-lived transactions.

Integrating transactions into the programming language and
implementing them with the high-efficiency techniques described
enables the creation of software with higher reliability.
Synchronization is more robust and its specification is modular and less
error-prone, and faults and exceptions in general can be
soundly handled with low overhead using the transaction mechanism.

%\bibliographystyle{plain}
%\bibliography{xaction}
\end{document}
% LocalWords:  csail mit edu AFRL LaMarca Alemany Felten Luchango Shavit JVM
% LocalWords:  Touitou Okasaki Qadeer Promela backoff transactified SPECjvm
% LocalWords:  atomicity transactifying transactional
