% -*- latex -*- This is a LaTeX document.
% $Id: proposal.tex,v 1.7 1999-09-03 23:12:06 cananian Exp $
%%%%%%%%%%%%%%%%%%%%%%%%%%%%%%%%%%%%%%%%%%%%%%%%%%%%%%%%
\documentclass[12pt,oneside]{article}
%\usepackage{pdffonts}
\usepackage{beton}\usepackage{euler}
\usepackage{comdef}

\linespread{1.2}

\title{{\large Master's Thesis proposal for}\\
FLEX: An Optimizing Java Compiler using SSI Form}
\author{C. Scott Ananian}
\date{March 1999\\\vspace{5pt}}

\begin{document}
\bibliographystyle{abbrv}
\begin{titlepage}
\setlength{\baselineskip}{0.9\baselineskip}
\newcommand{\nl}{\\[0.4\baselineskip]}
\newcommand{\tight}{\\[-0.1\baselineskip]}
\newcommand{\tighter}{\\[-0.2\baselineskip]}
\begin{centering}\large
Massachusetts Institute Of Technology\tight
Department of Electrical Engineering and Computer Science\nl
%
Proposal for Thesis Research in Partial Fulfillment\tight
of the Requirements for the Degree of\tight
Master of Engineering\nl
\end{centering}
\vspace{0.1cm}
\begin{tabbing}
Title: \textbf{FLEX: An Optimizing Java Compiler using SSI Form}\nl
Submitted By: \= C. Scott Ananian \hspace{3cm}\=\rule{6cm}{0.5pt}\tighter
              \> Ashdown House, Room 510A     \>(signature)\tighter
              \> Cambridge, MA 02139\nl
Date of Submission: March 1999\nl
Expected Date of Completion: September 1999\nl
Laboratory where thesis will be done: Laboratory for Computer
Science%\nl
\end{tabbing}
Brief Statement of the Problem:

An investigation into the use of novel intermediate representations is
proposed in the context of a sophisticated compiler for the Java
programming language.  The compiler will be constructed and both
theoretical and practical methods will be used to evaluate the
effectiveness of the intermediate representation.  Analysis and
optimization techniques based on the representation will be
implemented and presented.

~\\Supervision Agreement:

The program outlined in the proposal is adequate for a Master's
thesis.  The supplies and facilities are available, and I am willing
to supervise the research and evaluate the thesis report.

\begin{tabbing}
\hspace{3.5in}\=\kill
\>\rule{2.5in}{0.5pt}\\
\>M. Rinard, Prof. of Comp. Sci.\\
\end{tabbing}

\end{titlepage}

The FLEX compiler project is an ongoing effort to create intelligent
compiler technology to address embedded system and hardware design
issues for safe high-level languages, in particular, Java.
Current compilers have difficulty understanding  enough of the
high-level structure 
of a program to do reasonable optimization of threading and memory
management code, especially when pointers are present. Among the goals
of the FLEX project are automatic code partitioning for distributed
systems, memory allocation strategy conversion for embedded systems,
and the use of Java as a high-level hardware description language.

This thesis concentrates on the code representation used in the
FLEX compiler; specifically, on Static Single Information (SSI)
form, which is introduced as a more general replacement for the
prevalent SSA form.  Despite industry adoption
\cite{chow97:ssapre,chow96:hssa}, SSA form \cite{cytron91:ssa} is
widely acknowledged to have weaknesses, particularly for backwards
dataflow analyses \cite{johnson93:dfg}.  The FLEX project plans to utilize
\emph{predicated analysis}, which uses control-path information to
derive constraints on values; SSA form falls short in this application
as well.  In addition, there are no clearly defined semantics for SSA
form for use with the techniques of abstract interpretation
\cite{pingali90:dfg}.

We are proposing SSI form as a remedy for these deficiencies. The
proposed form is symmetrical for both forward and
reverse dataflow analysis, and represents information generated both
at variable definitions and at branches in control flow.  It seems
possible to create efficient creation and analysis algorithms based on
cycle equivalence properties of the underlying control-flow graph;
this work will attempt to construct and describe these algorithms and
prove their correctness.  Further, we will attempt to enumerate a
precise semantics for SSI form that will enable its use for abstract
interpretation and hardware compilation.

SSI form seems to enable simple and efficient code optimization, with obvious
benefits over SSA form for constructs (such as null-pointer and
array-bounds checks) common in Java code.  In addition, a novel
bitwidth analysis will be attempted which is suitable for both hardware
synthesis and code generation for vector units such as Intel's MMX.
Loop analyses, reverse dataflow analysis, and constraint systems can be
reexamined using SSI form; we hope to be able to derive new analyses
and more efficient versions of existing analyses.  Furthermore, it
seems that there exists a connection to the dataflow computing work of
Traub \cite{traub86:ttda}; we will investigate and describe the ways
our work intersects the field of dataflow computing.

\begin{table}[t]\centering
\newcommand{\major}[2]{\textbf{#1}\dotfill\textbf{#2}}
\newcommand{\minor}[2]{~~#1\dotfill #2\hspace*{0.75cm}}
\begin{tabular}{|l|}\hline
Task\hfill Hours\\\hline
\major{Implementation (total)}{130}\\
 \minor{Compiler infrastructure}{50}\\
 \minor{SSI construction algorithms}{40}\\
 \minor{Analysis and optimization algorithms}{40}\\
\major{Analysis (total)}{100}\\
 \minor{Properties of form}{30}\\
 \minor{Properties of construction algorithms}{20}\\
 \minor{Power and complexity of SSI-based analyses\ldots}{20}\\
 \minor{Properties of extended \ssiplus\ form}{30}\\
\major{Data collection (total)}{50}\\
 \minor{Time complexity of algorithms}{20}\\
 \minor{Effectiveness of optimization techniques}{20}\\
 \minor{Interpretation of collected data}{10}\\
\major{Preparation of report}{80}\\\hline
%\hfill Total:\hspace{1cm}\textbf{180}\\\hline
\end{tabular}
\caption{Time budget for proposed tasks.}\label{tab:time}
\end{table}
The thesis procedure will involve analysis, implementation, and data
collection.  Table~\ref{tab:time} summarizes the amount of time we
plan to spend on each task.  The next paragraphs will lay out the work
roughly chronologically.

The first task will be the completion of portions of
the existing FLEX compiler infrastructure in order to support our
research.  We have budgeted 50 hours for this over the course of the
project, but only about 30 hours will be needed before we can begin
implementing the SSI algorithms.  Prior to implementation---and
probably during, as bugs are discovered---we will need to formalize the
definitions of the proposed SSI form and prove properties needed for
our algorithms.  We will probably iterate the
analysis/implementation cycle in order to improve our implementation's
efficiency.  In sum, we have allotted 90 hours to the SSI construction
algorithms and analysis of the form.

We will then use the constructed SSI representation as the base for
optimization and analysis techniques.  Again, the time complexity and
effectiveness properties of the algorithms need to be analyzed
concurrent with their implementation.  For this we have allocated a
total of 60 hours.

Somewhat orthogonally, the extended \ssiplus\ form will be
investigated as a foundation for abstract analysis and hardware
compilation techniques.  It is not clear that time limitations will
permit extensive implementation for this extended form; we have
allocated 30 hours for the analysis.  As time permits we will attempt
implementation.

In order to support our complexity and effectiveness claims, we plan
on using the FLEX compiler infrastructure to generate dynamic
statistics for real-world Java code.  The remainder of the ``general
infrastructure'' time
budget will probably be spent improving the compiler back-end to allow
our benchmarks to generate useful data.  Twenty hours for collecting time
complexity data and another twenty for optimization effectiveness data
seem reasonable.  Another ten hours is allotted for visualization and
interpretation of the data.

Finally, we plan on spending about 80 hours clarifying algorithms,
generating figures, and writing proofs and text in the preparation of
the thesis report.

Only computer facilities will be needed for this work; the research
will be conducted in the Laboratory for Computer Science.

The favorable analysis properties of SSI form are, at this time,
central to the advanced analysis goals of the FLEX compiler project.
In addition, SSI form seems to enable new applications outside the
scope of typical compilers.  It is our goal to present a report
confirming the usefulness of the SSI form with hard data on
efficient construction algorithms and effective sparse optimizations
using the form.  In addition, the graph-theoretic structure and
properties of the form will be enumerated to spark further applications.

\bibliography{harpoon}
\end{document}
