% -*- latex -*- This is a LaTeX document.
% $Id: exec.tex,v 1.9 1999-05-09 02:05:35 cananian Exp $
%%%%%%%%%%%%%%%%%%%%%%%%%%%%%%%%%%%%%%%%%%%%%%%%%%
\documentclass[12pt,notitlepage,twoside]{article}
\usepackage{comdef}

\title{Executability in the Static Single Information Form}
\author{C. Scott Ananian}
\date{\today \\ $ $Revision: 1.9 $ $}

\begin{document}
\bibliographystyle{plain}

\maketitle

The Static Single Information (SSI) form, as originally presented,
requires control-flow graph information in order to be executable.
Executability is desirable because it enables the use of abstract
interpretation when designing algorithms \cite{pingali90:dfg}.  We
would like to have a demand-driven operational semantics for SSI form
that does not require control-flow information; thus freeing us to
more flexibly reorder execution.

In particular, we would like a representation that eliminates
unnecessary control dependencies such as exist in the program of
figure \ref{fig:ctrldep}.  A control-flow graph for this program, as
it is written, will explicitly specify that no assignments to
\texttt{B[]} will take place until all elements of \texttt{A[]} have
been assigned; that is, the second loop will be
\emph{control-dependent} on the first.  We would like to remove this
control dependence in order to provide greater parallelism---in this
case, to allow the assignments to \texttt{A[]} and \texttt{B[]} to
take place in parallel, if possible.

\begin{figure}[t]
\begin{samplecode}
for (int i=0; i<10; i++)\\
\>A[i] = x;\\
for (int j=0; j<10; j++)\\
\>B[j] = y;\\
\end{samplecode}
\caption{An example of unnecessary control dependence: the second loop
is \emph{control-dependent} on the first and so assignments to
\texttt{A[]} and \texttt{B[]} cannot take place in parallel.}
\label{fig:ctrldep}
\end{figure}

The modifications outlined here will extend SSI form in order to
provide a useful and descriptive operational semantics.  We will call
the extended form \ssiplus.  For clarity, SSI form as originally
presented we will call \ssizero.

\section{SSI extensions}
Although a demand-driven execution model can be constructed for
\ssizero,  it fails to handle loops and imperative
constructs well. \ssiplus\ form addresses these deficiencies.

\subsection{Imperative constructs, pointer variables, and side-effects}
The presentation of \ssizero\ ignored pointers,
concentrating on so-called register variables.  Extending \ssizero\ to
handle these imperative constructs is quite easy: we simply define a
``variable'' $S$ to represent an updatable store.  This variable is
renamed and numbered as before, so that $S_0$ represents the initial
contents of the store and $S_i, i>0$ represents the contents of the
store after some sequence of writes.  Figure \ref{fig:store} shows a
simple imperative program in \ssiplus\ form.  Note that modifications
to the store typically take the previous contents of the store as
input, and that subroutines with side-effects modifying the store must
be written in \ssiplus\ form such that they both take a store and
return a store.

\begin{figure}[t]
\begin{samplecode}[2]
\com{swap A[i] and B[j]} & \com{\ssiplus\ form:}\\
x = A[i];	& $x_0$ = FETCH($S_0$, $A_0 + i_0$) \\
y = B[j];	& $y_0$ = FETCH($S_0$, $B_0 + j_0$) \\
A[i] = y;	& $S_1$ = STORE($S_0$, $A_0 + i_0$, $y_0$); \\
B[j] = x;	& $S_2$ = STORE($S_1$, $B_0 + j_0$, $x_0$); \\
\end{samplecode}
\caption{Use of the ``store variable'' $S_x$ in \ssiplus\ form.}
\label{fig:store}
\end{figure}

It is worth noting that the store may be decomposed for added
precision.  In type-safe languages, it is trivial to define multiple
stores for differing type sets.  Even greater precision may be
obtained by using the \emph{may-point} relation from pointer
analysis to define a very fine-grained model of the store.
Figure \ref{fig:manystore} shows a simple example of this type of
decomposition using two different subtypes (\texttt{Integer} and
\texttt{Float}) of a common base class (\texttt{Number}).

\begin{figure}[t]
\centering$\vdash N:\mathbf{Number},I:\mathbf{Integer},F:\mathbf{Float}$\\
\centering$I \subset N \mbox{ and } F \subset N$\\
\begin{samplecode}[2]
if(P)           & \com{\ssiplus\ form:}\\
\>N=I;\\
else            & $N_0=\phi(I_0,F_0)$ \\
\>N=F;\\
F.add(3.14159); & $S^F_1=\mbox{\tt CALL}(\mbox{\tt add},S^F_0,F_0,3.14159)$\\
N.add(5);       & $\tuple{S^I_1,S^F_2}=
                         \mbox{\tt CALL}(\mbox{\tt add},S^I_0,S^F_1,N_0,5)$\\
\end{samplecode}
\caption{Factoring the store ($S_x$) using type information
         in a type-safe language.}
\label{fig:manystore}
\end{figure}

Statements with side-effects may be handled in a similar way: a
special SSI name is used/defined where side-effects occur to impose an
implicit ordering.  For maximum symmetry with the `store' case, we
will name this special variable $S^{fx}$.

\begin{figure}[t]
\begin{samplecode}[2]
int x=1;	& $x_0 = 1$ \\
int y=2;	& $y_0 = 2$ \\
int *p = \&x; 	& $p_0 = \set{\mbox{x}}$ \com{P is of type ``location set''} \\
if (P)  	& \com{Some \sigfunction{} goes here?} \\%$\tuple{S^{fx}_1,S^{fx}_2}=\sigma(P, S^{fx}_0)$ \\
\>p = \&y;	& $p_1 = \set{\mbox{y}}$ \\
		& $p_2 = \phi(p_0, p_1)$ \\
*p = 3; 	& $\tuple{x_1,y_1} = \mbox{\tt DEREF}(p_2, 3)$\\
return x;	& $\mbox{\tt return } x_1$ \\
\end{samplecode}
\caption{Pointer manipulation of local variables in C.}
\label{fig:messyC}
\end{figure}

Pointer handling is substantially easier in Java-like languages.  The
example on the left in figure \ref{fig:messyC} shows the difficulties
one may encounter in dealing with pointer variables that may rewrite
SSI temporaries.  It is possible to deal with this in the manner of
figure \ref{fig:manystore} using explicit stores\footnote{Pointer+SSA
references here.}, or with sufficient analysis one may write the SSI
representation on the right in the figure.  When compiling Java (or
some other object-oriented languages\footnote{which?}) this difficulty
is avoided: there are no pointers to base types, so the IR does not
have to worry about values changing ``behind its back'' like in the
example.\textbf{PROBABLY BETTER NOT TO ATTEMPT THE RIGHT-HAND SIDE.}

\textbf{SIGMA FUNCTIONS HAVE A HOLE: SEE FIGURE \ref{fig:problem}.}
\begin{figure}[t]
\begin{samplecode}[2]
int x=0;	& $x_0 = 0$ \\
if (P)		& \com{what \sigfunction{} goes here?} \\
\>x=1;		& $x_1 = 1$ \\
else		& \\
\>x=2;		& $x_2 = 2$ \\
		& $x_3 = \phi(x_1, x_2)$ \\
return x;	& $\mbox{\tt return } x_3$ \\
\end{samplecode}
\caption{A problem with SSI form. Need to hoist the constants.}
\label{fig:problem}
\end{figure}

\subsection{Loop constructs}
\begin{figure}[t]
\begin{samplecode}[3]
\com{a simple loop} & \com{\ssizero\ form:} & \com{\ssiplus\ form:}\\
j=1;	& \>$j_0 = 1$			& $j_0=1$\\
i=0;	& \>$i_0 = 0$			& $i_0=0$\\
do	& L1:			& $\xivec{j_1}{i_5}=\xi(\xivec{j_0}{i_3})$\\
$\{$	& \>$i_1 = \phi(i_0, i_3)$	& $i_1 = \phi(i_0, i_5)$ \\
\>i+=j; & \>$i_2 = i_1 + j_0$		& $i_2 = i_1 + j_1$ \\
$\}$ while (i<5);	& \>$P_0=(i_2<5)$	& $P_0=(i_2<5)$\\
	& \>if $P_0$ goto L1		& $\tuple{i_3,i_4}=\sigma(P_0, i_2)$ \\
	& \>~~$\tuple{i_3,i_4}=\sigma(i_2)$ &\\
\end{samplecode}
\caption{A simple loop, in \ssizero\ and \ssiplus\ forms.}
\label{fig:loop}
\end{figure}

The center column of figure \ref{fig:loop} shows a typical loop in
\ssizero\ form.  Note first that an explicit ``control flow''
expression (\texttt{goto L1}) is required in order to make sense of
the program.  Note also that $i_1$, $i_2$ and $i_3$ are potentially
\emph{dynamically} assigned many times, although \emph{statically}
they have only one definition each.  This complicates any sort of
demand-driven semantics: should the \phifunction{} demand the value of
$i_0$, or $i_3$, when it is evaluated the first time?  Which of the
values of $i_3$ does it receive when the \phifunction{} is
subsequently evaluated?  A token-based dataflow interpretation fails
as well: it is easy to see that tokens for $i_x$ flow around the loop
before flowing out at the end, but the token for $j_0$ seems to be
``used up'' in the first iteration.

\ssiplus\ introduces a \xifunction\ in the block of \phifunction{s} to
clarify the loop semantics. The left-hand column of figure
\ref{fig:loop} illustrates the nature of this function.  The
\xifunction{} arbitrates loop iteration, and will be defined precisely
in the operational semantics of \ssiplus\ form.  For now note that it
relates iteration variables (the top tuple of the parameter and result
vectors) to loop invariants (the bottom tuple of the vectors).  We
followed the statement ordering of \ssizero\ in the figure, but unlike
\ssizero, the statements of \ssiplus\ could appear in any order
without affecting their meaning---and so the statement label
\texttt{L1} of the \ssizero\ representation is unnecessary in \ssiplus.

There is at most one \xifunction{} per \phifunction{} block, and it
always precedes the \phifunction{s}.  Construction of \xifunction{s}
takes place before the renaming step associated with SSI form, and the
\xifunction{s} are then renamed in the same manner as any other
definition.  The top tuple of the constructed \xifunction{} contains
the names of all variables reaching the guarded \phifunction{} via a
backedge, and the bottom tuple contains all variables used inside the
guarded loop that are {\em not} mentioned in the header's
\phifunction{}.\footnote{I can give a rigorous algorithm: the top
tuple is found with a depth-first search of the source
control-flow graph, and the bottom tuple can be determined by a
traversal of the same SESE-region tree which generates \ssizero\
form.}

\subsection{Size complexity of \ssiplus}
\textbf{WRITE ME.}

\section{Operational Semantics}
We will base the operational semantics of \ssiplus\ on a demand-driven
dataflow model.  We will define both a cycle-oriented semantics and an
event-driven semantics, which (incidentally) correspond to synchronous
and asynchronous hardware models.

Following the lead of Pingali \cite{pingali90:dfg}, we present Plotkin-style
semantics \cite{plotkin81:opsem} in which \emph{configurations} are
rewritten instead of programs.  The configurations represent program
state and transitions correspond to steps in program execution.

The semantics operate over a lifted value domain
$V=\mathit{Int}_\bot$. When some variable $t = \bot_V$ we say it is
\emph{undefined}; conversely $t\succ\bot_V$ indicates that the
variable is \emph{defined}.  ``Store'' metavariables $S_x$ are not
explicitly handled by the semantics, but the extension is trivial with
an appropriate redefinition of the value domain $V$.  Floating-point
and other types are also trivial extensions.  The
metavariables $c$ and $v$ stand for elements of $V$.

We also define a domain of \emph{variable names},
$\mathit{Nam}=\set{n_0,n_1,\ldots}$.  The metavariables $t$ and $P$ stand for
elements in $\mathit{Nam}$, although $P$ will be reserved for naming branch predicates.

\subsection{Cycle-oriented semantics}
\begin{figure}[t]
\begin{transitions}
t=c,\,c\:\mbox{a constant}:
& \trule{\rho[t]=\bot \wedge t\notin D}
        {\tuple{\rho, D} \to
         \tuple{\rho[t \mapsto c],D}} \\

t=\mathbf{op}(t_1,\ldots,t_n):
& \trule{\rho[t]=\bot \wedge 
         \left(
          \rho[t_1]\succ\bot \wedge \ldots \wedge \rho[t_n]\succ\bot
	 \right)}
	{\tuple{\rho,D} \to
         \tuple{\rho[t \mapsto
                     \mathbf{op}(\rho[t_1],\ldots,\rho[t_n])],
                D \cup \set{t_1,\ldots,t_n} } } \\

t=\phi(t_1,\ldots,t_n):
& \trule{\rho[t]=\bot \wedge
         \rho[t_j]\succ\bot \wedge
         \mbox{all other}\,\rho[t_1],\ldots,\rho[t_n]=\bot}
        {\tuple{\rho,D} \to
         \tuple{\rho[t \mapsto \rho[t_j]],
                D \cup \set{t_j} } } \\

\tuple{t_1,\ldots,t_n}=\sigma(P,t):
& \trule{\rho[P]=v\succ\bot \wedge
         \rho[t_{v-1}]=\bot \wedge
         \rho[t]\succ\bot
         \mbox{ where } (0\leq v\leq n-1)}
	{\tuple{\rho,D} \to
         \tuple{\rho[t_{v-1} \mapsto \rho[t]],
                D \cup \set{t} } } \\

\footnotesize % latex complains, but does the right thing.
\xivec{t_1,\ldots,t_n}{t_{n+1},\ldots,t_{m}}=\xi(\xivec{t'_1,\ldots,t'_n}{t'_{n+1},\ldots,t'_m}):
& \trule{\rho[t_j]=\bot \wedge
         \rho[t'_j]\succ\bot \mbox{ where } (1\le j\le n)}
        {\tuple{\rho,D} \to
         \tuple{\rho[t_j \mapsto \rho[t'_j]], D \cup \set{t'_j} } } \\

\footnotesize % latex complains, but does the right thing.
\xivec{t_1,\ldots,t_n}{t_{n+1},\ldots,t_{m}}=\xi(\xivec{t'_1,\ldots,t'_n}{t'_{n+1},\ldots,t'_m}):
& \footnotesize % latex complains, but does the right thing.
  \trule{\rho[t'_{n+1}]\succ\bot \wedge \ldots \wedge \rho[t'_m]\succ\bot}
        {\myarray{r}{
         \tuple{\rho,D} \to
	 \hfill \\ \: % line-break
         \tuple{\rho_\emptyset
                [t_1 \mapsto \rho[t_1]]\ldots[t_n \mapsto \rho[t_n]]
%               \right.\right.\\ \left.\left. % line-break
                [t_{n+1} \mapsto \rho[t'_{n+1}]]\ldots[t_m \mapsto \rho[t'_m]],
                \right.\\ \left. % line-break
	        D \cup \set{t_1,\ldots,t_n,t'_{n+1},\ldots,t'_m} } } }\\
\end{transitions}
\caption{Cycle-oriented transition rules for \ssiplus.}
\label{fig:cyclesemantics}
\end{figure}

In the cycle-oriented semantics, configurations consist of an
\emph{environment} and a \emph{done set}.  The environment $\rho$ maps
names in $\mathit{Nam}$ to values in $V$; the done set $D$ is used for constant
value bookkeeping.

\begin{definition}~\\*[-1\baselineskip]
\begin{enumerate}
\item An \emph{environment} $\rho: N \to V$ is a finite function ---
its domain $N \subset \mathit{Nam}$ is finite.  The \emph{done set}
$D$ is a subset of $N$.
The notation $\rho[t\mapsto c]$ represents an environment
identical to $\rho$ except for name $t$ which is mapped to $c$.
\item The null environment $\rho_\emptyset$ maps every $t\in N$ to
$\bot_V$.
\item A \emph{configuration} is a pair $\tuple{\rho,D}$ consisting of
an environment and a done set.  The initial configuration is
$\tuple{\rho_\emptyset, \{\}_{\mathit{Nam}}}$ ---
that is, all names in $N$ are
mapped to $\bot_V$ and the done set $D$ is empty.
\end{enumerate}
\end{definition}

Figure \ref{fig:cyclesemantics} shows the cycle-oriented transition
rules for \ssiplus\ form.  The left column consists of definitions and
the right column shows a precondition on top of the line, and a
transition below the line.  If the definition in the left column is
present in the \ssiplus\ form and the precondition on top of the line
is satisfied, then the transition shown below the line can be performed.

\textbf{EXPLAIN THE RATIONALE BEHIND THE RULES HERE.}

\subsection{Event-driven semantics}
\begin{figure}[t]\small
\begin{transitions}
t=c,\,c\:\mbox{a constant}:
& \trule{t\notin D}
  {\tuple{E,D,S} \to \tuple{E[t=c],D\cup\{t\},S} }\\

t=\mathbf{op}(t_1,\ldots,t_n):
& \tuple{E[t_1=v_1]\ldots[t_n=v_n],D,S} \to \tuple{E[t=\mathbf{op}(v_1,\ldots,v_n)],D,S}\\

t=\phi(t_1,\ldots,t_n):
& \tuple{E[t_i=v],D,S} \to \tuple{E[t=v],D,S}\\

\tuple{t_1,\ldots,t_n}=\sigma(P,t):
& \tuple{E[t=v][P=i],D,S} \to \tuple{E[t_i=v],D,S}\\

\xivec{t_1,\ldots,t_n}{t_{n+1},\ldots,t_m}=\xi_K(\xivec{t'_1,\ldots,t'_n}{t'_{n+1},\ldots,t'_m}):
& \myarray{r}{
  \tuple{E[t'_i=v],D,S} \to \hfill\\\quad\quad\quad\quad\quad
  \tuple{E[t_i=v],D,S[K\mapsto S[K]\cup\tuple{t_i,v}]}\\
   \mbox{where }1\le i \le n\\
%  \mbox{where }K\mbox{ is a unique constant corresponding to}\\
%  \mbox{this \ssiplus\ statement}\\
  }\\

\xivec{t_1,\ldots,t_n}{t_{n+1},\ldots,t_m}=\xi_K(\xivec{t'_1,\ldots,t'_n}{t'_{n+1},\ldots,t'_m}):
& \trule{S[K]=\left\{\tuple{t_1,v_1},\ldots,\tuple{t_n,v_n}\right\}}
  {\myarray{r}{
   \tuple{E[t'_{n+1}=v_{n+1}]\ldots[t'_m=v_m],D,S} \to\quad\quad\quad\quad\\
   \tuple{E[t_1=v_1]\ldots[t_m=v_m],D,S}\\
%   \mbox{where }S[K]=\bigcup_{i=1}^n \{\tuple{t_i,v_i}\}\\
%   \mbox{where }S[K]=\left\{\tuple{t_1,v_1},\ldots,\tuple{t_n,v_n}\right\}\\
  } }
\end{transitions}
\caption{Event-driven transition rules for \ssiplus.  Note the
unfortunate synchronization in the last rule. $K$ is a
statement-identifier constant which is unique for each source \xifunction.}
\label{fig:eventsemantics}
\end{figure}

In the event-driven semantics, configurations consist of an
\emph{event set}, a \emph{done set}, and an \emph{invariant store}.
The event set $E$ contains definitions of the form $t=c$, and the done
set $D$ is defined as for the cycle-oriented semantics.  The invariant
store is a mapping from numbered \xifunction{s} in the source
\ssiplus\ form to a set of tuples representing saved values for loop
invariants.

We define the following domains:
\begin{itemize}
\item $\mathit{Evt} = \mathit{Nam} \times V$ is the event domain.  An
event consists of a name-value pair.  The metavariable $e$ stands for
elements of $\mathit{Evt}$.
\item $\mathit{Xif} \subset \mathit{Int}$ is used to number
\xifunction{s} in the source \ssiplus\ form.  There is some mapping
function which relates \xifunction{s} to unique elements of
$\mathit{Xif}$.  The metavariable $K$ stands for an element in
$\mathit{Xif}$.
\end{itemize}

A formal definition of our configuration domain is now possible:
\begin{definition}~\\*[-1\baselineskip]
\begin{enumerate}
\item An \emph{event set} $E$ is a finite subset of $\mathit{Evt}$.
The notation $E[t=c]$ represents an event set identical to $E$ except
that it contains the event $\tuple{t,c}$.  We say a name $t$ is
\emph{defined} if $\tuple{t,v} \in E$ for some $v$.  For all
$\tuple{t_1,v_1},\tuple{t_2,v_2} \in E$, $t_1$ and $t_2$ differ.  This is
equivalent to saying that no name $t$ is multiply defined in an event
set.  This constraint is enforced by the transition rules, 
not by the definition of $E$.
\item An \emph{invariant store} $S: K \to E$ is a finite mapping from
\xifunction{s} to event sets.
\item A \emph{configuration} is a tuple $\tuple{E, D, S}$ consisting
of an event set, a done set, and an invariant store.  The initial
configuration is
$\tuple{\{\}_{\mathit{Evt}},
        \{\}_{\mathit{Nam}},
        \{\}_{\mathit{Xif} \times E}}$ ---
that is, it consists of an empty event set, an empty done set, and
an empty mapping for the invariant store.
\end{enumerate}
\end{definition}

Figure \ref{fig:eventsemantics} shows the event-driven transition
rules for \ssiplus\ form.  As before, the left column consists of
definitions and the right column shows an optional precondition above
a line, and a transition.  If the definition in the left column is
present in the \ssiplus\ form and the precondition (if any) above the
line is satisfied, then the transition can be performed.  Note that
most transitions remove some event from the event set $E$, replacing
it with a new event.  The done set $D$ ensures that only one event is
generated for constant assignments.  The invariant store $S$ stores
the values of loop invariants for regeneration at each loop iteration.

\textbf{MORE DESCRIPTION OF EVENT-DRIVEN SEMANTICS HERE.}

\section{Hardware correspondence}

The operational semantics defined above for \ssiplus\ happen to
correspond to two different circuit interpretations of a program with
cyclic dependencies (i.e., loops).  The cycle-oriented semantics can
be read as a synchronous circuit with `clock cycles' stretching
between transitions involving \xifunction{s}.  The event-driven
semantics describe a more asynchronous dataflow-like machine.

\textbf{More text here.}

\bibliography{harpoon}
\appendix
\end{document}
