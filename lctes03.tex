% -*- latex -*- This is a LaTeX document.
% $Id: lctes03.tex,v 1.6 2003-04-17 19:44:58 cananian Exp $
%%%%%%%%%%%%%%%%%%%%%%%%%%%%%%%%%%%%%%%%
% lctes03.tex
% Data Size Optimizations for Java Programs
% Author: C. Scott Ananian / Martin Rinard
% Apr 2003
\documentclass{sig-alt-full}
%\usepackage[section,plain]{algorithm}
%\usepackage{amsthm} % proof environment
%\usepackage{amstext} % the \text command for math mode (replaces \mbox)
\usepackage{varioref} % \vref command
\usepackage{graphicx} % for eps figures
\usepackage{color}
\usepackage{comdef}
\usepackage{support}
\newcommand{\figscale}{1.0}

%setup varioref package
\renewcommand{\reftextbefore}{on the preceding page}\vrefwarning

\newcommand{\mycomment}[1]{}
\newcommand{\meet}{\ensuremath{\sqcap}}

\renewcommand{\floatpagefraction}{0.8}
\renewcommand{\topfraction}{0.25}

% MISSING IDEAS:
%  pointer compression? NO
%  more discussion of multithreading semantics? NO
%  atomic update of 3-byte objects NO
%  hashtable picture? NO
%  more discussion of exectution time? YES
%  more back-patting on bitwidth analysis?  signed integers. NO
%  which fields got split/specialized? YES
%  make the case for constant field elimination *in the compiler* YES
%  talk about static elimination of hash/locks? NO
%  kava paper in related work? YES

%\renewcommand{\baselinestretch}{0.97}
\begin{document}
\bibliographystyle{plain}
\conferenceinfo{LCTES'03,} {June 11--13, 2003, San Diego, California, USA.}
\CopyrightYear{2003}
\crdata{1-58113-647-1/03/0006}  
\title{Data Size Optimizations for Java Programs}
\numberofauthors{2}
\author{
\alignauthor C.~Scott~Ananian\\
\affaddr{Laboratory for Computer Science}\\
\affaddr{Massachusetts Institute of Technology}\\
\affaddr{Cambridge, MA 02139}\\
\email{cananian@lcs.mit.edu}
\alignauthor Martin~Rinard\\
\affaddr{Laboratory for Computer Science}\\
\affaddr{Massachusetts Institute of Technology}\\
\affaddr{Cambridge, MA 02139}\\
\email{rinard@lcs.mit.edu}
}
\maketitle
\support{This research was supported by DARPA/AFRL Contract F33615-00-C-1692.}
%
% abstract
\begin{abstract}
We present a set of techniques for reducing the memory consumption of
object-oriented programs which are particularly applicable to
memory-limited embedded systems. These techniques 
include optimizations that eliminate fields with
constant values, reduce the sizes of fields based on the range
of values that can appear in each field, and 
eliminate fields with common default values
or usage patterns.
We apply these optimizations both 
to fields declared by the programmer and to implicit fields in
the runtime object header. We describe analysis
algorithms to extract the information required to apply these
optimizations. 

We have implemented these techniques in the MIT FLEX compiler
system and applied them to the programs in the SPECjvm98 
benchmark suite. Our experimental results show that 
our combined techniques can reduce the maximum live heap size required
for the programs in our benchmark suite by as much as 40\%.
Some of the optimizations reduce the overall execution time;
others may impose modest performance penalties.
\end{abstract}
%\category{}%XXX
%\keywords{}%XXX
%
\section{Introduction}
%
This paper presents a set of techniques for reducing the
amount of data space required to represent objects
in object-oriented programs. Our techniques optimize
the representation of both the programmer-defined fields
within each object and the header information used by the
run-time system:
\begin{itemize}
\item {\bf Field Reduction:} 
Our flow-sensitive, interprocedural bitwidth analysis
computes the range of values that the program
may assign to each field. The compiler then transforms the program
to reduce the size of the field to the smallest type
capable of storing that range of values. 
\item {\bf Unread and Constant Field Elimination:} 
If the bitwidth analysis finds that a field always holds
the same constant value, the compiler eliminates the field. 
It removes each write to the field, and replaces each read
with the constant value.  Fields without executable reads are
also removed.
\item {\bf Static Specialization:} Our analysis finds 
classes with fields whose values do not change after initialization,
even though different instances of the object may
have different values for these fields. It then generates 
specialized versions of each class which omit these fields,
substituting accessor methods which return constant values.
\item {\bf Field Externalization:} Our analysis uses profiling
to find fields that almost always have the same default value. 
It then removes these fields from their enclosing class, 
using a hash table to store only values of the field that differ
from the default value. It replaces writes to the field with
an insertion into the hash table (if the written value is not the
default value) or a removal from the hash table (if the written value
is the default value). It replaces reads with hash table lookups; 
if the object is not present in the hash table, the lookup simply
returns the default value. 
\item {\bf Class Pointer Compression:} We use rapid type analysis
to compute an upper bound on the number of classes that the
program may instantiate. Objects in standard 
Java implementations have a header field, commonly called {\tt claz},
which contains a pointer
to the class data for that object,
such as inheritance information and method dispatch tables.
Our compiler uses the results of the 
analysis to replace the reference with a smaller
offset into a table of pointers to the class data. 
\item {\bf Byte Packing:} All of the above transformations may
reduce or eliminate the amount of space required to store each
field in the object or object header. Our byte packing algorithm
arranges the fields in the object to minimize the object size.
\end{itemize}
All of these transformations reduce the space required to store
objects, but some potentially increase the running time of the program.
Our experimental results show that, for our set of benchmark
programs, all of our techniques combined can reduce the peak amount of memory
required to run the program by as much as 40\%, although the running
time may increase.  In a memory-limited embedded system where
performance is not critical, cost savings may directly result from the
reduced minimum heap size.

\subsection{Contributions}

This paper makes the following contributions:
\begin{itemize}
\item {\bf Space Reduction Transformations:} It presents a set
of novel transformations for reducing the memory required to 
represent objects in object-oriented programs.

\item {\bf Analysis Algorithms:} It presents a set of 
analysis algorithms that automatically extract the 
information required to apply the space reduction 
transformations.

\item {\bf Implementation:} We have fully 
implemented all of the analyses and techniques 
presented in the paper. Our experience with this
implementation enables us to discuss the pragmatic
details necessary for an effective implementation 
of our techniques. 

\item {\bf Experimental Results:} This paper presents a set
of experimental results that characterize the impact
of our transformations, revealing the extent of the
savings available and the performance cost of attaining them.
\end{itemize}
%
\section{Example}
%
We next present a pair of examples that illustrate the kinds of 
analyses and transformations that our compiler performs.
%
\subsection{Field Reduction and Constant Field Elimination}
%
Figure~\ref{fig:value} presents the {\tt JValue} class, which is 
a wrapper around either an {\tt Integer} object or a {\tt Float}
object. The {\tt type} field indicates which kind of object
is stored in the {\tt value} field of the class, 
essentially implementing a tagged 
union.\footnote{This class is a simplfied version of similar
classes that appear in some of our benchmarks.
See for example the {\tt jess.Value} class in SPECjvm98 benchmark
{\tt jess}.} 
The class also maintains the {\tt positive} field, which is
{\tt 1} if the wrapped number is positive and {\tt 0} otherwise. 

Our bitwidth analysis uses an interprocedural
value-flow algorithm to compute upper and lower bounds for the
values that can appear in each variable. This analysis tracks
the flow of values across procedure boundaries via parameters,
into and out of the heap via instance variables of classes, and through
intermediate temporaries and local variables in the program.
It also reasons about the semantics of arithmetic operators such
as {\tt +} and {\tt *} to obtain bounds for the values computed
by arithmetic expressions. 
This analysis discovers the following facts about 
how the program uses this class: a) the {\tt integerType} 
field always has the value {\tt 0}, b) the {\tt floatType} 
field always has the value {\tt 1}, c) the {\tt type} 
field always has a value between {\tt 0} and {\tt 1} (inclusive),
and d) the {\tt positive} field always has a value between 
{\tt 0} and {\tt 1} (also inclusive).

\begin{figure}[tp]
\renewcommand{\baselinestretch}{0.6}%
{\small\tt\renewcommand{\>}{~~~}\begin{tabular}[t]{l}%
public class JValue \{ \\
\>int integerType = 0;\\
\>int floatType = 1;\\
\>int type, positive;\\
\>Object value;\\
\>void setInteger(Integer i) \{ \\
\>\>type = integerType; value = i;\\
\>\>positive = (i.intValue() > 0) ? 1 : 0;\\
\>\}\\
\>void setFloat(Float f) \{ \\
\>\>type = floatType; value = f;\\
\>\>positive = (f.floatValue() > 0) ? 1 : 0;\\
\>\}\\
\}
\end{tabular}}%
\caption{The {\tt JValue} class.}
\label{fig:value}
\end{figure}
\begin{figure}[tp]
\renewcommand{\baselinestretch}{0.6}%
{\small\tt\renewcommand{\>}{~~~}\begin{tabular}[t]{l}%
public final class String \{\\
\>private final char value[];\\
\>private final int offset;\\
\>private final int count;\\
\>\ldots\\
\>public char charAt(int i) \{\\
\>\>return value[offset+i];\\
\>\}\\
\>public String substring(int start)\\
\>\{\\
\>\>int noff = offset + start;\\
\>\>int ncnt = count - start;\\
\>\>return new String(noff, ncnt, value);\\
\>\}\\
\}\\
\end{tabular}}%
\caption{Portions of the {\tt java.lang.String} class.}
\label{fig:string}
\end{figure}

Our compiler uses this information to remove all occurrences
of the {\tt integerType} and {\tt floatType} fields from the
program. It replaces each read of the {\tt integerType} field
with the constant {\tt 0}, and each read of the {\tt floatType}
field with the constant {\tt 1}. It also uses the bounds on the 
values of the {\tt type} and {\tt positive} variables to reduce the size of the 
corresponding fields. Our currently implemented compiler rounds
field sizes to the nearest byte required to hold the range
of values that can occur. Our byte packing algorithm then 
generates a dense packing of the values, attempting to preserve
the alignment of the variables if possible. In this case, the
algorithm can reduce the field sizes by six bytes and the overall
size of the object by one four-byte word.  If the runtime can support
unaligned objects without external fragmentation, we can reduce the
object size by  the full six bytes.

\subsection{Static Specialization} 

Figure~\ref{fig:string} presents portions of the implementation
of the {\tt java.lang.String} class from the Java standard class
library. The {\tt value} field in this
class refers to a character array that holds the characters
in the string; the {\tt count} field holds the length of the
string. In some cases, instances of the {\tt String} class
are derived substrings of other instances 
(see the {\tt substring} method in Figure~\ref{fig:string}), in
which case the
{\tt offset} field provides the offset of the starting 
point of the string within a shared {\tt value} character array. 
Note that the {\tt value}, {\tt offset}, and {\tt count} 
fields are all initialized when the string is constructed
and do not change during the lifetime of the string.

\begin{figure}[tp]
\renewcommand{\baselinestretch}{0.6}%
{\small\tt\renewcommand{\>}{~~~}\begin{tabular}[t]{l}%
public final class SmallString \{\\
\>private final char value[];\\
\>private final int count;\\
\>int getOffset() \{ return 0; \}\\
\>\ldots\\
\>public char charAt(int i) \{\\
\>\>return value[getOffset()+i];\\
\>\}\\
\}\\
public final class BigString extends SmallString \{\\
\>private final int offset;\\
\>int getOffset() \{ return offset; \}\\
\}\\
\end{tabular}}%
\caption{Static specialization of {\tt java.lang.String}.}
\label{fig:split}
\end{figure}
\begin{figure}[tp]
\renewcommand{\baselinestretch}{0.6}%
{\small\tt\renewcommand{\>}{~~~}\begin{tabular}[t]{l}%
public SmallString substring(int start)\\
\{\\
\>int noff = offset + start;\\
\>int ncnt = count - start;\\
\>if (noff==0)\\
\>\>return new SmallString(value, noff, ncnt);\\
\>else\\
\>\>return new BigString(value, noff, ncnt);\\
\}\\
\end{tabular}}%
\caption{Dynamic selection among specialized classes in a method
             from {\tt java.lang.String}.
}
\label{fig:dynsel}
\end{figure}

In practice, most strings are not created as explicit substrings of other
strings, so
the {\tt offset} field in most strings is zero.
In fact, all of the public {\tt String} constructors create
strings with {\tt offset} zero; only the {\tt substring} method
creates strings with a non-zero offset. And even at 
calls to the private {\tt String(int, int, char[])} constructor
inside the {\tt substring} method, it is possible to dynamically
test the values of the parameters at the allocation site to determine
if the newly
constructed string will have a zero or non-zero offset.

Our analysis exploits this fact by splitting the 
{\tt String} class into two classes: a superclass {\tt SmallString}
that omits the {\tt offset} field, and a subclass {\tt BigString} that
extends {\tt SmallString} and includes the {\tt offset} field. 
Each of these two new classes implements a {\tt getOffset()} method
to replace the field: the {\tt getOffset()} method
in the {\tt SmallString} class simply returns zero; but the {\tt
getOffset()} method in the {\tt BigString} class returns the 
value of the {\tt offset} field in {\tt BigString}.
Figure~\ref{fig:split} illustrates this transformation.

At every allocation site except the one inside the {\tt substring}
method, the transformed program allocates a {\tt SmallString} 
object. Inside the {\tt substring} method, the program generates
code that dynamically tests if the offset in the substring
will be zero. If so, it allocates a {\tt SmallString} object;
if not, it allocates a {\tt BigString} object.
(See Figure~\ref{fig:dynsel}.)
This transformation
therefore eliminates the {\tt offset} field in the majority
of strings. 

The analysis required to support this transformation takes place
in two phases. The first phase scans the program
to identify fields that
are amenable to transformation.\footnote{See
  Section~\ref{sec:subclass-final} for a more precise definition.}
In our example, the analysis
determines that the {\tt offset} field is never written after
it is initialized. The next phase determines if the value
of this field is determined either by the constructor that
initialized it or if it is a simple function of the parameters
of the constructor. In our example, the analysis determines
that the {\tt offset} field is zero for all constructors
except the private constructor invoked within the {\tt substring}
method. It also determines that, for objects created within substring,
the value of the {\tt offset} field is simply
the value of the {\tt noff} parameter to this constructor. 

This analysis identifies a set of candidate fields. 
The analysis chooses one of the candidate fields, then 
splits the class along the possible values
that can appear in the field. Our current implementation uses
profiling to select the field that will provide the largest
space savings; our policy takes both the size of the field
and the percentage of objects that have the same value for 
that field. In our example, the analysis identifies the 
{\tt offset} field as the best candidate and splits the class
on that field. We can apply this idea recursively to the 
new program to obtain the benefits of splitting on multiple
fields. 

\subsection{Field Externalization}

In the string example discussed above, it was possible to determine
which version of the specialized class to use at object allocation
time. In some cases, however, a given field may almost always have
a given value, even though it is not possible to statically determine
when the value might be changed or which objects will contain fields of that 
value. In such cases we apply another optimization, 
{\em field externalization}. This optimization removes the field
from the class, replacing fields whose values differ from the default 
value with hash table entries that map objects to values. If an object/value
mapping is present in the hash table, that entry provides the 
value of the removed field. If there is no mapping for a given object,
the field is assumed to have the default value. 
In our current implementation, we use profiling 
to identify the default value. 

In this scheme, writes to the field are converted into a check to see
if the new value of the field is the default value. If so, the 
generated code simply removes any old mappings for that 
object from the hash table.
If not, the generated code replaces any old mapping with a new
mapping recording the new value. 

\subsection{Hash/Lock Externalization}

Our currently implemented system applies field externalization
in a general way to any field in the object. We would, however,
like to highlight an especially useful extension of the basic
technique. Java implementations typically store an object
hash code and lock information in the object header. For many
objects, however, the program never actually uses the hash code
or lock information. Our implemented system therefore uses
a variant of field externalization called {\em hash/lock 
externalization}. This variant allocates all objects 
without the hash code and lock information fields in the header,
then lazily creates the fields when necessary. 
Specifically, if the program ever uses the hash code or lock information, 
the generated code creates the hash code or lock information
for the object, then stores this information in a table
mapping objects to their hash code or lock information.%
\footnote{The object's address is used as its key when
  field externalization is done.  The garbage-collector is responsible
  for updating the field entries if it moves objects, by rehashing on
  the new address.}

Note that, in general, this transformation (as well as field
externalization) may actually increase space usage. But in practice,
we have found that our set of benchmark programs
rarely uses these fields. The overall result is a substantial
space savings. The combination of class pointer compression 
and hash/lock elimination can produce a common-case object header
size of one byte --- one byte for a class index and no
space at all for hash code or lock.

\section{Analysis Algorithms}
%
In this section we will present details of the analyses that enable
our transformations.
%
\subsection{Rapid Type Analysis}
We start with a rapid type analysis \cite{bacon96} to collect the set of
instantiated classes and callable methods.  This analysis allows us to generate
a conservative call graph for the program, using the known receiver
type at the call-site and its set of instantiated subclasses in the
hierarchy.  Based on the class hierarchy, we can also tag all leaf
classes as {\tt final}, regardless of whether the source code contained
this modifier.  Methods which are not overridden, based on
the hierarchy, are also marked {\tt final}, and calls with a single
receiver method are devirtualized.  We also remove uncallable methods
and assign non-conflicting slots to interface methods using a
graph-coloring algorithm.  The results of some class casts and {\tt
  instanceof} operations can also be determined statically using
these results.

Our analysis keeps separate the set of {\it mentioned} and
{\it instantiated} classes.  Although type-checks can be made and methods
invoked on abstract, interface, or otherwise uninstantiated classes,
every object
in the heap must belong to one of the instantiated class types.
The size of the set of instantiated classes is quite small for a
typical Java program, and over half of the benchmarks in SPECjvm98
have less than 256 instantiated class types.%
\footnote{note that \emph{all} have more than 256 \emph{total} class types.}
We use this information
to replace the class pointer in the object header, which
identifies the type of the object, with a one-byte {\it index} into a
small lookup table.  The {\tt jess}, {\tt javac}, and {\tt jack}
benchmarks require more than one byte of index, but a two byte index
amply suffices in these two cases.
%
% Our implementation does claz numbering so that we can use the
% claz index for a fast instanceof test, too!  But we don't
% actually use it for that, yet, so I'm not going to describe it here.

\subsection{Bitwidth Analysis}
% this allows us to reduce fields and remove unused/const fields.
We use a flow-sensitive interprocedural bitwidth analysis to
find constant values, unread and constant fields, and to reduce
field sizes where possible.  Our dataflow framework uses
Wegman and Zadeck's Sparse Conditional Constant (SCC) propagation
algorithm \cite{wegman91:scc} as a basis.  We then extend their
analysis interprocedurally, add coverage of Java language features,
and extend the value lattice to handle bitwidths.
Since almost all types in Java are signed (with the exception of the
16-bit {\tt char}), we must be able to describe bitwidths of both
negative and positive numbers, which we do by splitting the set of
values into negative, zero, and positive parts, and describing the
bitwidth of each individually.

We abstract non-singleton sets of integer values into a tuple
\tuple{m,p} where $m\ge 1+\lfloor\log_2 N\rfloor$ for all negative $N$
in the set, and $p\ge 1+\lfloor\log_2 N\rfloor$ for positive $N$.  We
use $m=p=0$ to represent the constant zero.
Some combination rules for arithmetic operations are
shown in Figure~\ref{fig:bitrules}.  The rules for simple arithmetic
operators should be self-evident upon examination (adding two $N$ bit
integers yields at most an $N+1$-bit integer, for example) although
care must be taken to ensure that combinations of negative and
positive integers are handled correctly.  Our implementation contains
additional rules giving it greater precision for common special cases,
such as multiplication by
a one-bit quantity, division by a constant, and (as the figure shows)
bitwise operations on positive numbers.

\begin{figure}[tp]
\begin{eqnarray*}
-\tuple{m,p} &=& \tuple{p,m}\\
\tuple{m_l,p_l} + \tuple{m_r,p_r} &=& \tuple{1+\max(m_l,m_r),1+\max(p_l,p_r)}\\
%\tuple{m_l,p_l} \times \tuple{m_r,p_r} &=& \langle\max(m_l+p_r,p_l+m_r),\\
%                                       &&  \max(m_l+m_r,p_l+p_r)\rangle\\
\tuple{m_l,p_l} \times \tuple{m_r,p_r} &=&
\tuple{\begin{array}{l}\max(m_l+p_r,p_l+m_r),\\
                       \max(m_l+m_r,p_l+p_r)\end{array}}\\
\tuple{0,p_l} \wedge \tuple{0,p_r} &=& \tuple{0,\min(p_l,p_r)}\\
\tuple{m_l,p_l}\wedge \tuple{m_r,p_r} &=& \tuple{\max(m_l,m_r),\max(p_l,p_r)}
\end{eqnarray*}%
\caption{Some combination rules for bitwidth analysis of arithmetic
  and bitwise-logical operators.
  Note that the penultimate entry is a
  special-case rule that only applies if the neither of the
  arguments can be negative.
}\label{fig:bitrules}
\end{figure}

\subsubsection{Treatment of Fields}
Dataflow on this bitwidth lattice is performed on the entire Java
program interprocedurally.  The analysis is what Heintze and Tardieu
\cite{heintze01}
would call {\it field-based}; that is, given a field $f$ defined in
class $X$, and an instance of $X$ named $x$, we consider an assignment
to $x.f$ to be an assignment to the field $X.f$ and ignore the base
object $x$.\footnote{An obvious extension is to use pointer
analysis to discriminate between fields allocated at different sites
in the program.}  The result of the analysis is a bitwidth
specification for each variable and field in the program.  As the
analysis is based on SCC, we also identify constant variables and
fields; we replace reads of constant fields with their constant
value and the field is eliminated.  Fields for which no reads are
found (even if writes are present) are also eliminated.%
\footnote{Note that checks which may throw exceptions on reads and
  writes are preserved.}

\subsubsection{Other Details}
Our analysis handles method calls by merging the lattice values of the
method parameters at the call site with the formal parameters of the
method.  Similarly, the return value of the method is propagated back
to all call-sites.  Our compiler's intermediate representation handles
thrown exceptions by treating the method return value as a tuple, and
the call site as a conditional branch.  The ``normal return value'' is
assigned and the first branch taken on a normal method return, and the
``exceptional return value'' is assigned and the second branch taken when an
exception is thrown from the method.

Our implementation of this analysis is actually context-sensitive,
with a user-defined context length.  All results presented here were
obtained with the context set to zero; we saw no clear benefit from 1-
or 2-deep calling contexts, and the increase in analysis
time was considerable.

Space does not permit us to describe the remaining details of the full
analysis, including the extension of the value
lattice to handle the full range of Java types, the class hierarchy,
{\tt null} and {\tt String} constants, and fixed-length arrays.
We refer the interested reader to \cite{ananian99:tech} for an
exhaustive description of the intraprocedural analysis.

The number of unread and constant fields found by this analysis in our
benchmark set is shown in Table~\ref{tab:const-unused}.
The space reductions due to bitwidth analysis and field reduction
using our byte packing strategy are shown in Table~\ref{tab:bitwidth-results}.
\begin{table}[t]
\begin{tabular}{lcccr@{.}l}
&\bf total&&&\multicolumn{2}{c}{\bf\% alloc'ed}\\
\bf Benchmark&\bf fields &\bf unread &\bf constant &
\multicolumn{2}{c}{\bf space saved} \\\hline
%200\_check      & 279 &   79   &   35   &  2&6\% \\
201\_compress   & 298 &   75   &   31   &  2&5\% \\
202\_jess       & 485 &   91   &   43   &  9&9\% \\
205\_raytrace   & 341 &   75   &   30   &  0&0\% \\
209\_db         & 286 &   75   &   35   &  0&0\% \\
213\_javac      & 531 &   85   &   34   &  0&6\% \\
222\_mpegaudio  & 286 &   75   &   35   &  1&4\% \\
227\_mtrt       & 341 &   75   &   30   &  0&0\% \\
228\_jack       & 378 &   77   &   31   & 10&2\% \\
\end{tabular}
\caption{Number of unused and constant fields in SPEC benchmarks,
  and the savings realized (in \% of total dynamic allocated bytes) by
  removing them.}
\label{tab:const-unused}
\end{table}
\begin{table}[t]
\centering\begin{tabular}{lccr@{.}l}
&\multicolumn{2}{c}{\bf static field bits}&\multicolumn{2}{c}{\bf\% alloc'ed}\\
\bf Benchmark &
\bf before & \bf after &
\multicolumn{2}{c}{\bf space saved} \\ \hline
% data from 11-15 build of NO and 11-01 build of BW.
%200\_check      & 7343& 4958&  3&4\% \\
201\_compress   & 7591& 5430&  3&0\% \\
202\_jess       &13349&10634& 30&1\% \\
205\_raytrace   & 7467& 5296&  0&9\% \\
209\_db         & 6777& 4983&  0&3\% \\
213\_javac      &11560& 8161&  5&4\% \\
222\_mpegaudio  & 6777& 4983&  1&5\% \\
227\_mtrt       & 7467& 5296&  0&9\% \\
228\_jack       & 8356& 6037& 17&2\% \\
% THE JACK NUMBER IS VERY QUESTIONABLE!!!
\end{tabular}
\caption{Number of field bits in SPEC benchmarks statically removed
  due to bitwidth analysis,
  and the dynamic savings (in \% of total allocated bytes) of
  field bitwidth reduction using byte packing.}
\label{tab:bitwidth-results}
\end{table}

\subsection{Definite Initialization Analysis}
Java field semantics dictate that uninitialized fields must have
the value zero (or {\tt null}, for pointer fields).  It may seem,
then, that the starting lattice value for every integer field should
be $\mathbf{0}$.  This, however, prevents us from finding non-zero field
constants in the program: a simple initialization statement like {\tt
  x=5} will assign {\tt x} the value $\mathbf{0}\meet\mathbf{5}$,
which is not equal to $\mathbf{5}$!\footnote{On the SCC lattice of
\cite{wegman91:scc}, $\mathbf{0}\meet\mathbf{5}=\top$ (but see
footnote \ref{ft:topbot}).}

We perform a {\it definite initialization} analysis to remedy this
problem and restore precision to our analysis.
For example, with only constructor {\tt A$_1$} in the following code,
field {\tt f} will get the lattice value $\mathbf{5}$:
{\small\renewcommand{\baselinestretch}{0.5}\begin{samplecode}
public class A \{\\
\>int f;\\
\>A$_1$(\ldots) \{ f = 5; \}\\
\>A$_2$(\ldots) \{ /* no assignment to f */ \}\\
\}
\end{samplecode}%
}
Without constructor {\tt A$_2$} in the class,
we say that field {\tt f} is {\it definitely initialized} because
every constructor of {\tt A} assigns a value to {\tt f} before
returning or calling an unsafe method.
Adding constructor {\tt A$_2$} allows the
default $\mathbf{0}$ value of {\tt f} to be seen; {\tt f} is then no longer
definitely-initialized.

We actually allow the constructor great flexibility with regard to
definite initialization; it is free to call any method which does not
read {\tt A.f} before finally executing a definite initializer.
We construct a mapping from methods to all
fields which they may read, in a flow-insensitive manner, and compute
a transitive closure of this map over the call graph to determine 
a ``safe set'' of
methods which the constructor may call before a definite
initialization of {\tt f}.  As long as control flow may not pass to a
method not in the safe set before {\tt f} is written, then {\tt f} is
definitely initialized.

When performing bitwidth analysis,
definitely-initialized fields are allowed to start at $\bot$ in the
dataflow lattice.%
\footnote{We use $\bot$ for ``nothing known'' and $\top$ for
  ``under-constrained''; another segment of the compiler community
  commonly reverses these definitions.\label{ft:topbot}}
  All other fields must start at value
$\mathbf{0}$, which will make it impossible for the field to represent a
non-zero constant value.  The results of the definite initialization
analysis are also used when profiling mostly-constant fields, as described
in the next section.
%
\subsection{Profiling Mostly-Constant Fields}
To inform the static specialization and field externalization
transformations, we instrument a profiling build of the code
to determine which fields are {\it mostly-constant}.  Our implementation
builds one binary per examined constant, that is, one binary to look
for ``mostly-zero'' fields, a separate binary to look for fields which
are usually ``one'', a third binary to look for fields commonly
``two'', and so forth.  We built eleven binaries for each benchmark,
looking for
field default values in the interval $[-5,5]$.
For pointer fields, we only look for {\tt null} as a default value.
Although our use of multiple binaries is by no means necessary,
for ease of exposition we will discuss our profiling technique
as if there is a single default value $N$ which we are looking for.

Our instrumentation pass starts by
adding a counter per class
to record the number of times each exact class type is instantiated.
We also add per-field counters which are incremented the first
time a non-$N$ value is stored into a certain field.%
\footnote{Note that this requires storing an additional bit per field
  during profiling
  to record whether a non-$N$ value has been seen previously.}
By comparing the
number of times the class (thus field) is instantiated and the number
of times the field is set to a non-$N$ value, we can determine the
amount of memory recoverable by applying a ``mostly-$N$''
transformation to the field, whether static specialization or field
externalization.  We use this potential savings to guide our selection
of fields for static specialization, using the field and default value
which the profile indicates will yield the largest gain.  If static
specialization isn't an option, the
proportion of non-$N$ fields helps indicate whether externalization is
likely to result in a net savings; see Section~\ref{sec:extern-impl}
for further discussion.

There is one last detail to attend to:  when looking for non-zero $N$
values, the default zero value of
uninitialized fields becomes a problem.  For these cases, we use the
definite-initialization analysis described in the previous section to
increment the
``non-$N$'' counter on any path where the field in question is not
definitely initialized.

\subsection{Finding Subclass-Final Fields}
\label{sec:subclass-final}
Our static specialization transformation can only be applied to what
we call {\it subclass-final} fields.  Subclass-finality is a less strict
but similar constraint to Java's {\tt final} modifier.  We do a
single-pass analysis to determine subclass-finality, using the results
from the bitwidth analysis to improve our precision.%
\footnote{By using analysis rather than relying on programmer
  specification, the author need not restrict \emph{all} users of
  their code in order to obtain maximum efficiency for \emph{some}
  constrained uses of it.}

A \emph{subclass-final} field {\tt f} of a class {\tt A} can be
written to from any method of a \emph{subclass} of {\tt A}, as well as
in any constructor of {\tt A}.  In each write, the receiver's type
must be a subtype of {\tt A}, except inside {\tt A}'s constructors,
where the receiver may also be the method's {\tt this} parameter.
Other writes are disallowed.  Unlike Java-{\tt final} fields,
multiple writes to {\tt f} are permitted, as long as each write
satisfies the above constraints.

Subclass-finality matches the requirements of the static
specialization transformation.  Since we always insert a ``big''
version of the class between the specialized class and its children,
subclasses can write to the field present in objects of the ``big'' type
without restriction.
We need only restrict writes which occur in the class proper.

Our analysis constructs the set of subclass-final fields by finding
its dual, the set of {\it non}-subclass-final fields.  We scan every
method and collect all fields with illegal writes; all
fields found are added to the set of non-subclass-final fields.

\subsection{Constructor Classification}
% includes MustParamOracle and ConstMap
The final requirement to enable static specialization is to identify
constructors which always initialize certain fields in a given way.
In particular, we wish to find constructors which always give fields
statically-known constant values, as well as constructors which
initialize fields with simple functions of their input parameters.
The first case enables us to unconditionally replace an instantiated
class with a smaller split version; the second case allows us to wrap
the constructor in an appropriate conditional to enable the creation
of the small version when possible.

This analysis builds upon our previous results.
In a single pass over the constructor, we merge the values written
to a selected subclass-final field, treating $\text{\tt Param}N$ as an
abstract value for the $N$th constructor parameter.
We treat any call to a {\tt this()}
constructor as if it were inlined.  By the properties of
subclass-final fields, we know that all writes to the
field are to the {\tt this} object and that there are no bad writes to
the field outside of the constructor.
If the merged value at the end of the pass 
is a {\tt Param} value or a constant equal to the desired ``default''
value of the selected field,
then we can statically specialize on the field for calls to this
particular constructor.
Further, we rule out specialization on any otherwise-suitable
fields for which there is not at least one callable constructor
amenable to static specialization.
%
\section{Implementation Issues}
%
In this section we will talk briefly about some of the practical
issues arising in an implementation of our space-saving techniques.
%
\subsection{Byte Packing}
%
A typical Java implementation may waste large amounts of space by
aligning fields for the most efficient memory access.  Fields are
often aligned to their widths (a 4-byte field will be placed at an address
which is an even multiple of 4, for example), and the object as a
whole is often placed on a double-word boundary. Our implementation
places object fields at the nearest byte boundary, although
the information provided by our bitwidth
analysis is sufficient to {\it bit}-pack the fields in the object
when space is truly at a premium.  Preliminary investigation
indicated that the amount of additional space gained by bit-packing 
is typically only a few percent, because there aren't enough sub-byte
fields to fill the space ``wasted'' by byte alignment.%
\footnote{Note also that ``bit-packing'' may lead to the loss of
atomicity on concurrent writes to adjacent fields packed within a byte,
typically the processor's smallest atomic write size.  An escape
analysis would be sufficient to ensure that fields accessed from
differing threads are not packed within the same atomic unit.}

Some architectures penalize unaligned accesses to fields.  It is
worthwhile to {\it attempt} to align fields to their preferred
alignment while not allowing this to cause the object size to grow.
Further, there are often {\it forced} alignment constraints on
(for example) pointers.  Our Java runtime uses a conservative garbage
collector; its efficiency decreases markedly if pointers are not
word-aligned.\footnote{This means that objects have to be word-aligned
  as well.}

Our ``byte-packing'' heuristic achieves
tight packing of fields while respecting forced alignments.  Packing
proceeds recursively through superclasses, and returns a list of
free-space intervals available between the fields of the superclass.
The algorithm first places all {\it forced-alignment} fields in
the class, from largest to smallest.  The aim is for the
alignment-induced spaces left by the large fields to be fillable by
the following smaller fields.

When there are no more forced-alignment fields, we attempt to allocate
fields on their ``preferred'' alignment boundaries, largest first.
At this stage fields are not allowed to introduce an alignment gap at
the end of the object.  If their preferred alignment does not allow
them to be placed flush
against the last field of the object, they are skipped.

Finally, when there are no more fields satisfying
preferred-alignments, we
allocate the {\it smallest} available field at the lowest possible
byte boundary.
The aim is that the small fields will fill space and nudge
the end of the object out so that a larger field may be allocated on
its preferred alignment.  After each field is placed, we begin again
by attempting to place fields on preferred boundaries.

This heuristic strategy has been observed to work well in practice,
and the penalties for occasionally placing an unaligned non-pointer
field have not
been observed to have a material adverse effect on performance (see
Section~\ref{sec:byte-pack}). 

% header optimizations.
% java-vs-byte-vs-bit packing.
% byte-packing strategy.
% garbage collection; dynamic methods.  conservative gc. ole aggeson's blah.
% efficiency of field virtualization.
% single-inheritance when splitting.
% efficiency of external hashtable
% distribution of mostly-constants.
% array allocation?
% pointer size improvements.
% parameter widening; reflection; interface to native code. stoplist

% header optimizations.
% pointer compression?

\subsection{External Hashtable Implementation}
\label{sec:extern-impl}

Close attention to the implementation of the hashtable used for
field and hash/lock externalization is
necessary to realize the identified possible gains.  To
maximize space savings, we need to use as little space
per field stored in the table as possible.  The overhead of
dynamically allocated buckets and the required {\it next} pointers
makes separate chaining impractical as a hashtable implementation technique.
Open-addressing
implementations are preferable: in addition to the value being stored,
all that is necessary is a key value and the empty space required to
limit the load-factor.  A load factor of two-thirds and one-word keys
and values yield an average space consumption of three words per
field.  This implementation breaks even when the mostly-zero fields
identified are zero over 66\% of the time.  This break-even point is
compared to the profiling data to allow our field externalization
transformation to intelligently choose targeted fields.

Key-size reduction is an important component of the implementation:
a na{\"\i}ve approach
would combine a one-word reference to the virtual-container object and a
one-word field identifier for a two-word key.  The large key will
shift the break-even point up so that only fields which are 82\% zero
will profit.  Instead, we can offset the
object reference (up to the limit of its size) by small integers
to discriminate the externalized fields of the object, yielding
a single-word key.

Our implementation leverages the garbage-collector to remove unneeded
entries from the hashtable.
%
\subsection{Class Loading and Reflection}
This research was conducted using the MIT FLEX compiler infrastructure,%
\footnote{Available from {\tt http://flexc.lcs.mit.edu/}.} which is a
whole-program static compiler.  Although the analyses as described reflect
this, it would be straightforward to use \emph{extant analysis}
\cite{sreedhar00} to apply transformations to only the closed-world
portions of a program which used dynamic class loading.  The space
allocated to the class index could be updated during garbage collection
as new classes are discovered.  Concurrent profiling could actually
expose more opportunities for space compression in a JIT environment.
And our various transformations need not be user-visible if the
reflection implementation is carefully written.
%
\section{Experimental Results}
\label{sec:results}
%
We have implemented all of the analyses and transformations described
in this paper in FLEX.
We measure the effectiveness
of our optimizations by using FLEX to analyze the SPECjvm98 benchmarks
and apply our transformations,
then measuring the resulting space savings and
performance.   All benchmarks were run with the full input size
on a dual-processor 900 MHz
Pentium III running Debian Linux.
%
\subsection{Memory Savings}
%
To evaluate the effectiveness of our technique at reducing the
amount of memory required to execute the program, 
we first ran an instrumented version of each
application with no space optimizations. We used this
instrumented version to compute the maximum amount of live data on the
heap at any point during the execution.  
We then ran an instrumented version of our program after each stage of
optimization.
These versions enabled us to calculate the amount by which each 
technique reduced the size of the live heap data.%
\footnote{The instrumented versions collect all non-live data before
each allocation, so that our computed maximum heap sizes are accurate.}

\begin{figure*}
\begin{center}%
\parbox[b]{8cm}{\centering%
\includegraphics[width=7.8cm,clip=true]{Figures/sas-ttllive.eps}\\
(a) Reduction in the maximum live heap achieved with our transformations.}%
~~\parbox[b]{8cm}{\centering%
\includegraphics[width=7.8cm,clip=true]{Figures/sas-ttlalloc.eps}\\
(b) Cumulative reduction in dynamic allocation achieved with
    our transformations.
}%
\\[.3cm]
\parbox[b]{8cm}{\centering%
\includegraphics[width=7.8cm,clip=true]{Figures/sas-objalloc.eps}\\
(c) Reduction in non-array dynamic allocation achieved with
    our transformations.
\\~}%
~~\parbox[b]{8cm}{\centering%
\includegraphics[width=7.8cm,clip=true]{Figures/sas-objarrptr.eps}\\
(d) Pre-transformation allocation breakdown between arrays and objects,
with allocations attributable to fields of pointer type split out.}%
\\[.3cm]
\parbox[b]{8cm}{\centering%
\includegraphics[width=7.8cm,clip=true]{Figures/oopsla-speed.eps}\\
(e) Runtime performance of space optimizations.}%
\end{center}
\caption{Experimental results of space optimization transformations.}
\label{fig:results}
\end{figure*}
%
Figure~\ref{fig:results}a presents the total space savings. This
figure contains a bar for each application, with the bar broken
down into categories that indicate the percentage of live data from 
the original unoptimized execution that we were able to eliminate
with each optimization. The black section of each bar indicates the
amount of live heap data remaining after all optimizations. 
We obtain as much as 40\% reduction in live data on the {\tt
  javac} benchmark, with almost all of this reduction coming from our
bitwidth-driven field reductions and static specialization.  In fact
we obtain more than 15\% reduction on all of the ``object-oriented''
benchmarks.  The {\tt compress} benchmark allocates a small
number of very large arrays, limiting the optimization opportunities
discoverable by our analysis.  Likewise, the {\tt raytrace} and
{\tt mtrt} benchmarks make heavy use of floating-point numbers,
limiting the applicability of our integer bitwidth analysis.
However, these raytracing benchmarks allocate a large number
of small arrays to represent vectors and matrices, and so our header
optimizations still allow us to reduce the maximum live data size by
over 20\%.

We also used an instrumented executable to determine the total amount
of memory allocated during the entire execution of the program, in
both the optimized and unoptimized versions.  Reducing this total
allocation decreases the load on the garbage collector.
Figure~\ref{fig:results}b presents the space savings according to this
metric.  Comparison to the previous figure reveals that
long-lived objects provide proportionally more opportunities for
optimization.
%
\subsection{Objects Versus Arrays}
%
The majority of our optimizations are designed to optimize
object fields rather than arrays. For context, we present numbers 
that characterize the reductions in total allocation for objects only,
rather than for both objects and arrays. Figure~\ref{fig:results}c
presents space savings numbers for objects alone, omitting
any storage required for arrays.
Figure~\ref{fig:results}d explains the difference by showing how the total
program allocation for each benchmark is broken down into array and
object allocations.
%The space required for pointer fields, which can
%not be compressed, are further broken out of
%the object allocations.
The reason for our poor performance on {\tt compress} is now
  obvious---a few large uncompressible integer arrays
  account for over 99\% of the total space allocated.
%
\subsection{Execution Times} 
\label{sec:byte-pack}
%
We next evaluate the execution time impact of applying our space
optimizations. Figure~\ref{fig:results}e presents the normalized execution 
times of each benchmark after the application of our sequence
of optimizations. These numbers show that the first several
optimizations (class pointer compression, field reduction, and byte packing)
typically reduce the execution times, while the
remainder (static specialization, field externalization, and  hash/lock
externalization) generate modest increases in the execution times.
The speedup is due to reduced GC times, despite the indirection and
misalignment costs.
Static specialization's virtualization of fields is responsible for
its slowdown; it is likely that an optimized speculatively-inlined
implementation of the field accessors which it adds to the program
would improve its performance.  Field externalization (including
hash/lock externalization) causes the expected penalty for hashtable
lookup; note that synchronization elimination would greatly reduce the
cost of hash/lock externalization in the four cases where the overhead
is unreasonable.
%
%%% distribution of mostly-N fields?
%
\section{Related Work}
%
Many researchers have focused on the problem of reducing the amount of
header space required to represent Java
locks~\cite{bacon98,OK99,ADGKRW99}. The vast
majority of programs do not use the lock associated with every object
in its full generality, so it is possible to develop improved
algorithms optimized for the common case.
The idea is to represent the lock with the minimum amount
of state (typically a bit) required to support the common usage
pattern of an acquire followed by a release,
and to back off to a more elaborate scheme only when the thread
exhibits a more complex pattern such as nested locking. The
primary focus has been on improving performance rather than on
reducing space; however, many of the algorithms also eliminate the
need to store the complicated locking objects required to support the
most general lock usage pattern possible in a Java program. These
techniques typically reduce the lock space overhead to 24 header bits
\cite{bacon98}; Bacon et al. in \cite{bacon02:efficient} show speed
improvements from header-size reduction, in agreement with the results
presented here.

Research on escape analysis and related analyses can enable the
compiler to find objects whose locks are never 
acquired~\cite{ACSE99,BH99,whaley99,CGSSM99,Ruf00:PLDI00,salcianu01}.
This information can enable the compiler to remove the space
reserved for synchronization support in these objects. 
Our hash/lock removal algorithm uses a totally dynamic approach
based on our field externalization mechanism. 

Several researchers have used bitwidth analysis to reduce the size
of the generated circuits for compilers that generate hardware
implementations of programs written in C or similar programming 
languages~\cite{ananian:siliconc,ananian99:tech,RR00:PLDI00,stephenson00,BGSW00}.

Dieckmann and H\"olzle have performed an in-depth analysis of the
memory allocation behavior of Java programs~\cite{DH99}. Although 
space is not their primary focus, their study does quantify 
the space overhead associated with the use of a two-word header
and of 8-byte alignment. In general, our measurements of the 
memory system behavior of Java programs broadly agree with their
measurements. 

Sweeney and Tip \cite{SweeneyTip98DeadDataMembers} did a study of dead
members of C++ programs, which is similar to the unread field
elimination done by our bitwidth analysis.  However, they
fail to identify constant members, which our SCC-based algorithm
does.  Further, our results show that unread and constant field
elimination is very dependent on the coding style of a particular
application.  The collection of techniques we have presented here
gives much more consistent savings over a wide range of benchmarks.

Aggarwal and Randall \cite{aggarwal01} described an array bounds check
removal method using {\it related fields}.  This work attempted to
discover fields, such as {\tt Vector.size}, which are guaranteed to be
less than or equal to the length of some array, for example, the
backing array stored in {\tt Vector.data}.  Tests against the related
field could then provide information about bounds checks on accesses
to the array.  This technique could be used to infer additional
bitwidth information on related fields from our analysis.

Marinov and O'Callahan have submitted a paper for publication
\cite{marinov03} which determines when several instances of an
object may be safely represented by a single representative instance.
This is an additional and orthogonal memory-saving technique which
could be used in addition to the ones described here.

Zhang and Gupta describe a runtime technique that recognizes two
special cases when an integer or a pointer field in a designated C
data structure may be compressed \cite{zhang02}.  For all but two of
their benchmarks, their heap savings (on these benchmarks, an average
of 27\%) are entirely due to a pointer compression techique which is
orthogonal to the transformations described in this paper.  The
techniques could be combined for greater savings.

\section{Conclusions}
%
We have presented a set of techniques for reducing the memory
consumption of object-oriented programs.  Our techniques include
program analyses to detect unused, constant, or overly-wide fields,
and transformations to eliminate fields with common default values
or usage patterns.  These techniques apply equally well to
both user-defined fields and
fields implicit in the runtime's object header, and can reduce the
maximum heap required for a program by as much as 40\%. 
Our experimental results from our fully-implemented system validate the
opportunity for space savings on typical object oriented programs.

\renewcommand{\baselinestretch}{1}
\bibliography{harpoon}

%\appendix
%\input{pldi02-appendix}
\end{document}
