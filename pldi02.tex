% -*- latex -*- This is a LaTeX document.
% $Id: pldi02.tex,v 1.4 2001-11-15 09:43:17 cananian Exp $
%%%%%%%%%%%%%%%%%%%%%%%%%%%%%%%%%%%%%%%%
\documentclass[preprint]{acmconf}
% don't forget to turn off 'preprint' before submission!
%\usepackage[section,plain]{algorithm}
%\usepackage{amsthm} % proof environment
%\usepackage{amstext} % the \text command for math mode (replaces \mbox)
\usepackage{varioref} % \vref command
%\usepackage{graphicx} % for SCC bargraph
\usepackage{comdef}
\newcommand{\figscale}{1.0}

%setup varioref package
\renewcommand{\reftextbefore}{on the preceding page}\vrefwarning

\newcommand{\mycomment}[1]{}

\title{\bf Size Optimizations for Java Programs}

\author{C. Scott Ananian \\
	Laboratory for Computer Science\\
	Massachusetts Institute of Technology\\ 
	Cambridge, MA 02139 \\ 
	{\tt cananian@lcs.mit.edu} }

\begin{document}
% in preprint mode, tag pages with a revision identifier.
\pagestyle{myheadings}\markboth{$ $Revision: 1.4 $ $}{$ $Revision: 1.4 $ $}
\bibliographystyle{plain}

\maketitle

% abstract
% -*- latex -*- This is a LaTeX document.
% $Id: pldi02-abstract.tex,v 1.1 2001-11-15 01:26:01 cananian Exp $
%%%%%%%%%%%%%%%%%%%%%%%%%%%%%%%%%%%%%%%%
\begin{abstract}

This paper presents a set of Java optimizations targetted at
memory-constrained embedded devices.  We aggressively transform
the program using both source-level transformations and
specializations of the runtime environment to save as much
as {\bf XX}\% of the storage requested by the allocator.

Our techniques fall into three broad categories: header optimizations,
bitwidth analyses, and mostly-zero field elimination.  Header
optimizations reduce the size penalties associated with virtual
dispatch and Java hashcode and locking semantics.  Bitwidth
analyses allow static compile-time reduction of field sizes where the
full range of a datatype is unused.  Mostly-zero field elimination
attempts to factor out fields which are ``usually zero'' (as
determined by profiling) to obtain savings.  Combined, these
techniques free the programmer from the burden of space accounting
and allow the use of general-purpose software on extremely
memory-constrained embedded devices.

\end{abstract}


% outline for paper.
% introduction
\section{Introduction}
The devices around us are getting dramatically smarter, but the
processing capabilities of your average toaster will always remain far
removed from those of a desktop computer.  Nevertheless, the designers
of embedded devices would like to use the high-level software
languages and paradigms designed for the desktop environment when
programming their much smaller systems.  The past few years have
demonstrated a remarkable industry desire to run software written in
languages such as Java on cellphones, PDAs, and similar platforms.
At the moment, much of this desire is for downloaded applets and
extension programs and upgrades; for firmware the slight
inefficiencies of the high-level languages, multiplied by the large
quantities of devices deployed, warrant lower-level approaches.

The aim of this paper is to present optimization and analysis
techniques to close this gap.  There is as much as a four-fold
% maybe more! reference here?
cost difference between ROM and RAM costs in embedded devices,
so we will concentrate on reducing the dynamic heap usage of
programs written in Java.  We will bypass the mire of Real-Time Java
for the moment, and assume that running-time of the optimized programs
is not important (so long as it is reasonable); we will trade-off
slight increases in execution time and ROM usage for dramatically
reduced RAM usage.

We will first present header optimizations which tweak the runtime's
representation of non-field information in Java objects in order to
reduce memory consumption.  We will then turn our attention to
compressing the information stored in the fields.  Using static
analysis, we will determine the minimal number of bits required to
represent a field's contents, and then present three alternate
field-packing techniques to take advantage of the unused bits
identified.  For systems with small memories, pointer compression is
also possible.

Finally, we will attempt to identify 'mostly-constant' information in
objects, specializing the object to reduce the space required for
the common case.  For fields which are not mutated, statically
specializing the class is possible; for mutable fields we utilize an
external hashtable to store values which are rarely unpredictable.

We present the results of applying these optimizations to the SPEC
benchmark suite --- a set of general tasks, which, although they may bear
little relation to actual programs one would want on an embedded
device, span a gamut sufficient to adequately show the strengths and
weaknesses of our techniques.

% graph showing breakdown of allocated memory: header, pointers, fields.

\section{Header Optimizations}
% header optimizations
%  discussion of typical object layout.
%  claz compression.
%  hashcode/lock compression.
%   analyses for doing so; static/dynamic hash counts.

The Java language specification requires that several pieces of
information about an object be stored in addition to the data in its
fields.  Most obviously, information about the {\it class} or {\it
  type} of the object must be available in order to implement dynamic
dispatch and primitives such as {\tt instanceof}.  In addition, there
must be some way to associate a synchronization lock with each object,
and each object must have a {\it identity hash code} associated with
it to facilitate hashtable implementations of various kinds.%
\footnote{Arrays must, in addition, contain a specification of their
  length; we treat the length as a standard {\tt int} field of the
  object in this paper.}
A typical implementation uses two words of header information in
addition to the fields contained by the object.  The first, often
designated {\tt claz}, is a pointer to a class descriptor for the
object's type.  Dynamic dispatch is usually implemented as an
indirection through a table contained in this descriptor.
The second word is usually overloaded to contain both hashtable and
lock information.  Although the declaration of the {\tt
  Object.hashCode() } and {\tt System.identityHashCode()} methods
return a 32-bit {\tt int}, implementations usually return some
more restricted range of values --- one implementation of IBM's JDK
\cite{bacon98}
returned as little as 8 bits of hashcode.  The remaining bits of the
second word is used to represent the lock and any other information
kept by the runtime.  In the FLEX compiler system used for the
research in this paper \cite{flexweb}, 30 bits of hashcode are kept, and the
remaining 2 bits indicate whether the value is actually a hashcode or
a pointer to an ``inflated'' object structure containing other
information: a copy of the displaced hashcode, pthreads locks,
and other runtime information.

\subsection{{\tt claz} compression}
The first and most obvious means of reducing the size of the header
kept for each object is to replace the direct class descriptor
pointer with an index into a table.  For the closed-world systems
expected on embedded devices, the number of program classes is known
and may be compiled into the program.  For extensible systems, one may
instead define epochs of execution in which the number of classes
in the system is bounded; between epochs a full-system garbage
collection may be necessary to expand the size of the index in the
header, but epoch transistions should be rare.%
\footnote{Alternatively, a
variable length encoding could be used so that ``young'' class types
(hopefully the most common classes) may be given short indices without
limiting the index size as the class universe grows.  The runtime cost
of such a scheme is likely to be prohibitive, however.}

Table {\bf XX} shows the number of classes required to build each of
the spec benchmarks, as determined by a standard class hierarchy
analysis at compile-time.  Also enumerated are the heap savings to be
expected from reducing the size of the {\tt claz} information using
table-lookup; the meanings of the various alignment alternatives will
be discussed in section {\bf XX}.

\subsection{Hashcode/lock compression}
Many objects in a typical Java program are neither locked nor inserted
into hashtables; thus the second word of the header is unused.
If we can determine that synchronization on an object is unnecessary
or never performed, that the hashcode is likewise never accessed,
and that any other runtime features signalled in the second header
word are unused, we can eliminate it from the object.

Determining that the lock features are unnecessary is straight-forward.
We use our previously published escape analysis \cite{whaley99,vivien01}
% alex reference here
to determine on which objects synchronization must be performed; we
can discard the lock features of all the rest.  Hashcode utilization
is a little more involved: it is not sufficient to identify the
objects in which the system {\tt Object.hashCode()} method is
overridden and not invoked, because there is also a {\tt
  System.identityHashCode()} method in the library which provides the
same value on any object.  Standard type analysis around the
callsites in question will not help, because generic collection
classes such as {\tt java.util.Hashtable} typically invoke
the {\tt hashCode} method on the generic {\tt Object} type contained
in the collection.  Either context-sensitivity or a type-cone analysis
will suffice to obtain the needed information.

{\bf XXX.  In the SPEC benchmark suite, it is possible to statically
  determine that {\tt System.identityHashCode()} is never invoked.
  Thus, we are able to simply identify those classes which override
  Object.hashCode() and determine that the hashcode information in the
  second header word is unnecessary for those objects.}

{\bf XXX. In the results presented here, we will use a dynamic tally
of the object types on which hashCode is {\it actually} invoked, which
represents a best-case bound for the analysis.}

In addition, only the SPEC benchmark {\tt 227\_mtrt} is multithreaded;
it is trivial to determine that synchronization is unnecessary for all
objects in the other benchmarks.

Table {\bf XX} presents the number of objects on which both
synchronization and hashcodes can be eliminated in each benchmark,
along with the space savings for so doing.

\section{Field compression}
% field compression.
%  description of bitwidth analysis.
%   treatment of loops.
%   constant/unread fields.
%   scc.  crib from thesis.
%   interprocedural.  no appreciable gain from context sensitivity.
%   no pointer analysis to discriminate object classes (yea, type-safety)
%  implementation: bit/byte/java-type packing.
After the header has been reduced in size as much as possible, we turn
our attention to the fields of the object.  The key idea here is to
do a {\it bitwidth analysis} \cite{stephenson00,ananian99:tech}
to determine a
smaller number of bits sufficient to represent the values stored in
the field.  We can then shrink the size of the field, subject to
limitations on field and object alignment.  The largest savings can be
realized by discarding alignment restrictions, a choice which requires
substantial changes to the runtime environment to realize.  More
modest reductions may be obtained with much less effort by
more-strongly aligning objects and restricting field widths to
those of standard java types.

\subsection{Bitwidth analysis}
Our bitwidth analysis is an inter-procedural extension of that
described in {\bf reference here}.  It is based on Wegman and Zadeck's
Sparse Conditional Constant (SCC) propagation algorithm
\cite{wegman91:scc}\ldots

{\bf write me}
% field-based

\subsection{Field packing}
If the program is going to run on a standard java virtual machine, or
for some other reason modifications to the runtime environment are not
possible, then optimization based on the bitwidth analysis can only
shrink each field to the size of a standard java type: 1,%
\footnote{If the runtime implements booleans as bit vectors, which the
  Java language specification allows but does not mandate.}
8, 16, 32, or 64 bits.  The entire object will be padded so that
consecutive objects are properly aligned, most often to a word or
double-word boundary.  This scenario is labelled ``std'' in our tables.
% XXX: describe our field packing algorithm?

If we can modify the runtime, however, much better use may be made of
the information from the bitwidth analysis.  For example, we can
allow field widths of any integral number of bytes, and pad/align
objects to byte boundaries, instead of words.  This entails some
cycle overhead on machines which disallow unaligned accesses, but is
fairly straight-forward to implement.  We can take much better
advantage of the bitwidth analysis results: a 33-bit wide field will
end up occupying a byte-aligned 40 bits (5 bytes), instead of the 64
bits (8 bytes) that the ``std'' scenario would require.  We will refer
to this scenario as ``byte''.

However, we can even improve this result.  If maximum space reduction
is desired, we can {\it bit}-align fields within a word.  Further, we
can bit-align objects as well by shifting our pointers left by 3 and
using the new lower-order bits to encode a bit-within-the-byte offset.
This requires much more time overhead to decode pointers and fields.
This scenario is called ``bit''.

Table {\bf XXX} shows the allocation reductions due to our bitwidth
analysis under each of the above scenarios.  We can see that {\bf blah
  blah blah.  ``bit'' does best, of course.}

\subsection{Pointer compression}
% sub-topic: pointer compression.
%  present numbers: how your memory stacks up.
The bitwidth analysis only produces results for fields of integer type.
These comprise on average {\bf XX}\% of the field bytes allocated in
our benchmarks.  Fields with reference type constitute {\bf XX}\% of
the field bytes allocated.  These fields can be compressed, too:
typically an embedded system will not require a full 32-bit address
space,\footnote{Not to mention a 64-bit address space!} due to
limitations on the amount of memory and I/O address space actually attached
to the device.  If we are byte- or bit-packing our fields already,
it is straight-forward to implement sub-word widths for fields of
reference type.  Figure {\bf XX} shows the expected space savings
across our benchmarks as one shrinks the pointer size.  A
space-optimizing java compiler ought to be able to specialize the code
emitted for the address space size of the target device to take
advantage of these potential savings.

\section{Mostly-zero field analysis}
% mostly-zero field analysis.
%  as extension: mostly-'N'
%  techniques:
%   dynamic specialization
%    analyses required.
%   ``external fields''
%    profiling/analyses required.
%    implementation of external hashtable.
\subsection{Dynamic specialization of final fields}
\subsection{Creating external fields}
\subsubsection{External hashtable implementation}

\section{Results}
% results/discussion.
%  present sum totals.
%  compress doesn't do so hot: show why.
\section{Conclusion}
% conclusion/future work.
%  better pointer analysis/separate out classes of objects.
%   (strings as motivating example: compress to byte[])
%  pointer encoding: more common pointers at lower addresses?
%  other stuff?


\bibliography{harpoon}

%\appendix
%\input{pldi02-appendix}
\end{document}
