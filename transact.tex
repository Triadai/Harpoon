% -*-latex-*- This is a LaTeX document.
% $Id: transact.tex,v 1.2 2000-09-25 22:15:35 cananian Exp $
\documentclass[11pt,notitlepage]{article}
\usepackage{alltt}
\usepackage{pdffonts} % PDF-friendly fonts
\author{C.~Scott~Ananian}
\title{Software Atomic Transactions}
\date{\today \\ $ $Revision: 1.2 $ $}

\begin{document}
%\bibliographystyle{abbrv}
\maketitle
\section{Data Structures}

Transaction support uses the same ``inflated object'' extension
mechanism which is used for thread synchronization, clustered heaps,
JNI data, and finalization for certain garbage collectors.
The basic object data structure, as shown in figure \ref{fig:oobj}, is
unchanged, although fields containing the specified
\texttt{FLAG\_VALUE} have different semantics.  The flag value is used
to indicate that the field value is ``not here''; that is, the code
must consult the transaction information to find the field's current
value.  This happens very rarely even when no transaction is
associated with the object; Shasta \cite{shasta} has shown that the
overhead entailed by such ``false'' transactions is expected to be
extremely low.

The ``inflated object'' data structure, shown in figure
\ref{fig:infl}, has a simple ``object versions'' array associated with
it.  Each entry conceptually stores information about a different,
possibly-uncommitted, version of the object.  One of these versions is
guaranteed to be ``most recent committed'', and it is guaranteed that
the ``most recent committed'' version will be the first committed
version in the list.

\begin{figure}
\begin{alltt}\small
/* the oobj structure tells you what's inside the object layout. */
struct oobj \{
  struct claz *claz;
  /* if low bit is one, then this is a fair-dinkum hashcode. else, it's a
   * pointer to a struct inflated_oobj. this pointer needs to be freed
   * when the object is garbage collected, which is done w/ a finalizer. */
  union \{ ptroff_t hashcode; struct inflated_oobj *inflated; \} hashunion;
\textbf{#ifdef WITH_TRANSACTIONS
#define FLAG_VALUE 0xCACACACA
  /* consult transaction to determine value of any field with FLAG_VALUE */
#endif}
  /* fields below this point */
  char field_start[0];
\};
\end{alltt}
\caption{The object structure in the FLEX runtime.}
\label{fig:oobj}
\end{figure}
\begin{figure}
\begin{alltt}\small
/* the inflated_oobj structure has various bits of information that we
 * want to associate with *some* (not all) objects. */
struct inflated_oobj \{
  ptroff_t hashcode; /* the real hashcode, since we've booted it */
  void *jni_data; /* information associated with this object by the JNI */
  void (*jni_cleanup_func)(void *jni_data);
\textbf{  /* TRANSACTION SUPPORT */
#if WITH_TRANSACTIONS
  struct vinfo *versions[NESTED_TRANS_MAX];
#endif}
  /* locking information */
#if WITH_HEAVY_THREADS || WITH_PTH_THREADS
  pthread_t tid; /* can be zero, if no one has this lock */
  jint nesting_depth; /* recursive lock nesting depth */
  pthread_mutex_t mutex; /* simple (not recursive) lock */
  pthread_cond_t  cond; /* condition variable */
  pthread_rwlock_t jni_data_lock; /*read/write lock for jni_data field, above*/
#endif
#ifdef WITH_CLUSTERED_HEAPS
  struct clustered_heap * heap;
  void (*heap_release)(struct clustered_heap *);
#endif
#ifdef BDW_CONSERVATIVE_GC
  /* for cleanup via finalization */
  GC_finalization_proc old_finalizer;
  GC_PTR old_client_data;
#endif
\};
\end{alltt}
\caption{The ``inflated object'' structure in the FLEX runtime.}
\label{fig:infl}
\end{figure}

%\bibliography{harpoon}
\end{document}