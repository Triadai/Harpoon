% -*-latex-*- This is a LaTeX document.
% $Id: v.tex,v 1.2 2005-10-27 15:22:14 cananian Exp $
%%%%%%%%%%%%%%%%%%%%%%%%%%%%%%%%%%%%%%
\documentclass[11pt,notitlepage,twocolumn]{article}
\usepackage{pdffonts} % PDF-friendly fonts
\usepackage{xspace}
\newcommand{\vlang}{\textsf{v}\xspace}
\author{C.~Scott~Ananian}
\title{\vlang: Rebuilding Java}
\date{\today \\ $ $Revision: 1.2 $ $}

\begin{document}
\bibliographystyle{abbrv}
\maketitle

\section{Introduction}
The Java language, in its ``version 5'' incarnation, is a huge and
complex language, riddled with syntax and burdened with a baroque
albeit extremely capable type system.  Guy Steele's now-famous talk
``Growing a Language'' \cite{steele} argued convincingly that means
for extension must be designed into a language at a very basic level;
this goal was not achieved for the Java language, and so each new
iteration of the language sprouts new orthogonal features out of its
already-impressive ball of hair.

The goal of this present work is to disassemble Java and discover a
core language of complexity similar to that of the original Java
1.0 specification.  To this core language we will add a flexible means
of extension to create \vlang, our small base language.  We will then use
our extension mechanism to recreate the full hair ball of Java 5.0,
but this time built with a means to avoid further \emph{deus ex
  machina} iterations of the language.  From our new language built on
\vlang, we can add any new language feature we like, with or without the
blessing of Sun.  A thousand flowers may bloom, some in particular
niches, and the hardiest general ones may become blessed into
``official'' Java.

Our extension mechanism must encompass both the syntax of the
language, as well its type system.  We propose to allow extension or
replacement of \vlang's parser, and to handle types with abstract
interpretation.  The boundary between compile-time and run-time will
blur, and we propose to re-establish ``compile-time'' safety (and
other software-engineering niceties) via aggressive partial
evaluation, both of the program and of its abstractions.  We expect
that partial evaluations of the ``standard'' language extensions will
be cached and special optimizations made to enable reasonably
efficient compilation of the ``standard'' language, without precluding
the compilation and execution of arbitrarily modified extensions and
variations of it.

\section{Thoughts}
\begin{itemize}
\item Classes can be built on top of a prototype-based system, so we
  expect \vlang to be prototype-based.
\item Creating a full meta-object protocol for \vlang will allow arbitrary
  extension of the object inheritance and other properties of the
  class/type system.  Special optimizations will recreate the standard
  table-based lookup in the cases where it applies.
\item Types are just an abstract interpretation of the program.  New
  abstractions can be added to extend/modify the type system.  The
  types may modify the static semantics of the language via rewriting;
  method overloading (for example) is not a problem.  The \vlang language
  does not include static overloading; it is implement as an extension which
  renames method definitions and call sites according to the results
  of a ``type analysis'' (which is really the execution of an abstract
  version of the program).  Because the boundary between runtime and
  compile-time is blurred, this renaming can even occur ``at
  runtime'', although obviously the ability to generate compile-time
  error messages about improper overloading will be sacrificed in this
  case.  Instead the compiler may generate a warning that this
  evaluation has been deferred to runtime (or, more precisely, that
  the code generating the error message could not be completely
  evaluated at compile-time).
\item ``Grammars'' will probably be reified into the language, ala
  Perl 6, to enable programmatic access to and modification of the
  language.  Ideally this will be done via an integrated
  pattern-matching system in \vlang.
\item \vlang will probably appear as a funny type of dynamic language,
  that happens to have a complicated macro system which lets people
  program in a peculiar statically-typed syntax.  Most of the
  runtime checks and dispatch mechanisms of a traditional dynamic
  language will be optimized away.  It is not particularly
  necessary to be able to directly extend the compiler to add
  optimizations for new language extensions: the creation of
  sophisticated compilers which efficiently execute common constructs
  (read, common extensions to \vlang) will remain the domain of compiler
  authors.   The programmer just needs to be able to define and
  redefine their specific language; they don't always need it to run
  fast (although we hope that general optimization techniques will
  also allow the efficient execution of languages ``close enough'' to
  those language variants specially targeted for optimization).
\end{itemize}

\section{Conclusion}

\bibliography{harpoon}

\end{document}

% LocalWords:  reified programmatic
