% -*- latex -*- This is a LaTeX document.
% $Id: thesis.tex,v 1.39 1999-06-13 19:55:54 cananian Exp $
%%%%%%%%%%%%%%%%%%%%%%%%%%%%%%%%%%%%%%%%%%%%%%%%%%%%%%%%
\documentclass[12pt,notitlepage]{article}
%\documentclass[12pt,notitlepage,twocolumn,twoside]{article}
\usepackage{comdef}

% the name of the project
\newcommand{\magic}{\textsc{Magic}}

% double-space
\linespread{1.2} %one-and-a-half spacing is 1.3

\title{Static Single Information Form}
\author{C. Scott Ananian}
\date{\today \\ $ $Revision: 1.39 $ $}

\begin{document}
\pagestyle{myheadings}\markboth{$ $Revision: 1.39 $ $}{$ $Revision: 1.39 $ $}
\bibliographystyle{abbrv}

\maketitle

\section{Introduction}
The selection of a compiler internal respresentation from the many
littering the academic literature may at times seem a black art.  Each
IR is accompanied by papers trumpeting its superiority for one purpose
or another, and each IR feels compelled to introduce new terms and
structures into the jargon.  A pragmatic compiler writer will often
pick randomly from the plethora available, basing the decision on ease
of implementation or understandibility without realizing the extent to
which the IR influences the structure of the compiler.  Similarly, the
theorist will derive proofs based upon intermediate representations
(and languages) that are amenable to mathematical methods, without
regard to current implementation practices.%
\fixme{This paragraph will be revised to be less offensive, eventually.}

This paper introduces yet another intermediate
representation to the literature:  Static Single Information (SSI) form.
This IR is the core of the FLEX compiler project, which is primarily
investigating intelligent compilation techniques for distributed
systems.  This thesis, in presenting the IR,
attempts to keep both the mathematician and the programmer in mind.  
SSI form has both a rigorous mathematical semantics and a factored
form which aids efficient implementation of advanced analyses.
I believe that it best straddles the gap between dataflow-oriented,
graph-structured, and control-flow driven IRs, while maintaining the
sparsity needed to achieve practical efficiency.  The construction
algorithms are linear in the size of the program.

Our discussion of the Static Single Information form will be at times
tied to the source language of the FLEX compiler, Java.  Unlike many
abstract IRs, the choices made in the design of SSI form have been
dictated by the necessities of compiling a real-world imperative
language.  Java, however, has several theoretical properties that make
mathematical analysis more tractable.  In particular, we
will mention here Java's strict constraints on pointer variables.
Pointers in C can be abused in many ways that Java disallows.

Ultimately, the choice of compiler internal representation is fundamental.
Advances in IRs translate into advances in compilers.  SSI form
represents a clean and simple unification of many extant ideas, and
our hope is that it will allow the FLEX compiler to achieve a similar
integration of practical implementation and mathematical elegance.

\section{Intermediate Representations}
The first paper directly addressing an IR as a separate feature of a
compiler was Foo and Bar in \cite{foobar}.\fixme{Find real references.}
The Static Single Assignment (SSA) form was introduced by 
Alpern, Rosen, Wegman and Zadeck
as a tool for efficient optimization in a pair of POPL
papers \cite{alpern88:ssa,rosen88:gvn}, and three years later Cytron
and Ferrante joined Rosen, Wegman, and Zadeck in explaining how to
compute SSA form efficiently in what has since become the 
``canonical'' SSA paper \cite{cytron89:ssa}.  Johnson and Pingali
\cite{johnson93:dfg} trace the development of SSA form back to Shapiro
and Saint in \cite{shapiro70:ssa}, while Havlak \cite{havlak94:isa}
views \phifunction{s} as descendants of the ``birthpoints'' introduced
in \cite{reif81:sym}.

Despite industry adoption of SSA form in production compilers
\cite{chow96:hssa,chow97:ssapre}, academic research into alternative
representations continues.
Recent proposals have included Value Dependence Graphs
\cite{weise94:vdg}, Program Dependence Webs \cite{ballance90:pdw},
the Program Structure Tree \cite{johnson94:pst},
DJ graphs \cite{sreedhar96:dj}, and Depedence Flow Graphs
\cite{johnson93:dfg}.

In comparison to these representations, the dominant characteristics of
our Static Single Information form may be summarized as follows:
\begin{itemize}
\item It names information units.
\item It is complete.
\item It is simple.
\item It is efficient.
\item It has no explicit control dependencies.
\item It allows both forward and reverse dataflow analyses.
\end{itemize}
SSI form is used as an IR for the FLEX compiler for the Java
programming language, which informs some of these design decisions.
The FLEX compiler does deep analysis and will support
hardware/software co-design.  SSI addresses these needs.  We will
address each design point in turn.

\textbf{It names information units.}  SSA form (which we will describe
further in \ref{sec:ssa}) assigns unique names to unique \emph{static
values} of a variable.  However, it ignores the value information
which may be added to a variable at program branch points.  SSI form
renames at branch points which allows us to associate unique names
with unique \emph{information} about static values.  For example, a
program may test the value of an integer against zero before using it
as a divisor.  After the branch on the tested predicate, it is
possible to make statements about values (regarding equality or
inequality to zero) which were impossible to make previously.  SSI
form allows us to take advantage of this additional information.

\textbf{It is complete.}  By this we mean that there exists an
executable semantics for the IR that does not require the use of
information external to the IR.  The original SSA form---and most
derivatives---require use of the original program control flow graph
during analysis, translation, or direct execution.  In fact,
\phifunction{s} are intimately tied with the precise input edge
structure of the control flow graph, and switch nodes (where control
flow splits) are undecipherable without referring to the control flow
graph.

In practice, this seems not a great disadvantage---it merely forces up to
maintain a mapping of SSA statements to nodes (equivalently, basic
blocks) of the original control flow graph.  But maintaining this
correspondence complicates editing the IR.  Also, it complicates the
interpretation of the program as a set of simultaneous equations,
which SSI form will allow us to do.  Finally, explicit control flow
may limit the available parallelism of the program.

\ssiplus, as it will be presented in section \ref{sec:ssiplus},
overcomes these difficulties and presents a \emph{complete}
representation of program meaning as a set of simultaneous equations,
without resort to graph information.

\textbf{It is simple.}  A bestiary of new $\phi$-like functions have
been introduced in recent\fixme{Not really.  Some as early as 1990.} papers, including
$\mu$-, $\gamma$-, and $\eta$-functions in \cite{ballance90:pdw,tu95:gssa},
$\psi$- and $\pi$-functions in \cite{lee99:parssa},
interprocedural \phifunction{s} in \cite{liao99:issa},
$\mu$- and $\chi$-functions in \cite{chow96:hssa},
$\mu$- and $\eta$-functions in \cite{gerlek95:inductssa},\footnote{Compare to
\cite{ballance90:pdw,tu95:gssa}.}, among others.\footnote{Maybe.
Actually I've named all the ones I know about.  Except for TGSSA,
an early form of which apparently used $\mu$- and $\nu$-functions (mentioned in
\cite{weise94:vdg}).}
Some of these are orthogonal to our work---the techniques of
\cite{lee99:parssa} can be used to extend SSI form to explicitly
parallel source languages, and those of \cite{chow96:hssa} to
languages with local variable aliasing (absent in Java).  Our goal is
to achieve minimal conceptual complexity in SSI form; that is, to
introduce the minimum set of $\phi$-like functions necessary to
represent the ``interesting'' properties of the compiled program.

\textbf{It is efficient.}  Construction of SSI form should be fast,
and space requirements should be reasonable.  The original SSA
algorithms required $O(n^3)$ time\ldots
\fixme{Finish this.  The original time bound for SSA is given in
       \cite{dhamdhere92:large}.}
%Construction should be fast, and
%space should be reasonable.\footnote{Replace with actual numbers.}

\textbf{All explicit control dependencies are eliminated.}  Many IRs
require explicit control flow edges that serialize computation more
than is strictly necessary.\fixme{Explain better.}

\textbf{It is efficient for both forward and backward dataflow analyses.}
\fixme{Explain this: we can do backwards analyses.}

\textbf{Transition goes here: we are going to describe SSA form for
background on SSI form.}\fixme{Write this.}

\subsection{Static Single Assignment form}\label{sec:ssa}
Static Single Assignment (SSA) form, introduced by Cytron in
\cite{cytron89:ssa}\ldots

\textbf{Define SSA form here.}\fixme{Write this.}

Figure~\ref{fig:tossa} illustrates the conversion to SSA form.
\begin{myfigure}
\begin{center}
\input{Figures/THex1base} \vline\ \input{Figures/THex1ssa}
\end{center}
\caption{\phifunction{s} are added to the program on the left to
produce the SSA-form IR on the right.}
\label{fig:tossa}
\end{myfigure}

\subsection{Minimal and pruned SSA forms}
Cytron defines a minimal SSA form in \cite{cytron91:ssa}.  His minimal
form is defined to use the smallest number of \phifunction{s} such
that the following three conditions hold:
\begin{enumerate}
\item If two nonnull paths $X \pathplus Z$ and $Y \pathplus Z$
converge at a node $Z$, and nodes $X$ and $Y$ contain assignments to
[a variable] $V$ (in the original program), then a trivial
\phifunction{} $V \leftarrow \phi(V, \ldots, V)$ has been inserted at
$Z$ (in the new program). \label{criteria1}
\item Each mention of $V$ in the original program or in an inserted
\phifunction{} has been replaced by a mention of a new variable $V_i$,
leaving the new program in SSA form.
\item Along any control flow path, consider any use of a variable $V$
(in the original program) and the corresponding use of $V_i$ (in the
new program).  Then $V$ and $V_i$ have the same value.
\end{enumerate}
Of these criteria, the first %\ref{criteria1}
is the most important.
%The other two can be summarized by noting that in proper SSA form,
%there is exactly one reaching definition for a variable $V$ at every
%non-$\phi$ use of $V$.
The SSA form in figure~\ref{fig:tossa} is minimal.

A variation
on minimal SSA form, called \emph{pruned} form \cite{ferrante91:pruned},
avoids placing \phifunction{s} which define variables which are never used.
In most cases, the more regular properties of minimal SSA form
outweigh the pruned form's slight increase in space efficiency.
Figure~\ref{fig:prunedssa} compares minimal and pruned SSA form for
our example program.\fixme{Mention \cite{ferrante91:pruned}, which
contains definition and algorithm for pruned SSA.}
\begin{myfigure}
\begin{center}
\input{Figures/THex1ssa} \vline\ \input{Figures/THex1ssaPr}
\end{center}
\caption{Minimal SSA form on the left; pruned SSA form on the right.}
\label{fig:prunedssa}
\end{myfigure}

\section{Static Single Information form}

SSI form extends SSA form to achieve symmetry for both forward and
reverse dataflow.   SSI form recognizes that information about
variables is generated at branches and generates new names at these
points.  This provides us with a one-to-one mapping between variable
names and information about those variables which is independent of
control-flow graph context.  Analyses can then associate information
with variable names and propagate this information efficiently and
directly both with and against the dataflow direction.

\subsection{Definition of SSI form}
Building SSI form involves adding pseudo-assignments for a variable $V$:
\begin{enumerate}
\item[$(\phi)$] at a control-flow merge when disjoint paths from a
conditional branch come together and at least one of the paths
contains a definition of $V$; and
\item[$(\sigma)$] at locations where control-flow splits and at least
one of the disjoint paths from the split uses the value of $V$.
\end{enumerate}

Figure~\ref{fig:tossi} compares the SSA and SSI forms for our example program.
% add text here about how SSI is better.
\begin{myfigure}
\begin{center}
\input{Figures/THex1ssa} \vline\ \input{Figures/THex1ssi}
\end{center}
\caption{SSA form on the left; SSI form on the right.}
\label{fig:tossi}
\end{myfigure}

\subsection{Minimal and pruned SSI forms}
\emph{Minimal} and \emph{pruned} SSI forms can be defined which
parallel their SSA counterparts.  \emph{Minimal} SSI form would have the
smallest number of \phisigfunction{s} such that:
\begin{enumerate}
\item If two nonnull paths $X \pathplus Z$ and $Y \pathplus Z$
exist having only the node $Z$ where they converge in common,
and nodes $X$ and $Y$ contain either assignments to a variable $V$ in the
original program or a \phisigfunction[or] for $V$ in the new program,
then a \phifunction{} for $V$ has been inserted at $Z$ in the new program.
\item If two nonnull paths $Z \pathplus X$ and $Z \pathplus Y$
exist having only the node $Z$ where they diverge in common,
and nodes $X$ and $Y$ contain either uses of a variable $V$ in the
original program or a \phisigfunction[or] for $V$ in the new program,
then a \sigfunction{} for $V$ has been inserted at $Z$ in the new program.
\item There is exactly one reaching definition of $V$ at every
non-$\phi$ use of $V$ in the new program.
\item \textbf{FIXME: what's the equivalent condition for \sigfunction{s}?}
% has something to do with cycle-equivalence.
\end{enumerate}

A \emph{pruned} SSI form would be the minimal form without any unused
definitions, that is, \phisigfunction[or]s after which there are no
subsequent uses of any of the variables defined on the left-hand side,
except in other \phisigfunction[or]s.\footnote{An even more
compact SSI form may be produced by removing \sigfunction{s} for which
there are uses for \emph{exactly one} of the variables on the
left-hand side, but it does not seem profitable to split this
particular hair.}

Figure~\ref{fig:prunedssi} compares minimal and pruned SSI form for
our example program.
\begin{myfigure}
\begin{center}
\input{Figures/THex1ssi} \vline\ \input{Figures/THex1ssiPr}
\end{center}
\caption{Minimal SSI form on the left; pruned SSI form on the right.}
\label{fig:prunedssi}
\end{myfigure}

\subsection{Fast construction of SSI form}
Our construction algorithm begins with a program structure tree of
single-entry single-exit (SESE) regions, constructed as described by
Johnson, Pearson, and Pingali \cite{johnson94:pst}.  We will review
the algorithms involved, as their published descriptions
\cite{johnson93:sese} contain a number of errors.

We begin with a few definitions from \cite{johnson94:pst}.
\begin{definition}
Edges $a$ and $b$ are said to be \newterm{edge cycle-equivalent} in a
graph iff every cycle containing $a$ contains $b$, and vice-versa.
Similarly, two nodes are said to be \newterm{node cycle-equivalent} iff
every cycle containing one of the nodes also contains the other.
\end{definition}
\begin{definition}
A \newterm{SESE region} in a graph $G$ is an ordered edge pair
$\tuple{a,b}$ of distinct control flow edges $a$ and $b$ where
\begin{tightenum}
\item $a$ dominates $b$,
\item $b$ postdominates $a$, and
\item every cycle containing $a$ also contains $b$ and vice-versa.
\end{tightenum}
Edges $a$ and $b$ are called the \newterm{entry} and \newterm{exit} edges,
respectively.
\end{definition}
\begin{definition}
A SESE region $\tuple{a,b}$ is \newterm{canonical} provided
\begin{tightenum}
\item $b$ dominates $b'$ for any SESE region $\tuple{a,b'}$, and
\item $a$ postdominates $a'$ for any SESE region $\tuple{a',b}$.
\end{tightenum}
\end{definition}

\subsubsection{Cycle-equivalency}
\newcommand{\cyceq}{\equiv_{cq}}%{\equiv_{CYC}}
The identification of SESE regions begins by computing the
cycle-equivalency of the edges in the program control flow graph.  The
cycle-equivalency algorithm works on undirected graphs, so we prepare the
directed control flow graph $G$ as follows:
\begin{enumerate}
\item \textbf{Add an edge from END to START in $\mathbf{G}$.} It is common
practice to add an edge from START to END in order to root the control
dependence graph at START \cite{cytron89:ssa}.  However in this case
the edge serves to make the control flow graph into a single strongly
connected component, and for this reason the direction of the edge is
from END to START.
\item \textbf{Create an equivalent undirected graph.}  Johnson et al.\
prove that the node expansion illustrated in figure~\ref{fig:CQundir}
results in an undirected graph with the same cycle-equivalency
properties as the original directed graph.  More precisely, nodes $a$
and $b$ in directed graph $G$ are cycle-equivalent if and only if
nodes $a'$ and $b'$ are cycle-equivalent in transformed undirected
graph $G'$.  The nodes $n_i$ and $n_o$ generated by the expansion are
termed \emph{not representative}; the node $n'$ in $G'$ is said to be
\emph{representative} of node $n$ in $G$.  Obviously, this
correspondence must be recorded during the transformation so we may
properly attribute the cycle-equivalency properties of $n'$ to $n$
later.
\begin{myfigure}
\begin{center}
\input{Figures/THundir}
\end{center}
\caption{Transformation from directed to undirected graph
	 (from \cite{johnson93:sese}).}
\label{fig:CQundir}
\end{myfigure}
\item \textbf{Perform a pre-order numbering of nodes in $\mathbf{G'}$.}
This is done with a simple depth-first search of $G'$.  When we visit
a node $a_i$ or $a_o$, we prefer to visit $a'$ before any other
neighbor.  This ensures that representative nodes are interior nodes
in the DFS spanning tree. The START node is numbered 0, and succeeding
nodes in the traversal get increasing numbers.  Thus low-numbered
nodes are closest to START and we will call them ``highest'' in the
DFS spanning tree.
\end{enumerate}

The above steps form an undirected graph $G'$ from the control-flow
graph $G$.  The remainder of the cycle-equivalency algorithm is
presented in figure~\ref{fig:CQalg}, with the above procedure
corresponding to the statement \texttt{G':=Preprocess(G)}.  The
algorithm has been corrected from the published version in
\cite{johnson93:sese}; in addition it has been extended to compute
both node and edge equivalencies (in effect, merging the algorithm of
\cite{johnson94:pst}).  Lines modified from the presentation in
\cite{johnson93:sese} are indicated in the figure with a vertical bar
in the left margin.  The datatype \texttt{BracketList} and the node
and edge properties used in the algorithm are described in
figure~\ref{fig:CQdata}.  The interested reader is encouraged to consult
\cite{johnson93:sese} for additional detail on these data structures
and representations.%
\fixme{Um, change to compute \emph{edge}
equivalency as well as \emph{node} equivalency.  Merge algorithms from
\cite{johnson93:sese} and \cite{johnson94:pst}.}
Figure~\ref{fig:CQex} shows cycle-equivalent regions in a simple
control-flow graph.  We use the notation
$\tuple{a,b}\cyceq\tuple{c,d}$ to indicate that the CFG edge from node
$a$ to node $b$ is edge cycle-equivalent to the edge from node $c$ to
node $d$.

Calculating cycle-equivalent regions is based on a single reverse
depth-first traversal of $G$, so as long as all datatype operations in
figure~\ref{fig:CQdata} can be completed in constant time (and
\cite{johnson93:sese} shows how to do so), this computation is $O(E)$.

\begin{myfigure}\small
% data types for cycle-equivalency algorithm.
Data type \texttt{BracketList}:
\begin{tightdesc}
\item[create():~BracketList]
:	Make an empty BracketList structure
\item[size(bl:BracketList):~integer]
:	Number of elements in BracketList structure
\item[push(bl:BracketList,~e:bracket):~BracketList]
:	Push \textbf{e} on top of \textbf{bl}
\item[top(bl:BracketList):~bracket]
:	Topmost bracket in \textbf{bl}
\item[delete(bl:BracketList,~e:bracket):~BracketList]
:	Delete \textbf{e} from \textbf{bl}
\item[concat(bl1,bl2:BracketList):~BracketList]
:	Concatenate \textbf{bl1} and \textbf{bl2}
\end{tightdesc}

Operations on nodes:
\begin{tightdesc}
\item[Number(n:node):~integer]
:	DFS preorder number of node
\item[NQClass(n:node):~integer]
:	Cycle-equivalency class of node
\item[BList(n:node):~BracketList]
:	List of brackets of node
\item[Hi(n:node):~integer]
:	Highest destination node of any edge originating from a
	descendant of node \textbf{n}
\end{tightdesc}

Operations on edges:
\begin{tightdesc}
\item[EQClass(e:node):~integer]
:	Cycle-equivalency class of edge
\item[RecentSize(e:edge):~integer]
:	Size of bracket set when \textbf{e} was most recently the
	topmost bracket for a representative node
\item[RecentClass(e:edge):~integer]
:	Cycle-equivalency class number of representative node for
	which \textbf{e} was most recently the topmost bracket.
\end{tightdesc}

\caption{Datatypes and operations used in the cycle-equivalency algorithm.}
\label{fig:CQdata}
\end{myfigure}

\begin{myfigure}\small\linespread{0.75}
% algorithm for cycle-equivalency on nodes and edges
\begin{verbatim}
Procedure cycle_equiv (G: CFG)
{
  /* Preprocessing */
  G' := Preprocess (G); /* described in text */

  /* Compute CD equivalence classes */
  for each node n of G', in reverse depth-first order, do {
    /* Compute Hi(n) */
    /*  hi0 is highest using backedges only */
    hi0 := min{ Number(t) | (t,n) is a backedge };
    /*  hi1 is highest through children */
    hi1 := min{ Hi(c) | c is a child of n };
|   /*  hi2 is lowest through children */
|   hi2 := max{ Hi(c) | c is a child of n };

    Hi(n) := min{ hi0, hi1 };

    /* Compute BList(n) */
    BList(n) := create ();

    for each child c of n, do
      BList(n) := concat (BList(n), BList(c));

    for each backedge <d, n> from a descendant d of n to n, do
      BList(n) := delete (BList(n), <d, n>);
|   for each capping backedge <d, n> of n, do
|     BList(n) := delete (BList(n), <d, n>);

    for each backedge <n, a> from n to an ancestor a of n, do {
      BList(n) := push (BList(n), <n, a>)
      RecentSize(<n, a>) := -1; /* not a representative node */
    }

    if n has more than one child, then {
      BList(n) := push (BList(n), <n, hi2>); /* capping backedge */
|     RecentSize(<n, hi2>) := -1;
|     add <n, hi2> to capping backedges list of hi2;
    }

    /* Compute Class (n) */
    if n is a representative node, then {
      if RecentSize (top (BList(n))) != size (BList(n)), then {
        /* start a new equivalence class */
        RecentSize (top (BList(n))) := size (BList(n));
        RecentClass (top (BList(n))) := new-class-name();
      }
      Class (n) := RecentClass (top (Blist(n)));
    }
  } /* for each node */
}
\end{verbatim}

\caption{The cycle-equivalency algorithm
	 (corrected from \cite{johnson93:sese}).}
\label{fig:CQalg}
\end{myfigure}

\begin{myfigure}
\centering
\vertcenter{\input{Figures/THcqex}}
$\begin{array}[c]{cc}
\tuple{\mbox{START},1}\cyceq\tuple{16,\mbox{END}}\\
\tuple{1,2}\cyceq\tuple{8,16}\\
\tuple{2,3}\cyceq\tuple{3,4}\cyceq\tuple{7,8}\\
\tuple{4,5}\cyceq\tuple{5,7}\\
\tuple{4,6}\cyceq\tuple{6,7}\\
\tuple{1,9}\cyceq\tuple{9,10}\cyceq\tuple{14,15}\cyceq\tuple{15,16}\\
\tuple{10,11}\cyceq\tuple{11,13}\\
\end{array}$
\caption{Control flow graph and cycle-equivalent edges.}
\label{fig:CQex}
\end{myfigure}

\subsubsection{SESE regions and the program structure tree}
Johnson, Pearson, and Pingali show how to construct a tree structure
of nested SESE regions from the cycle-equivalency information in
\cite{johnson94:pst}.  The cycle-equivalent regions are sorted by
dominance using a simple depth-first traversal of the graph, and then
canonical SESE regions are found by taking adjacent pairs of
edges from the cycle-equivalence classes.  Another depth-first search
of the CFG suffices to obtain to nesting of these regions,
which is represented in a data structure called the 
\emph{program structure tree}.
The algorithm and data structures required are presented in figures
\ref{fig:SESEdata} and \ref{fig:SESEalg}.  Figure~\ref{fig:SESEex}
shows the SESE regions on the left and program structure tree on
the right for the example of figure~\ref{fig:CQex}.%
\footnote{In addition, the regions ${c,d,e}$ and ${f,g}$ are
\emph{sequentially composed} \cite{johnson94:pst}.  Does this matter?}

The time complexity for constructing the PST is easily seen to be
$O(E)$. The algorithm presented in figure \ref{fig:SESEalg} begins
with a depth first traversal of $G$ to construct an ordered edge list
for each cycle-equivalent region; the traversal is $O(E)$ and the
list-append operation can be done in constant time.  We then iterate
through the cycle-equivalence classes and the edge lists of each
constructing SESE regions.  No edge can be on more than one list, so
this step is $O(E)$.  Finally, we do a final $O(E)$ depth-first
traversal of $G$, performing the constant-time operations {\tt append}
and {\tt LinkRegion}.  All steps are $O(E)$ and their sequential
composition is also $O(E)$.

\begin{myfigure}\small
% data types for cycle-equivalency algorithm.
Data type \texttt{EdgeList}:
\begin{tightdesc}
\item[size(el:EdgeList):~integer]
:	Number of elements in EdgeList structure
\item[head(el:EdgeList):~edge]
:	First edge in \textbf{el}
\item[tail(el:EdgeList):~EdgeList]
:	EdgeList like \textbf{el} but missing first element
\item[append(el:BracketList, e:edge):~EdgeList]
:	Add \textbf{e} to the end of \textbf{el}
\end{tightdesc}

Data type \texttt{Region}:
\begin{tightdesc}
\item[NewRegion(e1:edge, e2:edge):~Region]
:	Creates a new region with entry \textbf{e1} and exit \textbf{e2}
        and no parent
\item[Entry(r:Region):~Edge]
:	The entry edge of \textbf{r}
\item[Exit(r:Region):~Edge]
:	The exit edge of \textbf{r}
\item[Parent(r:Region):~Region]
:	The parent of \textbf{r}, or \texttt{nil} if none
\item[Nodes(r:Region):~NodeList]
:	A list of nodes in \textbf{r}
\item[LinkRegion(r1,r2:Region):~void]
:	Sets the parent of \textbf{r2} to be \textbf{r1}
\end{tightdesc}

Operations on nodes:
\begin{tightdesc}
\item[Mark(n:node):~boolean]
:	Visited status during DFS
\item[SESE(n:node):~Region]
:	The canonical SESE of \textbf{n}
\end{tightdesc}

Operations on edges:
\begin{tightdesc}
\item[EntryRegion(e:edge):~Region]
:	the region with entry \textbf{e}, or \texttt{nil} if none exists
\item[ExitRegion(e:edge):~Region]
:	the region with exit \textbf{e}, or \texttt{nil} if none exists
\end{tightdesc}

\caption{Datatypes and operations used in construction of the PST.}
\label{fig:SESEdata}
\end{myfigure}

\begin{myfigure}\small\linespread{0.75}
% algorithm for computing nested single-entry single-exit regions.
\begin{verbatim}
Procedure nested_sese (G: CFG)
{
  /* initialize */
  for all nodes n of G, do
    Mark(n) := false
  for all edges e of G, do {
    EntryRegion(e) := nil
    ExitRegion(e) := nil
  }

  /* order edges within cycle-equivalency classes by dominance */
  for each edge e of G, in depth first order, do
    CQList(EQClass(e)) := append(CQList(EQClass(e)), e)

  /* get all canonical SESE regions */
  for all equivalency classes q, do {
    l := CQList(q)
    while ( size (l) > 1), do {
      r := NewRegion( head(l), head(tail(l)) )
      EntryRegion(Entry(r)) := ExitRegion(Exit(r)) := r
    }
  }

  /* determine proper nesting of SESE regions */
  visit_node(START, top-region)
}

Procedure visit_node(n:node, r:Region) {
  if Mark(n) = false, then {
    Mark(n) := true

    /* record mapping from n to r */
    SESE(n) := r
    Nodes(r):= append(Nodes(r), n)

    for each edge <n, n'> from n to n', do {
      r1 := EntryRegion(<n, n'>)
      r2 := ExitRegion(<n, n'>)

      if r = r1 or r = r2, then
        rN := Parent(r) /* exiting current region */
      else
        rN := r

      if r1 != nil and r1 != r, then {
        LinkRegion(rN, r1) /* entering new region */
        rN := r1
      }
      if r2 != nil and r2 != r, then {
        LinkRegion(rN, r2) /* entering new region */
        rN := r2
      }
      visit_node(n', rN)
    }
  }
}
\end{verbatim}

\caption{Computing nested SESE regions and the PST.}
\label{fig:SESEalg}
\end{myfigure}

\begin{myfigure}
\centering
\vertcenter{\input{Figures/THseseex}}
\hspace{1cm}
\vertcenter{\input{Figures/THpst}}
\caption{SESE regions and PST for the CFG of
         figure~\ref{fig:CQex} (from \cite{johnson94:pst}).}
\label{fig:SESEex}
\end{myfigure}

\subsubsection{Placing \phisigfunction{s}}
\textbf{Compute variable
liveness.  Traverse the tree from bottom up, adding phi/sigmas if:
the variable is used or defined inside the PST block, and the variable
is live.}

\subsubsection{Computing liveness}
We can compute pruned SSI form if we have access to liveness
information.
We get this for free in our compiler framework.  There's also an
$O(E+N^2)$ algorithm.  I hope to do better using the properties of the
PST.  I want an $O(EV)$ algorithm, which means solving the
one-variable case in $O(E)$.  Argh.

\subsubsection{Pruning SSI form}
The SSI algorithm can be run using any conservative approximation to
the liveness information
(including the function $\mbox{\bf Live}(v)=\mbox{\tt true}$) if
unused code elimination%
\footnote{We follow \cite{wegman91:scc} in distinguishing
\emph{unreachable code elimination}, which removes code that can never
be executed, from \emph{unused code elimination}, which deletes
sections of code whose results are never used.  Both are often called
``dead code elimination'' in the literature.}  is performed to remove
the extra \phisigfunction{s} added.  Figures \ref{fig:deaddata} and
\ref{fig:deadalg} present an algorithm to identify unused code in
$O(NV)$ time, after which a simple $O(N)$ pass suffices to remove it.
The complexity analysis is simple: nodes and variables are visited at
most once, raising their value in the analysis lattive from
\emph{unused} to \emph{used}.  Nodes marked \emph{used} are never
visted.  So \texttt{MarkNodeUseful} is invoked at most $N$ times, and
\texttt{MarkVarUseful} is invoked at most $V$ times.  Each call to
\texttt{MarkNodeUseful} examines each variable used by the node; a
node may use at most every variable, taking $O(NV)$ time.  Each call
to \texttt{MarkVarUseful} examines at most one node (the single
definition node for the variable, if it exists) and in constant time
pushes at most one node on to the worklist for a total of $O(V)$ time.
So the total run time of \texttt{FindUseful} is $O(NV+V)=O(NV)$, and
in fact it is typically much less.\footnote{If the number of instruction
operands and \phisigfunction{} arities are limited by a
constant, we get a time bound of $O(N)$.}  The total runtime for SSI
placement and subsequent pruning, including the time to construct the
PST, remains $O(EV)$.

\begin{myfigure}\small
% data types for cycle-equivalency algorithm.
Operations on nodes:
\begin{tightdesc}
\item[NodeUseful(n:node):~boolean]
:	Whether the results of this node are ever used
\item[Uses(n:node):~set of variables]
:	Variables for which this node contains a use
\end{tightdesc}

Operations on variables:
\begin{tightdesc}
\item[VarUseful(v:variable):~boolean]
:	Whether there is some \textbf{n} for which 
        \textbf{Uses(n)} contains \textbf{v} and
        \textbf{NodeUseful(n)} is \texttt{true}
%:	$\exists n\in\domain{Nodes}\;.\;v\in\mbox{Uses}(n)\wedge\mbox{NodeUseful}(n)$
%:	Whether the value of this variable is ever used
\item[Definitions(v:variable):~set of nodes]
:	Nodes which contain a definition for~\textbf{v}
\end{tightdesc}

\caption{Datatypes and operations used in unused code elimination.}
\label{fig:deaddata}
\end{myfigure}

\begin{myfigure}\small\linespread{0.75}
% fast algorithm for dead code elimination (unused code elimination)
\begin{verbatim}
Procedure FindUseful (G: CFG) {
  let W be an empty work list;
  for each variable v, do
    VarUseful(v) := false;
  for each node n in G, in any order, do {
    NodeUseful(n) := false;
    if n is a CALL, RETURN, or other special node, then
      add n to W;
  }

  while W is not empty, do {
    let n be any element from W;
    remove n from W;
    MarkNodeUseful(n, W);
  }
}
Procedure MarkNodeUseful(n: node, W: WorkList) {
  NodeUseful(n) := true;
  /* everything used by a useful node is useful */
  for each v in Uses(n), do
    if not VarUseful(v), then
      MarkVarUseful(v, W);
}
Procedure MarkVarUseful(v: variable, W: WorkList) {
  VarUseful(v) := true;
  /* The definition of a useful variable is useful */
  for each n in Definitions(v), then
    /* size(Definitions(v)) <= 1 in SSI form */
    if not NodeUseful(n), then
      add n to W;
}
\end{verbatim}
\caption{Identifying unused code using SSI form.}
\label{fig:deadalg}
\end{myfigure}

\subsubsection{Discussion}
Note that our algorithm for placing \phisigfunction{s} in
SSI form is
\emph{optimistic}; that is, we at first assume every node in the
control-flow graph with input arity larger than one requires a
\phifunction{} for every variable and every node with out-arity larger
than one requires a \sigfunction{} for every variable, and then use
the PST and liveness information to determine safe places to
\emph{omit} \phisigfunction[or]s.  Most SSA construction
algorithms, by contrast, are \emph{pessimistic}; they assume no
\phisigfunction[or]s are needed and attempt to determine where
they are necessary.  In my experience, pessimistic algorithms tend to
have poor time bounds because of the possibility of input graphs like
figure~\ref{fig:evil}.  Proving that all but two nodes require
\phisigfunction[and/or]s for the variable $a$ in this example seems to
inherently
require $O(N)$ passes over the graph; the \phifunction{s} ``spread''
one node left from the circled node on each pass.  On the other hand,
an optimistic algorithm assumes the correct answer at the start, fails
to show that any \phisigfunction[or]s can be removed, and
terminates in one pass.\fixme{Are we \emph{sure} similar worst cases
don't exist for the optimistic algorithm?}

\begin{myfigure}[t]
\centering\input{Figures/evil}
\caption{A worst-case CFG for ``pessimistic'' algorithms.}
\label{fig:evil}
\end{myfigure}

\subsection{Uses and applications}
\textbf{Discuss predicated analysis, anticipatibility, PRE, etc.}%
\fixme{Write this.}

\section{An executable representation}\label{sec:ssiplus}
The Static Single Information (SSI) form, as presented in the first
half of this thesis,
requires control-flow graph information in order to be executable. We
would like to have a demand-driven operational semantics for SSI form
that does not require control-flow information; thus freeing us to
more flexibly reorder execution.

In particular, we would like a representation that eliminates
unnecessary control dependencies such as exist in the program of
figure~\ref{fig:ctrldep}.  A control-flow graph for this program, as
it is written, will explicitly specify that no assignments to
\texttt{B[]} will take place until all elements of \texttt{A[]} have
been assigned; that is, the second loop will be
\emph{control-dependent} on the first.  We would like to remove this
control dependence in order to provide greater parallelism---in this
case, to allow the assignments to \texttt{A[]} and \texttt{B[]} to
take place in parallel, if possible.\footnote{Note that Arvind's
dataflow compiler \cite{traub86:ttda} looked for exactly the opposite type of
parallelism.  By concentrating on intra-loop dependencies, as he did, you get
fine-grain parallelism suitable for VLIW-type machine with many
functional units.  We try to remove dependencies \emph{between} loops,
which gets us coarser parallelism that does not require as many
functional units to take advantage of.  This should be in the paper
body, but I'm not quite sure where a discussion of the differences
between Arvind's work and mine belongs.}

\begin{myfigure}[t]
\begin{samplecode}
for (int i=0; i<10; i++)\\
\>A[i] = x;\\
for (int j=0; j<10; j++)\\
\>B[j] = y;\\
\end{samplecode}
\caption{An example of unnecessary control dependence: the second loop
is \emph{control-dependent} on the first and so assignments to
\texttt{A[]} and \texttt{B[]} cannot take place in parallel.}
\label{fig:ctrldep}
\end{myfigure}

In addition, an executable representation allows us to apply the
techniques of abstract interpretation \cite{idunno}.  Although abstract
interpretation may be applied to the original SSI form using
information extracted from the control flow graph, as in \cite{foo}
and \cite{bar}, an executable SSI form allows more concise (and thus,
more easily derived and verified) interpretation
algorithms.\fixme{I'm still working on understanding abstract
interpretation techniques.  It may be that we can do stuff with
\ssiplus\ that is plain impossible with \ssizero\ (instead of merely
more difficult); if so, I'll mention that here, of course.
\textbf{ADDENDUM:} seems we might be able to \emph{prove} properties
of \ssiplus\ more easily than \ssizero; still working on this.}

The modifications outlined here will extend SSI form in order to
provide a useful and descriptive operational semantics.  We will call
the extended form \ssiplus.  For clarity, SSI form as originally
presented we will call \ssizero.  We will describe algorithms to
contruct \ssiplus{} efficiently,\fixme{I will make this more precise
as soon as I get the math done; construction should be $O(N)$, and space
$O(N^2)$, but $\Theta(N)$.} and illustrate analyses and
optimizations using the form.\fixme{maybe something here like: we
will also show how to apply the techniques of abstract interpretation
to \ssiplus?}

\subsection{Discussion}
\textbf{Why an executable representation is desirable.  Also,
deficiencies in \ssizero{} form.}

\textbf{A couple of paragraphs here on why an executable
representation is desirable.  Many some of the information above
should be moved down here.}

\textbf{A couple paragraphs describing deficiencies in \ssizero{} form.
Structure question: I'd like to describe the fixes \ssiplus{} proposes
right next to the bits saying what's broken in \ssizero{}; but the way
the outline is laid out currently, the fixes are described later.  I'm
going to try it the way the outline has it for now.}

\subsection{Definition}
\begin{itemize}
\item \textbf{\xifunction{s}.}
\item \textbf{Triggered constants and CALLs.}
\end{itemize}

\subsection{Semantics}\label{sec:semantics}
We will base the operational semantics of \ssiplus\ on a demand-driven
dataflow model.  We will define both a cycle-oriented semantics and an
event-driven semantics, which (incidentally) correspond to synchronous
and asynchronous hardware models.

Following the lead of Pingali \cite{pingali90:dfg}, we present Plotkin-style
semantics \cite{plotkin81:opsem} in which \emph{configurations} are
rewritten instead of programs.  The configurations represent program
state and transitions correspond to steps in program execution.  The
set of valid transitions is generated from the program text.

The semantics operate over a lifted value domain
$\domain{V}=\domain{Int}_\bot$. When some variable
$t = \bot_\domainsm{V}$ we say it is
\emph{undefined}; conversely $t\succ\bot_\domainsm{V}$ indicates that the
variable is \emph{defined}.  ``Store'' metavariables $S_x$ are not
explicitly handled by the semantics, but the extension is trivial with
an appropriate redefinition of the value domain $\domain{V}$.  Floating-point
and other types are also trivial extensions.  The
metavariables $c$ and $v$ stand for elements of $\domain{V}$.

We also define a domain of \emph{variable names},
$\domain{Nam}=\set{n_0,n_1,\ldots}$.  The metavariables $t$ and $P$ stand for
elements in $\domain{Nam}$, although $P$ will be reserved for naming branch predicates.

A fixed set of ``built-in'' operators, \textbf{op}, is defined,
of type $\domain{V}^* \to \domain{V}$.  If any operator argument is $\bot$, the
result is also $\bot$.  Constants are implemented as a special case of
the general operator rule: an \textbf{op} producing a constant has a
single input (which does not affect the output) called its \emph{trigger}.

\subsubsection{Cycle-oriented semantics}
\begin{myfigure}[t]
\begin{transitions}
t=\mathbf{op}(t_1,\ldots,t_n):
& \trule{\rho[t]=\bot \wedge 
         \left(
          \rho[t_1]\succ\bot \wedge \ldots \wedge \rho[t_n]\succ\bot
	 \right)}
	{\rho \to \rho[t \mapsto
		         \mathbf{op}(\rho[t_1],\ldots,\rho[t_n])] } \\

t=\phi(t_1,\ldots,t_n):
& \trule{\rho[t]=\bot \wedge
         \rho[t_j]\succ\bot \wedge
         \mbox{all other}\,\rho[t_1],\ldots,\rho[t_n]=\bot}
        {\rho \to \rho[t \mapsto \rho[t_j]] } \\

\tuple{t_1,\ldots,t_n}=\sigma(P,t):
& \myarray{r}{
  \trule{\rho[P]=v\succ\bot \wedge
         \rho[t_{v-1}]=\bot \wedge
         \rho[t]\succ\bot
         %\mbox{ where } (0\leq v\leq n-1)
	}
	{\rho \to \rho[t_{v-1} \mapsto \rho[t]] }
  \quad \hfill \\ \footnotesize \mbox{where } (0\leq v\leq n-1) }
 \\

\footnotesize % latex complains, but does the right thing.
\xivec{t_1,\ldots,t_n}{t_{n+1},\ldots,t_{m}}=\xi(\xivec{t'_1,\ldots,t'_n}{t'_{n+1},\ldots,t'_m}):
& \myarray{r}{
  \trule{\rho[t_j]=\bot \wedge
         \rho[t'_j]\succ\bot %\mbox{ where } (1\le j\le n)}
	}
        {\rho \to \rho[t_j \mapsto \rho[t'_j]] }
  \qquad\qquad\qquad \hfill \\ \footnotesize \mbox{where } (1\le j\le n)}
 \\

\footnotesize % latex complains, but does the right thing.
\xivec{t_1,\ldots,t_n}{t_{n+1},\ldots,t_{m}}=\xi(\xivec{t'_1,\ldots,t'_n}{t'_{n+1},\ldots,t'_m}):
& \footnotesize % latex complains, but does the right thing.
  \trule{\rho[t'_{n+1}]\succ\bot \wedge \ldots \wedge \rho[t'_m]\succ\bot}
        {\myarray{r @{} l}{
         \rho \to
%	 \hfill \\ \: % line-break
         \rho_\emptyset &
                [t_1 \mapsto \rho[t_1]]\ldots[t_n \mapsto \rho[t_n]]
		\hfill \\ & \quad % line-break
                [t_{n+1} \mapsto \rho[t'_{n+1}]]\ldots[t_m \mapsto \rho[t'_m]]
		} } \\
\end{transitions}
\caption{Cycle-oriented transition rules for \ssiplus.}
\label{fig:cyclesemantics}
\end{myfigure}

In the cycle-oriented semantics, configurations consist of an
\emph{environment}, $\rho$, which maps
names in $\domain{Nam}$ to values in $\domain{V}$.

\begin{definition}~\\*[-1\baselineskip]
\begin{enumerate}
\item An \emph{environment} $\rho: \domain{N} \to \domain{V}$ is a
finite function---its domain $\domain{N} \subseteq \domain{Nam}$ is
finite.  The notation $\rho[t\mapsto c]$ represents an environment
identical to $\rho$ except for name $t$ which is mapped to $c$.
\item The null environment $\rho_\emptyset$ maps every $t\in\domain{N}$ to
$\bot_\domainsm{V}$.
\item A \emph{configuration} consists of an environment.  The initial
configuration is $\rho_\emptyset$, that is, all names in $\domain{N}$
are mapped to $\bot_\domainsm{V}$.\fixme{START CONDITION.}
\end{enumerate}
\end{definition}

Figure~\ref{fig:cyclesemantics} shows the cycle-oriented transition
rules for \ssiplus\ form.  The left column consists of definitions and
the right column shows a precondition on top of the line, and a
transition below the line.  If the definition in the left column is
present in the \ssiplus\ form and the precondition on top of the line
is satisfied, then the transition shown below the line can be performed.

\textbf{EXPLAIN THE RATIONALE BEHIND THE RULES HERE.}

\subsubsection{Event-driven semantics}
\begin{myfigure}[t]\small
\begin{transitions}
t=\mathbf{op}(t_1,\ldots,t_n):
& \tuple{E[t_1=v_1]\ldots[t_n=v_n],S} \to \tuple{E[t=\mathbf{op}(v_1,\ldots,v_n)],S}\\

t=\phi(t_1,\ldots,t_n):
& \tuple{E[t_i=v],S} \to \tuple{E[t=v],S}\\

\tuple{t_1,\ldots,t_n}=\sigma(P,t):
& \tuple{E[t=v][P=i],S} \to \tuple{E[t_i=v],S}\\

\xivec{t_1,\ldots,t_n}{t_{n+1},\ldots,t_m}=\xi_K(\xivec{t'_1,\ldots,t'_n}{t'_{n+1},\ldots,t'_m}):
& \myarray{r}{
  \tuple{E[t'_i=v],S} \to \hfill\\\quad\quad\quad\quad\quad
  \tuple{E[t_i=v],S[K\mapsto S[K]\cup\tuple{t_i,v}]}\\
   \mbox{where }1\le i \le n\\
%  \mbox{where }K\mbox{ is a unique constant corresponding to}\\
%  \mbox{this \ssiplus\ statement}\\
  }\\

\xivec{t_1,\ldots,t_n}{t_{n+1},\ldots,t_m}=\xi_K(\xivec{t'_1,\ldots,t'_n}{t'_{n+1},\ldots,t'_m}):
& \trule{S[K]=\left\{\tuple{t_1,v_1},\ldots,\tuple{t_n,v_n}\right\}}
  {\myarray{r}{
   \tuple{E[t'_{n+1}=v_{n+1}]\ldots[t'_m=v_m],S} \to\quad\quad\quad\quad\\
   \tuple{E[t_1=v_1]\ldots[t_m=v_m],S}\\
%   \mbox{where }S[K]=\bigcup_{i=1}^n \{\tuple{t_i,v_i}\}\\
%   \mbox{where }S[K]=\left\{\tuple{t_1,v_1},\ldots,\tuple{t_n,v_n}\right\}\\
  } }
\end{transitions}
\caption{Event-driven transition rules for \ssiplus.  Note the
unfortunate synchronization in the last rule. $K$ is a
statement-identifier constant which is unique for each source \xifunction.}
\label{fig:eventsemantics}
\end{myfigure}

In the event-driven semantics, configurations consist of an
\emph{event set} and an \emph{invariant store}.  The event set
$E$ contains definitions of the form $t=c$,
and the invariant store is a mapping from numbered \xifunction{s} in
the source \ssiplus\ form to a set of tuples representing saved values
for loop invariants.

We define the following domains:
\begin{itemize}
\item $\domain{Evt} = \domain{Nam} \times \domain{V}$ is the event
domain.  An event consists of a name-value pair.  The metavariable $e$
stands for elements of $\domain{Evt}$.
\item $\domain{Xif} \subset \domain{Int}$ is used to number
\xifunction{s} in the source \ssiplus\ form.  There is some mapping
function which relates \xifunction{s} to unique elements of
$\domain{Xif}$.  The metavariable $K$ stands for an element in
$\domain{Xif}$.
\end{itemize}

A formal definition of our configuration domain is now possible:
\begin{definition}~\\*[-1\baselineskip]
\begin{enumerate}
\item An \emph{event set} $E:\domain{Evt}^*$.
The notation $E[t=c]$ represents an event set
identical to $E$ except that it contains the event $\tuple{t,c}$.  We
say a name $t$ is \emph{defined} if $\tuple{t,v} \in E$ for some $v$.
For all $\tuple{t_1,v_1},\tuple{t_2,v_2} \in E$, $t_1$ and $t_2$
differ.  This is equivalent to saying that no name $t$ is multiply
defined in an event set.  This constraint is enforced by the
transition rules, not by the definition of $E$.
\item An \emph{invariant store} $S: \domain{Xif} \to
\domain{Evt}^*$ is a finite mapping from \xifunction{s} to event sets.
\item A \emph{configuration} is a tuple
$\tuple{E, S}:\domain{Evt}^* \times (\domain{Xif}\to\domain{Evt}^*)$ consisting
of an event set and an invariant store.  The initial
configuration is
$\tuple{\{\}_{\domainsm{Evt}},
        []_{\domainsm{Xif} \to \domainsm{Evt}^*}}$
that is, it consists of an empty event set%
\footnote{Read, ``who cares about Maria'' \cite{marinov99}.}
and an empty mapping for the invariant store.%
\fixme{start condition.}
\end{enumerate}
\end{definition}

Figure~\ref{fig:eventsemantics} shows the event-driven transition
rules for \ssiplus\ form.  As before, the left column consists of
definitions and the right column shows an optional precondition above
a line, and a transition.  If the definition in the left column is
present in the \ssiplus\ form and the precondition (if any) above the
line is satisfied, then the transition can be performed.  Note that
most transitions remove some event from the event set $E$, replacing
it with a new event.  The invariant store $S$ stores
the values of loop invariants for regeneration at each loop iteration.

\textbf{MORE DESCRIPTION OF EVENT-DRIVEN SEMANTICS HERE.}

\subsection{Construction algorithms}
\textbf{Presentation of algorithms; complexity analysis.}
\footnote{Basic algorithm for generating \xifunction{s}: the top
tuple is found with a depth-first search of the source
control-flow graph, and the bottom tuple can be determined by a
traversal of the same SESE-region tree which generates \ssizero\
form.}
\footnote{Constant trigger is found via traversal of the SESE-region tree.}

\subsection{Uses and applications}
\subsubsection{Analysis and optimization.}
\subsubsection{Abstract Interpretation.}
\subsubsection{Hardware compilation.}\label{sec:hardware}
The observant reader may have noticed that the two 
operational semantics given in section \ref{sec:semantics} closely
resemble circuit implementations for the program according to
synchronous and asynchronous design methodologies.  In fact,
\ssiplus{} was designed specifically to facilitate rendering a
high-level program into hardware.  The two semantics differ primarily
on how cyclic dependencies (i.e. loops) are handled.

In this section, we will describe how to translate \ssiplus{} to
hardware, glossing over details of how the store is handled, which is
outside the scope of this work.  Upon concluding this section, it will
be obvious how to generate hardware for simple functional constructs
using \ssiplus.

\textbf{Er, at this point I will do what I just said I would do.  I'm
working on generating the figures to illustrate this section.}

\section{Results}
\textbf{Numbers, numbers, and more numbers.}

\section{Future work}
\textbf{Get a PhD.}\footnote{This involves finishing section
\ref{sec:hardware} by describing how memory is handled in hardware
compilation.  Answer: using lots and lots of sophisticated analysis,
that's how.}

\section{Conclusions}
\textbf{Scott is done.}

\newpage
\bibliography{harpoon}
\appendix
\end{document}
