% -*- latex -*- This is a LaTeX document.
% $Id: thesis.tex,v 1.10 1999-05-27 10:52:44 cananian Exp $
%%%%%%%%%%%%%%%%%%%%%%%%%%%%%%%%%%%%%%%%%%%%%%%%%%%%%%%%
\documentclass[12pt,notitlepage,twoside]{article}
%\documentclass[12pt,notitlepage,twocolumn,twoside]{article}
\usepackage{comdef}

% the name of the project
\newcommand{\magic}{\textsc{Magic}}

% double-space
\renewcommand{\baselinestretch}{1.2}

\title{Static Single Information Form}
\author{C. Scott Ananian}
\date{\today \\ $ $Revision: 1.10 $ $}

\begin{document}
\pagestyle{myheadings}\markboth{$ $Revision: 1.10 $ $}{$ $Revision: 1.10 $ $}
\bibliographystyle{plain}

\maketitle

\section{Introduction}
The selection of a compiler internal respresentation from the many
littering the academic literature may at times seem a black art.  Each
IR is accompanied by papers trumpeting its superiority for one purpose
or another, and each IR feels compelled to introduce new terms and
structures into the jargon.  A pragmatic compiler writer will often
pick randomly from the plethora available, basing the decision on ease
of implementation or understandibility without realizing the extent to
which the IR influences the structure of the compiler.  Similarly, the
theorist will derive proofs based upon intermediate representations
(and languages) that are amenable to mathematical methods, without
regard to current implementation practices.

This paper introduces yet another intermediate
representation to the literature:  Static Single Information (SSI) form.
This IR is the core of the FLEX compiler project, which is
investigating intelligent compilation techniques for distributed
systems among other things.  This thesis, in presenting the IR,
attempts to keep both the mathematician and the programmer in mind.  
SSI form has both a rigorous mathematical semantics and a factored
form which aids efficient implementation of advanced analyses.
I believe that it best straddles the gap between dataflow-oriented,
graph-structured, and control-flow driven IRs, while maintaining the
sparsity needed to achieve practical efficiency.  The construction
algorithms are linear in the size of the program.

Our discussion of the Static Single Information form will be at times
tied to the source language of the FLEX compiler, Java.  Unlike many
abstract IRs, the choices made in the design of SSI form have been
informed by the necessities of compiling a real-world imperative
language.  Java, however, has several theoretical properties that make
mathematical analysis more tractable.  In particular, we
will mention here Java's strict constraints on pointer variables.
Pointers in C can be abused in many ways that Java disallows.

Ultimately, the choice of compiler internal representation is fundamental.
Advances in IRs translate into advances in compilers.  SSI form
represents a clean and simple unification of many extant ideas, and
our hope is that it will allow the FLEX compiler to achieve a similar
integration of practical implementation and mathematical elegance.

\section{Intermediate Representations}
The first paper directly addressing an IR as a separate feature of a
compiler was Foo and Bar in \cite{foobar}.
The Static Single Assignment (SSA) form introduced by Shapiro and
Saint in \cite{shapiro70:ssa} is the foundation of most of the
work we will discuss.  Alpern, Rosen, Wegman and Zadeck reintroduced
SSA form as a tool for efficient optimization in a pair of POPL
papers \cite{alpern88:ssa,rosen88:gvn}, and three years later Cytron
and Ferrante joined Rosen, Wegman, and Zadeck in explaining how to
compute SSA form efficiently in what has since become the 
``canonical'' SSA paper \cite{cytron89:ssa}.

Despite industry adoption of SSA form in production compilers
\cite{chow96:hssa,chow97:ssapre}, academic research into alternative
representations continues.
Recent proposals have included Value Dependence Graphs
\cite{weise94:vdg}, Program Dependence Webs \cite{ballance90:pdw},
the Program Structure Tree \cite{johnson94:pst},
DJ graphs \cite{sreedhar96:dj}, and Depedence Flow Graphs
\cite{johnson93:dfg}.

In comparison to these representations, the domaniant characteristics of
our Static Single Information form may be summarized as follows:
\begin{itemize}
\item It is complete.  There exists an executable semantics
for the IR that does not require the use of information external to
the IR.
\item It is simple.  The number of added constructs is kept to a
minimum.  Many SSA-form variants spawn multitudes of new
constructs; for example, Gated SSA form \cite{ballance90:pdw,tu95:gssa}
has four different \phifunction{} variants.\footnote{$\phi$-functions,
$\mu$-functions, $\gamma$-functions, and $\eta$-functions.}
\item It is compact and efficient.  Construction should be fast, and
space should be reasonable.
\item Unnecessary control dependencies are eliminated.  Many IRs
require explicit control flow edges that serialize computation more
than is strictly necessary.
\end{itemize}

\textbf{Transition goes here: we are going to describe SSA form for
background on SSI form.}

\subsection{Static Single Assignment Form}
Static Single Assignment (SSA) form, introduced by Cytron in
\cite{cytron89:ssa}\ldots

\textbf{Define SSA form here.}

Figure \ref{fig:tossa} illustrates the conversion to SSA form.
\begin{figure}\label{fig:tossa}
\begin{center}
\input{Figures/THex1base} \vline\ \input{Figures/THex1ssa}
\end{center}
\caption{\phifunction{s} are added to the program on the left to
produce the SSA-form IR on the right.}
\end{figure}

\subsection{Minimal and pruned SSA forms}
Cytron defines a minimal SSA form in \cite{cytron91:ssa}.  His minimal
form is defined to use the smallest number of \phifunction{s} such
that the following three conditions hold:
\begin{enumerate}
\item If two nonnull paths $X \pathplus Z$ and $Y \pathplus Z$
converge at a node $Z$, and nodes $X$ and $Y$ contain assignments to
[a variable] $V$ (in the original program), then a trivial
\phifunction{} $V \leftarrow \phi(V, \ldots, V)$ has been inserted at
$Z$ (in the new program). \label{criteria1}
\item Each mention of $V$ in the original program or in an inserted
\phifunction{} has been replaced by a mention of a new variable $V_i$,
leaving the new program in SSA form.
\item Along any control flow path, consider any use of a variable $V$
(in the original program) and the corresponding use of $V_i$ (in the
new program).  Then $V$ and $V_i$ have the same value.
\end{enumerate}
Of these criteria, the first %\ref{criteria1}
is the most important.
%The other two can be summarized by noting that in proper SSA form,
%there is exactly one reaching definition for a variable $V$ at every
%non-$\phi$ use of $V$.
The SSA form in figure \ref{fig:tossa} is minimal.

A variation
on minimal SSA form, called \textit{pruned} form \cite{ferrante91:pruned},
avoids placing \phifunction{s} which define variables which are never used.
In most cases, the more regular properties of minimal SSA form
outweigh the pruned form's slight increase in space efficiency.
Figure \ref{fig:prunedssa} compares minimal and pruned SSA form for
our example program.
\begin{figure}\label{fig:prunedssa}
\begin{center}
\input{Figures/THex1ssa} \vline\ \input{Figures/THex1ssaPr}
\end{center}
\caption{Minimal SSA form on the left; pruned SSA form on the right.}
\end{figure}

\section{Static Single Information Form}

SSI form extends SSA form to achieve symmetry for both forward and
reverse dataflow.   SSI form recognizes that information about
variables is generated at branches and generates new names at these
points.  This provides us with a one-to-one mapping between variable
names and information about those variables which is independant of
control-flow graph context.  Analyses can then associate information
with variable names and propagate this information efficiently and
directly both with and against the dataflow direction.

\subsection{Definition of SSI form}
Building SSI form involves adding pseudo-assignments for a variable $V$:
\begin{enumerate}
\item[$(\phi)$] at a control-flow merge when disjoint paths from a
conditional branch come together and at least one of the paths
contains a definition of $V$; and
\item[$(\sigma)$] at locations where control-flow splits and at least
one of the disjoint paths from the split uses the value of $V$.
\end{enumerate}

Figure \ref{fig:tossi} compares the SSA and SSI forms for our example program.
% add text here about how SSI is better.
\begin{figure}\label{fig:tossi}
\begin{center}
\input{Figures/THex1ssa} \vline\ \input{Figures/THex1ssi}
\end{center}
\caption{SSA form on the left; SSI form on the right.}
\end{figure}

\subsection{Minimal and pruned SSI forms}
\textit{Minimal} and \textit{pruned} SSI forms can be defined which
parallel their SSA counterparts.  \textit{Minimal} SSI form would have the
smallest number of $\phi$- and \sigfunction{s} such that:
\begin{enumerate}
\item If two nonnull paths $X \pathplus Z$ and $Y \pathplus Z$
exist having only the node $Z$ where they converge in common,
and nodes $X$ and $Y$ contain either assignments to a variable $V$ in the
original program or a $\phi$- or \sigfunction{} for $V$ in the new program,
then a \phifunction{} for $V$ has been inserted at $Z$ in the new program.
\item If two nonnull paths $Z \pathplus X$ and $Z \pathplus Y$
exist having only the node $Z$ where they diverge in common,
and nodes $X$ and $Y$ contain either uses of a variable $V$ in the
original program or a $\phi$- or \sigfunction{} for $V$ in the new program,
then a \sigfunction{} for $V$ has been inserted at $Z$ in the new program.
\item There is exactly one reaching definition of $V$ at every
non-$\phi$ use of $V$ in the new program.
\item \textbf{FIXME: what's the equivalent condition for \sigfunction{s}?}
% has something to do with cycle equivalence.
\end{enumerate}

A \textit{pruned} SSI form would be the minimal form without any unused
definitions, that is, $\phi$ or \sigfunction{s} after which there are no
subsequent uses of any of the variables defined on the left-hand side,
except in other $\phi$ or \sigfunction{s}.\footnote{An even more
compact SSI form may be produced by removing \sigfunction{s} for which
there are uses for \textit{exactly one} of the variables on the
left-hand side, but it does not seem profitable to split this
particular hair.}

Figure \ref{fig:prunedssi} compares minimal and pruned SSI form for
our example program.
\begin{figure}\label{fig:prunedssi}
\begin{center}
\input{Figures/THex1ssi} \vline\ \input{Figures/THex1ssiPr}
\end{center}
\caption{Minimal SSI form on the left; pruned SSI form on the right.}
\end{figure}

\subsection{Theory and algorithms}

\subsection{Uses and applications}

\section{An executable representation}

\subsection{Discussion}

\subsection{Definition}

\subsection{Semantics}

\subsection{Construction algorithms}

\subsection{Uses and applications}

\section{Results}

\section{Future Work}

\section{Conclusions}

\bibliography{harpoon}
\appendix
\end{document}
