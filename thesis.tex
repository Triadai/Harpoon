% -*- latex -*- This is a LaTeX document.
% $Id: thesis.tex,v 1.120 1999-09-08 15:40:36 cananian Exp $
%%%%%%%%%%%%%%%%%%%%%%%%%%%%%%%%%%%%%%%%%%%%%%%%%%%%%%%%
\documentclass[12pt,titlepage,twoside]{article}
%\usepackage[light,first,bottomafter]{draftcopy}
% fonts
\usepackage{beton}\usepackage{euler}
%\usepackage{pdffonts}\usepackage{euler}
% useful stuff
\usepackage[section,plain]{algorithm} % algorithm environment,\listofalgorithms
\usepackage{amsthm} % proof environment
\usepackage{amstext} % the \text command for math mode (replaces \mbox)
\usepackage{varioref} % \vref command
\usepackage{graphicx} % for SCC bargraph
\usepackage{bibcontents} % MIT thesis spec.
% common definitions
\usepackage{comdef}
\newcommand{\figscale}{1.0}

%setup varioref package
\renewcommand{\reftextbefore}{on the preceding page}\vrefwarning

% Meet symbol
\newcommand{\meet}{\ensuremath{\sqcap}}
% lattice inequalities
\newcommand{\latlt}{\ensuremath{\sqsubset}}
\newcommand{\latleq}{\ensuremath{\sqsubseteq}}
% what to use for executable semantics
\newcommand{\sgt}{\sqsupset}%{\succ}

% get rid of cruft
\newcommand{\ignore}[1]{}

% number tables and figures according to section
\let\oldsection\section
\renewcommand{\section}{\setcounter{figure}{0}\setcounter{table}{0}\oldsection}
\renewcommand{\thefigure}{\thesection.\arabic{figure}}
\renewcommand{\thetable}{\thesection.\arabic{table}}

% double-space
\linespread{1.2} %one-and-a-half spacing is 1.3

\title{Static Single Information Form}
\author{C. Scott Ananian}
\date{\today \\ \ignore{$ $Revision: 1.120 $ $}}

\begin{document}
\bibliographystyle{abbrv}

%\pagestyle{empty}
%\maketitle
\newcommand{\nl}{\\[0.5\baselineskip]}
\newcommand{\tight}{\\[-0.2\baselineskip]}
\begin{titlepage}
\vspace*{0.5cm}
\begin{center}
\textbf{\LARGE The Static Single Information Form}\\
{\large by}\\
{\large C. Scott Ananian}\nl
B.S.E. Electrical Engineering\\
Princeton University, 1997
\end{center}
Submitted to the Department of Electrical Engineering and Computer Science
in partial fulfillment of the requirements for the degree of
Master of Science in Electrical Engineering and Computer Science
at the Massachusetts Institute of Technology.
\begin{center}
September 3, 1999\nl
Copyright 1999 Massachusetts Institute of Technology\\
All right reserved.
\end{center}

\vspace{0.5cm}
\begin{flushright}
Author\hrulefill\tight
\hfill Department of Electrical Engineering and Computer Science\tight
\hfill September 3, 1999\nl
Certified by\hrulefill\tight
\hfill Martin Rinard\tight
\hfill Thesis Supervisor\nl
Accepted by\hrulefill\\
\hfill Arthur C. Smith\tight
\hfill Chairman, Department Committee on Graduate Theses\nl
\end{flushright}
\end{titlepage}\setcounter{page}{2}
{% abstract page
\begin{center}
The Static Single Information Form\tight
by\tight
C. Scott Ananian\nl
Submmitted to the\tight
Department of Electrical Engineering and Computer Science\nl
September 3, 1999\nl
\end{center}

\begin{flushleft}
In partial fulfillment of the requirements for the Degree of Master of
Science in Electrical Engineering and Computer Science.
\end{flushleft}

\vspace{0.75cm}
\centerline{\LARGE\textbf{Abstract}}
\vspace{0.3cm}

The Static Single Information (SSI) form is a compiler intermediate
representation that allows efficient sparse implementations of
predicated analysis and backward dataflow algorithms.  It possesses
several attractive graph-theoretic properties which aid in program
analysis.  An extension to SSI form, \ssiplus, is also presented,
along with a complete executable abstract semantics for the
representation.  Applications to abstract interpretation and hardware
compilation are discussed.

The SSI form has been implemented on the FLEX compiler infrastructure,
and it has been used to implement several analyses and optimizations.
Details on these predicated analysis techniques are presented, as well
as data from the practical implementation.

\vspace{0.5cm}
\begin{flushleft}
Thesis Supervisor: Martin Rinard\\
Title: Professor, Laboratory for Computer Science\\
\end{flushleft}
}\clearpage
%\pagenumbering{roman}
\tableofcontents\listoffigures\listoftables\listofalgorithms\cleardoublepage
%\pagenumbering{arabic}

\section{Introduction}
\ignore{
The selection of a compiler internal representation from the many
littering the academic literature may at times seem a black art.  Each
IR is accompanied by papers trumpeting its superiority for one purpose
or another, and each IR feels compelled to introduce new terms and
structures into the jargon.  A pragmatic compiler writer will often
pick randomly from the plethora available, basing the decision on ease
of implementation or understandibility without realizing the extent to
which the IR influences the structure of the compiler.  Similarly, the
theorist will derive proofs based upon intermediate representations
(and languages) that are amenable to mathematical methods, without
regard to current implementation practices.%
\dontfixme{This paragraph will be revised to be less offensive, eventually.}

This paper introduces yet another intermediate representation to the
literature:  
}
This paper introduces a compiler intermediate representation:
Static Single Information (SSI) form.
This IR is the core of the FLEX compiler project, which is primarily
investigating intelligent compilation techniques for distributed
systems.  This thesis, in presenting the IR,
attempts to keep both the mathematician and the programmer in mind.  
SSI form has both a rigorous mathematical semantics and a factored
form which aids efficient implementation of advanced analyses.
I believe that it effectively straddles the gap between dataflow-oriented,
graph-structured, and control-flow driven IRs, while maintaining the
sparsity needed to achieve practical efficiency.  The construction
algorithms are linear in the size of the program.

Our discussion of the Static Single Information form will be at times
tied to the source language of the FLEX compiler, Java.  Unlike many
abstract IRs, the choices made in the design of SSI form have been
dictated by the necessities of compiling a real-world imperative
language.  Java, however, has several theoretical properties that make
program analysis more tractable.  In particular, we
mention here Java's strict constraints on pointer variables.
Pointers in earlier languages such as C can be abused in many ways
that Java disallows.

Ultimately, the choice of compiler internal representation is fundamental.
Advances in IRs translate into advances in compilers.  SSI form
represents a clean and simple unification of many extant ideas, and
our hope is that it will allow the FLEX compiler to achieve a similar
integration of practical implementation and mathematical elegance.

\section{Context and goals}
Strong et al.\ \cite{strong58}\footnote{Attribution by 
Aho \cite{aho88:dragon}.} first advocated the use of compiler
intermediate representations in a 1958 committee report.  Their
idealistic ``universal intermediate language'' was called UNCOL.
Thirty years later, the Static Single Assignment (SSA) form was
introduced by  Alpern, Rosen, Wegman and Zadeck
as a tool for efficient optimization in a pair of POPL
papers \cite{alpern88:ssa,rosen88:gvn}, and three years after that Cytron
and Ferrante joined Rosen, Wegman, and Zadeck in explaining how to
compute SSA form efficiently in what has since become the 
``canonical'' SSA paper \cite{cytron89:ssa}.  Johnson and Pingali
\cite{johnson93:dfg} trace the development of SSA form back to Shapiro
and Saint in \cite{shapiro70:ssa}, while Havlak \cite{havlak94:isa}
views \phifunction{s} as descendants of the ``birthpoints'' introduced
in \cite{reif81:sym}.

Despite industry adoption of SSA form in production compilers
\cite{chow97:ssapre,chow96:hssa}, academic research into alternative
representations continues.
Recent proposals have included Value Dependence Graphs
\cite{weise94:vdg}, Program Dependence Webs \cite{ballance90:pdw},
the Program Structure Tree \cite{johnson94:pst},
DJ graphs \cite{sreedhar96:dj}, and Depedence Flow Graphs
\cite{johnson93:dfg}.

In comparison to these representations, the dominant characteristics of
our Static Single Information form may be summarized as follows:
\begin{itemize}
\item It names information units.
\item It is complete.
\item It is simple.
\item It is efficient.
\item It has no explicit control dependencies.
\item It supports both forward and reverse dataflow analyses.
\end{itemize}
SSI form is used as an IR for the FLEX compiler for the Java
programming language, which informs some of these design decisions.
The FLEX compiler does deep analysis and will support
hardware/software co-design.  SSI addresses these needs, concentrating
on analysis rather than optimization.  We will address each design
point in turn.

\textbf{It names information units.}  SSA form (which we will describe
further in section \ref{sec:ssa}) assigns unique names to unique \emph{static
values} of a variable.  However, it ignores the value information
which may be added to a variable at program branch points.  SSI form
renames variable at branch points, which allows us to associate unique names
with unique \emph{information} about static values.  For example, a
program may test the value of an integer against zero before using it
as a divisor.  After the branch on the tested predicate, it is
possible to make statements about values (regarding equality or
inequality to zero) which were impossible to make previously.  SSI
form allows us to exploit this additional information.

\textbf{It is complete.}\label{sec:complete}
By this we mean that there exists an
executable semantics for the IR that does not require the use of
information external to the IR.  The original SSA form---and most
derivatives---require use of the original program control flow graph
during analysis, translation, or direct execution.  In fact,
\phifunction{s} are intimately tied with the precise input edge
structure of the control flow graph, and switch nodes (where control
flow splits) are undecipherable without referring to the control flow
graph.

In practice, this seems not a great disadvantage---it merely forces us to
maintain a mapping of SSA statements to nodes (equivalently, basic
blocks) of the original control flow graph.  But maintaining this
correspondence complicates editing the IR.  Also, it complicates the
interpretation of the program as a set of simultaneous equations,
which SSI form will allow us to do.  Finally, explicit control flow
may limit the available parallelism of the program.

\ssiplus, as it will be presented in section \ref{sec:ssiplus},
overcomes these difficulties and presents a \emph{complete}
representation of program meaning as a set of simultaneous equations,
without resort to graph information.

\textbf{It is simple.}  A bestiary of new $\phi$-like functions have
been introduced in the past decade, including
$\mu$-, $\gamma$-, and $\eta$-functions in \cite{ballance90:pdw,tu95:gssa},
$\psi$- and $\pi$-functions in \cite{lee99:parssa},
interprocedural \phifunction{s} in \cite{liao99:issa},
$\mu$- and $\chi$-functions in \cite{chow96:hssa},
$\mu$- and $\eta$-functions in \cite{gerlek95:inductssa},\footnote{Compare to
\cite{ballance90:pdw,tu95:gssa}.} and $\Lambda$-functions in \cite{lo98:ssu},
among others.%\footnote{Maybe.
%Actually I've named all the ones I know about.  Except for TGSSA,
%an early form of which apparently used $\mu$- and $\nu$-functions (mentioned
%in \cite{weise94:vdg}).}
Some of these are orthogonal to our work---the techniques of
\cite{lee99:parssa} can be used to extend SSI form to explicitly
parallel source languages, and those of \cite{chow96:hssa} to
languages with local variable aliasing (absent in Java).  Our goal is
to achieve minimal conceptual complexity in SSI form; that is, to
introduce the minimum set of $\phi$-like functions necessary to
represent the ``interesting'' properties of the compiled program.

\textbf{It is efficient.}  Construction of SSI form should be fast,
and space requirements should be reasonable.  The original SSA
algorithms required $O(E+V_{SSA}|{DF}|+NV_{SSA})$ time.%
\footnote{See section~\ref{sec:defs} for definitions of the
variables used in the complexity bounds of these two paragraphs.}
This bound
was dominated by the time and space required to construct the
dominance frontier, as $|{DF}|$, the size of the dominance frontier,
could be $O(N^2)$ for common cases.  Taking the dominant term, we
abbreviate the time complexity of the Cytron's SSA-construction
algorithm as $O(N^2 V)$.

Our algorithms do not require the construction of a dominance
frontier---building on recent work on efficient SSA construction in
this regard---and run in so-called ``linear'' time.  A more
detailed analysis will be given in section \ref{sec:ssi_complexity},
but suffice for now to say that our construction and analysis
algorithms are efficient.\footnote{Dhamdhere \cite{dhamdhere92:large} quite
correctly states that Cytron's original algorithm has a worst-case
time bound of $O(N^3)$.  This is also true for our algorithms.
However, these worst-case time bounds are not tight; we will present
experimental evidence that run times on real programs are $O(N)$.}
       
%Construction should be fast, and
%space should be reasonable.\footnote{Replace with actual numbers.}

\textbf{All explicit control dependencies are eliminated.}
Some researchers (including \cite{appel:modern} and
\cite{pingali97:apt}) view control dependence as a fundamental
property of the CFG, and \cite{ballance90:pdw,appel:modern} suggest
that accurate knowledge of control-dependence relations is the sole
key to automatic parallelization.  Often, incomplete intermediate
representations\footnote{See page \pageref{sec:complete} for our
definition of ``completeness'' in an IR.} are augmented with
control-dependence edges to express proper program semantics---see
\cite{johnson93:dfg} on DFGs and \cite{weise94:vdg} on VDGs, for
example.

Unfortunately, explicit control-flow edges tend to serialize
computation more than strictly necessary.  Figure~\vref{fig:ctrldep},
for example, contains two parallel loops which would be serialized by
the explicit control dependency between them.  Prior work often
focused on fine-grain intra-loop parallelism and ignored this coarser
inter-loop parallelism.\footnote{We discuss the dataflow-architecture
work of Traub \cite{traub86:ttda} in particular in section
\ref{sec:parallelism}.} Our objective in this work is to fully utilize
coarse parallelism by removing source-language control-dependency artifacts.

\textbf{It is efficient for both forward and backward dataflow analyses.}
It is often observed that traditional SSA form cannot handle backward
dataflow analysis.  Johnson and Pingali note this, and suggest
\emph{anticipatability} as an example of a backwards dataflow analysis
where their dependence flow graph representation betters SSA form
\cite{johnson93:dfg}. Lo et al.\ suggest the use of an ``SSU'' form to
address much the same issue \cite{lo98:ssu}.  There are in fact many
analyses where both use and definition information is utilized, and
where dataflow in both forward and reverse directions occurs.  SSI
form is able to handle both of these cases, as we demonstrate in
section \ref{sec:bidirectional}.

\section{Definitions}\label{sec:defs}
We next provide some definitions.
Our complexity metrics will usually be in terms of the following
variables:
\begin{itemize}
\item[$N$] is the number of nodes in the program control flow graph.
Each node represents either a single statement or a basic block; the
difference is unimportant for complexity metrics.
\item[$E$] is the number of edges in the program control flow graph.
For most programs $E$ is reasonably assumed to be $O(N)$, since most
nodes have either one or two successors (simple assignments and
conditional branches, respectively).  Unusual use of computed-goto and
\code{switch} statements may invalidate this assumption; but in these
cases $E$ is generally a better metric of program ``complexity'' than
$N$.  For this reason, we will case $O(E)$ ``linear in program size''.
\item[$V$] is the number of variables in the program.
%\item[$D$] is the total number of variable definitions in the program.
\item[$U$] is the total number of variable uses in the program.
\end{itemize}
As the transformations we will describe split and rename variables, we
will use subscripts to denote the number of variables, uses, or
definitions in a particular transformed version of a program.  For
example, $U_{SSA}$ is the number of uses in the SSA form (see
section~\ref{sec:ssa}) of a program.  When it is necessary to
explicitly denote a metric on the untransformed program, a zero
subscript will be used; for example, $V_0$.

Graphs will be directed unless specified otherwise.  If $X$ and $Y$
are nodes in some graph $G$, an edge from $X$ to $Y$ is written
$X \path Y$. A path \mbox{$X=s_0\path s_1\path \ldots \path s_n=Y$} is
written $X \pathplus Y$.  A \newterm{simple} path is one in which all
the nodes $s_i$ in it are distinct.

Control-flow graphs are assumed to be connected, and to contain unique
\code{START} and \code{END} nodes marking procedure entry and exit
points, respectively.  To ensure that graphs representing
infinite loops are connected, an edge will typically exist between
the \code{START} and \code{END} nodes.  
The presence of unique \code{START} and \code{END} nodes
ensures that both the dominance and post-dominance relation define
trees rooted at \code{START} and \code{END}, respectively.

For simplicity, we will assume that every node in the control-flow
graph with one successor and one predecessor contains exactly one
statement.  A node with no predecessors and a node with no successors
(\code{START} and \code{END}) are empty; they contain no statements.
Nodes with multiple successors or multiple predecessors are also empty
for conventional program representations, but may contain multiple
\phisigfunction[or]{} assignment statements in the SSA and SSI forms
we will discuss.  No node may contain both multiple predecessors and
multiple successors.

The symbol $\meet$ will be used for the dataflow ``meet'' operator.
The operator $\latleq$ is the %reflexive, transitive, and anti-symmetric
partial ordering relation for a lattice, and $x \latlt y$ iff $x
\latleq y$ and $x \not= y$.

\section{Static Single Assignment form}\label{sec:ssa}
Static Single Information (SSI) form derives many features from Static
Single Assignment (SSA) form, as described by Cytron in
\cite{cytron89:ssa}.  To provide context for
our definition of SSI form in section~\ref{sec:ssi}, we review SSA form.

\subsection{Definition of SSA form}

Static Single-Assignment form is a sparse program representation in
which each variable has exactly one definition point.  As a
consequence, only one assignment can reach each use, which means that
SSA form can be viewed as a type of sparse \newterm{def-use chain}
\cite{aho88:dragon}.

For straight-line code, the SSA transformation is straightforward:
each assignment to a variable is given a unique name (conventionally
indicated by the use of a subscripted version of the original variable
name) and each use is renamed to match its reaching definition.
Special \newterm{\phifunction{s}} must be inserted at join points to
preserve the single-assignment property.  These \phifunction{s} have
the form $v_0\gets\phi(v_1,v_2)$ and perform an assignment
according to the path by which control flow reaches the \phifunction.
Figure~\vref{fig:tossa} shows a simple program and its SSA form; the
\phifunction $Y_3\gets\phi(Y_1,Y_2)$ in the SSA version on the
right assigns $Y_3$ the value of $Y_1$ if control flow reaches it
along the false branch of the \code{if} statement.  If the true branch
is taken, $Y_3$ will get the value of $Y_2$ at the \phifunction.

\begin{myfigure}
\begin{center}
\input{Figures/THex1base} \vline\ \input{Figures/THex1ssa}
\end{center}
\caption
[A simple program and its single assignment version.]
{A simple program (left) and its single assignment version (right).
\label{fig:tossa}}
\end{myfigure}

Formally, a program is said to be in SSA form if the following three
conditions hold:
\begin{enumerate}
\item If two nonnull paths $X \pathplus Z$ and $Y \pathplus Z$
converge at a node $Z$, and nodes $X$ and $Y$ contain assignments to
[a variable] $V$ (in the original program), then a trivial
\phifunction{} $V \leftarrow \phi(V, \ldots, V)$ has been inserted at
$Z$ (in the new program).
\item Each mention of $V$ in the original program or in an inserted
\phifunction{} has been replaced by a mention of a new variable $V_i$,
leaving the new program in SSA form.
\item Along any control flow path, consider any use of a variable $V$
(in the original program) and the corresponding use of $V_i$ (in the
new program).  Then $V$ and $V_i$ have the same value.
\end{enumerate}
This formulation of this definition is due to Cytron et al.\ 
\cite{cytron91:ssa}.  Note that the definition does not prohibit
``extra'' \phifunction{s} not strictly required by condition 1.

\subsection{Minimal and pruned SSA forms}
Cytron et al.\ \cite{cytron91:ssa} defines \newterm{minimal SSA form} as
an SSA form using the smallest number of \phifunction{s} such that the
above three conditions hold.  
%The first condition is obviously the
%most important for this definition.
The SSA form in the previous example (Figure~\vref{fig:tossa}) is minimal.

A variation
on minimal SSA form, called \newterm{pruned form},
avoids placing \phifunction{s} which define variables which are never used.
The \phifunction{s} in pruned form are a subset of those in minimal
form, and as such note that pruned form does not strictly satisfy the
given SSA criteria.
In most cases, the more regular properties of minimal SSA form
outweigh the pruned form's slight increase in space efficiency.
Choi, Cytron, and Ferrante \cite{ferrante91:pruned} give a formal
definition and construction algorithm for pruned SSA.

Figure~\vref{fig:prunedssa} compares minimal and pruned SSA form for
our example program.
\begin{myfigure}
\begin{center}
\input{Figures/THex1ssa} \vline\ \input{Figures/THex1ssaPr}
\end{center}
\caption[Minimal and pruned SSA forms.]
{Minimal (left) and pruned (right) SSA forms.}
\label{fig:prunedssa}
\end{myfigure}

\section{Static Single Information form}\label{sec:ssi}

SSI form extends SSA form to achieve symmetry for both forward and
reverse dataflow.   SSI form recognizes that information about
variables is generated at branches and generates new names at these
points.  This provides us with a one-to-one mapping between variable
names and information about the variables at each point in the program.
Analyses can then associate information
with variable names and propagate this information efficiently and
directly both with and against the control-flow direction.

\subsection{Definition of SSI form}\label{sec:ssidef}
Building SSI form involves adding pseudo-assignments for a variable $V$:
\begin{enumerate}
\item[$(\phi)$] at a control-flow merge when disjoint paths from a
conditional branch come together and at least one of the paths
contains a definition of $V$; and
\item[$(\sigma)$] at locations where control-flow splits and at least
one of the disjoint paths from the split uses the value of $V$.
\end{enumerate}

Figure~\vref{fig:tossi} compares the SSA and SSI forms for 
the example of Figure~\ref{fig:tossa}.  Note that $X$ is renamed at
the conditional branch, allowing the compiler to distinguish between
$X_1$ (which is always the constant 2) from $X_2$ (which is never
equal to 2).

\begin{myfigure}
\begin{center}
\input{Figures/THex1ssa} \vline\ \input{Figures/THex1ssi}
\end{center}
\caption[A comparison of SSA and SSI forms.]
{A comparison of SSA (left) and SSI (right) forms.}
\label{fig:tossi}
\end{myfigure}

Formally, a program transformation to SSI form satisfies the following
conditions:
\begin{enumerate}
\item If two nonnull paths $X \pathplus Z$ and $Y \pathplus Z$
exist having only the node $Z$ where they converge in common,
and nodes $X$ and $Y$ contain either assignments to a variable $V$ in the
original program or a \phisigfunction[or]{} for $V$ in the new program,
then a \phifunction{} for $V$ has been inserted at $Z$ in the new program.
[Placement of \phifunction{s}.] \label{crit_phiplace}

\item If two nonnull paths $Z \pathplus X$ and $Z \pathplus Y$
exist having only the node $Z$ where they diverge in common,
and nodes $X$ and $Y$ contain either uses of a variable $V$ in the
original program or a \phisigfunction[or]{} for $V$ in the new program,
then a \sigfunction{} for $V$ has been inserted at $Z$ in the new program.
[Placement of \sigfunction{s}.] \label{crit_sigplace}

\item For every node $X$ containing a definition of a variable $V$ in
the new program and node $Y$ containing a use of that variable, there
exists at least one path $X \pathplus Y$ and no such path contains a
definition of $V$ other than at $X$. % awkward.
[Naming after \phifunction{s}.] \label{crit_phiname}

\item For every pair of nodes $X$ and $Y$ containing uses of a
variable $V$ defined at node $Z$ in the new program, either every path
$Z \pathplus X$ must contain $Y$ or every path $Z \pathplus Y$ must
contain $X$.
[Naming after \sigfunction{s}.] \label{crit_signame}

\item For the purposes of this definition, the \code{START} node
is assumed to contain a definition and the \code{END} node a use for every
variable in the original program.
[Boundary conditions.] \label{crit_boundary}

\item Along any possible control-flow path in a program being executed
consider any use of a variable $V$ in the original program and the
corresponding use of $V_i$ in the new program.  Then, at every
occurance of the use on the path, $V$ and $V_i$ have the same value.
The path need not be cycle-free.
[Correctness.] \label{crit_correct}
\end{enumerate}

As with the SSA conditions, this definition does not prohibit
``extra'' \phisigfunction[or]{s} not required by conditions
\ref{crit_phiplace} and \ref{crit_sigplace}.

\begin{property}
There exists exactly one reaching definition of $V$ at every 
non-\phifunction{} use of $V$ in the new program.
\end{property}
\begin{proof}
Offner \cite{offner95} defines a reaching definition as follows:
\begin{quote}
A definition of a variable $v$ \newterm{reaches} the point $P$ in the
program iff there is a path from the definition to $P$ on which\ldots
there is no other definition of $v$\ldots.
\end{quote}
From this definition and condition \ref{crit_phiname} we directly
obtain the property.
\end{proof}

Note that condition \ref{crit_phiname} and this property do not require
there to be exactly one definition of any variable $V$, just that at
every use only a single definition is relevant.  The renaming
algorithm we will present enforces the stricter single-definition
constraint.

\begin{property}\label{pty:ssi_dom}
Every cycle-free path $S \pathplus Y$ from the \code{START} node to a node $Y$
containing a non-\phifunction{} use of a variable must contain exactly
one node $X$ defining that variable in the new program.  Likewise,
every path $X \pathplus E$ from a node $X$ containing a
non-\sigfunction{} definition of a variable to the \code{END} node
must contain every node $Y$ which is a use of that variable in the new
program.
\end{property}
\begin{proof}
Let us call the variable $v$.  Conditions~\ref{crit_boundary} and
\ref{crit_correct} ensure that there exists at least one definition
node $X$ for $v$ from which $Y$ is
reachable---conditions~\ref{crit_boundary} and \ref{crit_correct}
substitute the \code{START} node, from which every node is reachable,
for any use of $v$ not reachable by some other definition in the
original program.  So assume this definition node $X$ exists, but is
not on the path $S\pathplus Y$.  Then $X\pathplus Y$ and $S\pathplus
Y$ must have some earliest node $N$ in common.  But $N$ must then have
a \phifunction{} for $v$ by condition~\ref{crit_phiplace}, which
violates either our choice of $Y$ as a non-\phifunction{} use (if
$N=Y$) or else condition~\ref{crit_phiname} which prohibits
definitions other than at $X$.  If $S\pathplus Y$ contains more than
one node $X_i$ defining $v$, then the path $X_0\pathplus Y$ between
the first and $Y$ also violates condition~\ref{crit_phiname}.  So
$S\pathplus Y$ must contain exactly one definition $X$ of $v$.

The second part is symmetric.  Assume there exists some node $Y$ using
$v$ which is not contained on some path $X \pathplus E$.  The path
$X\pathplus Y$ must exist by conditions~\ref{crit_phiname} and
\ref{crit_boundary}.  And $X\pathplus E$ and $X\pathplus Y$ must
have some final node $N$ in common, which must have a \sigfunction{}
for $v$ by condition~\ref{crit_sigplace}. The case $N=X$ violates the
choice of $X$ as a non-\sigfunction{} definition.  But if $N\not= X$,
then condition~\ref{crit_phiname}, which prohibits paths with multiple
definitions, is violated.  Thus $X\pathplus E$ must contain every use
of $v$.
\end{proof}

\begin{property}
Every definition of a variable $V$ dominates all non-\phifunction{}
uses of $V$ and
every use of $V$ post-dominates any non-\sigfunction{}
reaching definition of $V$ in the
new program.
\end{property}
\begin{proof}
The dominance relation is defined in Offner \cite{offner95} as:
\begin{quote}
If $x$ and $y$ are two elements in a flow graph $G$, then $x$
\newterm{dominates} $y$ ($x$ is a \newterm{dominator} of $y$) iff
every path from $s$ [\code{START}] to $y$ includes $x$.
\end{quote}
Post-dominance is the dual on a flow graph with edges reversed: $x$
post-dominates $y$ iff every path from \code{END} to $y$ includes $x$.

The previous property showed that every path from \code{START} to a
non-\phifunction{} use contained a unique definition node $X$.  If two
paths from \code{START} to $Y$ contained different definition nodes
$X_i$, then $Y$ would be a \phifunction, which it was chosen not to
be.  So every non-\phifunction{} use is dominated by the single
definition node.  Likewise the previous property showed that every
path from a non-\sigfunction{} definition to \code{END} must include
every use; therefore every use post-dominates a non-\sigfunction{}
definition.
\end{proof}

\subsection{Minimal and pruned SSI forms}
\emph{Minimal} and \emph{pruned} SSI forms can be defined which
parallel their SSA counterparts.  \emph{Minimal} SSI form would have
the smallest number of \phisigfunction{s} such that the above
conditions are satisfied.  \emph{Pruned} SSI form is the minimal form
with any unused \phisigfunction{s} deleted; that is, it contains no
\phisigfunction[or]{s} after which there are no subsequent
non-\phisigfunction[or]{} uses of any of the variables defined on the
left-hand side.\footnote{An even more compact SSI form may be produced
by removing \sigfunction{s} for which there are uses for \emph{exactly
one} of the variables on the left-hand side, but by doing so one loses
the ability to perform renaming at control-flow splits which generate
additional value information.}  Figure~\vref{fig:prunedssi} compares 
minimal and pruned SSI form for our example program.
\begin{myfigure}
\begin{center}
\input{Figures/THex1ssi} \vline\ \input{Figures/THex1ssiPr}
\end{center}
\caption[Minimal and pruned SSI forms.]
{Minimal (left) and pruned (right) SSI forms.}
\label{fig:prunedssi}
\end{myfigure}

Note that, as in SSA form, pruned SSI
does not strictly satisfy the SSI constraints because it omits dead
\phisigfunction{s} otherwise required by conditions \ref{crit_phiplace} and
\ref{crit_sigplace} of the definition.  In practice, a subtractive
definition of pruned form --- generate minimal form and then removed
the unused \phisigfunction{s} --- is most useful, but a constructive
definition can be generated from the standard SSI form definition as
follows:
\begin{enumerate}
\item The convergence/divergence node $Z$ of conditions
\ref{crit_phiplace} and \ref{crit_sigplace} must also satisfy: ``and
there exists a path from $Z\pathplus U$ to a $U$, a use of $V$ in the
original program, which does not contain another definition of $V$.''
  \label{amend_place}
\item The boundary condition \ref{crit_boundary} at \code{END} can be
loosened as follows (emphasis indicates modifications):
``For the purposes of this definition, the \code{START} node is
assumed to contain a definition for every variable in the original
program and the \code{END} nodes a use \emph{for every variable live
at \code{END}} in the original program.''
  \label{amend_boundary}
\end{enumerate}

Pruned form is defined as having the minimal set of \phisigfunction{s}
that satisfy the amended conditions.  It can easily be verified that
the modifications suffice to eliminate unused \phisigfunction{s}: if
the variable defined in a \phisigfunction[or]{} is used, there must
exist a path $Z\pathplus U$ as mandated by amendment~\ref{amend_place},
where amendment~\ref{amend_boundary} lets $U = \code{END}$ for
variables live exiting the procedure and thus usefully defined.

\begin{property}\label{pty:pruned_live}
A node $Z$ gets a \phisigfunction[or]{} for some variable $V_i$ in
pruned SSI form only if the corresponding variable $V$ is live at $Z$
in the original program.
\end{property}
\begin{proof}
This is a trivial restatement of amendment~\ref{amend_place}.  A
variable $v$ is said to be live at some node $N$ if there exists
a node $U$ using $v$ and a path $N\pathplus U$ on which no definitions
of $v$ are to be found.  If $V$ is not live at $Z$ then no path
$Z\pathplus U$ satisfying the amended conditions~\ref{crit_phiplace}
and \ref{crit_sigplace} can be found and neither a
\phisigfunction[or]{} can be placed.  Amendment~\ref{amend_boundary}
ensures this holds true at boundaries.
\end{proof}

\subsection{Fast construction of SSI form}
The most common construction algorithm for SSA form
\cite{cytron91:ssa} uses dominance frontiers and suffers from a
possible quadratic blow-up in the size of the dominance frontier for
certain common programming constructs.  Various improved algorithms
use such things as DJ graphs \cite{sreedhar95:lintime} and the
dependence flow graph \cite{johnson93:dfg} to achieve $O(EV)$ time
complexity for \phifunction placement.  We build on this work to
achieve $O(EV)$ construction of SSI form, and present a new algorithm
for variable renaming in SSI form after \phisigfunction{s} are placed.

Our construction algorithm begins with a program structure tree of
single-entry single-exit (SESE) regions, constructed as described by
Johnson, Pearson, and Pingali \cite{johnson94:pst}.  We will review
the algorithms involved, as their published descriptions
\cite{johnson93:sese} contain a number of errors.

We begin with a few definitions from \cite{johnson94:pst}.
\begin{definition}
Edges $a$ and $b$ are said to be \newterm{edge cycle-equivalent} in a
graph iff every cycle containing $a$ contains $b$, and vice-versa.
Similarly, two nodes are said to be \newterm{node cycle-equivalent} iff
every cycle containing one of the nodes also contains the other.
\end{definition}
\begin{definition}
A \newterm{SESE region} in a graph $G$ is an ordered edge pair
$\tuple{a,b}$ of distinct control flow edges $a$ and $b$ where
\begin{tightenum}
\item $a$ dominates $b$,
\item $b$ postdominates $a$, and
\item every cycle containing $a$ also contains $b$ and vice-versa.
\end{tightenum}
Edges $a$ and $b$ are called the \newterm{entry} and \newterm{exit} edges,
respectively.
\end{definition}
\begin{definition}
A SESE region $\tuple{a,b}$ is \newterm{canonical} provided
\begin{tightenum}
\item $b$ dominates $b'$ for any SESE region $\tuple{a,b'}$, and
\item $a$ postdominates $a'$ for any SESE region $\tuple{a',b}$.
\end{tightenum}
\end{definition}

We will give time bounds in terms of $N$ and $E$, the number of nodes
and edges of the control-flow graph, respectively.
Placement of \phisigfunction{s}
is also dependent on $V$, the number of variables in the program.
Since SSI renaming increases the number of variables, we will use
$V_0$ and $V_{SSI}$ to indicate the number of variables in the
original program and SSI form, respectively.

Note that $V$ is $O(N)$ at most, since our representation only allows
a constant number of variable definitions per node.  Typically $V_0$
will be much smaller than $N$, but $V_{SSI}$ need not be.  Also $E$
may be as large as $O(N^2)$, but in most control-flow graphs is $O(N)$
instead, as node arities are typically limited by a constant.

\subsubsection{Cycle-equivalency}
\newcommand{\cyceq}{\equiv_{cq}}%{\equiv_{CYC}}
The identification of SESE regions begins by computing the
cycle-equivalency of the edges in the program control flow graph.  The
cycle-equivalency algorithm works on undirected graphs, so we prepare the
directed control flow graph $G$ as follows:
\begin{enumerate}
\item \textbf{Add an edge from \code{END} to \code{START} in
$\mathbf{G}$.} It is common practice to add an edge from \code{START}
to \code{END} in order to root the control dependence graph at
\code{START} \cite{cytron89:ssa}.  However, our goal is not rooted
control dependence but to make the control flow graph into a single
strongly connected component; for this reason the direction of the
edge is from \code{END} to \code{START} instead.
\item \textbf{Create an equivalent undirected graph.}  Johnson et al.\
prove that the node expansion illustrated in Figure~\vref{fig:CQundir}
results in an undirected graph with the same cycle-equivalency
properties as the original directed graph.  More precisely, nodes $a$
and $b$ in directed graph $G$ are cycle-equivalent if and only if
nodes $a'$ and $b'$ are cycle-equivalent in transformed undirected
graph $G'$.  The nodes $n_i$ and $n_o$ generated by the expansion are
termed \emph{not representative}; the node $n'$ in $G'$ is said to be
\emph{representative} of node $n$ in $G$.  Obviously, this
correspondence must be recorded during the transformation so we may
properly attribute the cycle-equivalency properties of $n'$ to $n$
later.
\begin{myfigure}
\begin{center}
\renewcommand{\figscale}{0.5}\input{Figures/THundir}
\end{center}
\caption{Transformation from directed to undirected graph
	 (from \cite{johnson93:sese}).}
\label{fig:CQundir}
\end{myfigure}
\item \textbf{Perform a pre-order numbering of nodes in $\mathbf{G'}$.}
This is done with a simple depth-first search of $G'$.  When we visit
a node $a_i$ or $a_o$, we prefer to visit $a'$ before any other
neighbor.  This ensures that representative nodes are interior nodes
in the DFS spanning tree. The \code{START} node is numbered 0, and succeeding
nodes in the traversal get increasing numbers.  Thus low-numbered
nodes are closest to \code{START} and we will call them ``highest'' in the
DFS spanning tree.
\end{enumerate}

\begin{myfigure}\small% data types for cycle-equivalency algorithm.
Data type \texttt{BracketList}:
\begin{tightdesc}
\item[create():~BracketList]
:	Make an empty BracketList structure
\item[size(bl:BracketList):~integer]
:	Number of elements in BracketList structure
\item[push(bl:BracketList,~e:bracket):~BracketList]
:	Push \textbf{e} on top of \textbf{bl}
\item[top(bl:BracketList):~bracket]
:	Topmost bracket in \textbf{bl}
\item[delete(bl:BracketList,~e:bracket):~BracketList]
:	Delete \textbf{e} from \textbf{bl}
\item[concat(bl1,bl2:BracketList):~BracketList]
:	Concatenate \textbf{bl1} and \textbf{bl2}
\end{tightdesc}

Operations on nodes:
\begin{tightdesc}
\item[Number(n:node):~integer]
:	DFS preorder number of node
\item[NQClass(n:node):~integer]
:	Cycle-equivalency class of node
\item[BList(n:node):~BracketList]
:	List of brackets of node
\item[Hi(n:node):~integer]
:	Highest destination node of any edge originating from a
	descendant of node \textbf{n}
\end{tightdesc}

Operations on edges:
\begin{tightdesc}
\item[EQClass(e:node):~integer]
:	Cycle-equivalency class of edge
\item[RecentSize(e:edge):~integer]
:	Size of bracket set when \textbf{e} was most recently the
	topmost bracket for a representative node
\item[RecentClass(e:edge):~integer]
:	Cycle-equivalency class number of representative node for
	which \textbf{e} was most recently the topmost bracket.
\end{tightdesc}

\caption{Datatypes and operations for the cycle-equivalency algorithm.}
\label{fig:CQdata}\end{myfigure}

\begin{myalgorithm}\small\linespread{0.75}% algorithm for cycle-equivalency on nodes and edges
\begin{verbatim}
Procedure cycle_equiv (G: CFG)
{
  /* Preprocessing */
  G' := Preprocess (G); /* described in text */

  /* Compute CD equivalence classes */
  for each node n of G', in reverse depth-first order, do {
    /* Compute Hi(n) */
    /*  hi0 is highest using backedges only */
    hi0 := min{ Number(t) | (t,n) is a backedge };
    /*  hi1 is highest through children */
    hi1 := min{ Hi(c) | c is a child of n };
|   /*  hi2 is lowest through children */
|   hi2 := max{ Hi(c) | c is a child of n };

    Hi(n) := min{ hi0, hi1 };

    /* Compute BList(n) */
    BList(n) := create ();

    for each child c of n, do
      BList(n) := concat (BList(n), BList(c));

    for each backedge <d, n> from a descendant d of n to n, do
      BList(n) := delete (BList(n), <d, n>);
|   for each capping backedge <d, n> of n, do
|     BList(n) := delete (BList(n), <d, n>);

    for each backedge <n, a> from n to an ancestor a of n, do {
      BList(n) := push (BList(n), <n, a>)
      RecentSize(<n, a>) := -1; /* not a representative node */
    }

    if n has more than one child, then {
      BList(n) := push (BList(n), <n, hi2>); /* capping backedge */
|     RecentSize(<n, hi2>) := -1;
|     add <n, hi2> to capping backedges list of hi2;
    }

    /* Compute Class (n) */
    if n is a representative node, then {
      if RecentSize (top (BList(n))) != size (BList(n)), then {
        /* start a new equivalence class */
        RecentSize (top (BList(n))) := size (BList(n));
        RecentClass (top (BList(n))) := new-class-name();
      }
      Class (n) := RecentClass (top (Blist(n)));
    }
  } /* for each node */
}
\end{verbatim}

\caption{The cycle-equivalency algorithm
	 (corrected from \cite{johnson93:sese}).}
\label{alg:CQalg}\end{myalgorithm}

The above steps form an undirected graph $G'$ from the control-flow
graph $G$.  The remainder of the cycle-equivalency algorithm is
presented as Algorithm~\vref{alg:CQalg}, with the above procedure
corresponding to the statement \code{G':=Preprocess(G)}.  The
algorithm has been corrected from the published version in
\cite{johnson93:sese}; in addition it has been extended to compute
both node and edge equivalencies (in effect, merging the algorithm of
\cite{johnson94:pst}).  Lines modified from the presentation in
\cite{johnson93:sese} are indicated in the figure with a vertical bar
in the left margin.  The datatype \code{BracketList} and the node
and edge properties used in the algorithm are described in
Figure~\vref{fig:CQdata}.  The interested reader is encouraged to consult
\cite{johnson93:sese} for additional detail on these data structures
and representations.%
\dontfixme{Um, change to compute \emph{edge}
equivalency as well as \emph{node} equivalency.  Merge algorithms from
\cite{johnson93:sese} and \cite{johnson94:pst}.}
Figure~\vref{fig:CQex} shows cycle-equivalent regions in a simple
control-flow graph.  We use the notation
$\tuple{a,b}\cyceq\tuple{c,d}$ to indicate that the CFG edge from node
$a$ to node $b$ is edge cycle-equivalent to the edge from node $c$ to
node $d$.

\begin{myfigure}\centering
\vertcenter{\input{Figures/THcqex2}}
$\begin{array}[c]{cc}
\tuple{\code{START},1}\cyceq\tuple{16,\code{END}}\\
\tuple{1,2}\cyceq\tuple{8,16}\\
\tuple{2,3}\cyceq\tuple{3,4}\cyceq\tuple{7,8}\\
\tuple{4,5}\cyceq\tuple{5,7}\\
\tuple{4,6}\cyceq\tuple{6,7}\\
\tuple{1,9}\cyceq\tuple{9,10}\cyceq\tuple{14,15}\cyceq\tuple{15,16}\\
\tuple{10,11}\cyceq\tuple{11,13}\\
\end{array}$
\caption{Control flow graph and cycle-equivalent edges.}
\label{fig:CQex}\end{myfigure}

Calculating cycle-equivalent regions is based on a single reverse
depth-first traversal of $G$, so as long as all datatype operations in
Figure~\ref{fig:CQdata} can be completed in constant time (and
\cite{johnson93:sese} shows how to do so), this computation is $O(E)$.

\subsubsection{SESE regions and the program structure tree}
\begin{myfigure}\small% data types for cycle-equivalency algorithm.
Data type \texttt{EdgeList}:
\begin{tightdesc}
\item[size(el:EdgeList):~integer]
:	Number of elements in EdgeList structure
\item[head(el:EdgeList):~edge]
:	First edge in \textbf{el}
\item[tail(el:EdgeList):~EdgeList]
:	EdgeList like \textbf{el} but missing first element
\item[append(el:BracketList, e:edge):~EdgeList]
:	Add \textbf{e} to the end of \textbf{el}
\end{tightdesc}

Data type \texttt{Region}:
\begin{tightdesc}
\item[NewRegion(e1:edge, e2:edge):~Region]
:	Creates a new region with entry \textbf{e1} and exit \textbf{e2}
        and no parent
\item[Entry(r:Region):~Edge]
:	The entry edge of \textbf{r}
\item[Exit(r:Region):~Edge]
:	The exit edge of \textbf{r}
\item[Parent(r:Region):~Region]
:	The parent of \textbf{r}, or \texttt{nil} if none
\item[Nodes(r:Region):~NodeList]
:	A list of nodes in \textbf{r}
\item[LinkRegion(r1,r2:Region):~void]
:	Sets the parent of \textbf{r2} to be \textbf{r1}
\end{tightdesc}

Operations on nodes:
\begin{tightdesc}
\item[Mark(n:node):~boolean]
:	Visited status during DFS
\item[SESE(n:node):~Region]
:	The canonical SESE of \textbf{n}
\end{tightdesc}

Operations on edges:
\begin{tightdesc}
\item[EntryRegion(e:edge):~Region]
:	the region with entry \textbf{e}, or \texttt{nil} if none exists
\item[ExitRegion(e:edge):~Region]
:	the region with exit \textbf{e}, or \texttt{nil} if none exists
\end{tightdesc}

\caption{Datatypes and operations used in construction of the PST.}
\label{fig:SESEdata}\end{myfigure}
\begin{myalgorithm}\small% algorithm for computing nested single-entry single-exit regions.
\begin{verbatim}
Procedure nested_sese (G: CFG)
{
  /* initialize */
  for all nodes n of G, do
    Mark(n) := false
  for all edges e of G, do {
    EntryRegion(e) := nil
    ExitRegion(e) := nil
  }

  /* order edges within cycle-equivalency classes by dominance */
  for each edge e of G, in depth first order, do
    CQList(EQClass(e)) := append(CQList(EQClass(e)), e)

  /* get all canonical SESE regions */
  for all equivalency classes q, do {
    l := CQList(q)
    while ( size (l) > 1), do {
      r := NewRegion( head(l), head(tail(l)) )
      EntryRegion(Entry(r)) := ExitRegion(Exit(r)) := r
    }
  }

  /* determine proper nesting of SESE regions */
  visit_node(START, top-region)
}

Procedure visit_node(n:node, r:Region) {
  if Mark(n) = false, then {
    Mark(n) := true

    /* record mapping from n to r */
    SESE(n) := r
    Nodes(r):= append(Nodes(r), n)

    for each edge <n, n'> from n to n', do {
      r1 := EntryRegion(<n, n'>)
      r2 := ExitRegion(<n, n'>)

      if r = r1 or r = r2, then
        rN := Parent(r) /* exiting current region */
      else
        rN := r

      if r1 != nil and r1 != r, then {
        LinkRegion(rN, r1) /* entering new region */
        rN := r1
      }
      if r2 != nil and r2 != r, then {
        LinkRegion(rN, r2) /* entering new region */
        rN := r2
      }
      visit_node(n', rN)
    }
  }
}
\end{verbatim}

\caption{Computing nested SESE regions and the PST.}
\label{alg:SESEalg}\end{myalgorithm}
\begin{myfigure}\centering
\vertcenter{\renewcommand{\figscale}{0.5}\input{Figures/THcqex}}\hspace{1cm}
\vertcenter{\renewcommand{\figscale}{0.7}\input{Figures/THpst}}
\caption{SESE regions and PST for the CFG of
         Figure~\ref{fig:CQex} (from \cite{johnson94:pst}).}
\label{fig:SESEex}\end{myfigure}

Johnson, Pearson, and Pingali show how to construct a tree structure
of nested SESE regions from the cycle-equivalency information in
\cite{johnson94:pst}.  The cycle-equivalent regions are sorted by
dominance using a simple depth-first traversal of the graph, and then
canonical SESE regions are found by taking adjacent pairs of
edges from the cycle-equivalence classes.  Another depth-first search
of the CFG suffices to obtain to nesting of these regions,
which is represented in a data structure called the 
\emph{program structure tree}.
The algorithm and data structures required are presented in Figure~%
\ref{fig:SESEdata} and Algorithm~\ref{alg:SESEalg}.  Figure~\vref{fig:SESEex}
shows the SESE regions on the left and program structure tree on
the right for the example of Figure~\vref{fig:CQex}.%
\footnote{In addition, the regions ${c,d,e}$ and ${f,g}$ are
\emph{sequentially composed} \cite{johnson94:pst}. 
However, our SSI construction algorithm doesn't use this property.}

The time complexity for constructing the PST is easily seen to be
$O(E)$. Algorithm~\vref{alg:SESEalg} begins
with a depth first traversal of $G$ to construct an ordered edge list
for each cycle-equivalent region; the traversal is $O(E)$ and the
list-append operation can be done in constant time.  We then iterate
through the cycle-equivalence classes and the edge lists of each
constructing SESE regions.  No edge can be on more than one list, so
this step is $O(E)$.  Finally, we do a final $O(E)$ depth-first
traversal of $G$, performing the constant-time operations {\tt append}
and {\tt LinkRegion}.  All steps are $O(E)$ and their sequential
composition is also $O(E)$.

\subsubsection{Placing \phisigfunction{s}}\label{sec:SSIplace}
\begin{myalgorithm}\small
% algorithm for phi/sigma placement.

Place($G$:~CFG) =
\begin{myalgorithmic}
\LET $r$ be the top-level region for $G$
\FOREACH{variable $v$ in $G$}
\STATE PlaceOne($r$, $v$, \code{false}) \COMMENT{place phis}
\STATE PlaceOne($r$, $v$, \code{true})  \COMMENT{place sigmas}
\ENDFOR
\end{myalgorithmic}
~\\
PlaceOne($r$:~region, $v$:~variable, $ps$:~boolean):~boolean =
\begin{myalgorithmic}
\STATE \COMMENT{Post-order traversal}
\STATE flag $\gets$ false
\FOREACH{child region $r'$}
  \IF{PlaceOne($r'$, $v$, $ps$)}
    \STATE flag $\gets$ true
   \ENDIF 
\ENDFOR
                             \EMPTYLINE
\FOREACH{node $n$ in region $r$ not contained in a child region}
  \IF{$ps$ is \code{false} and $n$ contains a definition of $v$}
    \STATE flag $\gets$ true
  \ENDIF
  \IF{$ps$ is \code{true} and $n$ contains a use of $v$}
    \STATE flag $\gets$ true
  \ENDIF
\ENDFOR
                             \EMPTYLINE
\STATE \COMMENT{add phis/sigmas to merges/splits where $v$ may be live}
\IF{flag = true}
  \FOREACH{node $n$ in region $r$ not contained in a child region}
    \IF{MaybeLive($n$) = true}
      \IF{$ps$ is \code{false} and the input arity of $n$ exceeds 1}
        \STATE place a phi function for $v$ at $n$
      \ENDIF
      \IF{$ps$ is \code{true} and the output arity of $n$ exceeds 1}         
        \STATE place a sigma function for $v$ at $n$
      \ENDIF
    \ENDIF
  \ENDFOR
\ENDIF
                             \EMPTYLINE
\RETURN flag
\end{myalgorithmic}

\caption{Placing \phisigfunction{s}.}\label{alg:SSIplace}
\end{myalgorithm}

As with the presentation of SSA form in \cite{cytron91:ssa}, we split
construction of SSI form into two parts: placing \phisigfunction{s}
and renaming variables.  The placement algorithm runs in $O(N V_0)$
time, and is presented as Algorithm~\vref{alg:SSIplace}.  No new node
properties or datatypes are required; however, it is parameterized on
a function called \code{MaybeLive}.  For minimal SSI form,
\code{MaybeLive} should always return \code{true}.  Faster practical
run-time may be obtained if pruned SSI form is the desired goal by
allowing \code{MaybeLive} to return any conservative approximation of
variable liveness information, which will allow early suppression of
unused \phisigfunction{s}.  Note that \code{MaybeLive} need not be
precise; conservative values will only result in an excess of
\phisigfunction{s}, not an invalid SSI form.  Section
\ref{sec:unusedcode} describes a post-processing algorithm to
efficiently remove the excess \phisigfunction{s}.%
\footnote{Note that equivalent results could be
obtained by adding a \phifunction{} for every variable at every merge
and a \sigfunction{} for every variable at every split, and
post-processing.  In fact the same time bounds ($O(N V_0)$) would be
obtained.  There is a large practical difference in actual runtime and
space costs, however, which motivates our more efficient approach.}
The remainder of this section will be devoted to a correctness proof
of Algorithm~\ref{alg:SSIplace}.

\begin{lemma}\label{lem:sese_child}
No \phifunction{s} (\sigfunction{s}) for a variable $v$ are needed in
an SESE region not containing a definition (use) of $v$.
\end{lemma}
\begin{proof}
Let us assume a \phifunction for $v$ is needed at some node $Z$
inside an SESE not containing a definition of $v$.  
Then by condition \ref{crit_phiplace} of the SSI
form definition, there exist paths $X \pathplus Z$ and $Y \pathplus Z$
having no nodes but $Z$ in common where $X$ and $Y$ contain either
definitions of $v$ or \phisigfunction[or]{s} for $v$.  Choose any such
paths:
\begin{description}
\item[Case I:] Both $X$ and $Y$ are outside the SESE.  Then, as there
is only one entrance edge into the SESE, the paths $X \pathplus Z$ and
$Y \pathplus Z$ must contain some node in common other than Z.  But
this contradicts our choice of $X$ and $Y$.
\item[Case II:] At least one of $X$ and $Y$ must be inside the SESE.
If both $X$ and $Y$ are not definitions of $v$ but rather
\phisigfunction[or]{s} for $v$, then by recursive application of this proof
there must exist some choice of $X$, $Y$, and $Z$ inside this SESE
where at least one of $X$ and $Y$ is a definition.  But $X$ or $Y$
cannot be a definition of $v$ because they are inside the SESE of $Z$ which
was chosen to contain no definitions of $v$.
\end{description}

A symmetric argument holds for \sigfunction{s} for $v$, using
condition \ref{crit_sigplace} of the SSI form definition, and the fact
that there exists one exit edge from the SESE.
\end{proof}

The above lemma justifies line~\ref{line:place_skip} of the algorithm
on page~\pageref{line:place_skip}, which skips over any SESE region
not containing a definition (use) of $v$ when placing
\phifunction{s} (\sigfunction{s}) for $v$.

\begin{lemma}\label{lem:ssi_place_dom}
If a definition (use) or a \phisigfunction[or]{} for a variable $v$ is
present at some node $D$ (\/$U$), then a \phifunction (\sigfunction) for
$v$ is needed at every node $N$:
\begin{enumerate}
\item of input (output) arity greater than 1,
\item reachable from $D$ (from which $U$ is reachable),
\item whose smallest enclosing SESE contains $D$ (\/$U$), and
\item which is not dominated by $D$ (not post-dominated by $U$).
\end{enumerate}
\end{lemma}
\begin{proof}
We will first prove that a node $N$ failing any one of the conditions does
not need a \phisigfunction[or].
\begin{itemize}
\item Conditions \ref{crit_phiplace} and \ref{crit_sigplace} of the
SSI form definition require node $N$ to be the first convergence
(divergence) of some paths $X \pathplus N$ and $Y \pathplus N$ ($N
\pathplus X$ and $N \pathplus Y$).  If the input arity is less than 2
or there is no path from a definition of $v$, than it fails the
$\phi$-placement criterion \ref{crit_phiplace}.  If the output arity
is less than 2 or there is no path to a use of $v$, then it fails the
$\sigma$-placement criterion \ref{crit_sigplace}.
\item If there exists a SESE containing $N$ that does not contain any
definition, \phisigfunction[or]{} $D$ for $v$, then $N$ does not require a
\phisigfunction[or]{} for $v$ by lemma~\ref{lem:sese_child}.
\item Let us suppose every $D_i$ containing a definition,
\phisigfunction[or]{} for $v$ dominates $N$.  If $N$ requires a
\phifunction for $v$,  there exist paths $D_1 \pathplus N$ and
$D_2 \pathplus N$ containing no nodes in common but $N$.  We use these
paths to construct simple paths $\code{START}\pathplus D_1 \pathplus N$ and
$\code{START}\pathplus D_2 \pathplus N$.  By the definition of a
dominator, every path from \code{START} to $N$ must contain every
$D_i$.  But $D_1 \pathplus N$ cannot contain $D_2$, and if
$\code{START} \pathplus D_1$ contains $D_2$, we can make a path
$\code{START} \pathplus D_2 \pathplus N$ which does not contain $D_1$
by using the $D_1$-free path $D_2 \pathplus N$.  The assumption leads
to a contradiction; thus, there must exist some $D_i$ which does not
dominate $N$ if $N$ is required to have a \phifunction for $v$.  The
symmetric argument holds for post-dominance and \sigfunction{s}.
\end{itemize}
This proves that the conditions are necessary.  It is obvious from an
examination of conditions \ref{crit_phiplace} and \ref{crit_sigplace}
of the SSI form definition and lemma \ref{lem:sese_child} that they are
sufficient.
\end{proof}

In practice, the conditions of lemma \ref{lem:ssi_place_dom} are too expensive
to implement directly.  Instead, we use a conservative approximation
to SSI form, which allows us to place more \phisigfunction{s} than
minimal SSI requires (for example, a \phifunction for $v$ at the
circled node in Figure~\ref{fig:placeerror}), while satisfying the
conditions of the SSI form definition.  
Our algorithm also allows us to do pre-pruning of the SSI
form during placement.  The result is not pruned SSI, but contains a
tight superset of the \phisigfunction{s} that pruned form requires.
\begin{myfigure}
\centering\renewcommand{\figscale}{0.30}\input{Figures/THmorephi}
\caption{An flowgraph where Algorithm~\ref{alg:SSIplace} places
\phifunction{s} conservatively.}\label{fig:placeerror}
\end{myfigure}

\begin{theorem}\label{thm:placeproof}
Algorithm~\ref{alg:SSIplace} places all the
\phisigfunction{s} required by conditions \ref{crit_phiplace} and
\ref{crit_sigplace} of the SSI form definition.
\end{theorem}
\begin{proof}
Lemma \ref{lem:sese_child} states that the child region exclusion of
Algorithm~\ref{alg:SSIplace} does not cause required \phisigfunction[or]{s} to
be omitted.  Property~\ref{pty:pruned_live} allows the omission of
\phisigfunction{s} for $v$ at nodes where $v$ is dead when creating
pruned form; \code{MaybeLive} may not return \code{false} for nodes
where $v$ is not dead, but may return \code{true} at nodes where $v$
is dead without harming the correctness of the \phisigfunction{}
placement.
\end{proof}
\dontfixme{It would be nice to be able to show a means of using the
algorithm and the conditions in \ref{lem:ssi_place_dom} to produce exactly
minimal form or exactly pruned form.  It doesn't hurt our time bounds
to do fixup later, though.}

\subsubsection{Computing liveness}
Incorporating liveness information into the creation of pruned SSI
form appears to lead to a chicken-and-egg problem: although the pruned
SSI framework allows highly efficient liveness analysis, obtaining the
liveness information from the original program can be problematic.
The fastest sparse algorithm has stated time bounds of $O(E+N^2)$
\cite{ferrante91:pruned}, which is likely to be more expensive than
the rest of the SSI form conversion.  Luckily, Kam and Ullman
\cite{kam76:dataflow}, in conjunction with an empirical study by Knuth
\cite{knuth74:fortran}, show that liveness analysis is highly likely
to be linear for reducible flow-graphs.  In our work this question is
avoided, as we obtain our liveness information directly from
properties of the Java bytecode files that are our input to the
compiler.  But in any case our algorithms allow conservative
approximation to liveness, so even in the case of non-reducible flow
graphs it should not be difficult to quickly generate a rough
approximation.

\subsubsection{Variable renaming}
\begin{myfigure}[p]\small
% data types for SSI renaming algorithm.
Data type \code{Environment}:
\begin{tightdesc}
\item[create():~Environment]
:\par   make an environment with no mappings.
\item[put($\mathcal{E}$:~Environment, $v_1$:~variable, $v_2$:~variable)]
:\par   extend environment $\mathcal{E}$ with a mapping from $v_1$ to $v_2$.
\item[get($\mathcal{E}$:~Environment, $v$:~variable):~variable]
:\par   return the current mapping in $\mathcal{E}$ for $v$.
\item[beginScope($\mathcal{E}$:~Environment)]
:\par   save the current mapping of $\mathcal{E}$ for later restoration.
\item[endScope($\mathcal{E}$:~Environment)]
:\par   restore the mapping of $\mathcal{E}$ to that present at the
        last \code{beginScope} on $\mathcal{E}$.
\end{tightdesc}

\caption{Environment datatype for the SSI renaming algorithm.}
\label{fig:SSIrename_data}
\end{myfigure}
\begin{myalgorithm}\small
% algorithm for phi/sigma renaming.

Rename($G$:~CFG) =
\begin{myalgorithmic}
\STATE $\text{Init}(G)$
\FOREACH{edge $e$ leaving \code{START}}
 \STATE $\text{Search}(e)$ \label{line:search1}
\ENDFOR
\end{myalgorithmic}
~\\
Init($G$:~CFG) =
\begin{myalgorithmic}
\FOREACH{edge $e$ in $G$}
 \STATE $\text{Marked}[e] \gets \code{false}$
\ENDFOR
\FOREACH{variable $V$ in $G$}
 \STATE $C(V) \gets 0$
\ENDFOR
\STATE $\mathcal{E} = \text{create}()$ \COMMENT{create a new environment}
\end{myalgorithmic}
~\\
Inc($\mathcal{E}$:~Environment, $V$:~variable):~variable =
\begin{myalgorithmic}
\STATE $i \gets C(V) + 1$
\STATE $C(v) \gets i$
\STATE $\mathcal{E}.\text{put}(V, V_i)$
\RETURN $V_i$
\end{myalgorithmic}

\caption{SSI renaming algorithm.}\label{alg:SSIrename1}
\end{myalgorithm}
\begin{myalgorithm}\small
% SSI renaming algorithm, cont.
Search($\tuple{s,d}$:~edge) =
\begin{myalgorithmic}
 \REQUIRE{$s$ to be a node containing \phisigfunction[or]{s}}
 \IF{$s$ is a node containing \phifunction{s}}
  \FOREACH{\phifunction $P$ in $s$}
   \STATE replace the destination $V$ of $P$ by $Inc(V)$
  \ENDFOR
 \ELSIF{$s$ is a node containing \sigfunction{s}}
  \FOREACH{\sigfunction $S$ in $s$}
   \STATE $j \gets \text{WhichSucc}(\tuple{s,d})$
   \STATE replace the $j$-th destination $V$ of $S$ by $Inc(V)$
  \ENDFOR
 \ENDIF
 \STATE\COMMENT{now rename inside basic block}
 \LOOP
  \IF{$d$ is a node containing \phifunction{s}}
   \FOREACH{\phifunction $P$ in $d$}
    \STATE $j\gets \text{WhichPred}(\tuple{s,d})$
    \STATE replace the $j$-th operand $V$ of $P$ by $Get(V)$
   \ENDFOR
   \STATE $\text{PushEdge}(\tuple{d,\text{Succ}[d]})$
   \RETURN
  \ELSIF{$s$ is a node containing \sigfunction{s}}
   \FOREACH{\sigfunction $S$ in $d$}
    \STATE replace the operand $V$ of $S$ by $Get(V)$
   \ENDFOR
   \STATE $\text{PushEdge}(\tuple{d,\text{Succ}[d]})$
   \RETURN
  \ENDIF
  \STATE\COMMENT{ordinary assignment}
  \FOREACH{variable $V$ in $LHS(d)$}
   \STATE replace $V$ by $Get(V)$ in $LHS(d)$
  \ENDFOR
  \FOREACH{variable $V$ in $RHS(d)$}
   \STATE replace $V$ by $Inc(V)$ in $RHS(d)$
  \ENDFOR
  \IF{$d$ has a successor}
   \STATE $s \gets d$
   \STATE $d \gets \text{Succ}[d]$
  \ELSE
   \RETURN
  \ENDIF
 \ENDLOOP
\end{myalgorithmic}

\caption{SSI renaming algorithm, cont.}\label{alg:SSIrename2}
\end{myalgorithm}
Algorithm~\ref{alg:SSIrename1} performs variable renaming on a
flow-graph with placed \phisigfunction{s} in a single depth-first
traversal.  When the algorithm is complete, the control flow-graph
will be in proper SSI form.  The variable renaming algorithm requires
an \code{Environment} datatype which is defined in
Figure~\ref{fig:SSIrename_data}.  Using an imperative programming
style, it is possible to perform a sequence of any $N$ operations on
\code{Environment} as defined in the figure in $O(N)$ time; in a
functional programming style any $N$ operations can be completed in
$O(N \log N)$ time.\footnote{The curious reader is referred to section
5.1 of Appel \cite{appel:modern} for implementation details.}  As the
coarse structure of Algorithm~\ref{alg:SSIrename1} is a simple
depth-first search, it is easy to see that the \code{Search} procedure
can be invoked from line~\fullref{line:search1} and
line~\fullref{line:search2} a total of $O(E)$ times; likewise its
inner loop (lines~\ref{line:searchloop_start} to
\ref{line:searchloop_end}) can be executed a total of $E$ times across
all invocations of \code{Search}.  A total of $U_{SSA}+D_{SSA}$ calls
to the operations of the \code{Environment} datatype will be made
within all executions of \code{Search}.  For the imperative
implementation of \code{Environment} a total time bounds of
$O(E+U_{SSA}+D_{SSA})$ for the variable renaming algorithm is
obtained.

We have shown that Algorithm~\ref{alg:SSIplace} places all the
required \phisigfunction{s} in the control-flow graph according to SSI
form conditions \ref{crit_phiplace}, \ref{crit_sigplace}, and
\ref{crit_boundary}; we will now show that this algorithm renames
variables consistent with conditions \ref{crit_phiname} and
\ref{crit_signame} to prove that these algorithms combined suffice to
convert a program into SSI form.  The SSI form is not necessarily
minimal, as we showed in section~\ref{sec:SSIplace}; the next
section will show how to post-process to create minimal or pruned SSI
form.

\begin{lemma}\label{lem:path_construct}
The stack trace of calls to \code{Search} defines a unique path
through $G$ from \code{START}.
\end{lemma}
\begin{proof}
We will prove this lemma by construction.  For every consecutive pair
of calls to \code{Search} we construct a path $X\pathplus Y$ starting with the
edge $\tuple{X,N_0}$ which is the argument of the first call, and
ending with the edge $\tuple{N_n, Y}$ which is the argument of the
second call.  From line~\ref{line:search_onesucc} of the \code{Search}
procedure on page~\pageref{line:search_onesucc} we note that every
edge $\tuple{N_i, N_{i+1}}$ between the first and last has exactly one
successor.  Furthermore, the call to search on line~\ref{line:search2}
defines a path starting with the edge which our segment $X\pathplus Y$
ends with; therefore the paths can be combined.  By so doing from the
bottom of the call stack to the top we construct a unique path from
\code{START}.
\end{proof}

For brevity, we will hereafter refer to the canonical path constructed
in the manner of lemma~\ref{lem:path_construct} corresponding to the
stack of calls to \code{Search} when an edge $e$ is first
encountered as $CP(e)$.  Every edge in the CFG is encountered exactly
once by \code{Search}, so $CP(e)$ exists and is unique for every edge
$e$ in the CFG.

\begin{lemma}\label{lem:renamephi}
SSI form condition~\ref{crit_phiname} (\phifunction{} naming) holds
for variables renamed according to Algorithm~\ref{alg:SSIrename1}.
\end{lemma}
\begin{proof}
We restate SSI form condition~\ref{crit_phiname} for reference:
\begin{quote}
For every node $X$ containing a definition of a variable $V$ in
the new program and node $Y$ containing a use of that variable, there
exists at least one path $X \pathplus Y$ and no such path contains a
definition of $V$ other than at $X$.
\end{quote}
We consider the canonical path 
$CP(\tuple{Y',Y})=\code{START}\pathstar Y' \path Y$
for some use of a variable $v$ at $Y$, constructed according to
lemma~\ref{lem:path_construct} 
from a stack trace of calls to \code{Search}.
is encountered.  This path is unique, although more than one canonical
path may terminate at $Y$ at nodes with more than one predecessor.
These paths are distinguished by the incoming edge to
$Y$.\footnote{Note that the notation \tuple{N,N'} for denoting edges
does not always denote an edge unambigiously; imagine a conditional
branch where both the \code{true} and \code{false} case lead to the
same label.  In such cases an additional identifier is necessary to
distinguish the edges.  Alternatively, one may split such edges to
remove the ambiguity.  We treat edges as uniquely identifiable and
leave the implementation to the reader.}  We identify each operand
$v_i$ of a \phifunction{} with the appropriate incoming edge $e$ to
ensure that $CP(e)$ is well defined and unique in the context of a
use of $v_i$.

The canonical path $\code{START}\pathplus Y$ must contain $X$, a definition of
$v$, if $Y$ uses a variable defined in $X$, as \code{Search} renames
all definitions (in lines \ref{line:rendef1}, \ref{line:rendef2}, and
\ref{line:rendef3}) and destroys the name mapping in $\mathcal{E}$
just before it returns.  The call to \code{Search} which creates the
definition of $v$ must therefore always be on the stack, and thus in
the path $CP(\tuple{Y',Y})$, for any use to receive a the name $v$.
Note that this is
true for \phifunction{s} as well, which receive names when the
appropriate incoming edge $\tuple{Y',Y}$ is traversed, not necessarily
when the node $Y$ containing the \phifunction{} is first encountered.

We have proved that $\code{START}\pathplus X\pathplus Y$ exists; now
we must prove that no other path from $X$ to $Y$ contains a definition
of $v$.  Call this other definition $D$.  Obviously $D$ cannot be on
our canonical path $\code{START}\pathplus X\pathplus Y$, or
line~\ref{line:rendef3} would have caused $Y$ to use a different name.
But as we just stated, all variable name mappings done by $D$ will be
removed when the call to \code{Search} which touched $D$ is taken off
the call stack.  So $D$ must be on the call stack, and thus on the
canonical path; a contradiction. %
\ignore{%ackackackacakack I HATE PROOFS!!! HATE EM HATE EM HATE EM!
So let us call this
definition node $D$, and consider the path $D\pathplus Y$.  Since
$D\pathplus Y$ and $\code{START}\pathplus X\pathplus Y$ share at least
node $Y$ in common, there must be an earliest node $N$ where they
converge.  By condition~\ref{crit_phiplace} of the SSI form
definition there should be a \phifunction for $v$ at $N$.
If $N\not= Y$, then line~\ref{line:rendef1} would have caused $Y$ to
use a different name, a contradiction.  If $N=Y$ then $Y$ is a
\phifunction for $v$ and thus present on multiple possible call stacks
for \code{Search}.  As every edge is visited exactly once, however,
there will be only one call stack-derived path that enters the
\phifunction{} through the same edge as terminates the path
$D\pathplus Y$.  Consider this path.  If it contains $D$, then
line~\ref{line:rendef3} would have caused $Y$\ldots

Consider the call stack corresponding to the path
$\code{START}\pathplus N=Y$ through the incoming edge of $Y$ which terminates
the path $D\pathplus Y$.  The node $Y$, as a \phifunction, will be
present on multiple call stacks, but as every edge is visited exactly
once there is only one call stack-derived path that enters the
\phifunction{} through the same edge\ldots

Consider the path $\code{START}\pathplus
X\pathplus D\pathplus N=Y$.  By the properties of depth-first search
every node reachable from $X$, including $D$, will be visited before
the call to \code{Search} which first reached $X$ is popped off the
stack.  Because the incoming edge to $Y$ from $D$ is also reachable
from $X$ and every edge will be visited exactly once,
$\code{START}\pathplus X\pathplus D\pathplus N=Y$ will correspond to
the algorithm call stack at some point in time.
Line~\ref{line:rendef3} of \code{Search} will ensure that the variable
$D$ defines will not be named $v$, contradicting our choice of $D$.
}
%
Since assuming the existence of some other path $X\pathplus Y$
containing a definition of $v$ leads to contradiction no other such
path may exist, completing the proof of the lemma.
\end{proof}

\begin{lemma}\label{lem:renamesig}
SSI form condition~\ref{crit_signame} (\sigfunction{} naming) holds for
variables renamed according to Algorithm~\ref{alg:SSIrename1}.
\end{lemma}
\begin{proof}
We restate SSI form condition~\ref{crit_signame} for reference:
\begin{quote}
For every pair of nodes $X$ and $Y$ containing uses of a
variable $V$ defined at node $Z$ in the new program, either every path
$Z \pathplus X$ must contain $Y$ or every path $Z \pathplus Y$ must
contain $X$.
\end{quote}
Let us assume there are paths $Z\pathplus X$ and $Z\pathplus Y$
violating this condition; that is, let us chose nodes $X$ and $Y$
which use $V$ and $Z$ defining $V$ such that there exists a path $P_1$
from $Z$ to $X$ not containing $Y$ and a path $P_2$ from $Z$ to $Y$ not
containing $X$.  By the argument of the previous lemma, there exists
a canonical path $P_3=CP(e)$ from \code{START} to $X$ through $Z$
corresponding to a stack
trace of \code{Search}; note that $P_3$ need not contain $P_1$.
There are two cases:
\begin{description}
\item[Case I:] $P_3$ does not contains $Y$.  Then there is some last
node $N$ present on both $P_2: Z\pathstar N\pathplus Y$ and
$P_3: \code{START}\pathplus Z\pathstar N\pathplus X$.  By SSI
condition~\ref{crit_sigplace} this node $N$ requires a \sigfunction{}
for $V$.  If $N\not=Z$ then line~\ref{line:rendef1} of
Algorithm~\ref{alg:SSIrename1} would rename $V$ along $P_3$
and $X$ would not use the same variable $Z$ defined; if
$N=Z$, then line~\ref{line:rendef2} would have ensured that $X$
and $Y$ used different names.  Either case contradicts our choices of
$X$, $Y$, and $Z$.
\item[Case II:] $P_3$ does contain $Y$.  Then consider the path
$\code{START}\pathplus Z\pathplus Y$ along $P_3$, which does not
contain $X$.  The argument of case I applies with $X$ and $Y$ reversed.
\end{description}
Any assumed violation of condition~\ref{crit_signame} leads to
contradiction, proving the lemma.
\end{proof}

Every path $CP(e)$ corresponds to a execution state in a call to
\code{Search} at the point where $e$ is first encountered.  The value
of the environment mapping $\mathcal{E}$ at this point in the
execution of Algorithm~\ref{alg:SSIrename1} we will denote as
$\mathcal{E}^e$.  For a node $N$ having a single predecessor $N_p$ and
single successor $N_s$, we will denote
$\mathcal{E}^{\tuple{N_p,N}}$ as $\mathcal{E}_{\text{before}}^N$ and 
$\mathcal{E}^{\tuple{N,N_s}}$ as $\mathcal{E}_{\text{after}}^N$.
It is obvious that 
$\mathcal{E}_{\text{after}}^{N_p} = \mathcal{E}_{\text{before}}^{N  }$ and
$\mathcal{E}_{\text{after}}^{N  } = \mathcal{E}_{\text{before}}^{N_s}$
when $N_p$ and $N_s$, respectively, are also single-predecessor
single-successor nodes.

\begin{lemma}\label{lem:correctness}
SSI form condition~\ref{crit_correct} (correctness) holds for
variables renamed according to Algorithm~\ref{alg:SSIrename1}.  That
is, along any possible control-flow path in a program being executed a
use of a variable $V_i$ in the new program will always have the same
value as a use of the corresponding variable $V$ in the original
program.
\end{lemma}
\begin{proof}
We will use induction along the path $N_0\path N_1\path\ldots\path N_n$.
We consider $e_k=\tuple{N_{k},N_{k+1}}$, the $(k+1)$th edge in the path,
and assume that, for all $j<k$, each variable $V$ in the original
program agrees with the value of $\mathcal{E}^{e_j}[V]=V_i$ in the new
program.  We show that $\mathcal{E}^{e_k}[V]$ agrees with $V$ at edge
$e_k$ in the path.
\begin{description}
\item[Case I:] $k=0$. The base case is trivial: the \code{START} node
($N_0$) contains no statements, and along each edge $e$ leaving start
$\mathcal{E}^e[V]=V_0$.  By definition $V_0$ agrees with $V$ at the
entry to the procedure.
\item[Case II:] $k>0$ and $N_k$ has exactly one predecessor and one successor.
If $N_k$ is single-entry single-exit, then it is not a \phisigfunction[or].
As an ordinary assignment, it will be handled by
lines~\ref{line:rename_ordinary1} to \ref{line:rename_ordinary2} of
Algorithm~\vref{alg:SSIrename2}.  By the induction hypothesis (which
tells us that the uses at $N_k$ correspond to the same values as the
uses in the original program) and the semantics of
assignment, the mapping $\mathcal{E}_{\text{after}}^{N_k}$ is easily
verified to be valid when $\mathcal{E}_{\text{before}}^{N_k}$ is
valid.  Thus the value of every original variable $V$ corresponds to
the value of the new variable 
$\mathcal{E}_{\text{after}}^{N_k}[V]=\mathcal{E}^{e_k}[V]$ on $e_k$.
\item[Case III:] $k>0$ and $N_k$ has multiple predecessors and one
successor.  In this case $N_k$ may have multiple \phifunction{s} in
the new program, and by the definition in section~\ref{sec:defs} $N_k$
has no statements in the original program.  Thus the value of any
variable $V$ in the original program along edge $e_k$ is identical to
its value along edge $e_{k-1}$.  We need only show that the value of
the variable $\mathcal{E}^{e_{k-1}}[V]$ is the same as the value of
the variable $\mathcal{E}^{e_k}[V]$ in the new program.  For any
variable $V$ not mentioned in a \phifunction{} at $N_k$ this is
obvious.  Each variable defined in a \phifunction{} will get the value
of the operand corresponding to the incoming control-flow path edge.
The relevant lines in Algorithm~\ref{alg:SSIrename2} start with
\ref{line:phisrc1} and \ref{line:phisrc2}, where we see that the
operand corresponding to edge $e_{k-1}$ of a \phifunction{} for $V$
correctly gets $\mathcal{E}^{e_{k-1}}[V]$.  At
line~\ref{line:rendef1}, we see that the destination of the
\phifunction{} is correctly $\mathcal{E}^{e_k}[V]$.  Thus the value of
every original variable $V$ correctly correponds to
$\mathcal{E}^{e_k}[V]$ by the induction hyptothesis and the semantics
of the \phifunction{s}.
\item[Case IV:] $k>0$ and $N_k$ has one predecessor and multiple
successors.  Here $N_k$ may have multiple \sigfunction{s} in the new
program, and is empty in the original program.  The argument goes as
for the previous case.  It is obvious that variables not mentioned in
the \sigfunction{s} correspond at $e_k$ if they did at $e_{k-1}$.  For
variables mentioned in \sigfunction{s}, line~\ref{line:sigsrc} shows
that operands correctly get $\mathcal{E}^{e_{k-1}}[V]$ and
line~\ref{line:rendef2} shows that the destination corresponding to
$e_k$ correctly gets $\mathcal{E}^{e_k}[V]$.  Therefore the values of
original variables $V$ correspond to the value of
$\mathcal{E}^{e_k}[V]$ by the induction hypothesis and the semantics
of the \sigfunction{s}.
\item[Case V:] $N_k$ has multiple predecessors and multiple
successors.  Forbidden by the CFG definition in section~\ref{sec:defs}.
\end{description}
Therefore, on every edge of the chosen path, the values of the
original variables correspond to the values of the renamed SSI form
variables. The value correspondence at the path endpoint (a use of
some variable $V$) follows.
\end{proof}

\begin{theorem}\label{thm:renameproof}
Algorithm~\ref{alg:SSIrename1} renames variables such that SSI form
conditions \ref{crit_phiname}, \ref{crit_signame}, and
\ref{crit_correct} hold.
\end{theorem}
\begin{proof}
Direct from lemmas~\ref{lem:renamephi}, \ref{lem:renamesig}, and
\ref{lem:correctness}.
\end{proof}

\begin{theorem}
Algorithms~\ref{alg:SSIplace} and \ref{alg:SSIrename1} correctly
transform a program into SSI form.
\end{theorem}
\begin{proof}
Theorem~\ref{thm:placeproof} proves that \phisigfunction{s} are placed
correctly to satisfy conditions~\ref{crit_phiplace},
\ref{crit_sigplace} and \ref{crit_boundary} of the SSI form
definition, and theorem~\ref{thm:renameproof} proves that variables
are renamed correctly to satisfy conditions~\ref{crit_phiname},
\ref{crit_signame} and~\ref{crit_correct}.
\end{proof}

\subsubsection{Pruning SSI form}\label{sec:unusedcode}
The SSI algorithm can be run using any conservative approximation to
the liveness information
(including the function $\code{MaybeLive}(v, n)=\code{true}$) if
unused code elimination%
\footnote{We follow \cite{wegman91:scc} in distinguishing
\emph{unreachable code elimination}, which removes code that can never
be executed, from \emph{unused code elimination}, which deletes
sections of code whose results are never used.  Both are often called
``dead code elimination'' in the literature.} is performed to remove
extra \phisigfunction{s} added and create pruned SSI.
Figure~\ref{fig:deaddata} and Algorithm~%
\ref{alg:deadalg} present an algorithm to identify unused code in
$O(N V_{SSI})$ time, after which a simple $O(N)$ pass suffices to remove it.
The complexity analysis is simple: nodes and variables are visited at
most once, raising their value in the analysis lattive from
\emph{unused} to \emph{used}.  Nodes marked \emph{used} are never
visted.  So \code{MarkNodeUseful} is invoked at most $N$ times, and
\code{MarkVarUseful} is invoked at most $V_{SSI}$ times.  The calls to
\code{MarkNodeUseful} may examine at most every variable use in the
program in lines~\ref{line:deadnode1}-\ref{line:deadnode2}, taking
$O(U_{SSI})$ time at worst. Each call
to \code{MarkVarUseful} examines at most one node (the single
definition node for the variable, if it exists) and in constant time
pushes at most one node on to the worklist for a total of $O(V_{SSI})$ time.
So the total run time of \code{FindUseful} is
$O(U_{SSI}+V_{SSI})=O(U_{SSI})$.
%\footnote{If the number of instruction
%operands and \phisigfunction{} arities are limited by a
%constant, we get a time bound of $O(N)$.}

\begin{myfigure}\small
% data types for cycle-equivalency algorithm.
Operations on nodes:
\begin{tightdesc}
\item[NodeUseful(n:node):~boolean]
:	Whether the results of this node are ever used
\item[Uses(n:node):~set of variables]
:	Variables for which this node contains a use
\end{tightdesc}

Operations on variables:
\begin{tightdesc}
\item[VarUseful(v:variable):~boolean]
:	Whether there is some \textbf{n} for which 
        \textbf{Uses(n)} contains \textbf{v} and
        \textbf{NodeUseful(n)} is \texttt{true}
%:	$\exists n\in\domain{Nodes}\;.\;v\in\mbox{Uses}(n)\wedge\mbox{NodeUseful}(n)$
%:	Whether the value of this variable is ever used
\item[Definitions(v:variable):~set of nodes]
:	Nodes which contain a definition for~\textbf{v}
\end{tightdesc}

\caption{Datatypes and operations used in unused code elimination.}
\label{fig:deaddata}
\end{myfigure}

\begin{myalgorithm}\small\linespread{0.75}
% fast algorithm for dead code elimination (unused code elimination)
\begin{verbatim}
Procedure FindUseful (G: CFG) {
  let W be an empty work list;
  for each variable v, do
    VarUseful(v) := false;
  for each node n in G, in any order, do {
    NodeUseful(n) := false;
    if n is a CALL, RETURN, or other special node, then
      add n to W;
  }

  while W is not empty, do {
    let n be any element from W;
    remove n from W;
    MarkNodeUseful(n, W);
  }
}
Procedure MarkNodeUseful(n: node, W: WorkList) {
  NodeUseful(n) := true;
  /* everything used by a useful node is useful */
  for each v in Uses(n), do
    if not VarUseful(v), then
      MarkVarUseful(v, W);
}
Procedure MarkVarUseful(v: variable, W: WorkList) {
  VarUseful(v) := true;
  /* The definition of a useful variable is useful */
  for each n in Definitions(v), then
    /* size(Definitions(v)) <= 1 in SSI form */
    if not NodeUseful(n), then
      add n to W;
}
\end{verbatim}
\caption{Identifying unused code using SSI form.}
\label{alg:deadalg}
\end{myalgorithm}

\subsubsection{Discussion}
Note that our algorithm for placing \phisigfunction{s} in
SSI form is
\emph{pessimistic}; that is, we at first assume every node in the
control-flow graph with input arity larger than one requires a
\phifunction{} for every variable and every node with out-arity larger
than one requires a \sigfunction{} for every variable, and then use
the PST, liveness information, and unused code elimination to
determine safe places to
\emph{omit} \phisigfunction[or]{s}.  Most SSA construction
algorithms, by contrast, are \emph{optimistic}; they assume no
\phisigfunction[or]{s} are needed and attempt to determine where
they are provably necessary.  In my experience, optimistic algorithms tend to
have poor time bounds because of the possibility of input graphs like
the one illustrated in Figure~\vref{fig:evil}.
Proving that all but two nodes require
\phisigfunction[and/or]{s} for the variable $a$ in this example seems to
inherently require $O(N)$ passes over the graph; each pass can prove
that \phisigfunction[or]{s} are required for only those nodes adjacent to
nodes tagged in the previous pass.  Starting with the circled node, the
\phisigfunction{s} spread one node left on each pass. On the other hand,
an pessimistic algorithm assumes the correct answer at the start, fails
to show that any \phisigfunction[or]{s} can be removed, and
terminates in one pass.\dontfixme{Are we \emph{sure} similar worst cases
don't exist for the pessimistic algorithm?}

\begin{myfigure}[t]
\centering\renewcommand{\figscale}{0.25}\input{Figures/evil}
\caption{A worst-case CFG for ``optimistic'' algorithms.}
\label{fig:evil}
\end{myfigure}

\subsection{Time and space complexity of SSI form}\label{sec:ssi_complexity}
%\begin{myfigure}%[t]
%\input{Figures/phisig}
%\caption{Number of \phisigfunction{s} added versus procedure length.}
%\label{fig:phisigdata}
%\end{myfigure}
\begin{myfigure}%[t]
\input{Figures/THussi}
\caption{Number of uses in SSI form as a function of
procedure~length.}
\label{fig:ussidata}
\end{myfigure}
\begin{myfigure}%[t]
\input{Figures/THv0}
\caption{Number of original variables as a function of
procedure~length.}
\label{fig:v0data}
\end{myfigure}
Discussions of time and space complexity for sparse evaluation
frameworks in the literature are often misleadingly called ``linear''
regardless of what the $O$-notation runtime bounds are.  A canonical
example is \cite{sreedhar95:lintime}, which states that
for SSA form, ``the number of $\phi$-nodes needed remains linear.''
Typically Cytron \cite{cytron91:ssa} is cited; however, that reference
actually reads:
\begin{quote}
For the programs we tested, the plot in [Figure 21 of Cytron's paper]
shows that the number of \phifunction{s} is also linear in the size of
the original program.
\end{quote}
It is important to note that Cytron's claim is based not on
algorithmic worst-bounds complexity, but on empirical evidence.  This
reasoning is not unjustified; Knuth \cite{knuth74:fortran} showed in
1974 that ``human-generated'' programs almost without exception show
properties favorable to analysis; in particular shallow maximum loop
nesting depth.  Wegman and Zadeck \cite{wegman91:scc} clearly make
this distinction by noting that:
\begin{quote}
In theory the size [of the SSA form representation] can be $O(EV)$,
but empirical evidence indicates that the work required to compute the
SSA graph is linear in the program size.
\end{quote}
Our worst-case space complexity bounds for SSI form are identical to
SSA form --- $O(EV)$ --- but in this section we will endeavour to show
that typical complexities are likewise ``linear in the program size.''

The total runtime for SSI placement and subsequent pruning, including
the time to construct the PST, is $O(E + N V_0 + U_{SSI})$.  For most
programs $E$ will be a small constant factor multiple of $N$; as
Wegman and Zadeck \cite{wegman91:scc} note, most control flow graph
nodes will have at most two successors.  For those graphs where $E$ is
not $O(N)$, it can be argued that $E$ is the more relevant measure of
program complexity.\footnote{We will not follow Cytron \cite{cytron91:ssa} in
defining a new variable $R$ to denote $\max(N,E,\ldots)$ to avoid
following him in declaring worst-case complexity $O(R^3)$ and leaving
it to the reader to puzzle out whether $O(N^6)$ (!) is really being implied.}

Thus the ``linearity'' of our SSI construction algorithm rests on the
quantities $N V_0$ and $U_{SSI}$.  Figures~\ref{fig:ussidata} and
\ref{fig:v0data} present empirical data for $V_0$ and $U_{SSI}$ on a
sample of 1,048 Java methods.  The methods varied in length from 4
to 6,642 statements and were taken from the dynamic call-graph of the
FLEX compiler itself, which includes large portions of the standard
Java class libraries.  Figure~\ref{fig:ussidata} shows convincingly
that $U_{SSI}$ grows as $N$ for large procedures, and
Figure~\ref{fig:v0data} supports an argument that $V_0$ grows very
slowly and that the quantity $N V_0$ would tend to grow as $N^{1.3}$.
This would argue for a near-linear practical run-time.

In contrast, Cytron's original algorithm for SSA form had theoretical
complexity $O(E + V_{SSA} |\text{DF}| + N V_{SSA})$.  Cytron does not
present empirical data for $V_{SSA}$, but one can infer from the data
he presents for ``number of introduced \phifunction{s}'' that
$V_{SSA}$ behaves similarly to $V_{SSI}$ --- that is, it grows as $N$,
not as $V_0$.  It is frequently pointed out\footnote{See Dhamdhere
\cite{dhamdhere92:large} for example.} that the $|\text{DF}|$
term, the size of the dominance frontier, can be $O(N^2)$ for common
programming constructs (\code{repeat-until} loops), which indicates that
the $V_{SSA} |\text{DF}|$ term in Cytron's algorithm will be $O(N^2)$
at best and at times as bad as $O(N^3)$.

Note that the space complexity of SSI form, which may be $O(EV)$ in
the worst case (\phisigfunction{s} for every variable inserted at
every node) is certainly not greater than $U_{SSI}$, and thus
Figure~\ref{fig:ussidata} shows linear practical space use.

\section{Uses and applications of SSI}
The principle benefits of using SSI form are the ability to do
predicated and backward dataflow analyses efficiently.
\newterm{Predicated analysis} means that we can use information
extracted from branch conditions and control flow.  The
\sigfunction{s} in SSI form provide an variable naming that
allows us to sparsely associate the predication information with
variable names at control flow splits.  The \sigfunction{s} also
provide a reverse symmetry to SSI form that allow efficient backward
dataflow analyses like \newterm{liveness} and
\newterm{anticipatability}.

In this section, we will briefly sketch how SSI form can be applied to
backwards dataflow analyses, including anticipatability, an important
component of partial redundancy elimination.  We will then describe in
detail our Sparse Predicated Typed Constant propagation algorithm,
which shows how the predication information of SSI form may be used to
advantage in practical applications, including the removal of array
bounds and null-pointer checks.  Lastly, we will describe an extension
to SPTC that allows \newterm{bitwidth analysis}, and the possible uses
of this information.

\subsection{Backward Dataflow Analysis}\label{sec:bidirectional}
\newterm{Backward dataflow analyses} are those in which information is
propagated in the direction opposite that of program execution
\cite{offner95}.  There is general agreement
\cite{johnson93:dfg,ferrante91:pruned,weise94:vdg}
that SSA form is unable to directly handle backwards dataflow
analyses; \newterm{liveness} is often cited as a canonical example.

However, SSI form allows the sparse computation of such backwards
properties.  Liveness, for example, comes ``for free'' from pruned SSI
form: every variable is live in the region between its use and
sole definition.  Property~\ref{pty:ssi_dom} states that every
non-\phifunction{} use of a variable is dominated by the definition;
Cytron \cite{cytron91:ssa} has shown that \phifunction{s} will always be
found on the dominance frontier.  Thus the live region between
definition and use can be enumerated with a simple depth-first search,
taking advantage of the topological sorting by dominance that DFS
provides \cite{offner95}.  Because of \phifunction{} uses, the DFS
will have to look one node past its spanning-tree leaves to see the
\phifunction{s} on the dominance frontier; this does not change the
algorithmic complexity.

Computation of other dataflow properties will use this same
enumeration routine to propagate values computed on the sparse SSI
graph to the intermediate nodes on the control-flow graph.  Formally,
we can say that the dataflow property for variable $v$ at node $N$ is
dependent only on the properties at nodes $D$ and $U$, defining and
using $v$, for which there is a path $D\pathplus U$ containing $N$.
There is a ``default'' property which holds for nodes on no such path
from a definition to use; for liveness the default property is ``not
live.''  The remainder of this section will concentrate on the
dataflow properties at use and definition points.

A slightly more complicated backward dataflow property is
\newterm{very busy expressions}; this analysis is somewhat obsolete as
it serves to save code space, not time.  This in turn is related to
partial and total \newterm{anticipatability}.

\begin{definition}
An expression $e$ is \newterm{very busy} at a point $P$ of the program iff it
is always subsequently used before it is killed \cite{offner95}.
\end{definition}
\begin{definition}
An expression $e$ is \newterm{totally (partially) anticipatable} at a
point $P$ if, on every (some) path in the CFG from $P$ to \code{END},
there is a computation of $e$ before an assignment to any of the
variables in $e$ \cite{johnson93:dfg}.
\end{definition}

Johnson and Pingali \cite{johnson93:dfg} show how to reduce these
properties of expressions to properties on variables.  We will
therefore consider properties $\text{BSY}(v,N)$, $\text{ANT}(v,N)$,
and $\text{PAN}(v,N)$ denoting very busy, totally anticipatable, and
partially anticipatable variables $v$ at some program point $N$.

To compute BSY, we start with pruned SSI form.  Any variable defined
in a \phisigfunction[or]{} is used at some point, by definition.
So for statements at a point $P$ we have the rules:
\begin{displaymath}
\begin{array}{ll}
v = \ldots & \text{BSY}_{\text{in}}(v,P)=\code{false} \\
\ldots = v & \text{BSY}_{\text{in}}(v,P)=\code{true} \\
x=\phi(y_0,\ldots,y_n) &
       \text{BSY}_{\text{in}}(y_i,P)=\text{BSY}_{\text{out}}(x,P) \\
\tuple{x_0,\ldots,x_n}=\sigma(y) &
       \text{BSY}_{\text{in}}(y,P)=\bigwedge_{i=0}^n \text{BSY}_{\text{out}}(x_i,P) \\
\end{array}
\end{displaymath}

Total anticipatability, in the single variable case, is identical to
BSY.  Partial anticipatability for a variable $v$ at point $P$ follows
the rules:
\begin{displaymath}
\begin{array}{ll}
v = \ldots & \text{PAN}_{\text{in}}(v,P)=\code{false} \\
\ldots = v & \text{PAN}_{\text{in}}(v,P)=\code{true} \\
x=\phi(y_0,\ldots,y_n) &
       \text{PAN}_{\text{in}}(y_i,P)=\text{PAN}_{\text{out}}(x,P) \\
\tuple{x_0,\ldots,x_n}=\sigma(y) &
       \text{PAN}_{\text{in}}(y,P)=\bigvee_{i=0}^n \text{PAN}_{\text{out}}(x_i,P) \\
\end{array}
\end{displaymath}

The present section is concerned more with feasibility than the
mechanics of implementation; we refer the interested reader to
\cite{offner95} and \cite{johnson93:dfg} for details on how to turn
the efficient computation of BSY, PAN and ANT into practical
code-hoisting and partial-redundancy elimination routines, respectively.

We note in passing that the sophisticated strength-reduction and code-motion
techniques of SSAPRE \cite{kennedy98:strength} are applicable to an SSI-based
representation, as well, and may benefit from the predication
information available in SSI.  The remainder of this section will
focus on practical implementations of predicated analyses using SSI form.

\subsection{Sparse Predicated Typed Constant Propagation}
Sparse Predicated Typed Constant (SPTC) Propagation is a powerful
analysis tool which
derives its efficiency from SSI form.  It is built on Wegman and
Zadeck's Sparse Conditional Constant (SCC) algorithm
\cite{wegman91:scc} and removes unnecessary array-bounds and
null-pointer checks, computes variable types, and performs
floating-point- and string-constant-propagation in addition to the
integer constant propagation of standard SCC.
\ignore{\footnote{It also washes
the dishes and does the laundry.}}

We will describe this algorithm incrementally, beginning with the
standard SCC constant-propagation algorithm for review.
Wegman and Zadeck's algorithm operates on a program in SSA form; we will
call this SCC/SSA to differentiate it from SCC/SSI, using the SSI
form, which we will describe in section \ref{sec:sccssi}. Section
\vref{sec:bitwidth} will discuss an extension to SPTC which does
\newterm{bit-width analysis}.

\subsubsection{Wegman and Zadeck's SCC/SSA algorithm}
\begin{myfigure}
\centering\renewcommand{\figscale}{0.5}\input{Figures/THlat1}
\caption[Value and executability lattices for SCC.]
{Three-level value lattice and two-level executability lattice for SCC.}
\label{fig:scclat1}
\end{myfigure}
\begin{mytable}\centering
$\begin{array}{|l|cccc|} \hline
\meet & \bot & c & d (\not= c) & \top \\ \hline
\bot  & \bot & c & d  & \top \\
c     &   c  & c &\top & \top \\
\top  & \top &\top&\top& \top \\ \hline
\end{array}%
\quad\quad%
\begin{array}{|l|ccc|}\hline
\oplus& \bot &   d  & \top \\ \hline
\bot  & \bot &   d  & \top \\
c     &   c  &c\oplus d& \top \\
\top  & \top & \top & \top \\ \hline
\end{array}$
\caption{Meet and binary operation rules on the SCC value lattice.}
\label{tab:sccmeet1}
\end{mytable}
\begin{myalgorithm}\small
Init($G$:CFG) =
\begin{algorithmic}[1]
\STATE $E_e \gets \emptyset$
\STATE $E_n \gets \emptyset$
\FOREACH {variable $v$ in $G$}
 \IF{some node $n$ defines $v$}
  \STATE $V[v] \gets \bot$
 \ELSE
  \STATE $V[v] \gets \top$
 \ENDIF
\ENDFOR
\end{algorithmic}
~\\
Analyze($G$:CFG) =
\begin{algorithmic}[1]
\LET $r$ be the start node of graph $G$
\STATE $E_n \gets E_n \cup \{ r \}$
\STATE $W_n \gets \{ r \}$
\STATE $W_v \gets \emptyset$
          \EMPTYLINE
\REPEAT
 \IF{$W_n$ is not empty}
  \STATE remove some node $n$ from $W_n$
  \IF{$n$ has only one outgoing edge $e$ leading to some node $n'$}
   \STATE RaiseE($e$)
  \ENDIF
  \STATE Visit($n$)
 \ENDIF
 \IF{$W_v$ is not empty}
  \STATE remove some variable $v$ from $W_v$
  \FOREACH{node $n$ containing a use of $v$}
   \STATE Visit($n$)
  \ENDFOR
 \ENDIF
\UNTIL{both $W_v$ and $W_n$ are empty}
\end{algorithmic}

\caption{SCC algorithm for SSA form.}\label{alg:scc}
\end{myalgorithm}
\begin{myalgorithm}\small
RaiseE($e$:edge) =
\begin{algorithmic}[1]
\STATE $E_e \gets E_e \cup \{ e \}$
\LET $n$ be the destination of edge $e$
\IF{$n \not\in E_n$}
\STATE $E_n \gets E_n \cup \{ n \}$
\STATE $W_n \gets W_n \cup \{ n \}$
\ENDIF
\end{algorithmic}
~\\
RaiseV($v$:variable, $L$:lattice value) =
\begin{algorithmic}[1]
\IF{$V[v] \sqsupset L$}
\STATE $V[v] \gets L$
\STATE $W_v \gets W_v \cup \{ v \}$
\ENDIF
\end{algorithmic}
~\\
Visit($n$:node) =
\begin{algorithmic}[1]
\FOREACH{assignment ``$v \gets x \oplus y$'' in $n$}
 \STATE RaiseV($v$, $V[x] \oplus V[y]$)
 \COMMENT{binop rule: see table~\ref{tab:sccmeet1}}
\ENDFOR
               \EMPTYLINE
\FOREACH{assignment ``$v \gets \text{MEM}(\ldots)$'' or 
                    ``$v \gets \text{CALL}(\ldots)$'' in $n$}
 \STATE RaiseV($v$, $\top$)
\ENDFOR
               \EMPTYLINE
\FOREACH{assignment ``$v \gets \phi(x_1,\ldots,x_n)$'' in $n$}
 \FOREACH{variable $x_i$ corresponding to predecessor edge $e_i$ of $n$}
  \IF{$e_i \in E_e$}
   \STATE RaiseV($v$, $V[v] \meet V[x_i]$)
   \COMMENT{meet rule: see table~\ref{tab:sccmeet1}}
  \ENDIF
 \ENDFOR
\ENDFOR
               \EMPTYLINE
\FOREACH{branch ``if $v$ goto $e_1$ else $e_2$'' in $n$}
 \STATE $L \gets V[v]$
 \IF{$L=\top$ or $L=c$ where $c$ signifies ``true''}
  \STATE RaiseE($e_1$)
 \ENDIF
 \IF{$L=\top$ or $L=c$ where $c$ signifies ``false''}
  \STATE RaiseE($e_2$)
 \ENDIF
\ENDFOR
\end{algorithmic}
\caption{SCC algorithm for SSA form, cont.}\label{alg:scc2}
\end{myalgorithm}
The SCC algorithm works on a simple three-level value lattice
associated with variable definition points and a two-level
executability lattice associated with flow-graph edges.  These
lattices are shown in Figure~\vref{fig:scclat1}.  Associating a lattice
value with a definition point is a conservative statement that, for
all possible program paths, the value of that variable has a certain
property.  The value lattice is, formally, $\domain{Int}_\bot^\top$;
the lattice value $\bot$ signifies that no information
about the value is known, the lattice value $\top$ indicates that it
is possible that the variable has more than one dynamic value, and the
other lattice entries (corresponding to integer constants and occuping
a flat space between $\top$ and $\bot$) indicate that the variable can
be proven to have a single constant value in all runs of the program.%
\footnote{Note that we follow the $\top$ and $\bot$ conventions used
in semantics and abstract interpretation; authors in dataflow analysis
(including Wegman and Zadeck in their SCC paper \cite{wegman91:scc})
often use contrary definitions, letting $\top$ mean undefined and
$\bot$ indicate overdefinition.  As section \ref{sec:semantics} will
discuss the semantics of \ssiplus at length, we thought it best to
adhere to one set of definitions consistently, instead of switching
mid-paper.}
Similarly, the executability lattice indicates whether it is possible
that the control flow edge is traversed in some execution of the
program (marked ``executable''), or if it can be proven that the edge
is never traversed in any valid program path (marked ``not
executable'').  The algorithm works with SSA form, and is presented
as Algorithm~\ref{alg:scc}.  Binary operations on lattice values and
combination at $\phi$-nodes follow the rules in
Table~\ref{tab:sccmeet1}; notice that the meet operation ($\meet$) is
simply the least upper bound on the lattice.
The time complexity of SCC/SSA can be found
easily: the procedure \code{RaiseE} puts each node on the $W_n$
worklist at most once, and \code{RaiseV} puts a variable on the $W_v$
worklist at most $D-1$ times, where $D$ is the maximum lattice depth.
The \code{Visit} procedure can thus be invoked a maximum of $N$ times
by line~\ref{line:visitWn} of the \code{Analyze} procedure of
Algorithm~\ref{alg:scc}, and a maximum of $U_{SSA}(D-1)$ times by
line~\ref{line:visitWv}, where $U_{SSA}$ is the number of variable
\newterm{uses} in the SSA representation of the program.  The lattice
depth $D$ is the constant 3 in this version of the algorithm, so it
drops out of the expression.  The \code{RaiseE} procedure itself is
called at most $E$ times.  The time complexity is thus
$O(E+N+U_{SSA}(D-1))$ which simplifies to $O(E+U_{SSA})$.

\subsubsection{SCC/SSI: predication using \sigfunction{s}.}\label{sec:sccssi}
\begin{myfigure}%
\begin{samplecode}[2]%
foo = f();        & \subvar{foo}{0} = f();\\
if (foo == 1)     & if (\subvar{foo}{0} == 1) \\
                  & $\tuple{\subvar{foo}{1},\subvar{foo}{2}}$ =
                    $  \sigma(\subvar{foo}{0})$ \\
\>bar = foo + 1;  & \>\subvar{bar}{0} = \subvar{foo}{2} + 1;\\
else              & else \\
\>bar = 2;        & \>\subvar{bar}{1} = 2;\\
                  & \subvar{bar}{2} =
                    $  \phi(\subvar{bar}{0},\subvar{bar}{1})$%\\
\end{samplecode}%
\caption{A simple constant-propagation example.}
\label{fig:ssa_vs_ssi}
\end{myfigure}
\begin{myalgorithm}\small
Visit($n$:node) =
\begin{myalgorithmic}
\newcommand{\Lx}{\ensuremath{L[x]}\xspace}
\newcommand{\Ly}{\ensuremath{L[y]}\xspace}
\STATE \COMMENT{Assignment rules as on page \pageref{alg:scc2}}
               \EMPTYLINE
\FOREACH{branch ``if $x=y$ goto $e_1$ else $e_2$'' in $n$}
 \IF{$L[x] \sqsupset \bot$ and $L[y] \sqsupset \bot$}
  \IF{$L[x]=c$ and $L[y]=d$}
   \IF{$c=d$}
    \STATE RaiseE($e_1$) ; RaiseV($x_1$, $c$) ; RaiseV($y_1$, $d$)
   \ELSE
    \STATE RaiseE($e_2$) ; RaiseV($x_2$, $c$) ; RaiseV($y_2$, $d$)
   \ENDIF
  \ELSE[Either \Lx or \Ly is $\top$]
   \STATE RaiseE($e_1$) ; RaiseE($e_2$)
   \STATE RaiseV($x_1$, min($L[x]$, $L[y]$)) ; RaiseV($x_2$, $L[x]$)
   \STATE RaiseV($y_1$, min($L[x]$, $L[y]$)) ; RaiseV($y_2$, $L[y]$)
  \ENDIF
 \ENDIF
\ENDFOR
\end{myalgorithmic}

\caption{A revised \code{Visit} procedure for SCC/SSI.}\label{alg:sccssi}
\end{myalgorithm}
Porting the SCC algorithm from SSA to SSI form immediately increases
the number of constants we can find.  A simple example is shown in
Figure~\ref{fig:ssa_vs_ssi}: the version of the program on the right
is in SSI form, and SCC/SSI---unlike SCC/SSA---can
determine that \code{\subvar{foo}{2}} is a constant with value 1
(although nothing can be said about the value of
\code{\subvar{foo}{0}} or \code{\subvar{foo}{1}}) and therefore that
\code{\subvar{bar}{0}}, \code{\subvar{bar}{1}}, and
\code{\subvar{bar}{2}} are constants with the value 2.
SSI form creates a new name for \code{bar} at the conditional branch
to indicate that more information about its value is known.

Only the \code{Visit} procedure must be updated for SCC/SSI: lattice
update rules for \sigfunction{s} must be added.
Algorithm~\ref{alg:sccssi} shows a new \code{Visit} procedure for the
two-level integer constant lattice of Wegman and Zadeck's SCC/SSA;
with this restricted value set only integer equality tests tap the
algorithm's full power.  The utility of SCC/SSI's \newterm{predicated
analysis} will become more evident as the value lattice is extended to
cover more constant types.

The time complexity of the updated algorithm is identical to that of
SCC/SSA: $O(E+U_{SSA})$, by the same argument as before.

\subsubsection{Extending the value domain}
\begin{myfigure}
\centering\renewcommand{\figscale}{0.5}\input{Figures/THlat2}
\caption{SCC value lattice extended to Java primitive value domain.}
\label{fig:scclat2}
\end{myfigure}
The first simple extension of the SCC value lattice enables us to
represent floating-point and other values.  For this work, we extended
the domain to cover the full type system of Java bytecode
\cite{gosling95:bytecode}; the extended lattice is presented in
Figure~\ref{fig:scclat2}.
The figure also introduces the abbreviated lattice notation we will use
through the following sections; it is understood that the
lattice entry labelled ``int'' stands for a finite-but-large set of
incomparable lattice elements, consisting (in this case) of the
members of the Java \code{int} integer type.
%\footnote{As the Java \code{int}
%type is a proper subset of the Java \code{long} integer type, we
%simplify our lattice by representing all integers as elements of
%\code{long}. Overflows are handled correctly after analysis.  Although
%the mechanics are slightly different, the same principle holds for
%the Java \code{double} and \code{float} types, and we similarly
%combine them in our lattice.}
Java \code{int}s are 32 bits long, so the ``int'' entry abbreviates
$2^{32}$ lattice elements.  Similarly, the ``double'' entry encodes not
the infinite domain of real numbers, but the domain spanned by the
Java \code{double} type which has fewer than $2^{64}$
members.\footnote{In IEEE-standard floating-point, some possible bit
patterns are not valid number encodings.}  The
Java \code{String} type is also included, to allow simple constant
string coalescing to be performed.  The propagation algorithm over
this lattice is a trivial modification to Algorithm~\ref{alg:sccssi}, and
will be omitted for brevity.  In the next sections, the ``int'' and ``long''
entries in this lattice will be summarized as ``Integer Constant'',
the ``float'' and ``double'' entries as ``Floating-point Constant'',
and the ``String'' entry as ``String Constant''.  As the lattice is
still only three levels deep, the asymptotic runtime complexity is
identical to that of the previous algorithm.

\subsubsection{Type analysis}
\begin{myfigure}[p]
\centering\renewcommand{\figscale}{0.5}\input{Figures/THlat3}
\caption{SCC value lattice extended with type information.}
\label{fig:scclat3}
\end{myfigure}
\begin{myfigure}[p]
\centering\renewcommand{\figscale}{0.5}\input{Figures/THlat4}
\caption{``Typed'' category of Figure~\ref{fig:scclat3} shown expanded.}
\label{fig:scclat4}
\end{myfigure}
\begin{myfigure}%
\begin{eqnarray*}
\code{int}\oplus\code{int}&=&\code{int}\\
\code{long}\oplus\{\code{int},\code{long}\}&=&\code{long}\\
\code{float}\oplus\{\code{int},\code{long},\code{float}\}&=&\code{float}\\
\code{double}\oplus\{\code{int},\code{long},\code{float},\code{double}\}&=&\code{double}\\
\code{String}\oplus\{\code{int},\code{long},\code{float},\code{double},\code{Object},\ldots\} &=& \code{String}
\end{eqnarray*}%
\caption{Java typing rules for binary operations.}
\label{fig:scc_typed_binop}
\end{myfigure}
\begin{mytable}%[t]
\begin{tabular}{|l|l|c|c|c|}\hline
\small Hierarchy & \small Source language & \small Classes & \small Avg. depth & \small Max. depth \\ \hline
FLEX infrastructure & Java  &   550   &    1.9     &     5      \\
\code{javac} compiler & Java&   304   &    2.8     &     7      \\
NeXTStep 3.2$^\dag$& Objective-C & 488 &   3.5     &     8      \\
Objectworks 4.1$^\dag$&Smalltalk & 774 &   4.4     &    10      \\ \hline
\end{tabular}\\%
{\small$\dag$ indicates data obtained from Muthukrishnan and M\"uller
 \cite{muthukrishnan96:ch}.}
\caption{Class hierarchy statistics for several large O-O projects.}
\label{tab:chstats}
\end{mytable}
\begin{myalgorithm}\small
Visit($n$:node) =
\begin{myalgorithmic}
\FOREACH{assignment ``$v \gets x \oplus y$'' in $n$}
 \STATE RaiseV($v$, $V[x] \oplus V[y]$)
 \COMMENT{binop rule: see table~\ref{tab:sccmeet2}}
\ENDFOR
               \EMPTYLINE
\FOREACH{assignment ``$v \gets \text{MEM}(\ldots)$'' or 
                    ``$v \gets \text{CALL}(\ldots)$'' in $n$}
 \LET $t$ be the type of the \code{MEM} or \code{CALL} expression
 \STATE RaiseV($v$, $t$)
\ENDFOR
               \EMPTYLINE
\FOREACH{assignment ``$v \gets \phi(x_1,\ldots,x_n)$'' in $n$}
 \FOREACH{variable $x_i$ corresponding to predecessor edge $e_i$ of $n$}
  \IF{$e_i \in E_e$}
   \STATE RaiseV($v$, $\bigsqcup_L\{V[v],V[x_i]\}$)
   \COMMENT{meet rule: use least upper bound}
  \ENDIF
 \ENDFOR
\ENDFOR
               \EMPTYLINE
\FOREACH{branch ``if $x=y$ goto $e_1$ else $e_2$'' in $n$}
 \IF{$\text{Typed}\sqsubseteq L[x]$ or $\text{Typed}\sqsubseteq L[y]$}
  \STATE RaiseE($e_1$)
  \STATE RaiseE($e_2$)
 \ELSIF[if $x$ and $y$ are constants\ldots]{$L[x]=c$ and $L[y]=d$}
  \IF{$c=d$}
   \STATE RaiseE($e_1$)
  \ELSE
   \STATE RaiseE($e_2$)
  \ENDIF
 \ENDIF
\FOREACH{assignment ``$\tuple{v_1,v_2}\gets\sigma(v_0)$'' associated
         with this branch}
 \IF{edge $e_1 \in E_e$ and variable $v_0$ is the $x$ or $y$ in the test}
  \STATE\COMMENT{type error in source program if $L[x]$ and $L[y]$ are incomparable}
  \STATE RaiseV($v_1$, min($L[x]$, $L[y]$))
 \ELSIF{edge $e_1 \in E_e$}
  \STATE RaiseV($v_1$, $L[v_0]$)
 \ENDIF
 \IF{edge $e_2 \in E_e$}%[False branch]
  \STATE RaiseV($v_2$, $L[v_0]$)
 \ENDIF
\ENDFOR
\ENDFOR
               \EMPTYLINE
\STATE \COMMENT{Obvious generalization applies for tests like ``$x\not=y$''}
\STATE \COMMENT{Obvious generalization applies for tests like ``$x
 \text{ instanceof } C$''}
\end{myalgorithmic}
\caption{\code{Visit} procedure for typed SCC/SSI.}
\label{alg:scctyped}
\end{myalgorithm}
In Figure~\ref{fig:scclat3} we extend the lattice to compute Java type
information.  The new lattice entry marked ``Typed'' is actually
forest-structured as shown in Figure~\ref{fig:scclat4}; it is as deep
as the class hierarchy, and the roots and leaves are all comparable to
$\top$ and $\bot$.  Only the \code{Visit} procedure must be modified;
the new procedure is given as Algorithm~\ref{alg:scctyped}.
Because the lattice $L$ is
deeper, the asymptotic runtime complexity is now $O(E+U_{SSA}D_c)$
where $D_c$ is the maximum depth of the class hierarchy.  
To form an estimate of the magnitude of $D_c$, Table~\ref{tab:chstats}
compares class hierarchy statistics for several large
object-oriented projects in various source languages. Our FLEX
compiler infrastructure, as a typical Java example, has an average
class depth of 1.91.\footnote{Measured August 2, 1999; the
infrastructure is under continuing development.}
In a forced example, of course, one can make the class depth $O(N)$;
however, one can infer from the data given that in real code the $D_c$
term is not likely to make the algorithm significantly non-linear.

A brief word on the roots of the hierarchy forest in Figure~%
\ref{fig:scclat4} is called for: Java has both a class hierarchy,
rooted at \code{java.lang.Object}, and several primitive types, which
we will also use as roots.  The primitive types include
\code{int}, \code{long}, \code{float}, and
\code{double}.\footnote{In the type system our infrastructure uses
(which is borrowed from Java bytecode) the \code{char},
\code{boolean}, \code{short} and \code{byte} types are folded into
\code{int}.}  Integer constants in the lattice are comparable to and
less than the \code{int} or \code{long} type; floating-point constants
are likewise comparable to and less than either \code{float} or
\code{double}.  String constants are comparable to and less than the
\code{java.lang.String} non-primitive class type.

The \code{void} type, which is the type of the expression \code{null},
is also a primitive type in Java; however we wish to keep $x \meet y$
identical to $\bigsqcup_L\{x, y\}$ (the least upper bound of $x$ and
$y$) while satisfying the Java typing rule that $\code{null} \meet x = x$
when $x$ is a non-primitive type and not a constant.  This requires
putting \code{void} comparable to but less than every non-primitive
leaf in the class hierarchy lattice.

The Java class hierarchy also includes \newterm{interfaces}, which are
the means by which Java implements multiple inheritance.  Base
interface classes (which do not extend other interfaces) are additional
roots in the hierarchy forest, although no examples of this are shown
in Figure~\ref{fig:scclat4}.

Since untypeable variables are generally forbidden, no operation
should ever raise a lattice value above ``Typed'' to $\top$.  The
otherwise-unnecessary $\top$ element is retained to indicate error
conditions.

This variant of the constant-propagation algorithm allows us to
eliminate unnecessary \code{instanceof} checks due to type-casting or
type-safety checks.  Section~\ref{sec:sptc_results} will provide
experimental validation of its utility.

Finally, note that the ability to represent \code{null} as the
\code{void} type in the lattice begins to allow us to address
null-pointer checks, although because $\code{null} \meet x =x$ for
non-primitive types we can only reason about variables which can be
proven to be null, not those which might be proven to be non-null
(which is the more useful case).  The next section will provide a more
satisfactory treatment.

\subsubsection{Addressing array-bounds and null-pointer checks}
\begin{myfigure}
\centering\renewcommand{\figscale}{0.5}\input{Figures/THlat5}
\caption{Value lattice extended with array and null information.}
\label{fig:scclat5}
\end{myfigure}
\begin{myfigure}
\[\begin{array}{l}
\forall C \in \text{Class},\:
  C_{\text{non-null}} \latlt C_{\text{possibly-null}}\\
\forall C \in \text{Class}_{\text{non-null}},\:
  \bigsqcup_L \{\code{void},C\} \in \text{Class}_{\text{possibly-null}}\\
\forall C \in \text{Class}_{\text{possibly-null}},\:
  \code{void} \latlt C\\
\forall C \in \text{Class}_{\text{non-null}},\:
  \tuple{\code{void},C}\notin\:\latleq\\
\end{array}\]%
Let $A(C, n)$ be a
function to turn a lattice entry representing a non-null array class
type $C$ into the lattice entry representing a said array class with
known integer constant length $n$.  Then for any non-null array class
$C$ and integers $i$ and $j$,
\[\begin{array}{l}
A(C, i) \latlt C\\
\tuple{A(C, i), A(C, j)} \in\:\latleq \text{ if and only if } i=j\\
\end{array}\]%
\caption{Extended value lattice inequalities.}\label{fig:arraynull_rules}
\end{myfigure}
\begin{myalgorithm}\small
Visit($n$:node) =
\begin{myalgorithmic}
\STATE\COMMENT{Binop and \phifunction rules as in algorithm~\ref{alg:scctyped}}
               \EMPTYLINE
\FOREACH{assignment ``$v \gets \text{MEM}(\ldots)$'' or 
                    ``$v \gets \text{CALL}(\ldots)$'' in $n$}
 \LET $t\in\text{Class}_{\text{possibly-null}}\cup\text{Class}_{\text{primitive}}$
      be the type of the \code{MEM} or \code{CALL}% expression
 \STATE RaiseV($v$, $t$)
\ENDFOR
               \EMPTYLINE
\FOREACH{array creation expression ``$v \gets \text{new T}[x]$''}
 \IF{$L[x]$ is an integer constant}
  \STATE RaiseV($v$, $A(T, L[x])$)
 \ELSE
  \STATE RaiseV($v$, $T_{\text{non-null}}$)
 \ENDIF
\ENDFOR
               \EMPTYLINE
\FOREACH{array length assignment ``$v \gets \text{arraylength}(x)$''}
 \IF{$L[x]$ is an array of known constant length $n$}
  \STATE RaiseV($v$, $n$)
 \ELSE
  \STATE RaiseV($v$, \code{int})
 \ENDIF
\ENDFOR
               \EMPTYLINE
\STATE \COMMENT{Branch rules as in algorithm~\ref{alg:scctyped}, with
               the obvious extension to allow tests against
               \code{null} to lower a lattice value from
               $\text{Class}_{\text{possibly-null}}$ to
               $\text{Class}_{\text{non-null}}$.}
\end{myalgorithmic}
\caption{\code{Visit} procedure outline with array and null information.}
\label{alg:arraynull_scc}
\end{myalgorithm}
\begin{myfigure}
\begin{samplecode}
x = 5 + 6;\\
do \{\\
\>y = new int[x];\\
\>z = x-1;\\
\>if (0 <= z \&\& z < y.length)\\
\>\>y[z] = 0;\\
\>else\\
\>\>x--;\\
\} while (P);\\
\end{samplecode}
\caption{An example illustrating the power of combined analysis.}
\label{fig:combined}
\end{myfigure}
At this point, we can expand the value lattice once more to allow
elimination of unnecessary array-bounds and null-pointer checks, based
on our constant-propagation algorithm.  The new lattice is shown in
Figure~\ref{fig:scclat5}; we have split the ``Typed'' lattice entry to
enable the algorithm to distinguish between non-null and possibly-null
values,\footnote{Values which are always-null were discussed in the
previous section; they are identified as having primitive type \code{void}.}
and added a lattice level for arrays of known constant length.  
Some formal definition of the new value lattice can be found in
Figure~\ref{fig:arraynull_rules}; the meet rule is still the least upper
bound on the lattice.  Modifications to the \code{Visit} procedure are
outlined in Algorithm~\ref{alg:arraynull_scc}.
Notice that we exploit the pre-existing integer-constant propagation to
identify constant-length arrays, and that our integrated approach
allows one-pass optimization of the program in Figure~\ref{fig:combined}.

\begin{myfigure}\newcommand{\implicitcheck}[1]{\underline{#1}}
\begin{samplecode}
\implicitcheck{if (10 < 0)}\\
\>\implicitcheck{throw new NegativeArraySizeException();}\\
int[] A = new int[10]; \\
\implicitcheck{if (0 < 0 || 0 >= A.length)}\\
\>\implicitcheck{throw new ArrayIndexOutOfBoundsException();}\\
A[0] = 1;\\
for (int i=1; i < 10; i++) \{\\
\>\implicitcheck{if (i < 0 || i >= A.length)}\\
\>\>\implicitcheck{throw new ArrayIndexOutOfBoundsException();}\\
\>A[i] = 0;\\
\}\\
\end{samplecode}
\caption[Implicit bounds checks on Java array references.]
{Implicit bounds checks (underlined) on Java array references.}
\label{fig:induction}
\end{myfigure}
Note that the variable renaming performed by the SSI form at
control-flow splits is essential in allowing the algorithm to do 
null-pointer check elimination.  However, the lattice we are using
can remove bound checks from an expression $A[k]$ when $k$ is a
constant, but not when $k$ is an bounded induction variable.
In the example of Figure~\vref{fig:induction}, the first two implicit
checks are optimized away by this version of the algorithm, but the
loop-borne test is not. 

A typical array-bounds check (as shown in the example \vpageref{fig:induction})
verifies that the index $i$ of the array reference satisfies the
condition $0\leq i < n$, where $n$ is the length of the
array.\footnote{Languages in which array indices start at 1 can be
handled by slight modifications to the same techniques.} By
identifying integer constants as either positive, negative, or zero
the first half of the bounds check may be eliminated.  This requires a
simple extension of the integer constant portion of the lattice,
outlined in Figure~\vref{fig:scclat6}, with
negligible performance cost.  However, handling upper bounds
completely requires a symbolic analysis that is out of the current
scope of this work.  Future work will use induction variable analysis
and integrate an existing integer linear programming approach
\cite{rugina99:divide} to fully address array-bounds checks.
\begin{myfigure}[t]
\centering\renewcommand{\figscale}{0.6}\input{Figures/THlat6}
\caption[An integer lattice for signed integers.]
{An integer lattice for signed integers. A classification into
negative (M), positive (P), or zero (Z) is grafted onto the standard
flat integer constant domain.  The \code{(M-P)} entry is duplicated to
aid clarity.}
\label{fig:scclat6}
\end{myfigure}

\subsubsection{Experimental results}\label{sec:sptc_results}
The full SPTC analysis and optimization has been implemented in the
FLEX java compiler platform.\footnote{See
section~\ref{sec:methodology} for details of methodology.}
Some quantitative measure of the utility
of SPTC is given as Figure~\ref{fig:sptc_numbers}.  The ``run-times''
given are intermediate representation dynamic statement counts
generated by the FLEX compiler SSI IR interpreter.  The FLEX
infrastructure is still under development, and its backends are not
stable enough to allow directly executable code.  As such, the numbers
bear a tenuous relation to reality; in particular branch delays on
real architectures, which the elimination of null-pointer checks seeks
to eliminate, are unrepresented.  Furthermore, the intermediate
representation interpreter gives the same cycle-count to two-operand
instructions as to loading constants, which tends to negate most of
the benefit of constant propagation.  As is obvious from the figure,
the standard Wegman-Zadeck SCC algorithm, which has proven utility in
practice, shows no improvement over unoptimized code due to the metric
used.  Even so, SPTC shows a 10\% speed-up.  It is expected that the
improvement given in actual practice will be greater.
\begin{myfigure}
\centering
\includegraphics[scale=0.4,angle=-90,clip=true,trim=1cm 0 0 0]%
{Figures/THscccomp.eps}
\caption{SPTC optimization performance.}
\label{fig:sptc_numbers}
\end{myfigure}

Note that the speed-up is constant despite widely differing test
cases.  The ``Hello world'' example actually executes quite a bit of
library code in the Java implementation; this includes numerous
element-by-element array initializations (due to the semantics of
java bytecode) which we expect SPTC to excel at optimizing.  But SPTC
does just as well on the full FLEX compiler (68,032 lines of source
at the time the benchmark was run), which shows that the speed-up is
not limited to constant initialization code.

\subsection{Bit-width analysis}\label{sec:bitwidth}
The SPTC algorithm can be extended to allow efficient
\newterm{bit-width analysis}.  Bit-width analysis is a variation of
constant propagation with the goal of determining value ranges for
variables.  In this sense it is similar to, but simpler than,
array-bounds analysis: no symbolic manipulation is required and the
value lattice has $N$ levels (where $N$ is the maximum bitwidth of the
underlying datatype) instead of $2^N$.  For C and Java programs, this
means that only 32 levels need be added to the lattice; thus the
bit-width analysis can be made efficient.

Bit-width analysis allows optimization for modern media-processing
instruction set extensions which typically offer vector processing of
limited-width types. Intel's MMX extensions, for example, offer packed
8-bit, 16-bit, 32-bit and 64-bit vectors \cite{peleg97:mmx}.
To take advantage of these functional units without explicit human
annotation, the compiler must be able to guarantee that the data in a
vector can be expressed using the limited bit-width available.  A
simpler bit-width analysis in a previous work \cite{ananian:siliconc}
showed that a large amount of width-limit information can be extracted
from appropriate source programs; however, that work was not able to
intelligently compute widths of loop-bound variables due to the
limitations of the SSA form.  Extending the bitwidth algorithm to SSI
form allows induction variables width-limited by loop-bounds to be
detected.

Bit-width analysis is also a vital step in compiling a high-level
language to a hardware description.  General purpose programming
languages do not contain the fine-grained bit-width information that a
hardware implementation can take advantage of, so the compiler must
extract it itself.  The work cited showed that this is viable
and efficient.

\begin{myfigure}
\begin{eqnarray*}
-\tuple{M,P} &=& \tuple{P,M}\\
\tuple{M_l,P_l} + \tuple{M_r,P_r} &=& \tuple{1+\max(M_l,M_r),1+\max(P_l,P_r)}\\
\tuple{M_l,P_l} \times \tuple{M_r,P_r} &=& \tuple{\max(M_l+P_r,P_l+M_r),
                                             \max(M_l+M_r,P_l+P_r)}\\
\tuple{0,P_l} \wedge \tuple{0,P_r} &=& \tuple{0,\min(P_l,P_r)}\\
\tuple{M_l,P_l}\wedge \tuple{M_r,P_r} &=& \tuple{\max(M_l,M_r),\max(P_l,P_r)}
\end{eqnarray*}%
\caption{Some combination rules for bit-width analysis.}\label{fig:bitrules}
\end{myfigure}
The bit-width analysis algorithm has been implemented in the FLEX
compiler infrastructure.  Because most types in Java are signed, it is
necessary to separate bit-width information into ``positive width''
and ``negative width.''  This is just an extension of the
signed value lattice of Figure~\ref{fig:scclat6} to variable
bit-widths.  In practice the bit-widths are represented by a tuple,
extending the integer constant lattice with
$(\domain{Int}\times\domain{Int})_{\bot}$ under the natural total
ordering of \domain{Int}.  The tuple $\tuple{0,0}$ is identical to the
constant 0, and the tuple $\tuple{0,16}$ represents an ordinary
unsigned 16-bit data type.  The $\top$ element is represented by an
appropriate tuple reflecting the source-language semantics of the
value's type.  Figure~\ref{fig:bitrules} presents bit-width
combination rules for some unary negation and binary addition,
multiplication and bitwise-and.  In practice, the rules would be
extended to more precisely handle operands of zero, one, and other
small constants.

\dontfixme{
\section{SSI is stronger than SSA}
In this section we will show that SSI is strictly more expressive than
SSA, by showing how an efficiently-generated name mapping can be used
to apply any SSA-based algorithm to a program in SSI form.  This, and
the existence of algorithms such as SPTC and even simple liveness
analysis which cannot be performed efficiently in SSA form, means that
the class of efficient SSI-based algorithms is a proper superset of
those efficient using SSA forms.

Although one could apply an SSA-based algorithm to a program in SSI
form by converting out of SSI and into SSA forms, using the fairly
efficient algorithms in this thesis and the literature, doing so
negates the advantages of maintaining the program in SSI form in the
first place.  Furthermore, properties derived from the SSI form
representation may not survive the translation process.  SSI form
provides an extremely elegant solution: a simple variable name mapping
is all that is required.  The SSA-based algorithm procedes using
mapped versions of the variable names and ignoring all
\sigfunction{s}.  Resulting SSA-derived variable properties may be
easily and efficiently mapped back onto their corresponding SSI
variables using the same name mapping. Note that the SSI form requires
a superset of the \phifunction{s} required in SSA form, due to
\sigfunction{s} counting as definitions in the $\phi$-placement
condition in SSI form.  So all SSI \phifunction{s} may be treated as
SSA \phifunction{s} without difficulty.  The other requirements of SSA
form, including single reaching definitions for uses and definitions
dominating all uses are maintained in SSI form, as the various
properties derived in section~\ref{sec:ssidef} showed.  In fact, 
simply replacing \sigfunction{s} by assignments on either side of a
control-flow branch would generate a valid SSA form program.

This simple replacement, aside from violating the integrity of the
SSI form, adds unnecessary \code{move}s to the SSA form which
interrupt dataflow and may impact the precision of some algorithms.
Moreover, it is not clear that the extra assignments don't adversely
impact the running-time of the SSA-based algorithm.

In addition, simple \sigfunction{}-renaming leaves ``extra''
\phifunction{s} in the SSA-view of the program, which merge identical
values created at (invisible) \sigfunction{s}.  These useless
\phifunction{s} take the form $x\gets\phi(y,y,\ldots)$ or, in the
case of a loop, $x\gets\phi(x,y,\ldots)$: the operands are either
identical to the destination or identical to each other.

\begin{myalgorithm}\small
ConstructMap($G$:~CFG):~variable$\rightarrow$variable =
\begin{myalgorithmic}
\STATE $N\gets \mathbf{I}_V$
       \COMMENT{$N$ initially maps every variable to itself}
\STATE\COMMENT{elide \sigfunction{s} via mapping $N$}\label{line:sigelide1}
\FOREACH{\sigfunction{} $s$ in $G$}
 \FOREACH{destination variable $v_d$ corresponding to a source operand $v_s$}
  \STATE $N \gets N[v_d \rightarrow v_s]$
         \COMMENT{add a definition to source mapping}\label{line:sigelide2}
 \ENDFOR
\ENDFOR
\STATE\COMMENT{now prune redundant \phifunction{s}}\label{line:phiprune1}
\FOREACH{variable $v$ in $G$}
 \STATE $\text{MakeSet}(N[v])$
\ENDFOR
\FOREACH{\phifunction{} $P$, in reverse depth-first order}
 \STATE $\text{CheckOne}(P)$\label{line:phiprune2}
\ENDFOR
\STATE\COMMENT{construct final mapping from union-find sets}
\FOREACH{variable $v$ in $G$}
 \STATE $N \gets N[v \rightarrow \text{FindSet}(N[v])]$
\ENDFOR
\RETURN $N$
\end{myalgorithmic}
~\\
CheckOne($P$:~\phifunction{}) =
\begin{myalgorithmic}
\LET $v$ be the variable defined by \phifunction{} $P$
\STATE $V\gets \text{FindSet}(v)$
       \COMMENT{defs don't have to be mapped by $N$}
\STATE $S\gets \{ V \}$ \COMMENT{set of union-find sets}
\FOREACH{operand $v_i$ of $P$}
 \STATE $V_i\gets \text{FindSet}(N[v_i])$
 \STATE $S\gets S \cup \{ V_i \}$
\ENDFOR
\IF[it's redundant:  merge it]{$|S| \leq 2$}
 \FOREACH{operand $v_i$ of $P$}
  \STATE $\text{Union}(v, N[v_i])$
 \ENDFOR
\ENDIF
\RETURN
\end{myalgorithmic}
\caption{SSI to SSA name mapping algorithm.}
\label{alg:ssi2ssa}
\end{myalgorithm}
Algorithm~\ref{alg:ssi2ssa} uses a two-phase approach and proper
ordering to remove both \sigfunction{s} and phantom \phifunction{s} to
create a useful SSA name mapping of an SSI form CFG.  If the SSI form
is pruned or minimal, the generated SSA view will be as well.

Lines~\ref{line:sigelide1} to \ref{line:sigelide2} of the
\code{ConstructMap} function rename variables in the
name-map-under-construction $N$ to remove \sigfunction{s}.
Lines~\ref{line:phiprune1} to \ref{line:phiprune2} prune
\phifunction{s} made redundant as a result.

Note that as \sigfunction{s} are removed the newly-redundant
\phifunction{s} may exist in any dominance relation to the
\sigfunction{}, as the guarantees of property~\ref{pty:ssi_dom} do not
hold for definitions at \sigfunction{s}.  However, \phifunction{s} can
only be made newly-redundant during the $\phi$-pruning stage of the
algorithm by the removal of \phifunction{s} which dominate them, as
every use post-dominates a \phifunction{} definition.

Note that as \sigfunction{s} are removed, unused \phifunction{s}
appear ``below'' them in the CFG, as the variables defined in the
\sigfunction{} must dominate (by the SSI constraints) any
newly-redundant use.  Contrariwise, any \phifunction{s} found
redundant causes ``higher'' \phifunction{s} to become redundant, as
the redundant use\ldots
%
In this section we will show a $O(N\alpha(N))$ algorithm for
efficiently convertin\ldots %um, no.
} % sorry, folks, that algorithm don't work.

\section{An executable representation}\label{sec:ssiplus}
The Static Single Information (SSI) form, as presented in the first
half of this thesis,
requires control-flow graph information in order to be executable. We
would like to have a demand-driven operational semantics for SSI form
that does not require control-flow information; thus freeing us to
more flexibly reorder execution.

In particular, we would like a representation that eliminates
unnecessary control dependencies such as exist in the program of
Figure~\vref{fig:ctrldep}.  A control-flow graph for this program, as
it is written, will explicitly specify that no assignments to
\code{B[]} will take place until all elements of \code{A[]} have
been assigned; that is, the second loop will be
\emph{control-dependent} on the first.  We would like to remove this
control dependence in order to provide greater parallelism---in this
case, to allow the assignments to \code{A[]} and \code{B[]} to
take place in parallel, if possible.

\begin{myfigure}[t]
\begin{samplecode}
for (int i=0; i<10; i++)\\
\>A[i] = x;\\
for (int j=0; j<10; j++)\\
\>B[j] = y;\\
\end{samplecode}
\caption[An example of unnecessary control dependence.]
{An example of unnecessary control dependence: the second loop
is \emph{control-dependent} on the first and so assignments to
\code{A[]} and \code{B[]} cannot take place in parallel.}
\label{fig:ctrldep}
\end{myfigure}

In addition, an executable representation allows us to more easily
apply the techniques of abstract interpretation \cite{pingali90:dfg}.
Although abstract interpretation may be applied to the original SSI
form using information extracted from the control flow graph, an
executable SSI form allows more concise (and thus, more easily derived
and verified) abstract interpretation algorithms.

The modifications outlined here extend SSI form to
provide a useful and descriptive operational semantics.  We will call
the extended form \ssiplus.  For clarity, SSI form as originally
presented we will call \ssizero.  We will describe algorithms to
contruct \ssiplus{} efficiently,\dontfixme{I will make this more precise
as soon as I get the math done; construction should be $O(N)$, and space
$O(N^2)$, but $\Theta(N)$.} and illustrate analyses and
optimizations using the form.

\subsection{Deficiencies in \ssizero}
Although a demand-driven execution model can be constructed for
\ssizero,  it fails to handle loops and imperative
constructs well. \ssiplus\ form addresses these deficiencies.

\subsubsection{Imperative constructs, pointer variables, and side-effects}
The presentation of \ssizero\ ignored pointers,
concentrating on so-called register variables.  Extending \ssizero\ to
handle these imperative constructs is quite easy: we simply define a
``variable'' $S$ to represent an updatable store.  This variable is
renamed and numbered as before, so that $S_0$ represents the initial
contents of the store and $S_i, i>0$ represents the contents of the
store after some sequence of writes.  Figure~\ref{fig:store} shows a
simple imperative program in \ssiplus\ form.  Note that modifications
to the store typically take the previous contents of the store as
input, and that subroutines with side-effects modifying the store must
be written in \ssiplus\ form such that they both take a store and
return a store.

\begin{figure}[t]
\begin{samplecode}[2]
\com{swap A[i] and B[j]} & \com{\ssiplus\ form:}\\
x = A[i];	& $x_0$ = FETCH($S_0$, $A_0 + i_0$) \\
y = B[j];	& $y_0$ = FETCH($S_0$, $B_0 + j_0$) \\
A[i] = y;	& $S_1$ = STORE($S_0$, $A_0 + i_0$, $y_0$); \\
B[j] = x;	& $S_2$ = STORE($S_1$, $B_0 + j_0$, $x_0$); \\
\end{samplecode}
\caption{Use of the ``store variable'' $S_x$ in \ssiplus\ form.}
\label{fig:store}
\end{figure}

The single monolithic store may provide aliasing at too coarse a
resolution to be useful.  Decomposing the store into smaller regions
is a straight-forward application of pointer analysis, which may
benefit from an initial conversion of register variables to \ssizero\
form.   In type-safe languages, defining multiple stores for differing
type sets is a trivial implementation of basic pointer analysis;
Figure~\ref{fig:manystore} shows a simple example of this form of
decomposition using  two different subtypes (\texttt{Integer} and
\texttt{Float}) of a common base class (\texttt{Number}).
Pointer analysis is a huge and rapidly-growing field which we cannot
attempt to summarize here; suffice to say that the \emph{may-point-to}
relation from pointer analysis may be used
to define a fine-grained model of the store.

\begin{figure}[t]
\centering$\vdash N:\mathbf{Number},I:\mathbf{Integer},F:\mathbf{Float}$\\
\centering$I \subset N \mbox{ and } F \subset N$\\
\begin{samplecode}[2]
if(P)           & \com{\ssiplus\ form:}\\
\>N=I;\\
else            & $N_0=\phi(I_0,F_0)$ \\
\>N=F;\\
F.add(3.14159); & $S^F_1=\mbox{\tt CALL}(\mbox{\tt add},S^F_0,F_0,3.14159)$\\
N.add(5);       & $\tuple{S^I_1,S^F_2}=
                         \mbox{\tt CALL}(\mbox{\tt add},S^I_0,S^F_1,N_0,5)$\\
\end{samplecode}
\caption{Factoring the store ($S_x$) using type information
         in a type-safe language.}
\label{fig:manystore}
\end{figure}

Proper sequencing among statements with side-effects may be handled in
a similar way: a special SSI name is used/defined where side-effects
occur to impose an implicit ordering.  For maximum symmetry with the
`store' case, we will name this special variable $S^{fx}$.  This
variable may be further decomposed using effect analysis for more
precision.

\begin{figure}[t]
\begin{samplecode}[2]
int x=1;	& $x_0 = 1$ \\
int y=2;	& $y_0 = 2$ \\
int *p = \&x; 	& $p_0 = \set{\mbox{x}}$ \com{P is of type ``location set''} \\
if (P)  	& \\%\com{Some \sigfunction{} goes here?} \\%$\tuple{S^{fx}_1,S^{fx}_2}=\sigma(P, S^{fx}_0)$ \\
\>p = \&y;	& $p_1 = \set{\mbox{y}}$ \\
		& $p_2 = \phi(p_0, p_1)$ \\
*p = 3; 	& $\tuple{x_1,y_1} = \mbox{\tt DEREF}(p_2, 3)$\\
return x;	& $\mbox{\tt return } x_1$ \\
\end{samplecode}
\caption{Pointer manipulation of local variables in C.}
\label{fig:messyC}
\end{figure}

Note that precise analysis of side-effects and the store is much more
important in C-like languages. The
example on the left in Figure~\ref{fig:messyC} shows the difficulties
one may encounter in dealing with pointer variables that may rewrite
SSI temporaries.  It is possible to deal with this in the manner of
Figure~\ref{fig:manystore} using explicit stores,
and with sufficient analysis one may write the SSI
representation on the right in the figure.  The source language for
our FLEX compiler does not encounter this difficulty:  Java has no
pointers to base types, and so the compiler does not have to worry
about values changing ``behind its back'' as in the example.

\subsubsection{Loop constructs}
\begin{figure}[t]
\begin{samplecode}[3]
\com{a simple loop} & \com{\ssizero\ form:} & \com{\ssiplus\ form:}\\
j=1;	& \>$j_0 = 1$			& $j_0=1$\\
i=0;	& \>$i_0 = 0$			& $i_0=0$\\
do	& L1:			& $\xivec{j_1}{i_5}=\xi(\xivec{j_0}{i_3})$\\
$\{$	& \>$i_1 = \phi(i_0, i_3)$	& $i_1 = \phi(i_0, i_5)$ \\
\>i+=j; & \>$i_2 = i_1 + j_0$		& $i_2 = i_1 + j_1$ \\
$\}$ while (i<5);	& \>$P_0=(i_2<5)$	& $P_0=(i_2<5)$\\
	& \>if $P_0$ goto L1		& $\tuple{i_3,i_4}=\sigma(P_0, i_2)$ \\
	& \>~~$\tuple{i_3,i_4}=\sigma(i_2)$ &\\
\end{samplecode}
\caption{A simple loop, in \ssizero\ and \ssiplus\ forms.}
\label{fig:loop}
\end{figure}

The center column of Figure~\vref{fig:loop} shows a typical loop in
\ssizero\ form.  Note first that an explicit ``control flow''
expression (\texttt{goto L1}) is required in order to make sense of
the program.  Note also that $i_1$, $i_2$ and $i_3$ are potentially
\emph{dynamically} assigned many times, although \emph{statically}
they have only one definition each.  This complicates any sort of
demand-driven semantics: should the \phifunction{} demand the value of
$i_0$, or $i_3$, when it is evaluated the first time?  Which of the
values of $i_3$ does it receive when the \phifunction{} is
subsequently evaluated?  A token-based dataflow interpretation fails
as well: it is easy to see that tokens for $i_x$ flow around the loop
before flowing out at the end, but the token for $j_0$ seems to be
``used up'' in the first iteration.

\ssiplus\ introduces a \xifunction\ in the block of \phifunction{s} to
clarify the loop semantics. The left-hand column of Figure~%
\ref{fig:loop} illustrates the nature of this function.  The
\xifunction{} arbitrates loop iteration, and will be defined precisely
by the operational semantics of \ssiplus\ form.  For now note that it
relates iteration variables (the top tuple of the parameter and result
vectors) to loop invariants (the bottom tuple of the vectors).  We
followed the statement ordering of \ssizero\ in the figure, but unlike
\ssizero, the statements of \ssiplus\ could appear in any order
without affecting their meaning---and so the statement label
\texttt{L1} of the \ssizero\ representation and its implicit
control-flow edge are unnecessary in \ssiplus.

\subsection{Definitions}

The signature characteristic of \ssiplus\ are the \xifunction{s}.
These \xifunction{s} exist in the same places \phifunction{s} do, and
control loop iteration.  The exact semantics may vary---the sections
below present two different valid semantics for a \xifunction{s}---but
informally they can be viewed as ``time-warp'' operators.  They take
values from the ``past'' (previous iterations of the loop or loop
invariants valid when the loop began) and project them into the
``future'' (the current loop iteration).

There is at most one \xifunction{} per \phifunction{} block, and it
always precedes the \phifunction{s}.  Construction of \xifunction{s}
takes place before the renaming step associated with SSI form, and the
\xifunction{s} are then renamed in the same manner as any other
definition.  The top tuple of the constructed \xifunction{} contains
the names of all variables reaching the guarded \phifunction{} via a
backedge, and the bottom tuple contains all variables used inside the
guarded loop that are {\em not} mentioned in the header's
\phifunction{}.

The \ssiplus\ form also has \newterm{triggered constants}.  The
time-oriented semantics of \ssiplus\ dictate that each constant must
be associated with a trigger specifying for what times (cycles/loop
iterations) the value of the constant is valid.  These are similar to
the constant generators in some dataflow machines \cite{traub86:ttda}.
The triggers for a constant $c$ come from the variables defined in the
earliest applicable instruction post-dominated by the constant
definition statement $v=c$.  This is designed to generate the trigger
as soon as it is known that the constant definition statement will
always execute.  In practice it is necessary to introduce a bogus
\newterm{trigger variable}, $\mathcal{C}_T$ which is generated at the
\code{START} node and used to trigger any constants otherwise without a
suitable generator.  If the use of the constant does not post-dominate
the \code{START} node, $\mathcal{C}_T$ will have to be threaded
through \phisigfunction{s} to reach the earliest post-dominated node.

\ignore{I was supposed to talk about CALLs here?}

\subsection{Semantics}\label{sec:semantics}
We will base the operational semantics of \ssiplus\ on a demand-driven
dataflow model.  We will define both a cycle-oriented semantics and an
event-driven semantics, which (incidentally) correspond to synchronous
and asynchronous hardware models.

Following the lead of Pingali \cite{pingali90:dfg}, we present Plotkin-style
semantics \cite{plotkin81:opsem} in which \emph{configurations} are
rewritten instead of programs.  The configurations represent program
state and transitions correspond to steps in program execution.  The
set of valid transitions is generated from the program text.

The semantics operate over a lifted value domain
$\domain{V}=\domain{Int}_\bot$. When some variable
$t = \bot_\domainsm{V}$ we say it is
\emph{undefined}; conversely $t\sgt\bot_\domainsm{V}$ indicates that the
variable is \emph{defined}.  ``Store'' metavariables $S_x$ are not
explicitly handled by the semantics, but the extension is trivial with
an appropriate redefinition of the value domain $\domain{V}$.  Floating-point
and other types are also trivial extensions.  The
metavariables $c$ and $v$ stand for elements of $\domain{V}$.

We also define a domain of \emph{variable names},
$\domain{Nam}=\set{n_0,n_1,\ldots}$.  The metavariables $t$ and $P$ stand for
elements in $\domain{Nam}$, although $P$ will be reserved for naming branch predicates.

A fixed set of ``built-in'' operators, \textbf{op}, is defined,
of type $\domain{V}^* \to \domain{V}$.  If any operator argument is $\bot$, the
result is also $\bot$.  Constants are implemented as a special case of
the general operator rule: an \textbf{op} producing a constant has a
single trigger input which does not affect the output.

\subsubsection{Cycle-oriented semantics}
\begin{myfigure}[t]
\begin{transitions}
t=\mathbf{op}(t_1,\ldots,t_n):
& \trule{\rho[t]=\bot \wedge 
         \left(
          \rho[t_1]\sgt\bot \wedge \ldots \wedge \rho[t_n]\sgt\bot
	 \right)}
	{\rho \to \rho[t \mapsto
		         \mathbf{op}(\rho[t_1],\ldots,\rho[t_n])] } \\

t=\phi(t_1,\ldots,t_n):
& \trule{\rho[t]=\bot \wedge
         \rho[t_j]\sgt\bot \wedge
         \text{all other}\,\rho[t_1],\ldots,\rho[t_n]=\bot}
        {\rho \to \rho[t \mapsto \rho[t_j]] } \\

\tuple{t_1,\ldots,t_n}=\sigma(P,t):
& \myarray{r}{
  \trule{\rho[P]=v\sgt\bot \wedge
         \rho[t_{v-1}]=\bot \wedge
         \rho[t]\sgt\bot
         %\text{ where } (0\leq v\leq n-1)
	}
	{\rho \to \rho[t_{v-1} \mapsto \rho[t]] }
  \quad \hfill \\ \footnotesize \mbox{where } (0\leq v\leq n-1) }
 \\

\footnotesize % latex complains, but does the right thing.
\xivec{t_1,\ldots,t_n}{t_{n+1},\ldots,t_{m}}=\xi(\xivec{t'_1,\ldots,t'_n}{t'_{n+1},\ldots,t'_m}):
& \myarray{r}{
  \trule{\rho[t_j]=\bot \wedge
         \rho[t'_j]\sgt\bot %\mbox{ where } (1\le j\le n)}
	}
        {\rho \to \rho[t_j \mapsto \rho[t'_j]] }
  \qquad\qquad\qquad \hfill \\ \footnotesize \mbox{where } (1\le j\le n)}
 \\

\footnotesize % latex complains, but does the right thing.
\xivec{t_1,\ldots,t_n}{t_{n+1},\ldots,t_{m}}=\xi(\xivec{t'_1,\ldots,t'_n}{t'_{n+1},\ldots,t'_m}):
& \footnotesize % latex complains, but does the right thing.
  \trule{\rho[t'_{n+1}]\sgt\bot \wedge \ldots \wedge \rho[t'_m]\sgt\bot}
        {\myarray{r @{} l}{
         \rho \to
%	 \hfill \\ \: % line-break
         \rho_\emptyset &
                [t_1 \mapsto \rho[t_1]]\ldots[t_n \mapsto \rho[t_n]]
		\hfill \\ & \quad % line-break
                [t_{n+1} \mapsto \rho[t'_{n+1}]]\ldots[t_m \mapsto \rho[t'_m]]
		} } \\
\end{transitions}
\caption{Cycle-oriented transition rules for \ssiplus.}
\label{fig:cyclesemantics}
\end{myfigure}

In the cycle-oriented semantics, configurations consist of an
\emph{environment}, $\rho$, which maps
names in $\domain{Nam}$ to values in $\domain{V}$.

\begin{definition}~\\*[-1\baselineskip]
\begin{enumerate}
\item An \emph{environment} $\rho: \domain{N} \to \domain{V}$ is a
finite function---its domain $\domain{N} \subseteq \domain{Nam}$ is
finite.  The notation $\rho[t\mapsto c]$ represents an environment
identical to $\rho$ except for name $t$ which is mapped to $c$.
\item The null environment $\rho_\emptyset$ maps every $t\in\domain{N}$ to
$\bot_\domainsm{V}$.
\item A \emph{configuration} consists of an environment.  The initial
configuration is $\rho_\emptyset[\mathcal{C}_T\to 0]$ extended with
mappings for procedure parameters.  That is, all names in $\domain{N}$
are mapped to $\bot_\domainsm{V}$ except for the default constant trigger
$\mathcal{C}_T$ mapped to 0,\footnote{Any $k\sgt\bot_\domainsm{V}$
would do.} and any procedure parameters mapped to their proper entry values.
\end{enumerate}
\end{definition}

Figure~\vref{fig:cyclesemantics} shows the cycle-oriented transition
rules for \ssiplus\ form.  The left column consists of definitions and
the right column shows a precondition on top of the line, and a
transition below the line.  If the definition in the left column is
present in the \ssiplus\ form and the precondition on top of the line
is satisfied, then the transition shown below the line can be performed.

\subsubsection{Event-driven semantics}
\begin{myfigure}[t]\small
\begin{transitions}
t=\mathbf{op}(t_1,\ldots,t_n):
& \tuple{E[t_1=v_1]\ldots[t_n=v_n],S} \to \tuple{E[t=\mathbf{op}(v_1,\ldots,v_n)],S}\\

t=\phi(t_1,\ldots,t_n):
& \tuple{E[t_i=v],S} \to \tuple{E[t=v],S}\\

\tuple{t_1,\ldots,t_n}=\sigma(P,t):
& \tuple{E[t=v][P=i],S} \to \tuple{E[t_i=v],S}\\

\xivec{t_1,\ldots,t_n}{t_{n+1},\ldots,t_m}=\xi_K(\xivec{t'_1,\ldots,t'_n}{t'_{n+1},\ldots,t'_m}):
& \myarray{r}{
  \tuple{E[t'_i=v],S} \to \hfill\\\quad\quad\quad\quad\quad
  \tuple{E[t_i=v],S[K\mapsto S[K]\cup\tuple{t_i,v}]}\\
   \text{where }1\le i \le n\\
%  \mbox{where }K\mbox{ is a unique constant corresponding to}\\
%  \mbox{this \ssiplus\ statement}\\
  }\\

\xivec{t_1,\ldots,t_n}{t_{n+1},\ldots,t_m}=\xi_K(\xivec{t'_1,\ldots,t'_n}{t'_{n+1},\ldots,t'_m}):
& \trule{S[K]=\left\{\tuple{t_1,v_1},\ldots,\tuple{t_n,v_n}\right\}}
  {\myarray{r}{
   \tuple{E[t'_{n+1}=v_{n+1}]\ldots[t'_m=v_m],S} \to\quad\quad\quad\quad\\
   \tuple{E[t_1=v_1]\ldots[t_m=v_m],S}\\
%   \mbox{where }S[K]=\bigcup_{i=1}^n \{\tuple{t_i,v_i}\}\\
%   \mbox{where }S[K]=\left\{\tuple{t_1,v_1},\ldots,\tuple{t_n,v_n}\right\}\\
  } }
\end{transitions}
\caption[Event-driven transition rules for \ssiplus.]
{Event-driven transition rules for \ssiplus.  In the last two rules $K$ is a
statement-identifier constant which is unique for each source \xifunction.}
\label{fig:eventsemantics}
\end{myfigure}

In the event-driven semantics, configurations consist of an
\emph{event set} and an \emph{invariant store}.  The event set
$E$ contains definitions of the form $t=c$,
and the invariant store is a mapping from numbered \xifunction{s} in
the source \ssiplus\ form to a set of tuples representing saved values
for loop invariants.

We define the following domains:
\begin{itemize}
\item $\domain{Evt} = \domain{Nam} \times \domain{V}$ is the event
domain.  An event consists of a name-value pair.  The metavariable $e$
stands for elements of $\domain{Evt}$.
\item $\domain{Xif} \subset \domain{Int}$ is used to number
\xifunction{s} in the source \ssiplus\ form.  There is some mapping
function which relates \xifunction{s} to unique elements of
$\domain{Xif}$.  The metavariable $K$ stands for an element in
$\domain{Xif}$.
\end{itemize}

A formal definition of our configuration domain is now possible:
\begin{definition}~\\*[-1\baselineskip]
\begin{enumerate}
\item An \emph{event set} $E:\domain{Evt}^*$.
The notation $E[t=c]$ represents an event set
identical to $E$ except that it contains the event $\tuple{t,c}$.  We
say a name $t$ is \emph{defined} if $\tuple{t,v} \in E$ for some $v$.
For all $\tuple{t_1,v_1},\tuple{t_2,v_2} \in E$, $t_1$ and $t_2$
differ.  This is equivalent to saying that no name $t$ is multiply
defined in an event set.  This constraint is enforced by the
transition rules, not by the definition of $E$.
\item An \emph{invariant store} $S: \domain{Xif} \to
\domain{Evt}^*$ is a finite mapping from \xifunction{s} to event sets.
\item A \emph{configuration} is a tuple
$\tuple{E, S}:\domain{Evt}^* \times (\domain{Xif}\to\domain{Evt}^*)$ consisting
of an event set and an invariant store.  The initial
configuration for procedure parameters $p_0,\ldots,p_n$ mapped to
non-$\bot$ values $v_0,\ldots,v_n$ is
$\tuple{\{\mathcal{C}_T=0,p_0=v_0,\ldots,p_n=v_n\}_{\domainsm{Evt}},
        []_{\domainsm{Xif} \to \domainsm{Evt}^*}}$
that is, it consists of an empty event set%
\ignore{\footnote{Read, ``who cares about Maria'' \cite{marinov99}.}}
extended with events for default constant trigger $\mathcal{C}_T$
and the procedure parameters,
and an empty mapping for the invariant store.%
\end{enumerate}
\end{definition}

Figure~\vref{fig:eventsemantics} shows the event-driven transition
rules for \ssiplus\ form.  As before, the left column consists of
definitions and the right column shows an optional precondition above
a line, and a transition.  If the definition in the left column is
present in the \ssiplus\ form and the precondition (if any) above the
line is satisfied, then the transition can be performed.  Note that
most transitions remove some event from the event set $E$, replacing
it with a new event.  The invariant store $S$ stores
the values of loop invariants for regeneration at each loop iteration.

\subsection{Construction}
Construction of \ssiplus\ is only a slight variation on the
construction algorithms for \ssizero.  First, dominator and
post-dominator trees are produced using the Lengauer-Tarjan
\cite{lengauer79} or Harel \cite{harel85} algorithm.  The nodes of the
dominator tree are numbered in pre-order such that for all nodes $N$,
$\text{num}[N]>\text{num}[\text{idom}[N]]$.
Then, in a single traversal of the post-dominator tree, we find the
lowest-numbered node post-dominated by any given node.  We add
triggers to constants from variables defined at this lowest node
post-dominated by the constant use; using the default trigger
$\mathcal{C}_T$ where necessary.  We then place \phisigfunction{s}
for all variables, including constant triggers, using
Algorithm~\ref{alg:SSIplace}.

We then generate \xifunction{s}.  A standard interval analysis
creates a loop nesting tree, and each loop is scanned for invariants
and other definitions/uses to create the proper \xifunction{} tuples.
Renaming is done using Algorithm~\ref{alg:SSIrename1}, as before.

\subsection{Dataflow and control dependence}\label{sec:parallelism}
The \ssiplus\ semantics are data-driven, and thus bring to mind work
on compilers for dataflow machines.  Beck, Johnson, and Pingali have
previously written \cite{beck91:ctrlflow} on the benefits of
dataflow-oriented intermediate representations.  However, the previous
work on dataflow compilers (Traub \cite{traub86:ttda}, for example)
has concentrated on intra-loop dependencies, often leaving in
pseudo-control-flow edges to serialize non-loop structures.  This
strategy results in the sort of fine-grain intra-loop parallelism
suitable for parallel dataflow machines, vector processors, and VLIW
machines.

The current work concentrates on removing unnecessary dependencies
\emph{between} loops, which allows a coarser parallelism which does
not require as many functional units to take advantage of.  Moreover,
we extract parallel sequential threads that are not loop-based.
Obviously both fine-grain and coarse-grain parallelism are important,
but we feel the current industry trends towards loosely coupled
multiprocessors support our coarser-grained approach which has, to
date, seemingly been neglected by dataflow approaches.

\subsection{Hardware compilation.}\label{sec:hardware}
The observant reader may have noticed that the two 
operational semantics given in section \ref{sec:semantics} closely
resemble circuit implementations for the program according to
synchronous and asynchronous design methodologies.  In fact,
\ssiplus{} was designed specifically to facilitate rendering a
high-level program into hardware.  The two semantics differ primarily
on how cyclic dependencies (i.e. loops) are handled.

Translation of high-level languages directly to hardware has long been
a goal of researchers.  Tanaka et al. constructed a system based on FORTRAN
\cite{tanaka89:harp}, and Galloway's C-based hardware description
language \cite{galloway95:transmogrifier} inspired a new interest in
applying general-purpose languages to the task.  The recent general
use of type-safe object-oriented languages has encouraged speculation
that the more favorable analysis properties of these stricter
languages would enable further advances in general-use hardware
compilation.  In this context, the well-defined semantics and
data-flow orientation of \ssiplus\ solve the local-level hardware
compilation problem and allow effort to be concentrated on the more
difficult intra-procedural analyses required.

\section{Methodology}\label{sec:methodology}
The SSI intermediate representation described in this paper is the
core IR for the FLEX compiler infrastructure project, started in July
1998 and currently containing about 70,000 lines of Java source code.
The FLEX compiler reads in Java bytecodes, and targets both the JVM
(for high-level portable code transformations) and several
combinations of machine architectures and runtime systems.  Currently
the bytecode and ARM processor backends are near completion.
Interpreters exist for the various intermediate representations used
in the compiler, allowing the correctness of the earlier passes of the
compiler to be verified.  The compiler will correctly compile itself
to IR and interpret itself.

The FLEX compiler implements the algorithms described in this paper,
validating their correctness.  Variable counting for the graphs of
section~\ref{sec:ssi_complexity} was done by a special statistics
module that could be applied to the results of any pass.  The full
bitwidth-extended SPTC constant propagation algorithm was implemented,
although we currently do not use the bitwidth information produced.
\ssiplus\ and hardware compilation are the focus of current work.

\section{Conclusions}
The Static Single Information form extends SSA without adding unneeded
complexity to allow efficient predicated analysis and backward
dataflow analyses.  Futher, the \ssiplus\ variant removes all explicit
control-dependence relations, allowing extraction of parallelism from
the code, and possesses a complete and straight-forward semantics
which makes it useful for, among other things, abstract interpretation
and hardware compilation.

We have demonstrated efficient construction of SSI form, and several
optimizations which use it to obtain efficiency improvements over
previous methods.  The many SSA-variant papers in the literature
attest to limitations of standard SSA form; we believe SSI form solves
these problems in a simple and symmetric manner.  The FLEX
compiler infrastructure demonstrates the practicality
of SSI form.

\newpage
\bibliography{harpoon}
\newpage
\appendix
\dontfixme{
\section{Common errors of IR research papers}
For some reason, the field of compiler intermediate representations
seems to attract more than its share of village idiots.
\begin{enumerate}
\item Quote: ``The number of $\phi$-nodes needed remains linear''
\cite{sreedhar95:lintime}.  Often \cite{cytron91:ssa} is cited in
support of this.  What Cytron \emph{actually} said was ``For the
programs we tested, the plot in Figure~21 shows that the number of
\phifunction{s} is also linear in the size of the original program.''
In other words, this is not a theoretical or hard bound, this is an
\emph{experimental} result.  By the same experimental technique that
showed the run-time of his algorithm to be linear ``in
practice''---despite the fact that nested \code{repeat-until} loops
yielded quadratic performance.  Wegman and Zadeck get points in
\cite{wegman91:scc} for stating ``in theory the size can be $O(EV)$,
but empirical evidence indicates that the work required to compute the
SSA graph is linear in the program size,'' which is rigorously correct.
\item Time bounds of Cytron's SSA algorithm stated as being $O(N^2)$.
Chow et al.\ \cite{chow97:ssapre} claim ``the $\phi$-insertion step is
$\Omega(v^2)$\ldots'' (where they use $v$ to refer to what we call
$N$), and further ``[but] there are linear-time SSA $\phi$-placement
algorithms that can be used to lower it to $O(e)$.''  Wrong, wrong,
wrong.  Cytron's algorithm (which Chow et al.\ use) is $O(N^2 V)$ and
the faster algorithms are $O(EV)$.  The $V$ cannot be omitted; $V$ can
be $O(N)$ (one definition per node).  Dhamdhere et al.\
\cite{dhamdhere92:large} blame this common error on an early paper by
Cytron et at.\ \cite{cytron89:ssa}, noting that this was corrected in
\cite{cytron91:ssa}.  As the error was published in POPL and the
correction in TOPLAS, one may understand the widespread confusion.  A
closely related sin is referring to an SSA algorithm as ``linear,''
since the SSA form itself may be cubic in the size of the input
program.  Chow et al.\ \cite{chow97:ssapre} do this, too.  Usually
what happens is a redefinition of the ``size of the program'' until
the desired algorithm can be called ``linear.''  Sreedhar and Gao
\cite{sreedhar95:lintime} do this in two ways: first, they restrict
their algorithm to ``constructing a single SEG,'' which they can do in
$O(E)$, instead of contructing the SEGs for all variables (otherwise
known as SSA form), which takes them $O(EV)$.  Then they redefine
``linear'' to be in terms of $E$ (which could be $N^2$), instead of in
terms of $N$.  Thus they manage to call an $O(N^3)$ algorithm ($E=N^2$
and $V=N$) ``linear.''  Cytron et al.\ \cite{cytron91:ssa} win the
most points by blatently declaring a variable $R$ to be ``the
largest'' of $N$, $E$, $V$, and a few other metrics.  They then spin
you in circles to distract you from the $O(R^3)$ figure they cite
(which may be as high as $O(N^6)$ if you've defined $R=E$!) before
landing at ``the entire translation process is effectively $O(R)$''!
Since the algorithm they are describing takes time equivalent to
$O(E+V_{SSA}|{DF}|+NV_{SSA})$ their conclusion is surprising, to say
the least.\footnote{Their justification, term by term: $O(E)$ is
$O(R)$.  Placing \phifunction{s}, which may take as much as
$V_{SSA}|{DF}|$ time, is linear because their empirical results say
so.  The total mentions of variables in the final program, which could
have $V_{SSA}$ variables mentioned in each of $N$ statements, is
$O(N)$---again, because that's the way it looks ``in practice.''  So,
everything is ``linear.''}
\end{enumerate}

\section{Canonical and derived definitions of the empty set}
\begin{enumerate}
\item \textbf{``Who cares about Maria?''} (Canonical) \cite{marinov99}
\item \textbf{``Who cares about Martin and Maria?''} (Radu Revised)
\item \textbf{``Who cares about \code{bash}?''} (Found on FLEX web page)
\end{enumerate}
} % end comment.
\end{document}
