% -*-latex-*- This is a LaTeX document.
% $Id: design.tex,v 1.6 1998-07-15 08:28:46 cananian Exp $
%%%%%%%%%%%%%%%%%%%%%%%%%%%%%%%%%%%%%%
\documentclass[10pt,notitlepage,twocolumn]{article}
\author{C.~Scott~Ananian}
\title{Turning Java into Hardware: \\ Caffinated Compiler Construction}
\date{\today \\ $ $Revision: 1.6 $ $}

% PDF-friendly fonts:
\renewcommand{\rmdefault}{ptm}
\renewcommand{\sfdefault}{phv}
%\renewcommand{\ttdefault}{pcr}

% Outline:
% ~~~~~~~
% 0. Introduction. 
%    A. Why Java?
%       - simplicity, strong types, oo, threading
%       - BUT no bitwide types, no real time, no i/o grammar
%    B. Why hardware?
%       - codesign, etc.
% I. Java/Hardware semantics.
%    A. The rules.
%       1. Classes are objects.
%       2. Allocations are static.
%       3. Threads/Monitors model inherent parallelism.
%       4. Method calls form hardware interfaces.
%       5. Exceptions are?
%          - just a means of transferring control.  An evil means.
%    B. A grammar for Java--.
%       (this is the implemented grammar, which might be more restrictive
%        than absolutely necessary. We might implement floating-point, more
%        powerful allocation analysis, etc, later).
%       1. No 'new' except in constructors.
%          (actually less draconian rule to guarantee A.2 above)
%       2. Floating-point is verboten.
%          (allowed, but not yet)
%       3. Throw and catch are not permitted? [we'll do them, but not yet]
%       4. Other features we don't care to implement?
%    C. An interface description language.
%       (more details on how methods map to chip pin-outs and timing)
%       1. Formal parameters map to chip ports.
%       2. Timing constraints on external events map to constraints on
%          class methods and executable code.
%       3. Syntax/Grammar/Examples.
% II. Useful analyses (compiler stages)
%    A. Static/Extended type analysis: not just type but which objs of the type
%       - reference 'fast interprocedural class analysis' paper.
%       - probably want to combine with SSA-based Conditional Constant
%         Propagation algorithm.
%    B. Static allocation.
%    C. Analyzing methods for side-effects.
%    D. Optimizations:
%       1. Common subexpression, constant-folding/propagation/subexpression
%       2. Partial Redundancy Elimination
%       3. Strength reduction (divide->multiply, multiple->add/rotate)
%          - see 'division by invariant integers using multiplication' paper.
%       4. Exposing parallelism (removing control dependencies, loop analysis)
%       5. Bit-width analysis (combine with type analysis?)
%          - also, floating-point equivalent.
%    E. Representation Transformations
%       1. Java source -> bytecode (use javac for the foreseeable future)
%       2. Java bytecode -> quads (remove stack abstraction)
%       3. Quads -> SSA form? (best for optimizations?)
%       4. SSA -> dataflow?  (get rid of control dependencies, more opts?)
%       5. dataflow->logic functions (low-level hardware-(in)dependent stage)
%          (``logic function'' is some form of BDD?)
%       6. Logic function->Netlist (let others technology map, place and route)
% III. More details on hardware synthesis.
%    A. Dataflow/SSA to hardware
%       1. Branches become multiplexors. 
%          (control condition becomes selector bit)
%       2. State machines for iteration, recursion.
%       3. No memory stores. Value driven data flow.
%    B. Interface timing.
%       1. Computing 'go' signals from input timing constraints.
%       2. Synthesizing logic to adhere to an output timing constraint.
%    C. Object-orientation and inheritance.
%       1. Handling aliased classes and overloaded methods (type bits)
%       2. Parameterized classes (compile-time constructor evaluation)
% IV. Related Research (useful stuff from other people)
%    A. Various Intermediate Representations
%    B. Compiling Real-Time Programs
%    C. Array Analysis Techniques.
%    D. Loop analysis techniques.
%    E. Logic Synthesis (technology-independent timing, etc)
% V. Conclusion
%    Lots of work to keep me very busy for quite a while.

\begin{document}
\bibliographystyle{abbrv}
\maketitle

\section{Introduction}
This paper explores the design of a silicon compiler for the Java
programming language.  Java's simplicity, object-orientation, and
strong typing make it well suited to an class-based hardware
translation.  It is also possible to leverage Java's thread interfaces
to model coarse-grain parallelism in hardware.  The goal is the
efficient generation of hardware from a well-known general-purpose
programming language.  The use of one specification language for both
the hardware and software components of a system could also aid
hardware-software codesign; this may be explored in future work.

Java, as a general-purpose portable programming language, also
presents obstacles to its use as a hardware specification language.
The lack of integer types with parameterized bit-widths is a
limitation that compiler technology can overcome.  It is not certain
that Java's inability to specify real-time constraints on code or
describe hardware interfaces at the wire and timing level can be
similarly smoothed over.

\section{Java/Hardware Semantics}
% describe object model.

Up to this point we have left unspecified issues of timing.

The efficient implementation of Java in hardware entails walking a
tight-rope between the usual sequential statement-execution semantics
and a radically synchronous reading similar to the languages Esterel
\cite{berry:esterel_primer}, Lucid \cite{missing_reference}, and
SIGNAL \cite{amagbegnon95:signal}.  One option is to create an
entirely asynchronous design \cite{emerson97:async_design}
from a dataflow graph of the Java program
that respects the original program's sequential order exactly.  Such a
design has favorable power consumption and execution time
characteristics, but suffers from high circuit complexity
\cite{cheng97:diclasp, nanda97:universal}.
% example here of multiple-emitter program
% also, async semantics are not quite clear.
Most silicon compilers instead generate synchronous circuits from
their input programs, allowing us great flexibility in the assignment
of clocked states to an compiled object.  It seems most reasonable to
use a synchrony model a bit looser than the \textit{perfect synchrony
hypothesis} used in Esterel, which Berry in
\cite{berry92:hardware_esterel} credits to \cite{benveniste91:synchrony}.
Instead of ``instantaneous broadcast'' we prefer a ``time plus delta''
approach similar to that used by VHDL 
\cite{one_of_the_books_i_used_for_the_silicon_c_paper} and formalized
in \cite{gagne97:nonstandard} citing \cite{gagne96:nonstandard}.  The
program should behave according to the standard sequential semantics,
but externally-visible events will be synchronized to a clock.
Conceptually, all statements take a finite but infinitesimal amount of
time, excepting at points where we wait for a the next clock tick.

% multiple-emitter example, again.

\subsection{The rules}

\subsection{A Grammar for \textsc{Java--}}
\subsection{An interface description language}
\section{Compiler analyses}
\subsection{Static analysis} % Type, Allocation
\subsection{Functional Methods}
\subsection{Optimizations}
\subsection{Representations}
\section{Hardware Synthesis Details}
\subsection{Translating Dataflow/SSA into Hardware}
\subsection{Interface Timing Issues}
\subsection{Object-orientation and Inheritance}
\section{Related Research}
\subsection{Intermediate Representations}
\subsection{Compiling Real-Time programs}
\subsection{Array Analysis Techniques}
\subsection{Loop Analysis Techniques}
\subsection{Logic Synthesis}
\section{Conclusion}

\bibliography{harpoon}

\end{document}
