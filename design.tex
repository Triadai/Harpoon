% -*-latex-*- This is a LaTeX document.
% $Id: design.tex,v 1.1 1998-04-08 02:57:10 cananian Exp $
%%%%%%%%%%%%%%%%%%%%%%%%%%%%%%%%%%%%%%
\documentclass[10pt,notitlepage]{article}
%twocolumn
\author{C.~Scott~Ananian}
\title{Turning Java into Hardware: \\ Caffinated Compiler Construction}
\date{\today \\ $ $Revision: 1.1 $ $}

% PDF-friendly fonts:
\renewcommand{\rmdefault}{ptm}
\renewcommand{\sfdefault}{phv}
%\renewcommand{\ttdefault}{pcr}

\begin{document}
\bibliographystyle{alpha}
\maketitle

\section{Java}

% Classes are hardware objects.  Allocations are static.
% References may be external (memory port)
% What subset we compile (dynamic allocation/dynamic types are evil)
% Thread/Monitor model.  Interface model.
% Exceptions???

\section{Interface Description Language}
% methods == ports
% specifying timing constraints.

\section{Steps}
% Computing type information.
% Creating a static allocation.
% Analyzing methods for side-effects.
% Useful optimizations:
%   CSE, constant-*, PRE
%   divide->multiply, multiply->add/rotate.
%   exposing parallelism (removing control dependencies)
% Bit-width analysis (type information?)
% Branch->multiplexor. (control bit->selector)
% Compiling aliased class families into hardware (more bits!)
% Calculating 'go' signal from input timing
% Adhering to an output timing spec.
% Counting states. Unrolling loops.

\section{IR}
% Bytecode->quads.
% Quads->ssa?
% SSA->dataflow?
% dataflow->logic functions?
% logic functions->netlists?

\nocite{*}
\bibliography{harpoon}

\end{document}



