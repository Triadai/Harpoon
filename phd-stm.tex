In this chapter we will specifically describe the implementation of
our software transaction system.  We will first outline the design
properties of the system, using quantitative data to validate our
goals.  Then we will formally describe an implementation which meets those
design goals, using the modelling language SPIN for concreteness.  We
will evaluate the performance of the basic implementation, and point
out some deficiencies.  By recasting the basic implementation as a
``small object protocol'', we will show how to extend it to a ``large
object protocol'', in the process addressing these performance
problems.

\section{Designing Efficient Transactions}\label{sec:efficient}

In this section I briefly describe some desired properties of the
software transaction system of this thesis.
\note{Work on this transition.}

\subsection{Object-oriented vs. flat TM}
This transaction system, unlike all current proposals
\cite{HarrisFr03,HerlihyMo93}\note{Include more} (including the
hardware system presented in \charef{hardimpl}), uses an
``object-oriented'' design.  Current research is focused on
implementing a flat (transactional) memory abstraction in software,
primarily because this solves some of the issues we will present in
\charef{largeobj}.  However, the object-oriented approach offers
several benefits:
\begin{description}
\item[Efficient execution of long-running transactions.]  As discussed
  briefly in Chapter \ref{sec:efficiency} and \ref{sec:tm}, flat
  word-oriented transaction schemes require overhead proportional to
  the number of words read/written in the transaction, even if these
  locations had been accessed before inside the transaction. Object-oriented
  schemes impose a cost proportional to the number of \emph{objects}
  touched by the transaction --- but once the cost of ``opening'' those
  objects is paid, the transaction can continue to work indefinitely
  upon those objects without paying any further penalty.
  Object-oriented schemes are thus seen to be more efficient for
  \emph{long-running transactions}.
\item[Preservation of optimization opportunities.]  Furthermore,
  transaction-local objects can be identified (statically or
  dynamically) and creation/updates to these objects can be done
  without any transaction tax at all.  Word oriented schemes discard
  the high-level information required to implement these optimizations.
%\item[Freedom from implementation restrictions.]
% thinking of multi-version consistency here.
\end{description}
I believe that the problems with previous object-oriented schemes can
be solved while preserving the inherent benefits of an object-oriented
approach, and the current thesis presents one such solution.

\epsfigput{bloat}{Application slowdown with increasing object bloat
for the SPECjvm98 benchmark applications.}
\subsection{Tolerable limits for object expansion}
I will need to add some additional information to each object to
track transaction state.  I measured the slowdown caused by various
amounts of object ``bloat'' to determine reasonable bounds on the
size of this extra information.  \figref{bloat} presents these
results for the SPECjvm98 applications; I determined that two words
(eight bytes) of additional storage per object would not impact
performance unreasonably.  This amount of bloat causes a geometric
mean of 2\% slowdown on these benchmarks.

\begin{figure}\sis%
\begin{center}
\begin{tabular}{lrrrr}
        & transactional & transactional\\
program & memory ops    & stores \% \\\hline
{\tt 201\_compress} & 50,029 & 26.2\% \\
{\tt 202\_jess} & 36,701,037 & 0.6\% \\
{\tt 205\_raytrace} & 7,294,648 & 23.2\% \\
{\tt 209\_db} & 195,374,420 & 6.3\% \\
{\tt 213\_javac} & 472,134,289 & 22.9\% \\
{\tt 222\_mpegaudio} & 41,422 & 18.6\% \\
{\tt 228\_jack} & 63,912,386 & 17.0\% \\
\end{tabular}
\end{center}
\caption{Comparison of loads and stores inside transactions for the
  SPECjvm98 benchmark suite, full input runs.}
\label{fig:writepercent}
\end{figure}
\subsection{Designing for Reads vs. Writes}
\figref{writepercent} shows that transactional reads typically
outnumber transactional writes by at least 4 to 1; in some cases reads
outnumber writes by over 100 to 1.  It is worthwhile, therefore, to
make reads more efficient than writes.  In particular, since the
flag-overwrite technique discussed in \charef{flagfield} requires us
to allocate additional memory to store the ``real'' value of the
field, I wish to avoid this process for transactional reads,
reserving the extra allocation effort for transactional writes.

\subsection{The Big Idea: waving FLAGs}\label{sec:flagfield}

\note{Missing: performance numbers for adding check.  Use ``no trans''
version of transaction app and add check into the access functions.}
We would like non-transactional code to execute with minimal overhead,
however, transactions should still appear atomic to non-transactional
code.  My basic mechanism is loosely based on the
distributed shared memory implementation of Scales and Gharachorloo
\cite{ScalesGh97}.  I pick a special ``flag'' value, and
``cross-out'' locations currently involved in a transaction by
overwriting them with the flag value.  Reading or attempting to
overwrite a flagged value will indicate to non-transactional code
that exceptional processing is necessary; all other non-transactional
operations proceed as usual.

Note that this technique explicitly allows safe access to fields
involved in a transaction from non-transactional code, which is
another design goal of the system.  This eases ``transactification''
of legacy code (but see \charef{semantic}!).

\section{Specifying the mechanism}
We now present an algorithm which has these desired properties.
Our algorithms will be completely non-blocking, which allows good
scaling and proper fault-tolerant behavior: one faulty or slow
processor cannot hold up the remaining good processors.

We will implement the synchronization required by our algorithm using
load-linked/store-conditional instructions.  We require a particular
variant of these instructions which allows the location of the
load-linked to be different from the target of the store-conditional:
this variant is supported on the PowerPC processor family, although it
has been deprecated in the newest chips.  This disjoint location
capability is essential to allow us to keep a finger on one location
while modifying another: a poor man's ``Double Compare And Swap''
instruction.

We will describe our algorithms in the Promela modeling language
\cite{Holzmann03},
which we used to allow mechanical model checking of the race-safety
and correctness of the design.  Portions of the model have been
abbreviated for this presentation;  the full Promela model is
given in \appref{full-model}, along with a brief primer on Promela
syntax and semantics.

%\subsecput{interface}{Interface}
\subsection{Object Structures}%datastruct
\figput[Implementing software transactions with version lists.]{tr-multi-obj}%
 {Implementing software transactions with version
  lists.  A transaction object consists of a single field {\it
    status}, which indicates if it has COMMITTED, ABORTED, or is WAITING.
  Each object contains two extra fields: {\it readers}, a
  singly-linked list of transactions which have read this object; and
  {\it versions} a linked list of version objects.  If an object field
  is FLAG, then the value for the field is obtained from the
  appropriate linked version object.}
\figref{tr-multi-obj} illustrates the basic data structures of our
software transaction implementation.  Objects are extended with two
additional fields.  The first field, {\tt versions}, points to a
singly-linked list of object versions.  Each one contains field values
corresponding to a committed, aborted, or in-progress transaction,
identified by its {\tt owner} field.  There is a single unique
transaction object for each transaction.

The other added field, {\tt readers}, points to a singly-linked list
of transactions which have read from this object.  Committed and
aborted transactions are pruned from this list.  The {\tt readers}
field is used to ensure that a transaction does not operate with
out-of-date values if the object is later written
non-transactionally.

There is a special flag value, here denoted by {\tt FLAG}.  It should be
an uncommon value, i.e. not a small positive or negative integer
constant, nor zero.  In our implementation, we have chosen the byte
\texttt{0xCA} to be our flag value, repeated as necessary to fill out
the width of the appropriate type.
The semantic value of an object field is the value in the original
object structure, \emph{unless that value is \texttt{FLAG}}, in which
case the field's value is the value of the field in the first
committed transaction in the object's version list.  A ``false flag''
occurs when the application wishes to ``really'' store the value FLAG
in a field; this is handled by creating a fully-committed version
attached to the object and storing FLAG in that version as well as in
the object field.

\subsection{Operations}%ops
We support transactional read/write and non-transactional read/write
as well as transaction begin, transaction abort, and transaction
commit.  Transaction begin simply involves the creation of a new
transaction identifier object.  Transaction commit and abort are simply
compare-and-swap operations which atomically set the transaction object's {\tt
  status} field appropriately if and only if it was previously in the
WAITING state.
The simplicity of commit and abort are appealing: our algorithm
requires no complicated processing, delay, roll-back or validate
procedure to commit or abort a transaction.

We will present the other operations one-by-one.

\subsubsection{Read}

The {\tt ReadNT} function does a non-transactional read of field $f$ from
object $o$, putting the result in $v$.
In the common case, the only overhead is to check that
the read value is not FLAG.  However, if the value read \emph{is}
FLAG, we copy back the field value 
from the most-recently committed transaction (aborting all other
transactions) and try again.  The copy-back procedure will notify
us if this is a ``false flag'', in which case the value of this
field really is FLAG.  We pass the {\tt kill\_writers} constant
to the copy-back procedure to
indicate that only transactional writers need be aborted, not
transactional readers.
All possible races are confined to the copy-back procedure.

\begin{inlinecode}
inline readNT(o, f, v) {
  do
  :: v = object[o].field[f];
     if
     :: (v!=FLAG) -> break /* done! */
     :: else
     fi;
     copyBackField(o, f, kill_writers, _st);
     if
     :: (_st==false_flag) ->
        v = FLAG;
        break
     :: else
     fi
  od
}
\end{inlinecode}

\subsubsection{Write}
The {\tt WriteNT} function does a non-transactional write of new value $nval$
to field $f$ of object $o$.  For correctness, we need to ensure that
the reader list is empty before we do the write.  We implement this
with a load-linked/store-conditional pair, which is modelled in
Promela slightly differently, ensuring that our write only succeeds
so long as the reader list remains empty.\footnote{Note that a
  standard CAS would not suffice, as the load-linked targets a
  different location than the store-conditional.}
If it is not empty, we
call the copy-back procedure (as in {\tt readNT}), passing the
constant {\tt kill\_all} to indicate that both transactional readers
and writers should be aborted during the copy-back.  The copy-back
procedure leaves the reader list empty.

If the value to be written is actually the FLAG value, things get a
little bit trickier.  This case does not occur often, and so the
simplest correct implementation is to treat this non-transactional
write as a short transactional write, creating a new transaction for
this one write, and attempting to commit it immediately after the
write.  This is slow, but adequate for this uncommon case.

\begin{inlinecode}
inline writeNT(o, f, nval) {
  if
  :: (nval != FLAG) ->
     do
     :: atomic {
          if /* this is a LL(readerList)/SC(field) */
          :: (object[o].readerList == NIL) ->
             object[o].fieldLock[f] = _thread_id;
             object[o].field[f] = nval;
             break /* success! */
          :: else
          fi
        }
        /* unsuccessful SC */
        copyBackField(o, f, kill_all, _st)
     od
  :: else -> /* create false flag */
     /* implement this as a short *transactional* write. */
     /* start a new transaction, write FLAG, commit the */
     /* transaction; repeat until successful. */
     /* Implementation elided. */
     ...
  fi;
}
\end{inlinecode}

\subsubsection{Field Copy-Back}
\begin{figure}
\sis\fontsize{6.5}{7.4}
\begin{verbatim}
inline copyBackField(o, f, mode, st) {
  _nonceV=NIL; _ver = NIL; _r = NIL; st = success;
  /* try to abort each version.  when abort fails, we've got a
   * committed version. */
  do
  :: _ver = object[o].version;
     if
     :: (_ver==NIL) ->
        st = saw_race; break /* someone's done the copyback for us */
     :: else
     fi;
      /* move owner to local var to avoid races (owner set to NIL behind
       * our back) */
     _tmp_tid=version[_ver].owner;
     tryToAbort(_tmp_tid);
     if
     :: (_tmp_tid==NIL || transid[_tmp_tid].status==committed) ->
        break /* found a committed version */
     :: else
     fi;
     /* link out an aborted version */
     assert(transid[_tmp_tid].status==aborted);
     CAS_Version(object[o].version, _ver, version[_ver].next, _);
  od;
  /* okay, link in our nonce.  this will prevent others from doing the
   * copyback. */
  if
  :: (st==success) ->
     assert (_ver!=NIL);
     allocVersion(_retval, _nonceV, aborted_tid, _ver);
     CAS_Version(object[o].version, _ver, _nonceV, _cas_stat);
     if
     :: (!_cas_stat) ->
        st = saw_race_cleanup
     :: else
     fi
  :: else
  fi;
  /* check that no one's beaten us to the copy back */
  if
  :: (st==success) ->
     if
     :: (object[o].field[f]==FLAG) ->
        _val = version[_ver].field[f];
        if
        :: (_val==FLAG) -> /* false flag... */
           st = false_flag /* ...no copy back needed */
        :: else -> /* not a false flag */
           d_step { /* LL/SC */
             if
             :: (object[o].version == _nonceV) ->
                object[o].fieldLock[f] = _thread_id;
                object[o].field[f] = _val;
             :: else /* hmm, fail.  Must retry. */
                st = saw_race_cleanup /* need to clean up nonce */
             fi
           }
        fi
     :: else /* may arrive here because of readT, which doesn't set _val=FLAG*/
        st = saw_race_cleanup /* need to clean up nonce */
     fi
  :: else /* !success */
  fi;
  /* always kill readers, whether successful or not.  This ensures that we
   * make progress if called from writeNT after a readNT sets readerList
   * non-null without changing FLAG to _val (see immediately above; st will
   * equal saw_race_cleanup in this scenario). */
  if
  :: (mode == kill_all) ->
     do /* kill all readers */
     :: moveReaderList(_r, object[o].readerList);
        if
        :: (_r==NIL) -> break
        :: else
        fi;
        tryToAbort(readerlist[_r].transid);
        /* link out this reader */
        CAS_Reader(object[o].readerList, _r, readerlist[_r].next, _);
     od;
  :: else /* no more killing needed. */
  fi;
  /* done */
}
\end{verbatim}
\caption{The field copy-back routine.}\label{fig:copyback}
\end{figure}
\figref{copyback} presents the field copy-back routine.  We create a
new version owned by a pre-aborted transaction which serves as a
reservation on the head of the version list.  We then write to the
object field with a load-linked/store-conditional pair if and only if
our version is still at the head of the versions list.\footnote{Note
  again that a CAS does not suffice.}  This addresses
the major race possible in this routine.

\subsubsection{Transactional Read}
A transactional read is split into two parts.  Before the read, we
must ensure that our transaction is on the reader list for the
object.  This is straight-forward to do in a non-blocking manner as
long as we always add ourselves to the head of the list.  We must also
walk the versions list, and abort any uncommitted transaction other
than our own.  These steps can be combined and hoisted so that they
are done once before the first read from an object and not repeated.

At read time, we initially read from the original object.  If the
value read is not FLAG, we use it.  Otherwise, we look up the version
object associated with our transaction (this will typically be at the
head of the version list) and read the appropriate value from that
version.  Note that the initial read-and-check can be omitted if we
know that we have already written to this field inside this transaction.

\begin{inlinecode}
inline readT(tid, o, f, ver, result) {
  do
  ::
     /* we should always either be on the readerlist or
      * aborted here */
     result = object[o].field[f];
     if
     :: (result==FLAG) ->
        if
        :: (ver!=NIL) ->
           result = version[ver].field[f];
           break /* done! */
        :: else ->
           findVersion(tid, o, ver);
           if
           :: (ver==NIL) ->/*use val from committed vers.*/
              assert (_r!=NIL);
              result = version[_r].field[f];/*false flag?*/
              moveVersion(_r, NIL);
              break /* done */
           :: else /* try, try, again */
           fi
        fi
     :: else -> break /* done! */
     fi
  od
}
\end{inlinecode}

\begin{figure}
\sis\fontsize{6.5}{8}
\begin{verbatim}
/* per-object, before write. */
inline ensureWriter(tid, o, ver) {
  assert(tid!=NIL);
  ver = NIL; _r = NIL; _rr = NIL;
  do
  :: assert (ver==NIL);
     findVersion(tid, o, ver);
     if
     :: (ver!=NIL) -> break /* found a writable version for us */
     :: (ver==NIL && _r==NIL) ->
        /* create and link a fully-committed root version, then
         * use this as our base. */
        allocVersion(_retval, _r, NIL, NIL);
        CAS_Version(object[o].version, NIL, _r, _cas_stat)
     :: else ->
        _cas_stat = true
     fi;
     if
     :: (_cas_stat) ->
        /* so far, so good. */
        assert (_r!=NIL);
        assert (version[_r].owner==NIL ||
                transid[version[_r].owner].status==committed);
        /* okay, make new version for this transaction. */
        assert (ver==NIL);
        allocVersion(_retval, ver, tid, _r);
        /* want copy of committed version _r.  No race because
         * we never write to a committed versions. */
        version[ver].field[0] = version[_r].field[0];
        version[ver].field[1] = version[_r].field[1];
        assert(NUM_FIELDS==2); /* else ought to initialize more fields */
        CAS_Version(object[o].version, _r, ver, _cas_stat);
        moveVersion(_r, NIL); /* free _r */
        if
        :: (_cas_stat) ->
           /* kill all readers (except ourself) */
           /* note that all changes have to be made from the front of the
            * list, so we unlink ourself and then re-add us. */
           do
           :: moveReaderList(_r, object[o].readerList);
              if
              :: (_r==NIL) -> break
              :: (_r!=NIL && readerlist[_r].transid!=tid)->
                 tryToAbort(readerlist[_r].transid)
              :: else
              fi;
              /* link out this reader */
              CAS_Reader(object[o].readerList, _r, readerlist[_r].next, _)
           od;
           /* okay, all pre-existing readers dead & gone. */
           assert(_r==NIL);
           /* link us back in. */
           ensureReaderList(tid, o);
           break
        :: else
        fi;
        /* try again */
     :: else
     fi;
     /* try again from the top */
     moveVersion(ver, NIL)
  od;
  /* done! */
  assert (_r==NIL);
}
\end{verbatim}
\caption{The per-object version-setup routine for transactional writes.}
\label{fig:ensurewriter}
\end{figure}
\begin{figure}
\sis\fontsize{6.5}{8}
\begin{verbatim}
/* per-field, before write. */
inline checkWriteField(o, f) {
  _r = NIL; _rr = NIL;
  do
  ::
     /* set write flag, if not already set */
     _val = object[o].field[f];
     if
     :: (_val==FLAG) ->
        break; /* done! */
     :: else
     fi;
     /* okay, need to set write flag. */
     moveVersion(_rr, object[o].version);
     moveVersion(_r, _rr);
     assert (_r!=NIL);
     do
     :: (_r==NIL) -> break /* done */
     :: else ->
        object[o].fieldLock[f] = _thread_id;
        if
        /* this next check ensures that concurrent copythroughs don't stomp
         * on each other's versions, because the field will become FLAG
         * before any other version will be written. */
        :: (object[o].field[f]==_val) ->
           if
           :: (object[o].version==_rr) ->
              atomic {
                if
                :: (object[o].fieldLock[f]==_thread_id) ->
                   version[_r].field[f] = _val;
                :: else -> break /* abort */
                fi
              }
           :: else -> break /* abort */
           fi
        :: else -> break /* abort */
        fi;
        moveVersion(_r, version[_r].next) /* on to next */
     od;
     if
     :: (_r==NIL) ->
        /* field has been successfully copied to all versions */
        atomic {
          if
          :: (object[o].version==_rr) ->
             assert(object[o].field[f]==_val ||
                    /* we can race with another copythrough and that's okay;
                     * the locking strategy above ensures that we're all
                     * writing the same values to all the versions and not
                     * overwriting anything. */
                    object[o].field[f]==FLAG);
             object[o].fieldLock[f]=_thread_id;
             object[o].field[f] = FLAG;
             break; /* success!  done! */
          :: else
          fi
        }
     :: else
     fi
     /* retry */
  od;
  /* clean up */
  moveVersion(_r, NIL);
  moveVersion(_rr, NIL);
}
\end{verbatim}
\caption{The per-field copy-through routine for transactional writes.}
\label{fig:copythrough}
\end{figure}
\subsubsection{Transactional Write}
Again, writes are split in two.  Once for each object we must traverse
the version list, aborting other versions and locating or creating a
version corresponding to our transaction.  We must also traverse the
reader list, aborting all transactions on the list except ourself.
This is shown in the {\tt ensureWriter} routine in \figref{ensurewriter}.

Once for each field we intend to write, we must perform a
copy-through: copy the object's field value into all the versions and
then write FLAG to the object's field.  We use
load-linked/store-conditional to update versions only if the object's
field has not already been set to FLAG behind our backs by another
copy-through.  The {\tt checkWriteField} routine is shown in
\figref{copythrough}.

Then, for each write, we simply write to the identified version.
\begin{inlinecode}
inline writeT(ver, f, nval) {
  /* easy enough: */
  version[ver].field[f] = nval;
}
\end{inlinecode}


%%%%%%%%%%%%%%%%%%%%%%%%%%%%%%%%%%%%%%%%%%%5
\section{Arrays and large objects}\label{sec:largeobj}
%% XXX

\subsection{A simple implementation of functional arrays}
Our atomic method implementation will use \emph{functional arrays} as
a building block.  Functional arrays are \emph{persistent}; that is,
after an element is updated both the new and the old contents of the
array are available for use.  Since arrays are simply maps from
integers (indexes) to values; any functional map datatype (for
example, a functional balanced tree) can be used to implement
functional arrays.

However, the distinguishing characteristic of an imperative array is its
theoretical complexity: $O(1)$ access or update of any element.
Implementing functional arrays with a functional balanced tree yields
$O(\lg n)$ worst-case access or update.
% We will address this
% discrepancy in Section~\ref{sec:large-obj}.

For concreteness, functional arrays have the following three
operations defined:
\begin{itemize}
\item $\funcname{FA-Create}(n)$: Return an array of size $n$.  The contents of
  the array are initialized to zero.
\item $\funcname{FA-Update}(A_j, i, v)$: Return an array $A_{j'}$ which is
  functionally identical to array $A_j$ except that $A_{j'}(i)=v$.
  Array $A_j$ is not destroyed and can be accessed further.
\item $\funcname{FA-Read}(A_j, i)$: Return $A_j(i)$.
\end{itemize}
We allow any of these operations to \emph{fail}.  Failed operations
can be safely retried, as all operations are idempotent by definition.

For the moment, consider the following na{\"\i}ve implementation:
\begin{itemize}
\item $\funcname{FA-Create}(n)$: Return an ordinary imperative array of size
  $n$.
\item $\funcname{FA-Update}(A_j, i, v)$: Create a new imperative array
  $A_{j'}$ and copy the contents of $A_j$ to $A_{j'}$.  Return $A_{j'}$.
\item $\funcname{FA-Read}(A_j, i)$: Return $A_j[i]$.
\end{itemize}
This implementation has $O(1)$ read and $O(n)$ update, so it matches
the performance of imperative arrays only when $n=O(1)$.  We will
therefore call these \emph{small object functional arrays}.  Operations
in this implementation never fail.  Every operation is non-blocking
and no synchronization is necessary, since the imperative arrays are
never mutated after they are created.

\subsection{A single-object protocol}
\begin{figure}\centering
\includegraphics[width=3.25in,clip=true]{Figures/nb-single-obj.eps}
\caption{Implementing non-blocking single-object concurrent operations
  with functional arrays.}
\label{fig:single-o}
\end{figure}
Given a non-blocking implementation of functional arrays, we can
construct an implementation of \atomic for single objects.  In
this implementation, fields of at most one object may be referenced
during the execution of the atomic method.

We will consider the following two operations on objects:
\begin{itemize}
\item $\funcname{Read}(o, f)$: Read field $f$ of $o$.  We will assume that
  there is a constant mapping function which given a field name
  returns an integer index.  We will write the result of mapping $f$
  as \fref{f}{index}.  For simplicity, and without loss of generality,
  we will assume all fields are of equal size.
\item $\funcname{Write}(o, f, v)$: Write value $v$ to field $f$ of $o$.
\end{itemize}
All other operations on Java objects, such as method dispatch and type
interrogation, can be performed using the immutable {\tt type}
field in the object.  Because the {\tt type} field is never changed
after object creation, non-blocking implementations of operations on
the {\tt type} field are trivial.

As Figure~\ref{fig:single-o} shows, our single-object implementation
of \atomic represents objects as a pair, combining {\tt type} and a
reference to a functional array.  When not inside an atomic
method, object reads and writes are implemented using the
corresponding functional array operation, with the array reference in
the object being updated appropriately:
\begin{itemize}
\item $\funcname{Read}(o, f)$:
  Return $\funcname{FA-Read}(\fref{o}{fields}, \fref{f}{index})$.
\item $\funcname{Write}(o, f, v)$: Replace \fref{o}{fields} with the
  result of $\funcname{FA-Update}(\fref{o}{fields}, \fref{f}{index}, v)$.
\end{itemize}

The interesting cases are reads and writes inside an atomic method.
At entry to our atomic method which will access (only) object $o$, we
store \fref{o}{fields} in a local variable $u$.  We create another
local variable $u'$ which we initialize to $u$.  Then our read and
write operations are implemented as:
\begin{itemize}
\item $\funcname{ReadAtomic}(o, f)$:
  Return $\funcname{FA-Read}(u', \fref{f}{index})$.
\item $\funcname{WriteAtomic}(o, f, v)$:
  Update variable $u'$ to the result of
  $\funcname{FA-Update}(u', \fref{f}{index}, v)$.
\end{itemize}

At the end of the atomic method, we use Compare-And-Swap to atomically
set \fref{o}{fields} to $u'$ iff it contained $u$.  If the CAS fails,
we back-off and retry.

With our na{\"\i}ve ``small object'' functional arrays, this implementation is
exactly the ``small object protocol'' of Herlihy \cite{Herlihy93}.
Herlihy's protocol is rightly criticized for an excessive amount of
copying.  We will address this with a better implementation of
functional arrays in Section~\ref{sec:large-obj}.
However, the restriction that only one object
may be referenced within an atomic method is overly limiting.

\subsection{Extension to multiple objects}
\begin{figure}[t]\centering
\includegraphics[width=3.25in,clip=true]{Figures/nb-multi-obj.eps}
\caption[Data structures to support non-blocking multi-object
  concurrent operations.]{Data structures to support non-blocking multi-object
  concurrent operations.  Objects point to a linked list of versions,
  which reference operation identifiers.  Versions created within the
  same execution of an atomic method share the same operation
  identifier.  Version structure also contain pointers to functional
  arrays, which record the values for the fields of the object.
  If no modifications have been made to the object, multiple versions
  in the list may share the same functional array.}
\label{fig:multi-o}
\end{figure}
\begin{figure}[p]
\sis%
\renewcommand{\>}{~~}%
\newcommand{\com}[1]{\hfill [{\sl #1}]}%
\begin{tabular}{l}%
$\funcname{Read}(o, f)$:\\
begin\\
retry:\\
\>$u \gets \fref{o}{versions}$ \\
\>$u' \gets \fref{u}{next}$ \\
\>$s  \gets \fref{\fref{u}{owner}}{status}$ \\
\>if ($s=\text{\sl DISCARDED}$) \com{Delete DISCARDED?}\\
\>\>CAS$(u, u', \addr{\fref{o}{versions}})$\\
\>\>goto retry \\
\>else if ($s=\text{\sl COMPLETE}$)\\
\>\>$a \gets \fref{u}{fields}$ \com{$u$ is COMPLETE}\\
\>\>$\fref{u}{next} \gets \text{\bf null}$ \com{Trim version list}\\
\>else\\
\>\>$a \gets \fref{u'}{fields}$ \com{$u'$ is COMPLETE}\\
\>return $\funcname{FA-Read}(a, \fref{f}{index})$ \com{Do the read}\\
end\\
\\
$\funcname{ReadAtomic}(o, f)$:\\
begin\\
\>$u \gets \fref{o}{versions}$\\
\>if ($\var{oid} = \fref{u}{owner}$) \com{My OID should be first}\\
\>\>return $\funcname{FA-Read}(\fref{u}{fields}, \fref{f}{index})$
\com{Do the read}\\
\>else \com{Make me first!}\\
\>\>$u' \gets \fref{u}{next}$\\
\>\>$s  \gets \fref{\fref{u}{owner}}{status}$\\
\>\>if ($s=\text{\sl DISCARDED}$) \com{Delete DISCARDED?}\\
\>\>\>CAS$(u, u', \addr{\fref{o}{versions}})$\\
\>\>else if ($\fref{\var{oid}}{status}=\text{\sl DISCARDED}$)
\com{Am I alive?}\\
\>\>\>fail\\
\>\>else if ($s=\text{\sl IN-PROGRESS}$) \com{Abort IN-PROGRESS?}\\
\>\>\>CAS$(s, \text{\sl DISCARDED}, \addr{\fref{\fref{u}{owner}}{status}})$\\
\>\>else \com{Link new version in:} \\
\>\>\>$\fref{u}{next} \gets \text{\bf null}$ \com{Trim version list}\\
\>\>\>$u' \gets \text{new \tt Version}(\var{oid}, u, \text{\bf null})$
\com{Create new version}\\
\>\>\>if (CAS$(u, u', \addr{\fref{o}{versions}}) \neq \text{\sl FAIL}$)\\
\>\>\>\>$\fref{u'}{fields} \gets \fref{u}{fields}$ \com{Copy old fields}\\
\>\>goto retry\\
end\\
\end{tabular}
\caption{\funcname{Read} and \funcname{ReadAtomic} implementations for the
  multi-object protocol.}\label{fig:reads}
\end{figure}

\begin{figure}[p]
\sis%
\renewcommand{\>}{~~}%
\newcommand{\com}[1]{\hfill [{\sl #1}]}%
\begin{tabular}{l}%
$\funcname{Write}(o, f, v)$:\\
begin\\
retry:\\
\>$u  \gets \fref{o}{versions}$\\
\>$u' \gets \fref{u}{next}$\\
\>$s  \gets \fref{\fref{u}{owner}}{status}$\\
\>if ($s=\text{\sl DISCARDED}$) \com{Delete DISCARDED?}\\
\>\>CAS$(u, u', \addr{\fref{o}{versions}})$\\
\>else if ($s=\text{\sl IN-PROGRESS}$) \com{Abort IN-PROGRESS?}\\
\>\>CAS$(s, \text{\sl DISCARDED}, \addr{\fref{\fref{u}{owner}}{status}})$\\
\>else \com{$u$ is COMPLETE}\\
\>\>$\fref{u}{next} \gets \text{\bf null}$ \com{Trim version list}\\
\>\>$a \gets \fref{u}{fields}$\\
\>\>$a' \gets \funcname{FA-Update}(a, \fref{f}{index}, v)$\\
\>\>if (CAS$(a, a', \addr{\fref{u}{fields}}) \neq \text{\sl FAIL}$)
\com{Do the write}\\
\>\>\>return \com{Success!}\\
\>goto retry\\
end\\
\\
$\funcname{WriteAtomic}(o, f, v)$:\\
begin\\
\>$u  \gets \fref{o}{versions}$\\
\>if ($oid = \fref{u}{owner}$) \com{My OID should be first}\\
\>\>$\fref{u}{fields} \gets \funcname{FA-Update}(\fref{u}{fields}, \fref{f}{index}, v)$\com{Do write}\\
\>else \com{Make me first!}\\
\>\>$u' \gets \fref{u}{next}$\\
\>\>$s  \gets \fref{\fref{u}{owner}}{status}$\\
\>\>if ($s=\text{\sl DISCARDED}$) \com{Delete DISCARDED?}\\
\>\>\>CAS$(u, u', \addr{\fref{o}{versions}})$\\
\>\>else if ($\fref{\var{oid}}{status}=\text{\sl DISCARDED}$)
\com{Am I alive?}\\
\>\>\>{\it fail}\\
\>\>else if ($s=\text{\sl IN-PROGRESS}$) \com{Abort IN-PROGRESS?}\\
\>\>\>CAS$(s, \text{\sl DISCARDED}, \addr{\fref{\fref{u}{owner}}{status}})$\\
\>\>else \com{Link new version in:} \\
\>\>\>$\fref{u}{next} \gets \text{\bf null}$ \com{Trim version list}\\
\>\>\>$u' \gets \text{new \tt Version}(\var{oid}, u, \text{\bf null})$
\com{Create new version}\\
\>\>\>if (CAS$(u, u', \addr{\fref{o}{versions}}) \neq \text{\sl FAIL}$)\\
\>\>\>\>$\fref{u'}{fields} \gets \fref{u}{fields}$ \com{Copy old fields}\\
\>\>goto retry\\
end\\
\end{tabular}
\caption{\funcname{Write} and \funcname{WriteAtomic} implementations for the
  multi-object protocol.}\label{fig:writes}
\end{figure}

We now extend the implementation to allow the fields of any number of
objects to be accessed during the atomic method.
Figure~\ref{fig:multi-o} shows our new object representation.
Objects consist of two slots, and the first represents the immutable
{\tt type}, as before.  The second field, {\tt versions}, points to a
linked list of {\tt Version} structures.  The {\tt Version} structures
contain a pointer {\tt fields} to a functional array, and a pointer
{\tt owner} to an \emph{operation identifier}.  The operation
identifier contains a single field, {\tt status}, which can be set to
one of three values: \textsl{COMPLETE}, \textsl{IN-PROGRESS}, or
\textsl{DISCARDED}.  When the operation identifier is created, the
status field is initialized to \textsl{IN-PROGRESS}, and it will be
updated exactly once thereafter, to either \textsl{COMPLETE} or
\textsl{DISCARDED}.  A \textsl{COMPLETE} operation identifier never
later becomes \textsl{IN-PROGRESS} or \textsl{DISCARDED}, and
a \textsl{DISCARDED} operation identifier never becomes
\textsl{COMPLETE} or \textsl{IN-PROGRESS}.

We create an operation identifier when we begin or restart an atomic
method and place it in a local variable \emph{oid}.  At the end of the
atomic method, we use CAS to set \fref{\var{oid}}{status} to
{\sl COMPLETE} iff it was {\sl IN-PROGRESS}.  If the CAS is successful,
the atomic method has also executed successfully; otherwise
$\fref{\var{oid}}{status}=\text{\sl DISCARDED}$ and we must
back-off and retry the atomic method.  All {\tt Version} structures
created while in the atomic method will reference \emph{oid} in
in their {\tt owner} field.

Semantically, the current field values for the object will be given by
the first version in 
the versions list whose operation identifier is {\sl COMPLETE}.
This allows us to link {\sl IN-PROGRESS} versions in at the head of
multiple objects' versions lists and atomically change the values of
all these objects by setting the one common operation identifier to
{\sl COMPLETE}.  We only allow one {\sl IN-PROGRESS} version on the
versions list, and it must be at the head, so
Therefore, before we can link a new version at the head, we
must ensure that every other version on the list is {\sl DISCARDED} or
{\sl COMPLETE}.

Since we will never look past the first {\sl COMPLETE} version in the
versions list, we can free all versions past that point.  In our
presentation of the algorithm, we do this by explicitly setting the
{\tt next} field of every {\sl COMPLETE} version we see to {\bf null};
this allows the versions past that point to be garbage collected.
An optimization would be to have the garbage collector do the list
trimming for us when it does a collection.
% always must read u.next before u.owner.status to ensure we don't
% get caught with a null pointer from a version that just committed.

We don't want to inadvertently chase the null {\tt next} pointer
of a {\sl COMPLETE} version, so we always load the {\tt next}
field of a version \emph{before} we load {\tt owner.status}.  Since
the writes occur in the reverse order ({\sl COMPLETE} to
{\tt owner.status}, then {\bf null} to {\tt next}) we have ensured that
our {\tt next} pointer is valid whenever the status is not {\sl COMPLETE}.

We begin an atomic method with \funcname{AtomicEntry} and attempt to
complete an atomic method with \funcname{AtomicExit}.  They are defined as
follows:
\begin{itemize}
\item $\funcname{AtomicEntry}$: create a new operation identifier, with
  its status initialized to {\sl IN-PROGRESS}.  Assign it to the
  thread-local variable \var{oid}.
\item $\funcname{AtomicExit}$:
  If
 $$\text{CAS}(\text{\sl IN-PROGRESS}, \text{\sl COMPLETE},
             \addr{\fref{\var{oid}}{status}})$$
  is successful, the atomic method as a whole has completed successfully,
  and can be linearized at the location of the CAS.
  Otherwise, the method has failed.  Back-off and retry from
  \funcname{AtomicEntry}.
\end{itemize}
Pseudo-code describing \funcname{Read}, \funcname{Write}, \funcname{AtomicRead},
and \funcname{AtomicWrite} is presented in Figures~\ref{fig:reads} and
\ref{fig:writes}.  In the absence of contention, all operations take
constant time plus an invocation of \funcname{FA-Read} or
\funcname{FA-Update}.

\subsection{Lock-free functional arrays}\label{sec:large-obj}
\begin{figure}\centering
\includegraphics[width=6in,clip=true]{Figures/chuang.eps}
\caption{Shallow binding scheme for functional arrays, from
  \cite[Figure~1]{Chuang94}.}
\label{fig:chuang}
\end{figure}
In this section we will present a lock-free implementation of functional
arrays with $O(1)$ performance in the absence of contention.  This
will complete our implementation of non-blocking \atomic methods for Java.

There have been a number of proposed implementations of functional
arrays, starting from the ``classical'' functional binary tree
implementation.  O'Neill and Burton \cite{ONeillBu97} give a fairly
inclusive overview.  Functional array implementations fall generally
into one of three categories: \emph{tree-based}, \emph{fat-elements},
or \emph{shallow-binding}.

Tree-based implementations typically have a logarithmic term in their
complexity.  The simplest is the persistent binary tree with $O(\ln
n)$ look-up time; Chris Okasaki 
\cite{Okasaki95} has implemented a purely-functional random-access list
with $O(\ln i)$ expected lookup time, where $i$ is the index of the
desired element.

Fat-elements implementations have per-element data structures indexed
by a master array. Cohen \cite{Cohen84} hangs a list of
versions from each element in the master array.
O'Neill and Burton \cite{ONeillBu97}, in a more sophisticated
technique, hang a splay tree off each element and achieve $O(1)$
operations for single-threaded use, $O(1)$ amortized cost when
accesses to the array are ``uniform'', and $O(\ln n)$ amortized worst
case time. 

Shallow binding was introduced by Baker \cite{Baker78} as a method to
achieve fast variable lookup in Lisp environments.  Baker clarified
the relationship to functional arrays in \cite{Baker91}.  Shallow
binding is also called \emph{version tree arrays}, \emph{trailer
  arrays}, or \emph{reversible differential lists}.  A typical
drawback of shallow binding is that reads may take $O(u)$ worst-case
time, where $u$ is the number of updates made to the array.  Tyng-Ruey
Chuang \cite{Chuang94} uses randomized cuts to the version tree to limit
the cost of a read to $O(n)$ in the worst case.  Single-threaded
accesses are $O(1)$.

Our use of functional arrays is single-threaded in the common case,
when transactions do not abort.  Chuang's scheme is attractive because
it limits the worst-case cost of an abort, with very little added
complexity.   In this section we will present a lock-free version of
Chuang's randomized algorithm.

In shallow binding, only one version of the functional array (the
\emph{root}) keeps its contents in an imperative array (the
\emph{cache}).   Each of the other versions is represented as a path
of \emph{differential nodes}, where each node describes the
differences between the current array and the previous array.  The
difference is represented as a pair \tuple{\text{\it index},\text{\it value}},
representing the new value to be stored at the specified index.
All paths lead to the root.  An update to the functional array is
simply implemented by adding a differential node pointing to the array it is
updating.

The key to constant-time access for single-threaded use is provided by the read
operation.  A read to the root simply reads the appropriate value from
the cache.  However, a read to a differential node triggers a series
of rotations which swap the direction of differential nodes and result
in the current array acquiring the cache and becoming the new root.
This sequence of rotations is called \emph{re-rooting}, and is
illustrated in Figure~\ref{fig:chuang}.  Each rotation
exchanges the root nodes for a differential node pointing to it, after
which the differential node becomes the new root and the root becomes
a differential node pointing to the new root. The cost of a read is
proportional to its re-rooting length, but after the first read
accesses to the same version are $O(1)$ until the array is re-rooted again.

Shallow binding performs badly if read operations ping-pong between two
widely separated versions of the array, as we will continually
re-root the array from one version to the other.
Chuang's contribution is to provide for \emph{cuts} to the chain of
differential nodes: once in a while we clone the cache and create a
new root instead of performing a rotation.  This operation takes
$O(n)$ time, so we amortize it over $n$ operations by randomly
choosing to perform a cut with probability $1/n$.

\begin{figure}\centering%
\includegraphics[width=5.5in,clip=true]{Figures/funarr.eps}
\caption[Atomic steps in $\funcname{FA-Rotate}(B)$.]%
 {Atomic steps in $\funcname{FA-Rotate}(B)$.  Time proceeds top-to-bottom
  on the left hand side, and then top-to-bottom on the right.
  Array $A$ is a root node, and $\funcname{FA-Read}(A, x)=z$.
  Array $B$ has the almost the same contents as $A$, but
  $\funcname{FA-Read}(B, x)=y$.}
\label{fig:funarr}
\end{figure}

\begin{figure}\centering%
\sis%
\renewcommand{\>}{~~}%
\newcommand{\com}[1]{\hfill [{\sl #1}]}%
\begin{tabular}{l}%
$\funcname{FA-Update}(A, i, v)$:\\
begin\\
\>$d \gets \text{new DiffNode}(i, v, A)$\\
\>$A'\gets \text{new Array}(\fref{A}{size}, d)$\\
\>return $A'$\\
end\\
\\
$\funcname{FA-Read}(A, i)$:\\
begin\\
retry:\\
\>$d_C \gets \fref{A}{node}$\\
\>if $d_C$ is a cache, then\\
\>\>$v \gets \fref{A}{node}[i]$\\
\>\>if $(\fref{A}{node} \neq d_C)$\com{consistency check}\\
\>\>\>goto retry\\
\>\>return $v$\\
\>else\\
\>\>\funcname{FA-Rotate}(A)\\
\>\>goto retry\\
end\\
\end{tabular}
\caption{Implementation of lock-free functional array using shallow
  binding and randomized cuts (part 1).}
\label{fig:fun-impl1}
\end{figure}
\begin{figure}\centering%
\sis%
\renewcommand{\>}{~~}%
\newcommand{\com}[1]{\hfill [{\sl #1}]}%
\begin{tabular}{l}%
$\funcname{FA-Rotate}(B)$:\\
begin\\
retry:\\
\>$d_B \gets \fref{B}{node}$\com{step (1): assign names as per Figure~\ref{fig:funarr}.}\\
\>$A \gets \fref{d_B}{array}$\\
\>$x \gets \fref{d_B}{index}$\\
\>$y \gets \fref{d_B}{value}$\\
\>$z \gets \funcname{FA-Read}(A, x)$\com{rotates A as side effect}\\
\\
\>$d_C \gets \fref{A}{node}$\\
\>if $d_C$ is not a cache, then \\
\>\>goto retry\\
\\
\>if $(0 = (\text{random} \bmod \fref{A}{size}))$\com{random cut}\\
\>\>$d_C' \gets \text{copy of }d_C$\\
\>\>$d_C'[x] \gets y$\\
\>\>$s\gets\text{DCAS}(d_C, d_C, \addr{\fref{A}{node}}, d_B, d_C', \addr{\fref{B}{node}})$\\
\>\>if $(s \neq \text{\sl SUCCESS})$ goto retry\\
\>\>else return\\
\\
\>$C \gets \text{new Array}(\fref{A}{size}, d_C)$\\
\>$d_A \gets \text{new DiffNode}(x, z, C)$\\
\\
\>$s \gets \text{CAS}(d_C, d_A, \addr{\fref{A}{node}})$\com{step (2)}\\
\>if $(s\neq \text{\sl SUCCESS})$ goto retry\\
\\
\>$s\gets\text{CAS}(A, C, \addr{\fref{d_B}{array}})$\com{step (3)}\\
\>if $(s\neq \text{\sl SUCCESS})$ goto retry\\
\\
\>$s \gets\text{CAS}(C, B, \addr{\fref{d_A}{array}})$\com{step (4)}\\
\>if $(s\neq \text{\sl SUCCESS})$ goto retry\\
\\
\>$s \gets \text{DCAS}(z, y, \addr{d_C[x]},  d_C, d_C, \addr{\fref{C}{node}})$\com{step (5)}\\
\>if $(s\neq \text{\sl SUCCESS})$ goto retry\\
\\
\>$s \gets \text{DCAS}(d_B, d_C, \addr{\fref{B}{node}}, d_C, {\bf nil}, \addr{\fref{C}{node}})$\com{step (6)}\\
\>if $(s\neq \text{\sl SUCCESS})$ goto retry\\
end\\
\end{tabular}
\caption{Implementation of lock-free functional array using shallow
  binding and randomized cuts (part 2).}
\label{fig:fun-impl2}
\end{figure}

Figure~\ref{fig:funarr} shows the data structures used for the
functional array implementation, and the series of atomic steps used
to implement a rotation.  The {\tt Array} class represents a
functional array; it consists of a {\tt size} for the array and a
pointer to a {\tt Node}.  There are two types of nodes: a {\tt
  CacheNode} stores a value for every index in the array, and a {\tt
  DiffNode} stores a single change to an array.  {\tt Array} objects
which point to {\tt CacheNode}s are roots.

In step 1 of the figure, we have a root array $A$ and an
array $B$ whose differential node $d_B$ points to $A$.  The functional
arrays $A$ and $B$ differ in one element: element $x$ of $A$ is $z$,
while element $x$ of $B$ is $y$.  We are about to rotate $B$ to give
it the cache, while linking a differential node to $A$.

Step 2 shows our first atomic action.  We have created a new {\tt
  DiffNode} $d_A$ and a new {\tt Array} $C$ and linked them between
$A$ and its cache.  The {\tt DiffNode} $d_A$ contains the value for
element $x$ contained in the cache, $z$, so there is no change in
the value of $A$.

We continue swinging pointers until step 5, when can finally set
the element $x$ in the cache to $y$.  We perform this operation with a
DCAS operation which checks that $\fref{C}{node}$ is still pointing to
the cache as we expect.  Note that a concurrent rotation would swing
$\fref{C}{node}$ in its step 1.  In general, therefore, the location
pointing to the cache serves as a reservation on the cache.

Thus in step 6 we need to again use DCAS to simultaneously swing
$\fref{C}{node}$ away 
from the cache as we swing $\fref{B}{node}$ to point to the cache.

Figure~\ref{fig:fun-impl} presents pseudocode for \funcname{FA-Rotate},
\funcname{FA-Read}, and \funcname{FA-Update}.  Note that \funcname{FA-Read} also
uses the cache pointer as a reservation,
double-checking the cache pointer after it finishes its read to ensure that the
cache hasn't been stolen from it.

Let us now consider cuts, where \funcname{FA-Read} clones the cache
instead of performing a rotation.   Cuts also check the cache pointer
to protect against concurrent rotations.  But what if the cut occurs
while a rotation is mutating the cache in step 5?  In this case the
only array adjacent to the root is $B$, so the cut must be occurring
during an invocation of $\funcname{FA-Rotate}(B)$.  But then the
differential node $d_B$ will be applied after the cache is copied,
which will safely overwrite the mutation we were concerned about.

Note that with hardware support for small transactions \cite{HerlihyMo93}
we could cheaply perform the entire rotation atomically, instead of
using this six-step approach.


%% \subsection{Optimizations}
%% Re-rooting is the most complicated part of the functional array
%% algorithm.  It can be optimized in a number of ways.  For example,

%% %unsync rotate for transaction-local data.
%% The first is to recognize that some array versions can only be seen by
%% a single thread.  In particular, when we are working on an {\sl
%%   IN-PROGRESS} operation, all array versions which it creates are
%% unreachable from other threads until the operation is committed.
%% We can add a field {\tt creator} to the {\tt Array} object which records what
%% operation created that version.  If the {\tt creator} field of both
%% $A$ and $B$ contains our own \var{oid} when we begin a rotate, we know
%% that these versions are both thread local

%% .. uh, no.  This doesn't work.

%% % scales method of tagging fields.



