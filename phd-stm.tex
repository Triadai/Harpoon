\secput{efficient}{Designing Efficient Transactions}
In this section I briefly describe some desired properties of the
software transaction system being designed for this thesis.

\subsection{Field Flags}\label{sec:flagfield}
\note{Missing: performance numbers for adding check.  Use ``no trans''
version of transaction app and add check into the access functions.}
We would like non-transactional code to execute with minimal overhead,
however, transactions should still appear atomic to non-transactional
code.  My basic mechanism is loosely based on the
distributed shared memory implementation of Scales and Gharachorloo
\cite{ScalesGh97}.  I pick a special ``flag'' value, and
``cross-out'' locations currently involved in a transaction by
overwriting them with the flag value.  Reading or attempting to
overwrite a flagged value will indicate to non-transactional code
that exceptional processing is necessary; all other non-transactional
operations proceed as usual.

Note that this technique explicitly allows safe access to fields
involved in a transaction from non-transactional code.
\punt{
Ensuring that transactional updates remain atomic to non-transactional
code eases ``transactification'' and 
Key idea is to allow safe access by non-transactional code, so as to
allow transactification.
}

\epsfigput{bloat}{Application slowdown with increasing object bloat
for the SPECjvm98 benchmark applications.}
\subsection{Object expansion}
I will need to add some additional information to each object to
track transaction state.  I measured the slowdown caused by various
amounts of object ``bloat'' to determine reasonable bounds on the
size of this extra information.  \figref{bloat} presents these
results for the SPECjvm98 applications; I determined that two words
(eight bytes) of additional storage per object would not impact
performance unreasonably.  This amount of bloat causes a geometric
mean of 2\% slowdown on these benchmarks.

\begin{figure}\sis%
\begin{center}
\begin{tabular}{lrrrr}
        & transactional & transactional\\
program & memory ops    & stores \% \\\hline
{\tt 201\_compress} & 50,029 & 26.2\% \\
{\tt 202\_jess} & 36,701,037 & 0.6\% \\
{\tt 205\_raytrace} & 7,294,648 & 23.2\% \\
{\tt 209\_db} & 195,374,420 & 6.3\% \\
{\tt 213\_javac} & 472,134,289 & 22.9\% \\
{\tt 222\_mpegaudio} & 41,422 & 18.6\% \\
{\tt 228\_jack} & 63,912,386 & 17.0\% \\
\end{tabular}
\end{center}
\caption{Comparison of loads and stores inside transactions for the
  SPECjvm98 benchmark suite, full input runs.}
\label{fig:writepercent}
\end{figure}
\subsection{Reads vs. Writes}
\figref{writepercent} shows that transactional reads typically
outnumber transactional writes by at least 4 to 1; in some cases reads
outnumber writes by over 100 to 1.  It is worthwhile, therefore, to
make reads more efficient than writes.  In particular, since the
flag-overwrite technique discussed in \secref{flagfield} requires us
to allocate additional memory to store the ``real'' value of the
field, I wish to avoid this process for transactional reads,
reserving the extra allocation effort for transactional writes.

\punt{
\subsection{Large objects}
See \secref{properties}.
}

